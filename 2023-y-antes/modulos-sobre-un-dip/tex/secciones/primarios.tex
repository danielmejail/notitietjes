\theoremstyle{plain}
\newtheorem{lemaPrimarios}{Lema}[section]
\newtheorem{teoDescomposicionPrimaria}[lemaPrimarios]{Teorema}
\newtheorem{coroDescomposicionPrimaria}[lemaPrimarios]{Corolario}
% \newtheorem{coroDivisoresElementales}[lemaPrimarios]{Corolario}
% \newtheorem{lemaDescomposicionDeJordan}[lemaPrimarios]{Lema}
% \newtheorem{teoDescomposicionDeJordan}[lemaPrimarios]{Teorema}

\theoremstyle{definition}
\newtheorem{ejemploPrimariosDescomposicion}[lemaPrimarios]{Ejemplo}
\newtheorem{defPrimario}[lemaPrimarios]{Definici\'{o}n}
\newtheorem{obsPrimario}[lemaPrimarios]{Observaci\'{o}n}
% \newtheorem{obsDivisoresElementales}[lemaPrimarios]{Observaci\'{o}n}

%-----------

\begin{ejemploPrimariosDescomposicion}\label{ejemplo:primarios:descomposicion}
	Supongamos que $m=a\,b$ con $(a,b)=1$ (coprimos), entonces existen
	$r,s\in\bb Z$ tales que $r\,a+s\,b=1$. Miramos la siguiente
	sucesi\'{o}n exacta:
	\begin{center}
		\begin{tikzcd}
			& & & \bb Z/b \arrow[dl,"{\multiplicar[a]}"'] & & \\
			0\arrow[r] & \bb Z/a \arrow[r,"{\multiplicar[b]}"] &
				\bb Z/m \arrow[rr,"{\reducirmod[b]}"'] & &
			\bb Z/b \arrow[r]
				\arrow[ul,"{\multiplicar[r]}"',"\sim"] & 0
		\end{tikzcd}
	\end{center}
	Dado que $r$ es una unidad m\'{o}dulo $b$, $\multiplicar[r]$ es un
	isomorfismo en $\bb Z/b$. Adem\'{a}s,
	\begin{align*}
		\reducirmod[b]\circ(\multiplicar[a]\circ
			\multiplicar[r]) & \,=\,\id[\bb Z/b]
		\text{ ,}
	\end{align*}
	%
	pues $r\,a\equiv 1\tmodulo[b]$. An\'{a}logamente, tenemos
	\begin{align*}
		\reducirmod[a]\circ(\multiplicar[b]\circ
			\multiplicar[s]) & \,=\,\id[\bb Z/a]
		\text{ .}
	\end{align*}
	%
	Pero tambi\'{e}n valen las igualdades
	\begin{align*}
		\reducirmod[a]\circ(\multiplicar[a]\circ
			\multiplicar[r]) \,=\,0 & \quad\text{y}\quad
		\reducirmod[b]\circ(\multiplicar[b]\circ
			\multiplicar[s]) \,=\,0
	\end{align*}
	%
	Definamos $\inc[a]=\multiplicar[b]\circ\multiplicar[s]$ e
	$\inc[b]=\multiplicar[a]\circ\multiplicar[s]$. Entonces, dado
	$x\in\bb Z$ un entero arbitario,
	\begin{align*}
		\inc[b]\circ\reducirmod[b](x) & \,=\,\multiplicar[a](
				\multiplicar[r](x\tmodulo[b]))
			\,=\,(r\,a)\,x\tmodulo[m]
		\text{ .}
	\end{align*}
	%
	De manera similar,
	\begin{math}
		\inc[a]\circ\reducirmod[a](x)=(s\,b)\,x\tmodulo[m]
	\end{math}. Sumando ambas expresiones, deducimos que
	\begin{align*}
		\inc[b]\circ\reducirmod[b]\,+\,
			\inc[a]\circ\reducirmod[a] & \,=\,
				\id[\bb Z/m]
		\text{ .}
	\end{align*}
	%
	En conclusi\'{o}n, vale la descomposici\'{o}n
	\begin{align*}
		\bb Z/m & \,\simeq\,\bb Z/a\,\oplus\,\bb Z/b
		\text{ .}
	\end{align*}
	%
\end{ejemploPrimariosDescomposicion}

Sea $D$ un DIP. El siguiente lema podr\'{\i}a pensarse como un refinamiento del
Lema fundamental~\ref{lema:ciclicos:fundamental} y generaliza la situaci\'{o}n
descripta en el Ejemplo~\ref{ejemplo:primarios:descomposicion}.

\begin{lemaPrimarios}\label{lema:primarios}
	Sea $A$ un $D$-m\'{o}dulo y sea $\nu\in D$ tal que
	$\Anulador(A)=\generado\nu$. Si $\nu=\lambda\,\kappa$, con $\kappa$ y
	$\lambda$ coprimos. Entonces
	\begin{align*}
		A & \,=\,\torsion[A]{\kappa}\,\oplus\,\torsion[A]{\lambda}
			\,=\,\lambda\cdot A\,\oplus\,\kappa\cdot A
		\text{ .}
	\end{align*}
	%
\end{lemaPrimarios}

\begin{proof}
	Como $\kappa$ y $\lambda$ son coprimos,
	$\generado\kappa+\generado\lambda=1$ y
	$A=\big(\kappa\cdot A\big)+\big(\lambda\cdot A\big)$. En particular,
	$\torsion[A]{\kappa}\cap \torsion[A]{\lambda}=0$. Pero
	$\kappa\cdot A\subset\torsion[A]{\lambda}$ y
	$\lambda\cdot A\subset\torsion[A]{\kappa}$, con lo cual,
	$A=\torsion[A]{\kappa}+\torsion[A]{\lambda}$ y, tambi\'{e}n,
	$\big(\kappa\cdot A\big)\cap\big(\lambda\cdot A\big)=0$.
\end{proof}

Dicho de otra manera, la descomposici\'{o}n del anulador minimal de un $D$-%
m\'{o}dulo en factores coprimos se traduce en una descomposici\'{o}n
an\'{a}loga del $D$-m\'{o}dulo.

\begin{defPrimario}\label{def:primario}
	Dado un elemento primo $p\in D$, se denomina \emph{$D$-m\'{o}dulo $p$-%
	primario} a todo $D$-m\'{o}dulo cuyos elementos tengan orden una
	potencia de $p$. Un \emph{$D$-m\'{o}dulo primario} es un m\'{o}dulo que
	es $p$-primario para alg\'{u}n primo $p$.
\end{defPrimario}

\begin{obsPrimario}\label{obs:primario}
	Todo $D$-m\'{o}dulo $p$-primario es de torsi\'{o}n. Si $A$ es $p$-%
	primario y $x\in A$, entonces $\Anulador(x)=\generado{p^e}$ para cierto
	$e\geq 1$, pero el exponente puede no estar acotado en $A$.
\end{obsPrimario}

\begin{teoDescomposicionPrimaria}[de descomposici\'{o}n primaria]%
	\label{thm:descomposicionprimaria}
	Sea $A$ un $D$-m\'{o}dulo de torsi\'{o}n f.g. y sea
	$\Anulador(A)=\generado\nu$ el anulador minimal de $A$. Si
	$\nu=p_1^{e_1}\cdots p_r^{e_r}$ es una descomposici\'{o}n de
	$\nu$ como producto de primos no asociados a potencias,%
	\footnote{
		Una expresi\'{o}n de la forma
		$\nu=p_1^{e_1}\cdots p_r^{e_r}$ es una descomposici\'{o}n o
		factorizaci\'{o}n de $\nu\in D$ como producto de primos no
		asociados a potencias, si los elementos de la lista
		$\lista{p}{r}$ son primos en $D$, coprimos de a pares y
		$\lista{e}{r}\geq 1$.
	}
	entonces
	\begin{equation}
		\label{eq:primarios:descomposicion}
		A \,=\,\primario[p_1]{A}\,\oplus\,\cdots\,\oplus\,
			\primario[p_r]{A}
		\text{ ,}
	\end{equation}
	%
	donde
	\begin{align*}
		\primario[p]{A} & \,=\,\bigcup_{k\geq 1}\,
			\Big\{x\in A\,:\,\Anulador(x)= \generado{p^k}\Big\}
		\,=\,\bigcup_{k\geq 1}\,\Big\{x\in A\,:\,p^k\,x=0\Big\}
	\end{align*}
	%
	es el subm\'{o}dulo $p$-primario maximal en $A$.
\end{teoDescomposicionPrimaria}

\begin{proof}
	Si $r=1$, $A=\primario[p_1]{A}$ y no hay nada m\'{a}s que probar. En
	otro caso, $\nu=\lambda\,\kappa$, con
	$\lambda=p_1^{e_1}\cdots p_{r-1}^{e_{r-1}}$ y
	$\kappa=p_r^{e_r}$. Por el Lema~\ref{lema:primarios},
	$A=\torsion[A]{\kappa}\,\oplus\,\torsion[A]{\lambda}$.
	Pero $\torsion[A]{\kappa}\subset\primario[p_r]{A}$. Esta inclusi\'{o}n
	es una igualdad, porque $\nu$ es el anulador minimal
	($\primario[p_r]{A}\subset \torsion[A]{\kappa}$). An\'{a}logamente,
	\begin{math}
		\Anulador(\torsion[A]{\lambda})=
			\generado{p_1^{e_1}\cdots p_{r-1}^{e_{r-1}}}
	\end{math} y vale que
	\begin{math}
		\primario[p_i]{\big(\torsion[A]{\lambda}\big)}=
			\primario[p_i]{A}
	\end{math}, para todo $i\leq r-1$. Inductivamente,
	\begin{math}
		\torsion[A]{\lambda}=\primario[p_1]{A}
			\,\oplus\,\cdots\,\oplus\, \primario[p_{r-1}]{A}
	\end{math}.
\end{proof}

Aplicando el Teorema~\ref{teo:torsion:estructura} a un m\'{o}dulo $p$-primario
f.g., cada sumando c\'{\i}clico en la descomposici\'{o}n es de orden una
potencia de $p$. Entonces, por ejemplo,
\begin{equation}
	\label{eq:primarioporfactoresinvariantes}
	\primario[p]{A} \,=\,C_1\,\oplus\,\cdots\,\oplus\,C_k
	\text{ ,}
\end{equation}
%
donde $\Anulador(C_i)=\generado{p^{d_i}}$ y $d_1\geq\cdots\geq d_k$.

\begin{coroDescomposicionPrimaria}\label{coro:descomposicionprimaria}
	Sea $A$ un $D$-m\'{o}dulo de torsi\'{o}n f.g. Entonces $A$ se
	descompone como suma directa de subm\'{o}dulos c\'{\i}clicos, cada uno
	de orden una potencia de un primo en $D$. La lista de los \'{o}rdenes
	de estos subm\'{o}dulos  es \'{u}nica, salvo permutaciones o
	reemplazo de un primo por un primo asociado.
\end{coroDescomposicionPrimaria}

% \begin{coroDivisoresElementales}\label{coro:divisoreselementales}
	% Sea $B\in F^{n\times n}$ una matriz cuadrada con coeficientes en un
	% cuerpo $F$. Existe una lista $L$ de potencias $f^{e}$ de polinomios
	% m\'{o}nicos irreducibles $f\in F[X]$, $e\geq 1$, tal que $B$ es similar
	% a la suma directa de las matrices compa\~{n}eras $\compa{f^{e}}$ de los
	% elementos de la misma. Cualquier otra lista de potencias de polinomios
	% m\'{o}nicos irreducibles (en $F[X]$) con esta propiedad es una
	% permutaci\'{o}n de $L$.
% \end{coroDivisoresElementales}
% 
% El cuerpo ``de base'' $F$ entra en juego en esta descomposici\'{o}n de una
% matriz $B$. Por un lado, la relaci\'{o}n de similitud mencionada en el
% corolario es realizada por una matriz con coeficientes en $F$. Por otra parte,
% los polinomios $f\in F[X]$ que son irreducibles sobre $F$ pueden dejar de serlo
% en una extensi\'{o}n $F'\supset F$. Los elementos de la lista $L$ (o sus
% ideales) se denominan \emph{divisores elementales de $B$} (o de $t_{B}$, o del
% m\'{o}dulo $A$, en el corolario previo).
% 
% \begin{obsDivisoresElementales}\label{obs:divisoreselementales}
	% La matriz compa\~{n}era $\compa{f^{e}}$ de una potencia $f^{e}$,
	% $e\geq 1$, de un polinomio m\'{o}nico irreducible $f\in F[X]$ posee un
	% \'{u}nico divisor elemental: $f^{e}$. La descomposici\'{o}n en sumandos
	% c\'{\i}clicos primarios (divisores elementales) es \'{u}nica y respeta
	% sumas directas. Adem\'{a}s, si $B$ y $L$ son como en el corolario
	% \ref{coro:descomposicionprimaria},
	% \begin{align*}
		% \caracteristico{B} \,=\,\pm\,\prod_{g\in L}\,g
		% & \quad\text{y}\quad
		% \minimal{B} \,=\,\mathsf{mcm}(L)
		% \text{ .}
	% \end{align*}
	% %
% \end{obsDivisoresElementales}
% 
% \begin{lemaDescomposicionDeJordan}\label{lema:descomposiciondejordan}
	% Sea $f=X-\lambda\in F[X]$ un polinomio m\'{o}nico, lineal (en
	% particular irreducible). Sea $C$ un $F[X]$-m\'{o}dulo c\'{\i}clico
	% $f$-primario. Se $f^{e}=(X-\lambda)^{e}$, $e\geq 1$, su orden. Existe
	% una base de $V=V_{C}$ tal que la matriz del endomorfismo
	% correspondiente a $t_{X}:\,c\mapsto cX$ con respecto a la misma sea una
	% matriz de Jordan \emph{elemental} de tama\~{n}o $e\times e$.
% \end{lemaDescomposicionDeJordan}
% 
% \begin{proof}
	% Como $f^{e}=(X-\lambda)^{e}$ es el orden de $C$, si $c_{0}\in C$ es un
	% generador de $C$ en tanto $F[X]$-m\'{o}dulo, entonces el conjunto
	% $\{c_{0},\,c_{0}\,(X-\lambda),\,\dots,\,c_{o}\,(X-\lambda)^{e-1}\}$ es
	% una $F$-base de $V_{C}$. La matriz de $t_{X}$ con respecto a esta base
	% est\'{a} dada por
	% \begin{equation}
		% \label{eq:bloquedejordanelemental}
		% [t_{X}] \,=\,
			% \begin{bmatrix}
				% \lambda & & & \\
				% 1 & \lambda & & \\
				% & \ddots & \ddots & \\
				% & & 1 & \lambda
			% \end{bmatrix}
		% \text{ .}
	% \end{equation}
	% %
% \end{proof}
% 
% \begin{teoDescomposicionDeJordan}[Descomposici\'{o}n de Jordan]%
	% \label{thm:descomposiciondejordan}
	% Sea $B\in\MM[n\times n]{F}$ una matriz cuadrada con coeficientes en un
	% cuerpo $F$. Si el polinomio caracter\'{\i}stico $\caracteristico{B}$ se
	% factoriza como producto de factores lineales en $F[X]$, entonces $B$ es
	% similar a una matriz de Jordan, es igual a la suma directa de matrices
	% de Jordan elementales asociadas a cada uno de los divisores elementales
	% de $B$. Esta matriz se denomina \emph{forma de Jordan de $B$}. Si
	% $(X-\lambda)^{e}$ es un divisor elemental de $B$, cualquier forma de
	% Jordan de $B$ posee un bloque de Jordan como en
	% \eqref{eq:bloquedejordanelemental}. La forma de Jordan de $B$ es
	% \'{u}nica, salvo permutaciones de los bloques.
% \end{teoDescomposicionDeJordan}
