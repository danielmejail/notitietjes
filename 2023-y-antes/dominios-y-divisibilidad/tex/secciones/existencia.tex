\theoremstyle{definition}
\newtheorem{obsExistencia}{Observaci\'{o}n}[section]
\newtheorem{ejemploExistencia}[obsExistencia]{Ejemplo}
\newtheorem{obsAsociados}[obsExistencia]{Observaci\'{o}n}

\theoremstyle{plain}
\newtheorem{coroExistencia}[obsExistencia]{Corolario}
\newtheorem{teoExistenciaYUnicidad}[obsExistencia]{Teorema}
\newtheorem{teoAsociadosFactorial}[obsExistencia]{Teorema}

%-------------

Como mencionamos en la secci\'{o}n~\ref{sec:factoriales}, no es cierto que en
un monoide cancelativo todo elemento admita una factorizaci\'{o}n como producto
de irreducibles. Dados un anillo $A$ y un elemento $a_0\in A$, si no fuese
irreducible, entonces existir\'{\i}an $a_1,b_1\in A$ tales que $a_0=a_1\,b_1$.
Si $a_1$, por ejemplo, tampoco fuese irreducible, $a_1=a_2\,b_2$, para ciertos
$a_2,b_2\in A$. Supongamos que existe una sucesi\'{o}n $\{a_i\}_{i\geq 0}$ de
elementos no nulos de $A$ tales que, para $i\geq 0$, existe una
factorizaci\'{o}n $a_{i+1}=a_i\,b_i$, para ciertos $b_i\in A$. Una sucesi\'{o}n
as\'{\i} da lugar a una cadena de ideales (principales) en $A$:
\begin{equation}
	\label{eq:cadenadedivisores}
	\generado{a_0}\,\subset\,\generado{a_1}\,\subset\,\cdots\,
		\subset\,A
	\text{ .}
\end{equation}
%

Un anillo $A$ se dice \emph{noetheriano (a izquierda)}, si satisface la
\emph{condici\'{o}n de cadenas ascendentes} de ideales (a izquierda), es decir,
dada una sucesi\'{o}n de ideales
\begin{align*}
	I_0 & \,\subset\,I_1\,\subset\,\cdots\,\subset\,A
\end{align*}
%
existe $k\geq 0$ tal que $I_i=I_k$ para todo $i\geq k$. Si $A$ es noetheriano,
la cadena \eqref{eq:cadenadedivisores} debe estabilizarse a partir de cierto
punto, es decir, $\generado{a_i}=\generado{a_k}$, para todo $i\geq k$, digamos.
Si $A=D$ es un dominio \'{\i}ntegro noetheriano, entonces $a_i\sim a_k$, para
$i\geq k$.

\begin{coroExistencia}\label{coro:existencia}
	Sea $D$ un dominio \'{\i}ntegro y sea $M=D\setmin\{0\}$ el monoide
	multiplicativo de elementos no nulos. La condici\'{o}n de cadenas de
	divisores en el monoide $M$ equivale a la condici\'{o}n de cadenas
	ascendentes \emph{de ideales principales} en $D$. En particular, si $D$
	es un dominio \'{\i}ntegro noetheriano, entonces el monoide $M$
	satisface la condici\'{o}n de cadenas de divisores.
\end{coroExistencia}

\begin{obsExistencia}\label{obs:existencia}
	En un monoide conmutativo y cancelativo que satisface la condici\'{o}n
	de cadenas de divisores, todo elemento que no es una unidad, admite al
	menos una factorizaci\'{o}n como producto de irreducibles. Para
	demostrar esta afirmaci\'{o}n, ser\'{a} suficiente probar que todo
	elemento distinto de una unidad posee, al menos, un factor irreducible.
	Sea $M$ es un monoide conmutativo y cancelativo. Si existe $a\in M$
	distinto de una unidad que no admite factores irreducibles, es posible
	definir una sucesi\'{o}n $\{a_i\}_{i\geq 0}$ de elementos de $M$ que
	verifica $a_{i+1}|a_i$ propiamente. En primer lugar, $a_0=a$ no es
	irreducible, con lo cual existen factores propios $a_1$ y $b_1$ de $a$
	(no asociados a $a$, ni unidades) y, como $a$ no admite factores
	irreducibles, $a_1$ no es irredcible. Dada una lista $\lista{a}{i}$ tal
	que cada elemento es un divisor propio del elemento anterior, como
	$a_i$ es factor de $a$, no puede ser irreducible y existe una
	factorizaci\'{o}n $a_i=a_{i+1}\,b_{i+1}$, con $a_{i+1}$ y $b_{i+1}$
	factores propios de $a_i$. En definitiva, existe una cadena de
	divisores propios en $M$.
\end{obsExistencia}

En particular, en un dominio \'{\i}ntegro noetheriano, todo elemento no nulo
admite \emph{alguna} factorizaci\'{o}n como producto de irreducibles. Aun
as\'{\i}, la existencia de factorizaciones no es una propiedad exclusiva de
los dominios noetherianos.

\begin{ejemploExistencia}\label{ejemplo:existencia:nonoetheriano}
	El anillo $A=k[X_1,\,X_2,\,\dots]$ de polinomios en infinitas
	indeterminadas con coeficientes en un cuerpo $k$ no es noetheriano,
	pero todo elemento admite una factorizaci\'{o}n como producto de
	irreducibles. M\'{a}s aun, dicha factorizaci\'{o}n es \'{u}nica, ya
	que, si $f\in A$, entonces $f$ pertenece a alguno de los subanillos
	$k[\lista{X}{n}]$ y cada uno de \'{e}stos es un dominio de
	factorizaci\'{o}n \'{u}nica.%
	\footnote{
		C.f. la Definici\'{o}n~\ref{def:factorial}.
	}
	% El anillo $A$ es un dominio no noetheriano, pero el monoide
	% $A\setmin\{0\}$ satisface la condici\'{o}n de cadenas de divisores.
\end{ejemploExistencia}

\begin{teoExistenciaYUnicidad}\label{teo:existenciayunicidad}
	Sea $M$ un monoide conmutativo y cancelativo. Si $M$ satisface la
	condici\'{o}n de cadenas de divisores y la condici\'{o}n de primalidad,
	entonces $M$ es factorial.
\end{teoExistenciaYUnicidad}

\begin{proof}
	Tenemos que probar que todo elemento $a\in M$ posee una
	factorizaci\'{o}n como producto de irreducibles y que dos
	factorizaciones son esencialmente iguales, es decir, difieren
	solamente en cambiar factores irreducibles por asociados e intercambiar
	el orden de los factores. Por la Observaci\'{o}n~\ref{obs:existencia},
	como $M$ satisface la condici\'{o}n de cadenas de divisores, dado
	$a\in M$ sabemos que existe al menos una factorizaci\'{o}n
	$a=p_1\cdots p_r$ en donde los $p_i$ son irreducibles. Sea
	$a=p_1'\cdots p_s'$ alguna otra factorizaci\'{o}n. Si $r=1$, entonces,
	por la condici\'{o}n de primalidad, $p_1$ es primo y divide a un
	producto de irreducibles $p_1'\cdots p_s'$. Como $p_1$ es primo,
	divide a $p_i'$ para alg\'{u}n $i$. Sin p\'{e}rdida de generalidad,
	podemos asumir que $i=1$, es decir, $p_1|p_1'$.
	Ahora, la igualdad $p_1=p_1'\cdots p_s'$ implica que $p_1'$
	tambi\'{e}n divide a $p_1$.
	Ahora, como $p_1'$ es irreducible y $p_1$ no es unidad, $p_1\sim p_1'$.
	Si, adem\'{a}s, $s\geq 2$, cancelando,
	\begin{align*}
		1 & \,=\,u\,p_2'\cdots p_s'
		\text{ ,}
	\end{align*}
	%
	para cierta unidad $u$. Pero esto contradice la irreducibilidad de los
	$p_i'$ (por definici\'{o}n, no son unidades). Entonces $r=s=1$ y
	$p_1\sim p_1'$, lo que significa que las factorizaciones son
	esencialmente la misma.

	Supongamos que $r\geq 2$ (por simetr\'{\i}a $s\geq 2$, tambi\'{e}n). De
	nuevo, $p_1|p_1'\cdots p_s'$ implica que $p_1$ divide a alguno de los
	$p_i'$. Permutando los factores, podemos asumir que $p_1|p_1'$. Como
	$p_1'$ es irreducible, $p_1\sim p_1'$. Cancelando, obetenemos la
	igualdad
	\begin{align*}
		p_2\cdots p_r & \,=\,u\,p_2'\cdots p_s'
		\text{ ,}
	\end{align*}
	%
	para cierta unidad $u$. Inductivamente, $r=s$ y existe una
	permutaci\'{o}n $i\mapsto j(i)$ tal que $p_i'\sim p_{j(i)}$.
\end{proof}

\begin{obsAsociados}\label{obs:asociados}
	Sea $M$ un monoide conmutativo y cancelativo y sean $a,a',b,b'\in M$
	tales que $a\sim a'$ y $b\sim b'$. Por definici\'{o}n, existen unidades
	$u,v\in M$ que verifican que $a=u\,a'$ y $b=v\,b'$. Tomando el
	producto, se ve que $a\,b=(u\,v)\,a'\,b'$. Pero $u\,v\in M$ es una
	unidad. En definitiva, $a\,b\sim a'\,b'$. Esto prueba que la
	relaci\'{o}n entre elementos de $M$ de ser asociados es una
	relaci\'{o}n de congruencia.
	% En particular, queda definido el monoide
	% cociente $M/\sim$ de clases m\'{o}dulo asociados.
	% 
	El cociente $M/\sim$ de un monoide conmutativo y cancelativo $M$ es un
	monoide conmutativo y cancelativo, tambi\'{e}n. Los elementos de
	$M/\sim$ son las clases $\clase a=\{b\in M\,:\,b\sim a\}$ de elementos
	de $M$ m\'{o}dulo asociados en $M$. Por la Observaci\'{o}n~%
	\ref{obs:irreducible}, las unidades de $M$ pertenecen a una misma
	clases. Esta clase es $\clase 1$, s\'{o}lo contiene unidades y,
	adem\'{a}s, es el elemento neutro para el producto en $M/\sim$ dado por
	$\clase a\,\clase b=\clase{a\,b}$. Una igualdad de la forma
	\begin{align*}
		\clase a & \,=\,\clase b\,\clase c
	\end{align*}
	%
	equivale a $a\sim b\,c$. Por lo tanto,
	\begin{itemize}
		\item la \'{u}nica unidad en $M/\sim$ es $\clase 1$,
		\item los irreducibles en $M/\sim$ son las clases $\clase p$,
			donde $p\in M$ es irreducible y
		\item si $\clase a$ y $\clase b$ son asociados \emph{en %
			$M/\sim$}, entonces $\clase a=\clase b$.
	\end{itemize}
	%
\end{obsAsociados}

La existencia de factorizaciones en un monoide se puede interpretar en
t\'{e}rminos de generadores. En un monoide conmutativo y cancelativo, todo
elemento distinto de una unidad admite una factorizaci\'{o}n como producto de
irreducibles, si y s\'{o}lo si el conjunto de elementos irreducibles genera el
monoide. En tal caso, podr\'{\i}a haber cierta ambig\"{u}edad en la
factorizaci\'{o}n de un elemento, es decir, la misma podr\'{\i}a no ser
\'{u}nica (ni esencialmente \'{u}nica). Un primer paso hacia la eliminaci\'{o}n
de esta ambig\"{u}edad es pasar al monoide cociente. Los elementos de un
monoide conmutativo y cancelativo $M$ admiten factorizaciones como productos de
irreducibles, si y s\'{o}lo si los elementos (clases) de $M/\sim$ admiten
factorizaciones como productos de (clases de) irreducibles. Adem\'{a}s, por la
Observaci\'{o}n~\ref{obs:asociados}, las factorizaciones en $M$ son
esencialmente \'{u}nicas, si y s\'{o}lo si las factorizaciones en $M/\sim$ son
\'{u}nicas, salvo por el orden de los factores.

\begin{teoAsociadosFactorial}\label{teo:asociados:factorial}
	Un monoide conmutativo y cancelativo $M$ es factorial, si y s\'{o}lo si
	el monoide cociente $M/\sim$ es el monoide abeliano libre en
	$\big\{\clase p\,:\,p\in M\text{ irreducible}\big\}$. En tal caso, si
	$I\subset M$ es un conjunto de representantes de las clases de
	irreducibles m\'{o}dulo asociados y $U\subset M$ es el subgrupo de
	unidades,
	\begin{align*}
		M & \,=\,U\times \generado I
		\text{ ,}
	\end{align*}
	%
	donde
	\begin{math}
		\generado I=\big\{p_1^{r_1}\cdots p_s^{r_s}\,:\,s\geq 1,\,
				p_i\in I,\,r_i\geq 1\big\}
	\end{math} es el submonoide generado por $I$.
\end{teoAsociadosFactorial}
