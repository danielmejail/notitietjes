\theoremstyle{plain}
\newtheorem{teoMCD}{Teorema}[section]
\newtheorem{propoMCD}[teoMCD]{Proposici\'{o}n}
\newtheorem{propoMCDPrimalidad}[teoMCD]{Proposici\'{o}n}
\newtheorem{lemaMCD}[teoMCD]{Lema}
\newtheorem{teoFactorialEquivalencias}[teoMCD]{Teorema}

\theoremstyle{definition}
\newtheorem{defMCD}[teoMCD]{Definici\'{o}n}
\newtheorem{obsMCD}[teoMCD]{Observaci\'{o}n}
\newtheorem{obsMCDFactorizacion}[teoMCD]{Observaci\'{o}n}
\newtheorem{defMCDCoprimos}[teoMCD]{Definici\'{o}n}
\newtheorem{obsMCDCoprimos}[teoMCD]{Observaci\'{o}n}

%-------------

\begin{defMCD}\label{def:mcd}
	Sea $M$ un monoide conmutativo y sean $a,b\in M$. Un \emph{m\'{a}ximo %
	com\'{u}n divisor} (M.C.D.) de $a$ y $b$ es un elemento $d\in M$ que
	verifica:
	\begin{enumerate}
		\item $d|a$ y $d|b$ y
		\item si $c|a$ y $c|b$, entonces $c|d$.
	\end{enumerate}
	%
	Denotamos por $\mcd{a}{b}$ un M.C.D. de $a$ y $b$.
\end{defMCD}

\begin{obsMCD}\label{obs:mcd}
	Si existe un M.C.D. para $a,b\in M$, entonces es \'{u}nico, salvo
	asociados. Por esta raz\'{o}n, si $d$ es un M.C.D. para $a$ y $b$,
	escribimos $\mcd{a}{b}\sim d$. Adem\'{a}s, si $a\sim a'$, entonces
	existe un M.C.D. para $a$ y $b$, si y s\'{o}lo si existe uno para $a'$
	y $b$, pues $d|a$, si y s\'{o}lo si $d|a'$. En tal caso,
	$\mcd{a}{b}\sim\mcd{a'}{b}$.
\end{obsMCD}

Hay dos casos en los que est\'{a} garantizada la existencia de un M.C.D.

\begin{propoMCD}\label{propo:mcd}
	Sea $M$ un monoide conmutativo y cancelativo. Entonces, para todo
	$b\in M$,
	\begin{enumerate}
		\item\label{propo:mcd:unidad}
			si $u\in M$ es una unidad, existe un M.C.D. para $u$ y
			$b$ y $\mcd{u}{b}\sim 1$;
		\item\label{propo:mcd:irreducible}
			si $p\in M$ es irreducible, existe un M.C.D. para $p$ y
			$b$ y
			\begin{align*}
				\mcd{p}{b} & \,\sim\,
					\begin{cases}
						p & \text{si }p\mid b
							\text{ ,}\\
						1 & \text{si }p\nmid b
							\text{ .}
					\end{cases}
			\end{align*}
			%
	\end{enumerate}
	%
\end{propoMCD}

\begin{proof}
	Sean $u,p,b\in M$, $u$ unidad, $p$ irreducible y $b$ arbitrario.
	Entonces $1|u$ y $1|b$. Si $c|u$ y $c|b$, en particular, $c|u$. Como
	$u\sim 1$, debe cumplirse que $c|1$, con lo cual $1$ es un M.C.D. para
	$u$ y $b$. En cuanto al irreducible, si $p|b$, entonces $p|p$ y $p|b$.
	Si $c|p$ y $c|b$, en particular $c|p$ y $p$ es un M.C.D. para $p$ y
	$b$. Si, en cambio, $p\nmid b$, entonces consideramos el divisor
	com\'{u}n $1$: $1|p$ y $1|b$. Si $c|p$ y $c|b$, en particular, $c|p$.
	Por definici\'{o}n, $c\sim1$ o $c\sim p$. Si fuese $c\sim p$,
	deducimos que $p|b$, contradiciendo $p\nmid b$. As\'{\i}, debe ser que
	$c\sim 1$ y, en particular, $c|1$. En definitiva, en este caso
	tambi\'{e}n existe un M.C.D.
\end{proof}
% Para agregar m\'{a}s irreducibles al M.C.D., hace falta suponer que todo
% irreducible es primo, con lo cual, las garant\'{\i}as se terminan ac\'{a}.

\begin{defMCDCoprimos}\label{def:mcd:coprimos}
	Dado un monoide conmutativo y cancelativo $M$, dos elementos
	$a,b\in M$ se dicen \emph{coprimos}, si existe un M.C.D. para $a$ y $b$
	y $\mcd{a}{b}\sim1$.
\end{defMCDCoprimos}

\begin{teoMCD}\label{teo:mcd}
	En un monoide factorial, todo par de elementos posee un M.C.D.
\end{teoMCD}

\begin{obsMCDFactorizacion}\label{obs:mcd:factorizacion}
	Sea $M$ un monoide factorial y sea $a\in M$. Por el Teorema~%
	\ref{teo:asociados:factorial}, existen una unidad $u$, irreducibles
	$\lista{p}{s}$ y enteros positivos $\lista{e}{s}\geq 1$ tales que
	$a=u\,p_1^{e_1}\cdots p_s^{e_s}$. Los divisores de $a$ en $M$ son
	exactamente los elementos de la forma:
	\begin{equation}
		\label{eq:mcd:factorizacion}
		c \,=\,w\,p_1^{h_1}\cdots p_s^{h_s}
		\text{ ,}
	\end{equation}
	%
	para cierta unidad $w$ y enteros $\lista{h}{s}$ que cumplen
	$0\leq h_i\leq e_i$.
\end{obsMCDFactorizacion}

\begin{proof}
	Sea $M$ un monoide factorial y sean $a,b\in M$. Si $a$ es una unidad o
	si $b$ es una unidad, entonces $\mcd{a}{b}=1$ es un M.C.D. de $a$ y de
	$b$. Supongamos que ni $a$, ni $b$ son unidades. Por el Teorema~%
	\ref{teo:asociados:factorial}, existen unidades $u,v$, irreducibles no
	asociados $\lista{p}{s}$ y enteros no negativos
	$\lista{e}{s},\lista{f}{s}\geq 0$ tales
	\begin{align*}
		a \,=\,u\,p_1^{e_1}\cdots p_s^{e_s} & \quad\text{y}\quad
			b\,=\,v\,p_1^{f_1}\cdots p_s^{f_s}
		\text{ .}
	\end{align*}
	%
	Por la Observaci\'{o}n~\ref{obs:mcd:factorizacion}, si $c|a$ y $c|b$,
	entonces $c=w\,p_1^{h_1}\cdots p_s^{h_s}$, para cierta unidad $w$ y
	enteros $\lista{h}{i}$ que cumplen $0\leq h_i\leq\min\,\{e_i,f_i\}$. En
	particular, nuevamente por la Observaci\'{o}n~%
	\ref{obs:mcd:factorizacion}, si $g_i=\min\,\{e_i,f_i\}$, entonces
	\begin{align*}
		d & \,=\,p_1^{g_1}\cdots p_s^{g_s}
	\end{align*}
	%
	es un M.C.D. de $a$ y de $b$.
\end{proof}

\begin{obsMCDCoprimos}\label{obs:mcd:coprimos}
	En un monoide factorial $M$, dos elementos $a$ y $b$ son coprimos, si
	y s\'{o}lo si $a$ es una unidad, $b$ es una unidad o no poseen
	factores irreducibles en com\'{u}n.
\end{obsMCDCoprimos}

\begin{defMCD}\label{def:mcd:condicion}
	Un monoide conmutativo y cancelativo \emph{satisface la condici\'{o}n %
	del m\'{a}ximo com\'{u}n divisor} (la condici\'{o}n del M.C.D.), si
	todo par de elementos admite un M.C.D.
\end{defMCD}

\begin{propoMCDPrimalidad}\label{propo:mcd:primalidad}
	Todo monoide conmutativo y cancelativo que satisface la condici\'{o}n
	del M.C.D. satisface la condici\'{o}n de primalidad.
\end{propoMCDPrimalidad}

Antes de demostrar la Proposici\'{o}n~\ref{propo:mcd:primalidad}, probamos
algunas propiedades del M.C.D.

\begin{lemaMCD}\label{lema:mcd}
	Sea $M$ un monoide conmutativo, cancelativo y que satisface la
	condici\'{o}n del M.C.D. Entonces,
	\begin{enumerate}
		\item\label{item:lema:mcd:multiples}
			dados $\lista{a}{r}\in M$ existe $d\in M$ que verifica
			que $d|a_i$ para todo $i$ y que, si $c|a_i$ para todo
			$i$, entonces $c|d$;
		\item\label{item:lema:mcd:multiples:biendefinido}
			dados $a,b,c\in M$,
			\begin{math}
				\mcd{\mcd{a}{b}}{c}\sim\mcd{a}{\mcd{b}{c}}
			\end{math};
		\item\label{item:lema:mcd:escalares}
			dados $a,b,c\in M$,
			\begin{math}
				\mcd{c\,a}{c\,b}\sim c\,\mcd{a}{b}
			\end{math};
		\item\label{item:lema:mcd:coprimos}
			dados $a,b,c\in M$, si $\mcd{a}{b}\sim1$, entonces
			$\mcd{a}{b\,c}\sim\mcd{a}{c}$.
	\end{enumerate}
	%
\end{lemaMCD}

\begin{proof}
	Sea $d_0=a_1$ y, en general, $d_i=\mcd{d_{i-1}}{a_i}$. Entonces
	$d=d_r|d_{r-1}|\cdots|d_2|d_1$. En particular, como $d_i|a_i$, vale que
	$d|a_i$ para todo $i$. Adem\'{a}s, si $c|a_i$ para todo $i$, entonces
	$c|d_i$ para todo $i$ y, por lo tanto, $c|d_r=d$.

	Si $a,b,c\in M$, entonces $d=\mcd{\mcd{a}{b}}{c}$ y
	$d'=\mcd{a}{\mcd{b}{c}}$ verifican que $d|\mcd{a}{b}|a,b$ y $d|c$,
	$d'|a$ y $d'|\mcd{b}{c}|b,c$. Entonces $d|\mcd{b}{c}$ y, por lo tanto,
	$d|d'$. An\'{a}logamente, $d'|\mcd{a}{b}$ y, por lo tanto, $d'|d$.

	Sean, ahora, $d=\mcd{a}{b}$ y $e=\mcd{c\,a}{c\,b}$. Entonces,
	$c\,d|c\,a$ y $c\,d|c\,b$, con lo cual, $c\,d|e$. Esto quiere decir que
	existe $x\in M$ tal que $e=(c\,d)\,x$. Como $e|c\,a$, se deduce que
	$d\,x|a$, porque $M$ es cancelativo. An\'{a}logamente, $d\,x|b$ y, en
	consecuencia, $d\,x|\mcd{a}{b}=d$. En definitiva, $d\,x\sim d$ y $x$
	era una unidad de $M$. Es decir, $e\sim c\,d$.

	Si $\mcd{a}{b}\sim 1$, entonces
	$\mcd{a\,c}{b\,c}\sim c\,\mcd{a}{b}\sim c$. Por otro lado, en general,
	$\mcd{a}{a\,c}\sim a\,\mcd{1}{c}\sim a$. As\'{\i}, por la
	Observaci\'{o}n~\ref{obs:mcd},
	\begin{align*}
		\mcd{a}{b\,c} & \,\sim\,\mcd{\mcd{a}{a\,c}}{b\,c}\,\sim\,
			\mcd{a}{\mcd{a\,c}{b\,c}}\,\sim\,
			\mcd{a}{\mcd{a}{b}\,c}\,\sim\,\mcd{a}{c}
		\text{ .}
	\end{align*}
	%
\end{proof}

Si $d\in M$ satisface las condiciones del \'{\i}tem~%
\ref{item:lema:mcd:multiples}, decimos que $d$ es un M.C.D. para $\lista{a}{r}$
y escribimos $d\sim\big(a_1:\,\cdots\,:a_r\big)$. El \'{\i}tem~%
\ref{item:lema:mcd:multiples:biendefinido} muestra que no importa el orden en
que se toman los divisores sucesivos.

\begin{proof}[Demostraci\'{o}n de \ref{propo:mcd:primalidad}]
	Sea $M$ un monoide conmutativo, cancelativo que satisface la
	condici\'{o}n del M.C.D. Si $p\in M$ es un irreducible y $a,b\in M$ son
	tales que $p\nmid a$ y $p\nmid b$, entonces, por la Proposici\'{o}n~%
	\ref{propo:mcd}, $\mcd{p}{a}\sim1$ y $\mcd{p}{b}\sim 1$. Por el
	\'{\i}tem~\ref{item:lema:mcd:coprimos} del Lema~\ref{lema:mcd}, vale
	que $\mcd{p}{a\,b}\sim 1$, tambi\'{e}n. Pero, nuevamente por la
	Proposici\'{o}n~\ref{propo:mcd}, $p\nmid a\,b$. En definitiva, $p$ es
	primo.
\end{proof}

\begin{teoFactorialEquivalencias}\label{teo:factorial:equivalencias}
	Sea $M$ un monoide conmutativo y cancelativo. Las siguientes
	propiedades son equivalentes:
	\begin{enumerate}
		\item\label{item:factorial:equivalencias:i}
			$M$ es factorial;
		\item\label{item:factorial:equivalencias:ii}
			$M$ satisface la condici\'{o}n de cadenas de divisores
			y la condici\'{o}n de primalidad;
		\item\label{item:factorial:equivalencias:iii}
			$M$ satisface la condici\'{o}n de cadenas de divisores
			y la condici\'{o}n del M.C.D.
	\end{enumerate}
	%
\end{teoFactorialEquivalencias}

\begin{proof}
	El Teorema~\ref{teo:factoriales:condiciondecadena} y el Teorema~%
	\ref{teo:factoriales:condiciondeprimalidad} muestran que
	\ref{item:factorial:equivalencias:i} implica
	\ref{item:factorial:equivalencias:ii}. El Teorema~%
	\ref{teo:existenciayunicidad} muestra que
	\ref{item:factorial:equivalencias:ii} implica
	\ref{item:factorial:equivalencias:i}. La Proposici\'{o}n~%
	\ref{propo:mcd:primalidad} muestra que
	\ref{item:factorial:equivalencias:iii} implica
	\ref{item:factorial:equivalencias:ii} y el Teorema~\ref{teo:mcd}, junto
	con el Teorema~\ref{teo:factoriales:condiciondecadena}, muestra que
	\ref{item:factorial:equivalencias:i} implica
	\ref{item:factorial:equivalencias:iii}.
\end{proof}
