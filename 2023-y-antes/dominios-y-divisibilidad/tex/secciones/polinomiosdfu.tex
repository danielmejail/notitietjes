\theoremstyle{plain}
\newtheorem{lemaGauss}{Lema}[section]
\newtheorem{propoContenido}[lemaGauss]{Proposici\'{o}n}
\newtheorem{lemaContenido}[lemaGauss]{Lema}
\newtheorem{coroContenido}[lemaGauss]{Corolario}
\newtheorem{lemaPolinomiosDFUIrreducible}[lemaGauss]{Lema}
\newtheorem{teoPolinomiosDFU}[lemaGauss]{Teorema}
\newtheorem{coroPolinomiosDFU}[lemaGauss]{Corolario}

\theoremstyle{definition}
\newtheorem{defCuerpoDeFracciones}[lemaGauss]{Definici\'{o}n}
\newtheorem{obsCuerpoDeFracciones}[lemaGauss]{Observaci\'{o}n}
\newtheorem{defContenido}[lemaGauss]{Definici\'{o}n}
\newtheorem{defPrimitivo}[lemaGauss]{Definici\'{o}n}
\newtheorem{obsPolinomiosUnidades}[lemaGauss]{Observaci\'{o}n}
\newtheorem{obsPrimitivos}[lemaGauss]{Observaci\'{o}n}
\newtheorem{obsGauss}[lemaGauss]{Observaci\'{o}n}

%-------------

En esta secci\'{o}n introducimos el \emph{contenido} de un polinomio con
coeficientes en un D.F.U. y demostramos el Lema de Gauss (Lema~%
\ref{lema:gauss}), con el objetivo de probar que el anillo de polinomios $D[X]$
es D.F.U., si $D$ lo es. Un corolario importante de este resultado es que todo
polinomio irreducible en $D[X]$ es irreducible sobre el cuerpo de fracciones.

Dado un anillo $A$ y un polinomio $f\in A[X]$, diremos que $f$ satisface una
propiedad ``\emph{sobre} $A$'' o ``\emph{en} $A[X]$'', si $f$ posee la
propiedad en tanto elemento del anillo de polinomios $A[X]$.

\begin{defCuerpoDeFracciones}\label{def:cuerpodefracciones}
	Dado un dominio \'{\i}ntegro $D$, existe un cuerpo,
	$\cuerpocociente D$, y un morfismo inyectivo de anillos
	$\inc[D]:\,D\hookrightarrow\cuerpocociente D$ tal que, para todo
	anillo $B$ y todo morfismo de anillos $\varphi:\,D\rightarrow B$ tal
	que $\varphi(D\setmin\{0\})\subset B^\times$, existe un \'{u}nico
	morfismo de anillos $\lconj\varphi:\,\cuerpocociente D\rightarrow B$
	tal que
	\begin{align*}
		\lconj\varphi\circ\inc[D] & \,=\,\varphi
		\text{ .}
	\end{align*}
	%
	% \begin{center}
		% \begin{tikzcd}
			% \cuerpocociente D \arrow[r,"\lconj\varphi"] & B \\
			% D \arrow[u,hook,"{\inc[D]}"] \arrow[ur,"\varphi"'] &
		% \end{tikzcd}
		% % \qquad
	% \end{center}
	El cuerpo $\cuerpocociente D$ se denomina \emph{cuerpo cociente} o
	\emph{cuerpo de fracciones} de $D$; es \'{u}nico salvo \'{u}nico
	isomorfismo.
\end{defCuerpoDeFracciones}
% \begin{wrapfigure}{r}{.5\textwidth}
	% \centering
	% \begin{tikzcd}
		% \cuerpocociente(D) \arrow[r,"\lconj\varphi"] & B \\
		% D \arrow[u,hook,"{\inc[D]}"'] \arrow[ur,"\varphi"'] &
	% \end{tikzcd}
% \end{wrapfigure}
% %

\begin{obsPolinomiosUnidades}\label{obs:polinomios:unidades}
	Si $D$ es un dominio \'{\i}ntegro, entonces $f\in D[X]$ es una unidad,
	si y s\'{o}lo si $f$ es constante e invertible en $D$. Es decir,
	$D[X]^\times=D^\times$. Adem\'{a}s, si $p\in D$ es irreducible, sigue
	siendo irreducible en $D[X]$.
\end{obsPolinomiosUnidades}

En lo que resta de esta secci\'{o}n asumimos que $D$ es un dominio \'{\i}ntegro
tal que $D\setmin\{0\}$ satisface la condici\'{o}n del M.C.D. Diremos que
\emph{$D$ posee M.C.D.}

\begin{obsCuerpoDeFracciones}\label{obs:cuerpodefracciones}
	Dado $\gamma\in\cuerpocociente D$, existen, por definici\'{o}n,
	$a,b\in D$ tales que $\gamma=a/b$. Asumiendo que $D$ posee M.C.D.,
	podemos elegir $a$ y $b$ coprimos: si $d=\mcd{a}{b}$, entonces
	$a=d\,\tilde a$ y $b=d\,\tilde b$ y $\gamma=\tilde a/\tilde b$, pero
	$\mcd{\tilde a}{\tilde b}\sim1$, pues, si $c$ es un divisor com\'{u}n
	de $\tilde a$ y de $\tilde b$, $d\,c$ es un divisor com\'{u}n de $a$ y
	de $b$ y, por definici\'{o}n, $d\,c$ divide a $d$, lo que implica que
	$c\sim 1$.
\end{obsCuerpoDeFracciones}

\begin{defContenido}\label{def:contenido}
	Dado un polinomio no nulo $f\in D[X]$, definimos el \emph{contenido %
	de $f$} como el M.C.D. de sus coeficientes: si
	$f=a_0+a_1\,X+\,\cdots\,+a_n\,X^n\neq 0$, $a_i\in D$, entonces el
	contenido de $f$ es
	\begin{align*}
		\contenido f & \,:=\,\big(a_0\,:\,a_1\,:\,\cdots\,:\,a_n\big)
		\text{ .}
	\end{align*}
	%
	El contenido est\'{a} definido salvo asociados en $D$.
\end{defContenido}

\begin{defPrimitivo}\label{def:primitivo}
	Un polinomio $f\in D[X]$ es un \emph{polinomio primitivo}, si
	$\contenido f\sim 1$.
\end{defPrimitivo}

\begin{propoContenido}\label{propo:contenido}
	Sea $D$ un dominio que posee M.C.D. Entonces:
	\begin{enumerate}
		\item\label{item:contenido:i}
			todo polinomio no nulo $f\in D[X]$ se puede expresar
			como $f=c\,g$, donde $c\in D$ y $g\in D[X]$ es
			primitivo;
		\item\label{item:contenido:ii}
			si $a\in D$ y $f\in D[X]$, entonces
			$\contenido{a\,f}\sim a\,\contenido f$;
		\item\label{item:contenido:iii}
			una expresi\'{o}n para $f\in D[X]$ como en
			\ref{item:contenido:i} es \'{u}nica m\'{o}dulo
			asociados: si $f=d\,h$ con $d\in D$ y $h\in D[X]$
			primitivo, entonces $d\sim c$ en $D$ y $h\sim g$ en
			$D[X]$.
	\end{enumerate}
	%
\end{propoContenido}

\begin{proof}
	Para ver \ref{item:contenido:i}, basta con notar que, si
	\begin{math}
		g=\tilde a_0+\tilde a_1\,X+\,\cdots\,+\tilde a_n\,X^n
	\end{math} y $\tilde a_i=a_i/\contenido f$, entonces
	$f=\contenido f\,g$. El \'{\i}tem~\ref{item:contenido:ii} se deduce del
	Lema~\ref{lema:mcd}:
	\begin{align*}
		\contenido{a\,f} & \,=\,\big(a\,a_0\,:\,a\,a_1\,\cdots\,
			a\,a_n\big)\,\sim\,
			\big(a\,a_0\,:\,\big(a\,a_1\,:\,\cdots\,:\,a\,a_n\big)
				\big) \\
			& \,\sim\,\big(a\,a_0\,:\,a\,\big(a_1\,:\,\cdots\,a_n
				\big)\big)
				\,\sim\,a\,\big(a_0\,:\,a_1\,:\,\cdots\,:\,a_n
					\big)
		\text{ .}
	\end{align*}
	%
	Finalmente, si $c\,g=d\,h$ con $c,d\in D$ y $g,h\in D[X]$, entonces,
	tomando contenido,
	\begin{align*}
		c\,\contenido g & \,\sim\,\contenido{c\,g}
			\,=\,\contenido{d\,h}\,\sim\,d\,\contenido h
		\text{ .}
	\end{align*}
	%
	Si $g$ y $h$ son primitivos, entonces $c$ y $d$ son asocicados. En tal
	caso, existe $u\in D^\times$ tal que $c=u\,d$. De la igualdad
	$c\,g=d\,h$, se deduce que $u^{-1}\,g=h$. Pero $u^{-1}$ es unidad en
	$D$ y, por lo tanto, unidad en $D[X]$.
\end{proof}

\begin{lemaContenido}\label{lema:contenido}
	Dado $f\in \cuerpocociente D[X]$ no nulo, existen
	$\gamma\in\cuerpocociente D$ y $g\in D[X]$ primitivo tales que
	\begin{equation}
		\label{eq:contenido}
		f \,=\,\gamma\,g
		\text{ .}
	\end{equation}
	%
	Si $f=\delta\,h$ con $\delta\in\cuerpocociente D$ y $h\in D[X]$
	primitivo, entonces existe una unidad $u\in D^\times$ tal que
	$\gamma=u\,\delta$. En particular, $g\sim h$ en $D[X]$.
\end{lemaContenido}

\begin{proof}
	Los coeficientes de $f$ pertenecen al cuerpo de fracciones de $D$. Si
	$b\,f\in D[X]$, tomando contenido, $a=\contenido{b\,f}\in D$ y
	\begin{align*}
		b\,f & \,=\,a\,g
		\text{ ,}
	\end{align*}
	%
	para alg\'{u}n polinomio primitivo $g\in D[X]$. As\'{\i},
	$f=\gamma\,g$, con $\gamma=a/b$, de lo que se deduce la existencia de
	una ``factorizaci\'{o}n'' de la forma \eqref{eq:contenido}. Sean,
	ahora, $\gamma,\delta\in\cuerpocociente D$ y $g,h\in D[X]$ polinomios
	primitivos tales que
	\begin{align*}
		\gamma\,g & \,=\,\delta\,h
		\text{ .}
	\end{align*}
	%
	Si $\gamma=a/b$ y $\delta=c/d$, deducimos que
	\begin{align*}
		a\,d & \,\sim\,\contenido{(a\,d)\,g} \,=\,
			\contenido{(b\,c)\,h} \,\sim\,b\,c
		\text{ .}
	\end{align*}
	%
	Si suponemos, adem\'{a}s, que $\mcd{a}{b}\sim 1$ y $\mcd{c}{d}\sim 1$,
	entonces
	\begin{align*}
		\mcd{a}{c} & \,\sim\,\mcd{a}{b\,c}\,\sim\,\mcd{a}{a\,d}
			\,\sim\, a
		\text{ .}
	\end{align*}
	%
	An\'{a}logamente, $\mcd{a}{c}\sim c$, de lo que se deduce que
	$a\sim c$. De la misma manera, se deduce que $b\sim d$. As\'{\i},
	\begin{align*}
		\frac{a}{b} & \,=\,\frac{v\,c}{w\,d}\,=\,u\,\frac{c}{d}
		\text{ ,}
	\end{align*}
	%
	donde $u=w^{-1}\,v$.
\end{proof}

\begin{defContenido}\label{def:contenido:denominadores}
	Se define el \emph{contenido} de $f\in\cuerpocociente D[X]$, como
	cualquier constante $\gamma=\contenido f\in\cuerpocociente D$ que
	verifique que existe $g\in D[X]$ primitivo tal que $f=\gamma\,g$. El
	contenido est\'{a} bien definido salvo unidades en $D$.
\end{defContenido}

Esta definici\'{o}n coincide con la Definici\'{o}n~\ref{def:contenido} para
$f\in D[X]$. Como las unidades en $\cuerpocociente D[X]$ son las constantes no
nulas, deducimos el siguiente resultado.

\begin{coroContenido}\label{coro:contenido}
	Sean $f,g\in D[X]$ polinomios primitivos. Si $f\sim g$ sobre
	$\cuerpocociente D$, entonces $f\sim g$ en $D[X]$.
\end{coroContenido}

\begin{obsPrimitivos}\label{obs:primitivos:descomposicion}
	Sea $f\in D[X]$ un polinomio primitivo. Si $f$ es constante, entonces
	$f=\contenido f\sim 1$ y $f\in D^\times$. Si $f$ no es constante,
	entonces admite una factorizaci\'{o}n como producto de irreducibles
	(necesariamente primitivos). Si $f$ es irreducible, no hay nada que
	probar. En otro caso, existen $g,h\in D[X]$ no unidades tales que
	$f=g\,h$ y $\grado(g),\grado(h)\leq\grado(f)$. Dado que los contenidos
	$\contenido g$ y $\contenido h$ dividen a $\contenido f\sim 1$, los
	polinomios $g$ y $h$ deben ser primitivos.%
	\footnote{
		No usamos multiplicatividad del contenido --esto no es cierto
		en general--, sino la propiedad de sacar constantes.
	}
	Como $g$ y $h$ son primitivos, no pueden ser constantes, pues, en caso
	contrario, ser\'{\i}an unidades. As\'{\i}, $\grado(g)$ y $\grado(h)$
	son estrictamente menores que $\grado(f)$. Inductivamente, existen
	irreducibles $\lista{f}{s}\in D[X]$ tales que $g=f_1\cdots f_r$ y
	$h=f_{r+1}\cdots f_s$. En particular, $f=f_1\cdots f_s$. Vale notar
	que, sin hip\'{o}tesis adicionales,
	\begin{itemize}
		\item la factorizaci\'{o}n de un polinomio primitivo como
			producto de irreducibles en $D[X]$ no es necesariamente
			\'{u}nica,
		\item los irreducibles en $D[X]$ no son necesariamente
			irreducibles en $\cuerpocociente D[X]$ y
		\item el conjunto de polinomios primitivos no es necesariamente
			un monoide pero
		\item si $f\in D[X]$ es primitivo y admite una
			factorizaci\'{o}n dentro de $D[X]$, la misma debe
			ocurrir dentro del conjunto de polinomios primitivos.
	\end{itemize}
	%
\end{obsPrimitivos}

\begin{lemaGauss}[Lema de Gauss]\label{lema:gauss}
	Sobre un D.F.U., el contenido es multiplicativo: si $D$ es D.F.U.,
	dados $f,g\in D[X]$, $\contenido{f\,g}=\contenido f\,\contenido g$. En
	particular, el producto de polinomios primitivos en $D[X]$ es
	primitivo.
\end{lemaGauss}

\begin{proof}
	Observamos que, si $f=\contenido f\,\tilde f$ y
	$g=\contenido g\,\tilde g$, donde $\tilde f$ y $\tilde g$ son
	polinomios primitivos con coeficientes en $D$, entonces
	\begin{align*}
		\contenido{f\,g} & \,=\,\contenido f\,\contenido g\,
			\contenido{\tilde f\,\tilde g}
		\text{ .}
	\end{align*}
	%
	En particular, la demostraci\'{o}n se reduce al caso en que $f$ y $g$
	son primitivos.
	
	Ahora bien, como $D$ es un D.F.U., satisface la condici\'{o}n de
	cadenas de divisores. \emph{En particular}, todo elemento no nulo que
	no es una unidad admite un factor irreducible.%
	\footnote{
		C.f. la demostraci\'{o}n del Teorema~%
		\ref{teo:existenciayunicidad} ?`Es, esta condici\'{o}n,
		equivalente a la condici\'{o}n de cadenas de divisores? Notemos
		que usamos que todo elemento no unidad posee, al menos, un
		factor irreducible, pero no que todo elemento no unidad posee
		una factorizaci\'{o}n como producto de irreducibles.
	}
	Si $\contenido{f\,g}\not\sim 1$, es decir, no es una unidad, existe
	alg\'{u}n irreducible $p\in D$ que lo divide. Como $D$ posee M.C.D.,
	por la Proposici\'{o}n~\ref{propo:mcd:primalidad}, satisface la
	condici\'{o}n de primalidad.
	% \footnote{
		% De hecho, como $D$ es un dominio, la condici\'{o}n del M.C.D.
		% es equivalente a la condici\'{o}n de primalidad.
	% }
	Esto implica que el factor $p$ es primo y que el cociente
	$D/\generado p$ es un dominio \'{\i}ntegro (m\'{a}s aun, un cuerpo). En
	particular, el anillo de polinomios $\big(D/\generado p\big)[X]$ es un
	dominio. Pasando, entonces, al cociente,
	\begin{align*}
		f\,g \,\equiv\,0 & \quad\text{implica}\quad f\,\equiv\,0
			\quad\text{o}\quad g\,\equiv\,0
	\end{align*}
	%
	en $\big(D/\generado p\big)[X]$. Pero esto significa que, o bien
	$p|\contenido f$, o bien $p|\contenido g$.
\end{proof}

\begin{obsGauss}\label{obs:gauss}
	En la demostraci\'{o}n del Lema~\ref{lema:gauss}, s\'{o}lo usamos la
	existencia de factores irreducibles (no de factorizaci\'{o}n en
	irreducibles) y la condici\'{o}n de primalidad, sobre un dominio.
	Puesto que no s\'{e} si estas condiciones son suficientes para
	garantizar que $D$ es D.F.U., seguir\'{e} asumiendo la condici\'{o}n
	m\'{a}s fuerte.
\end{obsGauss}

\begin{lemaPolinomiosDFUIrreducible}\label{lema:polinomiosdfu:irreducible}
	Sea $D$ un D.F.U. y sea $f\in D[X]$ un polinomio no constante. Si
	$f$ es irreducible en $D[X]$, entonces $f$ es irreducible en
	$\cuerpocociente D[X]$.
\end{lemaPolinomiosDFUIrreducible}

\begin{proof}
	Si $f$ no fuese irreducible en $\cuerpocociente D[X]$, existir\'{\i}an
	polinomios $g$ y $h$ con coeficientes en $\cuerpocociente D$, ambos no
	constantes y tales que $f=g\,h$. Limpiando denomiadores, podemos asumir
	que $f=\gamma\,g\,h$, donde $g$ y $h$ tienen coeficientes en $D$ y
	$\gamma\in\cuerpocociente D$ es una constante. Podemos asumir,
	tambi\'{e}n, que $g$ y $h$ son primitivos, incorporando el contenido de
	\'{e}stos a $\gamma$. Por el Lema~\ref{lema:gauss},
	\begin{align*}
		\contenido f & \,\sim\,\gamma
		\text{ ,}
	\end{align*}
	%
	es decir, $\contenido f=u\,\gamma$, donde $u\in D^\times$. Pero esto
	implica que $\gamma\in D$ y que $f$ se factoriza en $D[X]$.
\end{proof}

\begin{teoPolinomiosDFU}\label{teo:polinomiosdfu}
	Si $D$ es D.F.U., $D[X]$ es D.F.U., tambi\'{e}n.
\end{teoPolinomiosDFU}

\begin{proof}
	Sea $f\in D[X]$. Si $f$ es constante, entonces $f$ se factoriza de
	manera esencialmente \'{u}nica en $D$ y, por la Observaci\'{o}n~
	\ref{obs:polinomios:unidades}, tambi\'{e}n en $D[X]$. Supongamos que
	$f$ no es constante. Entonces $f=c\,g$, donde $c=\contenido f$ y
	$g\in D[X]$ es primitivo de grado positivo. Por la Observaci\'{o}n~%
	\ref{obs:primitivos:descomposicion}, existen $\lista{f}{s}\in D[X]$
	irreducibles no constantes tales que $g=f_1\cdots f_s$. Como $D$ es
	D.F.U., existen irreducibles $\lista{p}{k}\in D$ tales que
	$c=p_1\cdots p_k$. Por la Observaci\'{o}n~%
	\ref{obs:polinomios:unidades}, los $p_i$ son irreducibles en $D[X]$ y
	\begin{align*}
		f & \,=\,p_1\,\cdots\,p_k\,f_1\,\cdots\,f_s
	\end{align*}
	%
	es una factorizaci\'{o}n de $f$ como producto de irreducibles en
	$D[X]$. Notemos que $\contenido f\sim p_1\cdots p_k$ y que
	$\contenido{f_1\cdots f_s}=\contenido g\sim 1$.%
	\footnote{
		No es aqu\'{\i} que usamos la hip\'{o}tesis de
		factorizaci\'{o}n, sino antes para factorizar $c$. Esto es
		cierto, s\'{o}lo porque primero separamos $f$ como producto de
		su contenido y un primitivo. En general, debemos usar m\'{a}s
		insistentemente la hip\'{o}tesis.
	}

	Sean $\lista{p}{k},\,\lista{q}{l}\in D$ irreducibles y
	$\lista{g}{s},\,\lista{h}{t}\in D[X]$ de grado positivo e irreducibles,
	tales que
	\begin{align*}
		f & \,:=\,p_1\,\cdots\,p_k\,g_1\,\cdots\,g_s \,=\,
			q_1\,\cdots\,q_l\,h_1\,\cdots\,h_t
		\text{ .}
	\end{align*}
	%
	Como $D$ es D.F.U., por el Lema~\ref{lema:polinomiosdfu:irreducible},
	los polinomios $g_i$ y $h_j$ son irreducibles sobre
	$\cuerpocociente D$. Pero el anillo de polinomios
	$\cuerpocociente D[X]$ es D.I.P. y, en particular, D.F.U. En
	consecuencia, $s=t$ y, m\'{a}s aun, existe una permutaci\'{o}n
	de $\{1,\,\cdots,\,s\}$, $j$, tal que $g_i\sim h_{j(i)}$ en
	$\cuerpocociente D[X]$. Por el Corolario~\ref{coro:contenido},
	$g_i\sim h_{j(i)}$ en $D[X]$. Ahora, por el Lema~\ref{lema:contenido},
	existe una unidad $u\in D^\times$ tal que
	\begin{align*}
		p_1\,\cdots\,p_k & \,=\,u\,q_1\,\cdots\,q_l
		% & \quad\text{y}\quad
			% g_1\,\cdots\,g_s\,=\,u^{-1}\,h_1\,\cdots\,h_s
		\text{ .}
	\end{align*}
	%
	En particular, como $D$ es D.F.U., $k=l$, tambi\'{e}n. M\'{a}s aun,
	existe una permutaci\'{o}n de $\{1,\,\cdots,\,k\}$, $j$, tal que
	$p_i\sim q_{j(i)}$. En definitiva, la factorizaci\'{o}n de un polinomio
	no constante $f\in D[X]$ es esencialmente \'{u}nica, tambi\'{e}n.
\end{proof}

\begin{coroPolinomiosDFU}\label{coro:polinomiosdfu}
	Si $D$ es un D.F.U. y $f\in D[X]$ es m\'{o}nico, entonces todo factor
	m\'{o}nico de $f$ en $\cuerpocociente D[X]$ tiene coeficientes en $D$.
\end{coroPolinomiosDFU}

\begin{coroPolinomiosDFU}\label{coro:polinomiosdfu:trascendente}
	Sea $F$ un cuerpo y sea $t$ un elemento trascendente sobre $F$. Si
	$f\in F[X]$ es un polinomio irreducible, entonces $f$ es irreducible en
	$F(t)[X]$.
\end{coroPolinomiosDFU}

\begin{coroPolinomiosDFU}\label{coro:polinomiosdfu:enteros}
	Sea $f\in\bb Z[X]$ un polinomio m\'{o}nico con coeficientes enteros. Si
	$x\in\bb Q$ es una ra\'{\i}z de $f$, entonces $x\in\bb Z$.
\end{coroPolinomiosDFU}

\begin{coroPolinomiosDFU}[criterio de irreducibilidad de Eisenstein]%
	\label{coro:polinomiosdfu:eisenstein}
	Sea $f=a_0+a_1\,X+\,\cdots\,+a_n\,X^n\in\bb Z[X]$ para el cual existe
	un primo $p\in\bb Z$ tal que
	\begin{itemize}
		\item $p|a_i$, si $0\leq i< n$,
		\item $p\nmid a_n$ y
		\item $p^2\nmid a_0$.
	\end{itemize}
	%
	Entonces $f$ es irreducible sobre $\bb Q$.
\end{coroPolinomiosDFU}
