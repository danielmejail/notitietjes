\theoremstyle{definition}
\newtheorem{defFactorial}{Definici\'{o}n}[section]
\newtheorem{defCancelable}[defFactorial]{Definici\'{o}n}
\newtheorem{defCancelativo}[defFactorial]{Definici\'{o}n}
\newtheorem{obsCancelativo}[defFactorial]{Observaci\'{o}n}
\newtheorem{ejemploCancelativo}[defFactorial]{Ejemplo}
\newtheorem{defDivisibilidad}[defFactorial]{Definici\'{o}n}
\newtheorem{obsDivisibilidad}[defFactorial]{Observaci\'{o}n}
\newtheorem{ejemploDivisibilidad}[defFactorial]{Ejemplo}
\newtheorem{defAsociados}[defFactorial]{Definici\'{o}n}
% \newtheorem{obsAsociados}[defFactorial]{Observaci\'{o}n}
\newtheorem{ejemploAsociados}[defFactorial]{Ejemplo}
\newtheorem{defIrreducible}[defFactorial]{Definici\'{o}n}
\newtheorem{obsIrreducible}[defFactorial]{Observaci\'{o}n}
\newtheorem{ejemploIrreducible}[defFactorial]{Ejemplo}
\newtheorem{defFactorizacion}[defFactorial]{Definici\'{o}n}
% \newtheorem{obsFactorizacion}[defFactorial]{Observaci\'{o}n}
\newtheorem{obsEjemploIrreducible}[defFactorial]{Observaci\'{o}n}
\newtheorem{ejemploFactorial}[defFactorial]{Ejemplo}
\newtheorem{defFactorizacionEsencialmenteUnica}[defFactorial]{Definici\'{o}n}
\newtheorem{defFactorialesLongitud}[defFactorial]{Definici\'{o}n}
\newtheorem{obsFactorialesLongitud}[defFactorial]{Observaci\'{o}n}
\newtheorem{defFactorialesCondicionDeCadena}[defFactorial]{Definici\'{o}n}
\newtheorem{defPrimo}[defFactorial]{Definici\'{o}n}
\newtheorem{obsPrimo}[defFactorial]{Observaci\'{o}n}
\newtheorem{defFactorialesCondicionDePrimalidad}[defFactorial]{Definici\'{o}n}

\theoremstyle{plain}
\newtheorem{coroAsociadosEnterosDeGauss}[defFactorial]{Corolario}
\newtheorem{propoFactorialesLongitud}[defFactorial]{Proposici\'{o}n}
\newtheorem{teoFactorialesCondicionDeCadena}[defFactorial]{Teorema}
\newtheorem{teoFactorialesCondicionDePrimalidad}[defFactorial]{Teorema}

%-------------

\begin{defCancelable}\label{def:cancelable}
	Sea $M$ un monoide. Un elemento $x\in M$ se dice \emph{cancelable a %
	izquierda}, si, para todo par $y,z\in M$
	\begin{equation}
		\label{eq:cancelable}
		x\,y\,=\,x\,z \,\Rightarrow\,y\,=\,z
		\text{ .}
	\end{equation}
	%
	An\'{a}logamente, $x$ es \emph{cancelable a derecha}, si, para todo par
	$y,z\in M$, $y\,x=z\,x$ implica $y=z$. Decimos que $x$ es
	\emph{cancelable}, si es cancelable a izquierda y a derecha.
\end{defCancelable}

\begin{defCancelativo}\label{def:cancelativo}
	Un \emph{monoide cancelativo} es un monoide en el cual todo elemento es
	cancelable.%
	\footnote{
		Comparar con la definici\'{o}n de grupo: un grupo es un monoide
		en el cual todo elemento es invertible.
	}
\end{defCancelativo}

\begin{obsCancelativo}\label{obs:cancelativo}
	Todo elemento invertible a izquierda es cancelable a izquierda y todo
	invertible a derecha es cancelable a derecha. En particular, todo grupo
	es un monoide cancelativo.
\end{obsCancelativo}

\begin{ejemploCancelativo}\label{ejemplo:cancelativo}
	El monoide $\bb Z\setmin\{0\}$ con el producto usual es un monoide
	cancelativo: si $a\,b=a\,c$, entonces $a\,(b-c)=0$ y, como $\bb Z$ no
	posee divisores de $0$, debe cumplirse $b=c$. En general, si $D$ es un
	dominio \'{\i}ntegro, el monoide $D\setmin\{0\}$ es un monoide
	cancelativo.
\end{ejemploCancelativo}

De ahora en adelante $M$ denotar\'{a} un monoide conmutativo cancelativo y
$U\subset M$ el submonoide (grupo) de elementos invertibles. Abreviaremos el
producto en $M$ yuxtaponiendo los argumentos y denotaremos $1$ al elemento
neutro.

\begin{defDivisibilidad}\label{def:divisibilidad}
	Dados $a,b\in M$, decimos que \emph{$b$ es un divisor de $a$}, si
	existe $c\in M$ tal que $a=b\,c$. Decimos tambi\'{e}n que $b$ es un
	\emph{factor} de $a$, que $b$ divide a $a$ o que $a$ es un m\'{u}ltiplo
	de $b$. Ser divisor o m\'{u}ltiplo define una relaci\'{o}n en $M$,
	escribimos $b|a$ para indicar que $b$ divide a $a$. Nos referimos a
	esta relaci\'{o}n como la \emph{relaci\'{o}n de divisibilidad} en $M$.
\end{defDivisibilidad}

\begin{obsDivisibilidad}\label{obs:divisibilidad}
	La relaci\'{o}n de divisibilidad en $M$ es una realci\'{o}n reflexiva
	y transitiva, pero no necesariamente sim\'{e}trica. Por ejemplo, en
	$\bb Z\setmin\{0\}$, $2|(-2)$, $(-2)|2$, pero $2\not=-2$. Sin
	embargo, en $\bb N$, la relaci\'{o}n de divisibilidad s\'{\i} es
	sim\'{e}trica. La raz\'{o}n de que esto sea as\'{\i} es que, en
	$\bb N$, el \'{u}nico elemento invertible es $1$.
\end{obsDivisibilidad}

\begin{obsDivisibilidad}\label{obs:divisibilidad:unidades}
	Un elemento $u\in M$ es invertible, si y s\'{o}lo si $u|1$. Adem\'{a}s,
	los elementos invertibles son factores \emph{triviales} en el sentido
	de que son factores de cualquier elemento del monoide.
\end{obsDivisibilidad}

\begin{defAsociados}\label{def:asociados}
	Dos elementos $a,b\in M$ se dicen \emph{asociados}, si $a|b$ y $b|a$.
	Equivalentemente, $a$ y $b$ son asociados, si existe un elemento
	invertible $u\in U$ tal que $a=u\,b$.%
	\footnote{
		Si $a=u\,b$ con $u$ invertible, entonces $b|a$ --por
		definici\'{o}n-- y $b=u^{-1}\,a$, con lo que $a|b$.
		Rec\'{\i}procamente, si $a|b$ y $b|a$, entonces existen
		$u,v\in M$ tales que $a=u\,b$ y $b=v\,a$. As\'{\i},
		$a=(u\,v)\,a$ y, como $M$ es cancelativo, $u\,v=1$, lo que
		significa que $u$ y $v$ son unidades.
	}
	Usaremos la notaci\'{o}n $a\sim b$ para indicar que $a$ y $b$ son
	asociados.
\end{defAsociados}

\begin{ejemploAsociados}\label{ejemplo:asociados}
	Dado $k\in\bb Z$, los enteros $k$ y $-k$ son asociados en $\bb Z$.
	Los enteros $-1$ y $1$ son las \'{u}nicas unidades en $\bb Z$: si
	$a,b\in\bb Z$ y $a\not=0$,%
	\footnote{
		Necesitamos usar propiedades de los n\'{u}meros reales.
	}
	\begin{align*}
		a\,b \,=\,1 & \quad\Rightarrow\quad |a|\,\leq\,1
			\quad\Rightarrow\quad |a|\,=\,1
			\quad\Rightarrow\quad a\in\{-1,1\}
		\text{ .}
	\end{align*}
	%
	En particular, los \'{u}nicos asociados de $k\in\bb Z$ son $k$ y $-k$.
\end{ejemploAsociados}

\begin{ejemploAsociados}\label{ejemplo:asociados:enterosdegauss}
	El \emph{anillo de enteros de Gauss}, es el conjunto
	\begin{align*}
		\bb Z[i] & \,:=\,\big\{a+b\,i\,:\,a,\,b\in\bb Z\big\}
		\text{ ,}
	\end{align*}
	%
	donde $i\in\bb C$ es una ra\'{\i}z cuadrada de $-1$, con las
	operaciones usuales heredadas de $\bb C$. Observamos que
	\begin{itemize}
		\item dados $a,b,c,d\in\bb Z$, $a+b\,i=c+d\,i$, si y
			s\'{o}lo si $a=c$ y $b=d$ y que
		\item $(a+b\,i)\,(c+d\,i)=(a\,c-b\,d)+(a\,d+b\,c)\,i$.
	\end{itemize}
	%
	Este anillo es un dominio de integridad, pues $\bb C$ es un cuerpo.
	Adem\'{a}s de $-1$ y $1$, $\bb Z[i]$ posee otras unidades. Por ejemplo,
	$i$ y $-i$ son unidades, pues $i\,(-i)=-(-1)=1$. M\'{a}s aun, \'{e}stas
	son todas las unidades.%
	\footnote{
			}
	De las propiedades enunciadas arriba, vemos que
	\begin{align*}
		(a+b\,i)\,(c+d\,i) \,=\,1 & \quad\Leftrightarrow\quad
			a\,c\,-b\,d\,=\,1\quad\text{y}\quad
			a\,d\,=\,-b\,c
		\text{ .}
	\end{align*}
	%
	De la primera igualdad, deducimos que $a$ y $b$ deben ser coprimos en
	$\bb Z$. Pasando a la segunda, $a|b\,c$ implica $a|c$. Pero $c$ y $d$
	tambi\'{e}n deben ser coprimos y $c|a\,d$ implica $c|a$. Es decir,
	$a$ y $c$ son asociados en $\bb Z$. De acuerdo con el Ejemplo~%
	\ref{ejemplo:asociados}, $a=c$ o $a=-c$. An\'{a}logamente, $b=d$ o
	$b=-d$. Separando en casos, alguna de las siguientes igualdades debe
	ser cierta:
	\begin{align*}
		a^2\,-\,b^2 \,=\,1 & \quad\text{,}\quad
		a^2\,+\,b^2 \,=\,1 \text{ ,} \\
		-a^2\,-\,b^2 \,=\,1 & \quad\text{o}\quad
		-a^2\,+\,b^2 \,=\,1
		\text{ .}
	\end{align*}
	%
	Como $a,b\in\bb R$, la tercera no se puede ocurrir. Como $a,b\in\bb Z$,
	la segunda se cumple s\'{o}lo si uno de $a$ y $b$ es $0$ y el otro es
	$\pm1$. La primera es una diferencias de cuadrados:
	$a^2-b^2=(a+b)\,(a-b)$. Si este producto fuese igual a $1$,
	valdr\'{\i}a que $a+b=1$ o $a+b=-1$. En el primer caso, $a-b=1$, con lo
	que $b=0$, y, en el segundo, $a-b=-1$ y $b=0$, tambi\'{e}n. La cuarta
	es similar a la primera. En ese caso, deducir\'{\i}amos que debe ser
	$a=0$. En definitiva, las \'{u}nicas unidades son de la forma $a+b\,i$
	con $a=\pm1$ y $b=0$ o con $a=0$ y $b=\pm1$. Si ahora $x\in\bb Z[i]$,
	el conjunto de asociados de $x$ est\'{a} dado por
	\begin{align*}
		& \big\{x,\,-x,\,i\,x,\,(-i)\,x\big\}
		\text{ .}
	\end{align*}
	%
\end{ejemploAsociados}

El argumento es \emph{ad hoc} en el caso de los enteros de Gauss. En general,
el argumento es el siguiente. Si $x=a+b\,i\in\bb Z[i]$, llamamos
\emph{conjugado de $x$} al entero $\conj x=a-b\,i$. Se puede comprobar, usando
las propiedades de $\bb Z[i]$ mencionadas, que $x\,\conj x\in\bb Z$ y que, si
$x$ es una unidad en $\bb Z[i]$ con inverso $y$, entonces $\conj x$ es una
unidad, tambi\'{e}n, y su inverso es $\conj y$. Pero entonces, en tal caso,
como $x\,y=1$ y $\conj x\,\conj y=1$, vale que $(x\,\conj x)\,(y\,\conj y)=1$ y
el producto $x\,\conj x$ --la \emph{norma} de $x$-- es una unidad en $\bb Z$.
Si $x=a+b\,i$, entonces $x\,\conj x=a^2+b^2$. En particular, todas las unidades
de $\bb Z[i]$ tienen norma positiva. De esto se deduce que, o bien $a=0$ y
$b=\pm1$, o bien $a=\pm1$ y $b=0$. Rec\'{\i}procamente, si la norma de $x$ es
$1$ (o $-1$), entonces $x$ es una unidad.

\begin{coroAsociadosEnterosDeGauss}\label{coro:asociados:enterosdegauss}
	Conjugar es un morfismo de monoides en $\bb Z[i]\setmin\{0\}$, es
	decir, si $x\,y=z$, entonces $\conj z=\conj x\,\conj y$. La
	aplicaci\'{o}n dada por tomar norma, $x\mapsto x\,\conj x$, tambi\'{e}n
	es morfismo de monoides. En particular, $x\in\bb Z[i]$ es una unidad,
	si y s\'{o}lo si $x\,\conj x\in\bb Z$ es una unidad
\end{coroAsociadosEnterosDeGauss}

\begin{ejemploAsociados}\label{ejemplo:asociados:cuadraticoimaginario}
	Sea $\sqrt{-5}\in\bb C$ una ra\'{\i}z cuadrada de $-5$. El conjunto
	\begin{align*}
		\bb Z[\sqrt{-5}] & \,:=\,\big\{a+b\,\sqrt{-5}\,:\,
			a,\,b\in\bb Z\big\}
	\end{align*}
	%
	es un subanillo de $\bb C$ y, en particular, un dominio \'{\i}ntegro.
	Dado $x=a+b\,\sqrt{-5}\in\bb Z[\sqrt{-5}]$, su \emph{conjugado} es el
	elemento $\conj x=a-b\,\sqrt{-5}$ del anillo. La \emph{norma} de $x$
	es, en este caso,
	\begin{align*}
		x\,\conj x & \,=\,a^2\,+\,5\,b^2
		\text{ .}
	\end{align*}
	%
	En particular, si $x$ es una unidad, su norma es $\pm1$, de lo que se
	deduce que $b=0$ y $a=\pm1$. Esto quiere decir que, en el anillo
	$\bb Z[\sqrt{-5}]$, las \'{u}nicas unidades son $-1$ y $1$. En
	definitiva, los \'{u}nicos asociados de $x\in\bb Z[\sqrt{-5}]$ son $-x$
	y $x$.
\end{ejemploAsociados}

Como vimos en el Ejemplo~\ref{ejemplo:monoide:cociente:enteros}, ser asociados
es una relaci\'{o}n de congruencia en $M$.

\begin{defIrreducible}\label{def:irreducible}
	Dado $a\in M$, si $b\in M$ es tal que $b|a$, pero $a\nmid b$, decimos
	que $b$ es un \emph{factor propio} de $a$. Se dice que $a$ es
	\emph{irreducible (en $M$)}, si
	\begin{enumerate}[(i)]
		\item $a$ no es una unidad y
		\item los \'{u}nicos factores propios de $a$ son las unidades.
	\end{enumerate}
	%
\end{defIrreducible}

\begin{obsIrreducible}\label{obs:irreducible}
	Las unidades no poseen factores propios y son factores de todos los
	elementos de $M$. Si $a\in M$ no es una unidad, entonces las unidades
	son factores propios de $a$. Si $a$ es irreducible, sus asociados
	tambi\'{e}n lo son.
\end{obsIrreducible}

\begin{ejemploIrreducible}\label{ejemplo:irreducible}
	Los primos racionales son irreducibles en $\bb Z$. Un entero
	$p\in\bb Z$ se dice \emph{primo}, si $p|ab$ con $a,b\in\bb Z$ implica
	$p|a$ o $p|b$. Si $p=h\,k$ es primo, entonces $p|h\,k$. Por
	definici\'{o}n, o $p|h$ o $p|k$. Sin p\'{e}rdida de generalidad,
	podemos suponer que $p|h$ y que $h=p\,l$ con $l\in\bb Z$. Pero
	entonces, como $\bb Z$ es dominio \'{\i}ntegro y $p=(p\,l)\,k$,
	$1=l\,k$. Esto significa que $l$ y $k$ son unidades en $\bb Z$.
\end{ejemploIrreducible}

El argumento del Ejemplo~\ref{ejemplo:irreducible} es v\'{a}lido en cualquier
monoide cancelativo.

\begin{ejemploIrreducible}\label{ejemplo:irreducible:enterosdegauss}
	Sea $\bb Z[i]$ el anillo definido en el Ejemplo~%
	\ref{ejemplo:asociados:enterosdegauss}. El entero racional $2\in\bb Z$,
	si bien es irreducible en $\bb Z$, se factoriza en $\bb Z[i]$ de manera
	no trivial:
	\begin{equation}
		\label{eq:irreducible:enterosdegauss}
		2 \,=\,(1+i)\,(1-i) \,=\,(-i)\,(1+i)^2
		\text{ .}
	\end{equation}
	%
	Por lo visto en el Ejemplo~\ref{ejemplo:asociados:enterosdegauss},
	$1+i$ y $1-i$ no son unidades en $\bb Z[i]$. El factor $1+i$ es
	irreducible. Si $x\,y=1+i$, entonces, multiplicando por los conjugados,
	$(x\,\conj x)\,(y\,\conj y)\,=\,(1+i)\,(1-i)=2$. Dado que $2$ es
	irreducible en $\bb Z$, deducimos que
	\begin{itemize}
		\item $x\,\conj x\in\{-1,1\}$ e $y\,\conj y\in\{-2,2\}$, o
		\item $x\,\conj x\in\{-2,2\}$ e $y\,\conj y\in\{-1,1\}$.
	\end{itemize}
	%
	En el primer caso, $x$ es una unidad, y, en el segundo, $y$ lo es.
	En definitiva $1+i$ es irreducible. Como $1-i$ es asociado de $1+i$
	(multiplicando por la unidad $(-i)$), deducimos que $1-i$ es
	irreducible, tambi\'{e}n.
	% Por otro lado, $3\in\bb Z$ es irreducible en
	% $\bb Z[i]$. Si $3=r\,s$, con $r$ y $s$ no unidades, debe cumplirse,
	% aplicando la norma, que
	% \begin{align*}
		% 9 & \,=\,(r\,\conj r)\,(s\,\conj s)
	% \end{align*}
	% %
	% y $r\,\conj r=\pm 3$. La existencia de un factor no trivial $r$
	% equivale a la existencia de una soluci\'{o}n de la ecuaci\'{o}n
	% \begin{align*}
		% a^2\,+\,b^2 & \,=\,\pm3
		% \text{ ,}
	% \end{align*}
	% %
	% con $a,b\in\bb Z$. Pero esta ecuaci\'{o}n no tiene soluciones enteras.
\end{ejemploIrreducible}

\begin{obsEjemploIrreducible}\label{obs:ejemplo:irreducible}
	Del Corolario~\ref{coro:asociados:enterosdegauss}, se deduce que, si
	$x\in\bb Z[i]$ no es irreducible, entonces su norma
	$x\,\conj x\in\bb Z$ tampoco es irreducible. Equivalentemente, si
	$x\,\conj x$ es irreducible en $\bb Z$, entonces $x$ es irreducible en
	$\bb Z[i]$.
\end{obsEjemploIrreducible}

\begin{ejemploIrreducible}\label{ejemplo:irreducible:cuadraticoimaginario}
	Sea $\bb Z[\sqrt{-5}]$ el anillo del Ejemplo~%
	\ref{ejemplo:asociados:cuadraticoimaginario}. Veamos que los elementos
	de norma $9$ son irreducibles. Sea $x\in\bb Z[\sqrt{-5}]$ tal que
	$x\,\conj x=9$. Supongamos que existe una factorizaci\'{o}n $x=r\,s$,
	con $r$ y $s$ no unidades en el anillo. Aplicando norma, por el
	Corolario~\ref{coro:asociados:enterosdegauss}, $r\,\conj r=\pm 3$. Esto
	quiere decir que existe una soluci\'{o}n de
	\begin{equation}
		\label{eq:irreducible:cuadraticoimaginario}
		a^2\,+\,5\,b^2 \,=\,\pm3
		\text{ ,}
	\end{equation}
	%
	con $a,b\in\bb Z$. El lado izquierdo es no negativo y, si $b$ no es
	cero, entonces el resultado es al menos $5$ y, as\'{\i}, $b=0$. Pero
	no existe $a$ entero tal que $a^2=3$, con lo que no hay soluci\'{o}n de
	\eqref{eq:irreducible:cuadraticoimaginario} en $\bb Z$. En
	conclusi\'{o}n, no existen elementos de norma $\pm3$ en este anillo y
	los elementos de norma $9$ deben ser irreducibles. As\'{\i}, por
	ejemplo, $\pm3$ son irreducibles en $\bb Z$ y siguen siendo
	irreducibles en $\bb Z[\sqrt{-5}]$. Pero no son los \'{u}nicos
	irreducibles de norma $9$. En $\bb Z[\sqrt{-5}]$,
	\begin{equation}
		\label{eq:irreducible:cuadraticoimaginario:factorizacion}
		9 \,=\,3\cdot 3 \,=\,(2+\sqrt{-5})\,(2-\sqrt{-5})
		\text{ .}
	\end{equation}
	%
	De estas igualdades, se deduce que $2\pm\sqrt{-5}$ son irreducibles de
	norma $9$. Notamos que $3$ no es asociado de $2+\sqrt{-5}$ ni de
	$2-\sqrt{-5}$.%
	\footnote{
		Ni tampoco es $2+\sqrt{-5}$ asociado de $2-\sqrt{-5}$.
	}
	La Tabla~\ref{tab:irreducible:cuadraticoimaginario} contiene todos los
	elementos de $\bb Z[\sqrt{-5}]$ con un valor espec\'{\i}fico de la
	norma. Podemos ver que, en particular, los elementos de norma $4$, $5$,
	$6$, $9$ o $14$ son irreducibles.
	\begin{table}
		\centering
		\begin{tabular}{c|c}
			$x\,\conj x$ & $x$ \\
			\hline
			$2$ & $\varnothing$ \\
			$3$ & $\varnothing$ \\
			$4$ & $\pm2$ \\
			$5$ & $\pm\sqrt{-5}$ \\
			$6$ & $\pm(1+\sqrt{-5}),\pm(1-\sqrt{-5})$ \\
			$7$ & $\varnothing$ \\
			$8$ & $\varnothing$ \\
			$9$ & $\pm3,\pm(2+\sqrt{-5}),\pm(2-\sqrt{-5})$ \\
			$10$ & $\varnothing$ \\
			$11$ & $\varnothing$ \\
			$12$ & $\varnothing$ \\
			$13$ & $\varnothing$ \\
			$14$ & $\pm(3+\sqrt{-5}),\pm(3-\sqrt{-5})$ \\
		\end{tabular}
		\caption{Elementos de $\bb Z[\sqrt{-5}]$ de norma dada.}
		\label{tab:irreducible:cuadraticoimaginario}
	\end{table}
	%
\end{ejemploIrreducible}

\begin{defFactorizacion}\label{def:factorial:factorizacion}
	Dado $a\in M$, una \emph{factorizaci\'{o}n de $a$ como producto de
	irreducibles} es una expresi\'{o}n de la forma
	\begin{equation}
		\label{eq:factoriales:factorizacion}
		a \,=\,p_1\cdots p_s
		\text{ ,}
	\end{equation}
	%
	donde cada $p_i$ es un elemento irreducible en $M$.
\end{defFactorizacion}

No es cierto, dado un monoide cancelativo y conmutativo, que todo elemento
admita una factorizaci\'{o}n como producto de irreducibles. Pero, cuando
exista, nos importar\'{a} saber bajo qu\'{e} condiciones y en qu\'{e} sentido
es \eqref{eq:factoriales:factorizacion} una factorizaci\'{o}n ``\'{u}nica''.
Hay dos maneras ``triviales'' en las que podemos alterar los factores
$\lista{p}{s}$ para obtener otras factorizaciones posibles.

\begin{ejemploFactorial}\label{ejemplo:factorial}
	En $\bb Z$, $6=2\cdot 3$, pero tambi\'{e}n
	\begin{math}
		6=(-2)\cdot (-3)=3\cdot 2=(-3)\cdot (-2)
	\end{math}. El Teorema~\ref{teo:fundamental} garantiza que \'{e}stas
	son todas las posibles factorizaciones de $6$ como producto de
	irreducibles en $\bb Z$.
\end{ejemploFactorial}

\begin{ejemploFactorial}\label{ejemplo:factorial:enterosdegauss}
	Podemos utilizar la observaci\'{o}n del Ejemplo~\ref{ejemplo:factorial}
	para obtener condiciones sobre las posibles factorizaciones de $6$ en
	$\bb Z[i]$. Si $6=p_1\cdots p_s$, aplicando la norma, se obtiene la
	siguiente factorizaci\'{o}n en $\bb Z$:
	\begin{align*}
		36 & \,=\,(p_1\,\conj p_1)\,\cdots\,(p_s\,\conj p_s)
		\text{ .}
	\end{align*}
	%
	Si definimos $a_i=p_i\,\conj p_i$, entonces $a_i\geq 2$ y $a_i|36$ para
	todo $i$. Como $36=2\cdot 2\cdot 3\cdot 3$, por el Teorema~%
	\ref{teo:fundamental}, $s\leq 4$ y, adem\'{a}s,
	\begin{align*}
		a_i \,=\, 2^{u_i}\,3^{v_i} & \quad\text{,}\quad
			0\,\leq\,u_i,\,v_i\,\leq\,2 \quad\text{,}\quad
			u_i+v_i\,\geq\,1 \quad\text{para todo } i
			\quad\text{y} \\
		\sum_i\,u_i & \,=\,\sum_i\,v_i\,=\,2
		\text{ .}
	\end{align*}
	%
	Ahora bien, seg\'{u}n lo visto en el Ejemplo~%
	\ref{ejemplo:irreducible:enterosdegauss}, $2=(1+i)\,(1-i)$ en
	$\bb Z[i]$. Se puede comprobar, usando un argumento similar al del
	Ejemplo~\ref{ejemplo:irreducible:cuadraticoimaginario}, que $3$ es
	irreducible en este anillo. La identidad
	\eqref{eq:irreducible:enterosdegauss} proporciona las siguientes
	factorizaciones de $6$ en $\bb Z[i]$:
	\begin{align*}
		6 & \,=\,(1+i)\,(1-i)\,3\,=\,(1+i)^2\,(-3\,i)
		\text{ .}
	\end{align*}
	%
	M\'{a}s aun, no hay elementos de norma $3$, $6$ o $12$, con lo cual
	cada $a_i$ en la factorizaci\'{o}n de $36$ debe pertenecer al conjunto
	de divisores $\{1,\,2,\,4,\,9,\,18,\,36\}$. As\'{\i}, m\'{o}dulo
	cambiar un factor por un asociado o reordenarlos, la factorizaci\'{o}n
	anterior de $6$ es la \'{u}nica posible.
\end{ejemploFactorial}

En general, si \eqref{eq:factoriales:factorizacion} es una factorizaci\'{o}n de
$a$ como producto de irreducibles y si $\lista{u}{s}$ son unidades tales que
$u_1\cdots u_s=1$, entonces, cambiando cada $p_i$ por su asociado
$p_i'=u_i\,p_i$, se encuentra una factorizaci\'{o}n posiblemente distinta
\begin{align*}
	a & \,=\,p_1'\cdots p_s'
	\text{ .}
\end{align*}
%
Lo mismo ocurre, si reordenamos los factores: si $j$ es una permutaci\'{o}n de
$\{1,\,\dots,\,s\}$, entonces
\begin{align*}
	a & \,=\,p_{j(1)}\cdots p_{j(s)}
\end{align*}
%
tambi\'{e}n es una factorizaci\'{o}n de $a$ como producto de irreducibles,
posiblemente distinta.

\begin{defFactorizacionEsencialmenteUnica}%
	\label{def:factorizacionesencialmenteunica}
	Sea $M$ un monoide cancelativo y conmutativo. Dado $a\in M$, diremos
	que una factorizaci\'{o}n $a=p_1\cdots p_s$ como producto de
	irreducibles \emph{es esencialmente \'{u}nica}, si, dada cualquier
	factorizaci\'{o}n como producto de irreducibles $a=p_1'\cdots p_t'$ se
	cumple que
	\begin{enumerate}[(i)]
		\item $t=s$,
		\item existe una permutaci\'{o}n $j$ de $\{1,\,\dots,\,s\}$ tal
			que $p_i'\sim p_{j(i)}$ para todo $i$.
	\end{enumerate}
	%
\end{defFactorizacionEsencialmenteUnica}

En palabras, un elemento de $M$ se factoriza de manera esencialmente \'{u}nica
como producto de irreducibles, si toda factorizaci\'{o}n como producto de
irreducibles posee la misma cantidad de factores irreducibles y si la misma se
puede obtener a partir de cualquier otra reordenando los factores y tomando
asociados.

\begin{defFactorial}\label{def:factorial}
	Un monoide conmutativo y cancelativo se dice \emph{factorial} (o
	\emph{monoide de factorizaci\'{o}n \'{u}nica}), si todo elemento
	distinto de una unidad posee una factorizaci\'{o}n esencialmente
	\'{u}nica como producto de irreducibles. Un \emph{dominio de
	factorizaci\'{o}n \'{u}nica} (D.F.U.) es un dominio $D$ cuyo monoide
	multiplicativo $D\setmin\{0\}$ es factorial.
\end{defFactorial}

\begin{ejemploFactorial}\label{ejemplo:factorial:cuadraticoimaginario}
	El anillo $\bb Z$ de enteros racionales es un D.F.U., por el Teorema~%
	\ref{teo:fundamental}. El anillo $\bb Z[i]$ de enteros de Gauss
	tambi\'{e}n es un D.F.U.%
	\footnote{
		C.f. \ref{teo:}
	}
	La identidad \eqref{eq:irreducible:cuadraticoimaginario:factorizacion}
	muestra que el anillo $\bb Z[\sqrt{-5}]$ no es un D.F.U. Usando la
	Tabla~\ref{tab:irreducible:cuadraticoimaginario}, podemos obtener otros
	contraejemplos de la unicidad de la factorizaci\'{o}n.
\end{ejemploFactorial}

A continuaci\'{o}n, deducimos dos propiedades que todo monoide factorial posee
y probamos que son, tambi\'{e}n, condiciones suficientes.

\begin{defFactorialesLongitud}\label{def:factoriales:longitud}
	Sea $M$ un monoide factorial. Dado $a\in M$, definimos la
	\emph{longitud} de $a$ como la cantidad de factores irreducibles en
	cualquier factorizaci\'{o}n de $a$ como producto de irreducibles.
	Denotamos la longitud de $a$ por $\longitud(a)$.
\end{defFactorialesLongitud}

Si $M$ es factorial, la longitud de $a\in M$ est\'{a} bien definida: por un
lado, como $a$ admite al menos una factorizaci\'{o}n como producto de
irreducibles, podemos hablar de la longitud de la factorizaci\'{o}n, pero, por
otro lado, como toda tal factorizaci\'{o}n posee la misma cantidad de factores
irreducibles, no hay ambig\"{u}edad, si la longitud de $a$ se define como la
longitud de cualquiera de ellas. La longitud caracteriza a las unidades y a los
irreducibles.

\begin{propoFactorialesLongitud}\label{propo:factoriales:longitud}
	Sea $M$ un monoide factorial y sea $a\in M$. Entonces
	\begin{enumerate}
		\item $a$ es una unidad, si y s\'{o}lo si $\longitud(a)=0$;
		\item $a$ es irreducible, si y s\'{o}lo si $\longitud(a)=1$;
		\item si $a=b\,c$, entonces
			$\longitud(a)=\longitud(b)+\longitud(c)$;
		\item si $b$ es un factor propio de $a$, entonces
			$\longitud(a)>\longitud(b)$.
	\end{enumerate}
	%
\end{propoFactorialesLongitud}

\begin{proof}
	Si $a=b\,c$ y $\longitud(a)=r$, $\longitud(b)=s$ y $\longitud(c)=t$,
	entonces existen factorizaciones $a=p_1\cdots p_r$, $b=p_1'\cdots p_s'$
	y $c=p_{s+1}'\cdots p_{s+t}'$ como productos de irreducibles para $a$,
	$b$ y $c$, respectivamente. La igualdad
	\begin{align*}
		p_1\cdots p_r & \,=\,(p_1'\cdots p_s')\,
			(p_{s+1}'\cdots p_{s+t}')
	\end{align*}
	%
	y la unicidad de la longitud implican que $r=s+t$.
\end{proof}

\begin{obsFactorialesLongitud}\label{obs:factoriales:longitud}
	Sea $M$ es un monoide factorial y sea $a=b\,c\in M$. Dadas
	factorizaciones $a=p_1\cdots p_r$ y $b=p_1'\cdots p_s'$ como productos
	de irreducibles, $s\leq r$ y existe una funci\'{o}n inyectiva
	$j:\{1,\,\dots,\,s\}\rightarrow\{1,\,\dots,\,r\}$ tal que
	\begin{align*}
		p_i' & \sim p_{j(i)}
		\text{ .}
	\end{align*}
	%
	Es decir, la lista de los posibles divisores de $a$ se obtiene a partir
	de considerar los productos parciales de los factores irreducibles que
	aparecen en cualquier factorizaci\'{o}n de $a$ como producto de
	irreducibles y asociados de estos productos.
\end{obsFactorialesLongitud}

\begin{teoFactorialesCondicionDeCadena}%
	\label{teo:factoriales:condiciondecadena}
	Sea $M$ un monoide factorial y sean $a,b\in M$ tales que $b|a$. Si $b$
	es un factor propio de $a$, entonces $\longitud(a)>\longitud(b)$. En
	particular, en un monoide factorial no existen sucesiones infinitamente
	largas $\{a_i\}_{i\geq 0}\subset M$ tales que $a_{i+1}$ sea un divisor
	propio de $a_i$ para todo $i\geq 0$.
\end{teoFactorialesCondicionDeCadena}

\begin{defFactorialesCondicionDeCadena}%
	\label{def:factoriales:condiciondecadena}
	Sea $M$ un monoide (conmutativo). Decimos que \emph{$M$ satisface la %
	condici\'{o}n de cadenas de divisores},%
	\footnote{
		Cadenas \emph{ascendentes} de divisores. Comparar con la
		condici\'{o}n de cadenas ascendentes que caracterizan a los
		m\'{o}dulos/anillos noetherianos.
	}
	si, dada una sucesi\'{o}n
	$\{a_i\}_{i\geq 0}\subset M$ tal que $a_{i+1}|a_i$ para todo $i\geq 0$,
	entonces existe $k\geq 0$ tal que $a_i\sim a_k$ para todo $i\geq k$.
\end{defFactorialesCondicionDeCadena}

Dicho de otra manera, $M$ satisface la condici\'{o}n de cadenas de divisores,
si toda sucesi\'{o}n $\{a_i\}_{i\geq 0}$ en $M$ tal que $a_{i+1}|a_i$ es
eventualmente esencialmente constante.

La segunda condici\'{o}n necesaria tiene que ver con la noci\'{o}n de
primalidad.

\begin{defPrimo}\label{def:primo}
	En un monoide $M$, un elemento $p\in M$ se dice \emph{primo}, si
	\begin{enumerate}[(i)]
		\item\label{item:def:primo:i}
			$p$ no es una unidad y
		\item\label{item:def:primo:ii}
			si $p|a\,b$, entonces $p|a$ o $p|b$.
	\end{enumerate}
	%
\end{defPrimo}

\begin{obsPrimo}\label{obs:primo}
	Si $p\in M$ es un elemento primo en un monoide \emph{cancelativo},
	entonces $p$ es irreducible. Si $p=k\,h$ es una factorizaci\'{o}n en
	$M$, por el \'{\i}tem~\ref{item:def:primo:ii} de la Definci\'{o}n~%
	\ref{def:primo}, $p|k$ o $p|h$. Sin p\'{e}rdida de generalidad, podemos
	asumir que $p|k$ y que $k=p\,b$, para cierto $b\in M$. As\'{\i},
	$p=(p\,b)\,h$. Dado que $M$ es cancelativo, $b\,h=1$ y $h$ es una
	unidad.
\end{obsPrimo}

\begin{teoFactorialesCondicionDePrimalidad}%
	\label{teo:factoriales:condiciondeprimalidad}
	En un monoide factorial, todo elemento irreducible es primo.
\end{teoFactorialesCondicionDePrimalidad}

\begin{proof}
	Sea $M$ un monoide factorial y sea $p\in M$ un elemento irreducible.
	Dados $a,b\in M$, existen factorizaciones $a=p_1\cdots p_s$ y
	$b=p_{s+1}\cdots p_{s+t}$ como productos de irreducibles. En
	particular,
	\begin{align*}
		a\,b & \,=\,p_1\cdots p_s\cdot p_{s+1}\cdots p_{s+t}
	\end{align*}
	%
	es una factorizaci\'{o}n de $a\,b$ como producto de irreducibles. Si
	$p|a\,b$, entonces, por lo mencionado en la Observaci\'{o}n~%
	\ref{obs:factoriales:longitud}, debe existir $i\in\{1,\,\dots,\,s+t\}$
	tal que $p\sim p_i$. Si $i\leq s$, entonces $p|a$ y si $i>s$, entonces
	$p|b$.
	% Puesto que, por definici\'{o}n, $p$ no es una unidad, $p$ es primo.
\end{proof}

\begin{defFactorialesCondicionDePrimalidad}%
	\label{def:factoriales:condiciondeprimalidad}
	Sea $M$ un monoide (conmutativo). Decimos que \emph{$M$ satisface la %
	condici\'{o}n de primalidad}, si todo elemento irreducible es primo.
\end{defFactorialesCondicionDePrimalidad}

El Teorema~\ref{teo:factoriales:condiciondecadena} y el Teorema~%
\ref{teo:factoriales:condiciondeprimalidad} implican que todo monoide factorial
satisface la condici\'{o}n de cadenas de divisores y la condici\'{o}n de
primalidad.
