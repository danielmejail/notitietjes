\theoremstyle{plain}
\newtheorem{teoFundamental}{Teorema}[section]

\theoremstyle{definition}

%-------------

El tema central de estas notas es el problema de divisibilidad o
factorizaci\'{o}n en un dominio conmutativo. En pocas palabras, el objetivo
ser\'{a} poner en un contexto general el siguiente resultado fundamental:

\begin{teoFundamental}[Teorema fundamental de la aritm\'{e}tica]%
	\label{teo:fundamental}
	Dado un n\'{u}mero entero positivo $m$, no nulo y distinto de $1$,
	existe una lista no vac\'{\i}a de primos positivos $\lista{p}{k}$ tal
	que
	\begin{equation}
		\label{eq:fundamental}
		m \,=\, p_1\cdots p_k
		\text{ .}
	\end{equation}
	%
	La lista es \'{u}nica, salvo por el orden de los factores.
\end{teoFundamental}

Que la lista sea no vac\'{\i}a quiere decir que $k\geq 1$. En
\eqref{eq:fundamental}, los factores $p_i$ pueden aparecer repetidos, es decir,
con multiplicidad. Esta expresi\'{o}n se denomina \emph{factorizaci\'{o}n de %
$m$ como producto de primos}. A veces, usamos el t\'{e}rmino
``descomposici\'{o}n'' en lugar de ``factorizaci\'{o}n'' y tambi\'{e}n diremos
que \eqref{eq:fundamental} es la descomposici\'{o}n de $m$ como producto de
irreducibles.

En el anillo $\bb Z$ de enteros racionales, los conceptos de ``n\'{u}mero
irreducible'' y de ``n\'{u}mero primo'' coinciden. Esto es cierto en cualquier
dominio de factorizaci\'{o}n \'{u}nica.%
\footnote{
	\ref{:}.
}
Pero, a diferencia, de lo que ocurre con los enteros positivos, en general hay
que tener en cuenta que una descomposici\'{o}n de la forma $a=b\,c$ puede
variar de distintas maneras, no s\'{o}lo en el orden de los factores. Esto se
relaciona con la presencia de unidades en un dominio arbitrario. En $\bb Z$,
las \'{u}nicas unidades son $\{1,-1\}$, con lo que los primos racionales son
los primos positivos $2,\,3,\,5,\,7,\,\dots$ y los correspondientes negativos
$-2,\,-3,\,-5,\,-7,\,\dots$. En general, en un dominio $D$, si $p$ es
irreducible y $u$ es una unidad en $D$, entonces el producto $u\,p$ es
irreducible, tambi\'{e}n. Los elementos $p$ y $u\,p$ se dicen \emph{asociados}.
Se ve, as\'{\i}, que, si quisi\'{e}ramos generalizar el Teorema~%
\ref{teo:fundamental} a un dominio arbitrario, debemos tener en cuenta que es
posible obtener descomposiciones que difieren en cambiar un factor por un
asociado.

