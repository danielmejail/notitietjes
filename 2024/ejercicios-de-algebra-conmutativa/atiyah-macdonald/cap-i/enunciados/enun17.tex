For each $f\in A$, let $\principal f$ denote the complemento of $\ceros f$ in
$X=\espectro(A)$. The sets $\principal f$ are open. Show that they form a basis
of open sets for the Zariski topology, and that:
\begin{enumerate}[(i)]
	\item\label{item:ejer:capi:17:i}
		$\principal f\cap\principal g=\principal{f\,g}$;
	\item\label{item:ejer:capi:17:ii}
		$\principal f=\varnothing$, if and only if $f$ is nilpotent;
	\item\label{item:ejer:capi:17:iii}
		$\principal f=X$, if and only if $f$ is a unit;
	\item\label{item:ejer:capi:17:iv}
		$\principal f=\principal g$, if and only if
		$\rad(\generado f)=\rad(\generado g)$;
	\item\label{item:ejer:capi:17:v}
		$X$ is (quasi-) compact (that is, every open covering of $X$
		has a finite subcovering);%
		\hint{
			To prove \eqref{item:ejer:capi:17:v}, remark that it is
			enough to consider a convering of $X$ by basic open
			sets $\principal{f_i}$ ($i\in I$). Show that the $f_i$
			generate the unit ideal and hence that there is an
			equation of the form
			\begin{align*}
				1 & \,=\,\sum_j\,g_j\,f_j
			\end{align*}
			%
			($g_i\in A$), where $J$ is some \emph{finite} subset of
			$I$. Then the $\principal{f_j}$ ($j\in J$) cover $X$.
		}
	\item\label{item:ejer:capi:17:vi}
		more generally, each $\principal f$ is (quasi-) compact;
	\item\label{item:ejer:capi:17:vii}
		an open subset of $X$ is (quasi-) compact, if and only if it is
		a finite union of sets $\principal f$.
\end{enumerate}
%
The sets $\principal f$ are called \emph{basic open sets} of $X=\espectro(A)$.

