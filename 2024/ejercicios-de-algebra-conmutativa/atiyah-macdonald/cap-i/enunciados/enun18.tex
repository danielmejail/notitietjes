For psychological reasons it is sometimes convenient to denote a prime ideal of
$A$ by a letter such as $x$ or $y$ when thinking of it as a point of
$X=\espectro(A)$. When thiinking of $x$ as a prime ideal of $A$, we denote it
by $\frak p_x$ (logically, of course, it is the same thing). Show that:
\begin{enumerate}[(i)]
	\item\label{item:ejer:capi:18:i}
		the set $\{x\}$ is closed (we say that $x$ is a ``closed
		point'') in $\espectro(A)$, if and only if $\frak p_x$ is
		maximal;
	\item\label{item:ejer:capi:18:ii}
		$\clos{\{x\}}=\ceros{\frak p_x}$;
	\item\label{item:ejer:capi:18:iii}
		$y\in\clos{\{x\}}$, if and only if $\frak p_x\subset\frak p_y$;
	\item\label{item:ejer:capi18:iv}
		$X$ is a $T_0$-space (this means that, if $x$ and $y$ are
		distinct points of $X$, then either there is a neighbourhood of
		$x$ which does not contain $y$, or else there is a
		neighbourhood of $y$ which does not contain $x$).
\end{enumerate}
%

