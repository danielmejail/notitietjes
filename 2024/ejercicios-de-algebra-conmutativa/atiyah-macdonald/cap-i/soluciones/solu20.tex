Sea $A$ un anillo y sea $X=\espectro(A)$ su espectro. Por los Ejercicios~%
\ref{ejer:capi:18} \eqref{item:ejer:capi:18:ii} y \ref{ejer:capi:20}
\eqref{item:ejer:capi:20:i}, si $\frak p\subset A$ es primo, el conjunto
$\ceros{\frak p}$ es irreducible (los espacios con un \'unico punto son
irreducibles). Para probar que las componentes irreducibles (cerrados
irreducibles maximales de $X$) est\'an en correspondencia con los primos
minimales de $A$, ser\'a suficiente probar que, si $\frak b\subset A$ es un
ideal radical tal que $\ceros{\frak b}$ es irreducible, entonces $\frak b$ es
primo.

Sean $f,g\in A$ tales que $f\,g\in\frak b$, pero ni $f$ ni $g$ pertenecen a
$\frak b$. Por el Ejercicio~\ref{ejer:capi:17} \eqref{item:ejer:capi:17:i},
$\principal{f\,g}=\principal f\cap\principal g$ y, si $\frak p\supset\frak b$
es un ideal primo, entonces $f\,g\in\frak p$, es decir,
$\frak p\not\in\principal{f\,g}$. Ahora, dado que $f\not\in\frak b$ y que
$\frak b$ es radical, existe $\frak p\in\ceros{\frak b}$ tal que
$f\not\in\frak p$; an\'alogamente, existe $\frak q\in\ceros{\frak b}$ tal que
$g\not\in\frak q$. Esto implica que $\principal f\cap\ceros{\frak b}$ y
$\principal g\cap\ceros{\frak b}$ son abiertos no vac\'{\i}os de
$\ceros{\frak b}$, pero
\begin{align*}
	\big(\principal f\cap\ceros{\frak b}\big)\,\cap\,
		\big(\principal g\cap\ceros{\frak b}\big) & \,=\,
		\principal{f\,g}\,\cap\,\ceros{\frak b}\,=\,\varnothing
	\text{ .}
\end{align*}
%
En particular, $\ceros{\frak b}$ no puede ser irreducible, si $\frak b$ no es
primo.

