Asumiendo \eqref{item:ejer:capi:10:iii}, $\nilrad$ es un ideal maximal. En
particular, todo ideal primo es igual a $\nilrad$, lo que implica
\eqref{item:ejer:capi:10:i}.

Asumiendo \eqref{item:ejer:capi:10:ii}, para todo $x\not\in\nilrad$,
$x\in A^\times$, lo que implica \eqref{item:ejer:capi:10:iii}.

Asumiendo \eqref{item:ejer:capi:10:i}, $\nilrad$ es el \'unico ideal primo y,
en particular, el \'unico ideal maximal. En consecuencia, si $x\in A$ no es una
unidad, entonces $x\in\nilrad$, es decir, \eqref{item:ejer:capi:10:ii}.

