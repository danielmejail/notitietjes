Si $f,g\in A[\![x]\!]$ con coeficientes $a_i$ y $b_j$, respectivamente, el
coeficiente en grado $k$ del producto $f\,g$ est\'a dado por
$c_k=\sum_{i+j=k}\,a_i\,b_j$. En particular, si $f$ es una unidad,
$a_0\,b_0=1$ (en $A$), para cierto $b_0\in A$. Rec\'{\i}procamente, si
$a_0\,b_0=1$ en $A$, entonces es posible definir, inductivamente, coeficientes
$b_j$ de manera que $g=\sum_{j\geq 0}\,b_j\,x^j$ sea el inverso de $f$:
\begin{equation}
	\label{eq:ejer:capi:05:coeficienteinverso}
	b_k \,=\,\big(-\sum_{
			\begin{smallmatrix}
				i+j=k \\
				i\neq 0
			\end{smallmatrix}
			}\,a_i\,b_j\big)\,b_0
	\text{ .}
\end{equation}
%

Si $f$ es nilpotente, existe $k\geq 0$ tal que $f^k=0$. Para este valor de $k$,
$a_0^k=0$, es decir, $a_0$ es nilpotente en $A$. En particular, $a_0$ es
nilpotente en $A[\![x]\!]$ y $f-a_0$ es nilpotente. Si
$g=\sum_{i\geq 1}\,a_i\,x^{i-1}$, $f-a_0=g\,x$ y existe $l\geq 0$ tal que
$g^l\,x^l=0$. Dado que una serie es igual a $0$, si y s\'olo si sus
coeficientes son todos nulos (en $A$), se deduce que $g^l=0$, es decir, que $g$
es nilpotente. Inductivamente, $a_n$ es nilpotente para todo $n\geq 0$.

Elegir $\{a_n\}_{n\geq 0}\subset A$ tal que $a_n^{k(n)}=0$ para una funci\'on
creciente $k:\,\bb N\rightarrow\bb N$.

Si $f\in\jacrad(A[\![x]\!])$, cualquiera sea $a\in A\subset A[\![x]\!]$, vale
que $1-a\,f$ es una unidad en el anillo de series de potencias. El coeficiente
principal de $1-a\,f$ es igual a $1-a\,a_0$ y debe ser, por
\eqref{item:ejer:capi:05:i}, una unidad en $A$. Como $a\in A$ es arbitrario, se
deduce que $a_0\in\jacrad(A)$. Rec\'{\i}procamente, si $a_0$ pertenece al
radical de Jacobson de $A$ y $g=\sum_{j\geq 0}\,b_j\,x\in A[\![x]\!]$,
\begin{align*}
	1-g\,f & \,=\,(1-b_0\,a_0)+h\,x
	\text{ ,}
\end{align*}
%
para cierta $h\in A[\![x]\!]$. Dado que $1-b_0\,a_0\in A^\times$, se deduce que
$1-g\,f\in A[\![x]\!]^\times$. As\'{\i}, $f\in\jacrad(A[\![x]\!])$.

Sea $\frak m\subset A[\![x]\!]$ un ideal maximal y sea
$\contracted{\frak m}=\{a\in A\,:\,a\in\frak m\}=\frak m\cap A$ el ideal de $A$
que se obtiene por contracci\'on de $\frak m$ v\'{\i}a la inclusi\'on. Si
$b\not\in\contracted{\frak m}$, entonce $b\not\in\frak m$ y
$\generado b+\frak m=A[\![x]\!]$. Es decir, para $b\in A$ que no pertenece a la
contracci\'on $\contracted{\frak m}$, existe $f,h\in A[\![x]\!]$,
$h\in\frak m$, tales que
\begin{align*}
	1 & \,=\,f\,b+h
	\text{ .}
\end{align*}
%
Si $f=\sum_{k\geq 0}\,f_k\,x^k$ y $h=\sum_{k\geq 0}\,h_k\,x^k$, entonces, en
particular,
\begin{align*}
	1 & \,=\,f_0\,b+h_0
	\text{ ,}
\end{align*}
%
en $A$. Si $x\in\frak m$, entonces $h-h_0=\tilde h\,x\in\frak m$, para cierta
$\tilde h\in A[\![x]\!]$, de lo que se deduce que $h_0\in\contracted{\frak m}$
y, dado que $b$ fue elegido de manera arbitraria en el complemento de
$\contracted{\frak m}$, concluimos que este ideal es maximal en $A$. En
definitiva, todo lo que hay que verificar es que $x\in\frak m$, cualquiera sea
el ideal maximal $\frak m\subset A[\![x]\!]$. Esta afirmaci\'on es consecuencia
de \eqref{item:ejer:capi:05:i}: si $x\not\in\frak m$, entonces
\begin{align*}
	1 & \,=\, f\,x+h
	\text{ ,}
\end{align*}
%
para ciertas $f,h\in A[\![x]\!]$, $h\in\frak m$. Pero, entonces, $h_0=1$ y
$h$ es una unidad, contradiciendo el hecho de que $\frak m$ es un ideal propio
de $A[\![x]\!]$.

De esto \'ultimo, se deduce que, si $\frak m\subset A[\![x]\!]$ es maximal y
$h\in\frak m$, entonces, como $x\in\frak m$, $h_0\in\contracted{\frak m}$. En
conclusi\'on,
\begin{align*}
	\frak m & \,=\,\contracted{\frak m}+\generado x
	\text{ .}
\end{align*}
%

