Sean $f=a_0+a_1\,x+\,\cdots\,+a_m\,x^m$ y $h\in A[x]$ tal que $f=a_0+h\,x$, es
decir, $h=a_1+a_2\,x+\,\cdots\,+a_m\,x^{m-1}$. Supongamos, primero que $a_0$ es
unidad en $A$ y que $\lista a{m}$ son nilpotentes. En este caso, $h$ y $h\,x$
son nilpotentes en $A[x]$. Por el Ejercicio~\ref{ejer:capi:01}, $f$ es una
unidad.

Rec\'{\i}procamente, supongamos que $f$ es una unidad y que
$g=b_0+b_1\,x+\,\cdots\,b_n\,x^n$ es inverso de $f$. Supongamos, adem\'as, que
$m=\grado(f)$ y que $n=\grado(g)$. Sea
\begin{equation}
	\label{eq:ejer:capi:02:coeficienteproducto}
	c_k \,=\,\sum_{i+j=k}\,a_i\,b_j
\end{equation}
%
el coeficiente en grado $k$ del producto $f\,g$. Entonces, $c_0=1$ y $c_k=0$,
para todo $k\geq 1$. Dado que $a_i=0$ y que $b_j=0$, para $i>m$ y $j>n$, las
\'unicas condiciones no triviales sobre los $c_k$ y, en particular, sobre los
coeficientes de $f$, se obtienen con $k\leq m+n$. Por ejemplo,
\begin{align*}
	1 & \,=\,c_0 \,=\, a_0\,b_0
		\text{ ,} \\
	0 & \,=\,c_1 \,=\, a_0\,b_1+a_1\,b_0
		\text{ ,} \\
	& \vdots \\
	0 & \,=\,c_{m+n-1} \,=\, a_m\,b_{n-1}+a_{m-1}\,b_n
		\text{ ,} \\
	0 & \,=\,c_{m+n} \,=\, a_m\,b_n
	\text{ .}
\end{align*}
%
Supongamos, primero, que $m=0$. Entonces, $a_0\in A^\times$ y su inverso es
$b_0$. Pero $a_0\,b_j=0$ (en $A$), para $j\geq 1$ implica que $b_j=0$ y $n=0$,
tambi\'en. Sim\'etricamente, $n=0$ implica $m=0$. Si, en cambio, $m>0$ (y, por
lo tanto, $n>0$), entonces $0<1\leq m+n-1<m+n$. En particular, como $b_n\neq 0$
y $a_m\,b_n=0$, $a_m$ es divisor de cero. Ahora, multiplicando la ante\'ultima
condici\'on por $a_m$, se deduce
\begin{align*}
	0 & \,=\,a_m^2\,b_{n-1}+a_m\,a_{m-1}\,b_n\,=\,a_m^2\,b_{n-1}
	\text{ .}
\end{align*}
%
Supongamos, inductivamente, que $n>r_0\geq 0$ y que
\begin{equation}
	\label{eq:ejer:capi:02:coeficientesanulacion}
	a_m^{r+1}\,b_{n-r} \,=\,0
	\text{ ,}
\end{equation}
%
para todo $r\leq r_0$. En particular, $a_m^{r_0+1}\,b_j=0$, para todo
$j\geq n-r_0$. Para que $b_j$, con $j<n-r_0$, \emph{no} aparezca en la
sumatoria \eqref{eq:ejer:capi:02:coeficienteproducto} que define al coeficiente
$c_k$ --cuesti\'on v\'alida, pues $a_i=0$, si $i>m$--, es suficiente que
$k\geq m+n-r_0$ (pues, en ese caso, $i\leq m$ y $j<n-r_0$ implican
$i+j<m+n-r_0\leq k$). Eligiendo $k=m+n-r_0-1$, como $k\geq m>0$, $c_k=0$ y
\begin{align*}
	0 & \,=\,a_m^{r_0+1}\,c_k \,=\,
		\sum_{
			\begin{smallmatrix}
				i+j=k \\
				j\geq n-r_0
			\end{smallmatrix}
		}\,a_i\,a_m^{r_0+1}\,b_j \\
	& \,=\,a_m^{r_0+2}\,b_{n-(r_0+1)}
	\text{ .}
\end{align*}
%
La \'ultima igualdad se deduce como consecuencia de lo mencionado luego de
\eqref{eq:ejer:capi:02:coeficientesanulacion}. Esto demuestra que
$a_m^{r+1}\,b_{n-r}=0$ para todo $r\geq 0$ y, en particular, que
$a_m^{n+1}\,b_0=0$. Pero $b_0$ (al igual que $a_0$) es una unidad en $A$, con
lo cual $a_m^{n+1}=0$. Finalmente, como $a_m$ es nilpotente (en $A$),
$a_m\,x^m$ lo es en $A[x]$ y, apelando al Ejercicio~\eqref{ejer:capi:01},
nuevamente,
\begin{align*}
	f \,-\,a_m\,x^m
\end{align*}
%
es una unidad (en $A[x]$). Inductivamente en el grado de $f$, se deduce que
$a_0\in A^\times$ y que $\lista a{m}\in \nilrad(A)$. Esto prueba el \'{\i}tem~%
\eqref{item:ejer:capi:02:i}.

Si $f$ es nilpotente (en $A[x]$), entonces $1+f$ es una unidad en el anillo de
polinomios y los coeficientes $a_0,\,\lista a{m}$ son nilpotentes, por el
\'{\i}tem anterior, demostrando \eqref{item:ejer:capi:02:ii}.

En cuanto a \eqref{item:ejer:capi:02:iii}, si $a\,f=0$, $f$ es divisor de cero.
Rec\'{\i}procamente, si $f$ es divisor de $0$, existe $g\in A[x]$ no nulo, tal
que $f\,g=0$. M\'as aun, podemos asumir que $g=b_0+b_1\,x+\,\cdots\,b_n\,x^n$
con $n$ m\'{\i}nimo (en particular, $b_n\neq 0$ y $n=\grado(g)$). Dado que
$a_m\,b_n=0$ y que $(a_m\,g)\,f=0$, por minimalidad de $n$, $a_m\,g=0$. Si
$h_r\in A[x]$ es el polinomio que se obtiene truncando $f$ en grado $r$ ($h_r$
es de grado $r$ o menor, o cero),
\begin{align*}
	0 & \,=\,f\,g \,=\,h_{m-1}\,g +a_m\,g\,x
	\text{ ,}
\end{align*}
%
de lo que se deduce que $h_{m-1}\,g=0$. Esto implica que $a_{m-1}\,b_n=0$ y que
$a_{m-1}\,g=0$, por la minimalidad de $n$. Inductivamente, $a_r\,g=0$, para
cada $r\geq 0$, y $a_r\,b_n=0$. Esto implica que $f\,b_n=0$.

Finalmente, para probar \eqref{item:ejer:capi:02:iv}, sean
$f=a_0+a_1\,x+\,\cdots\,+a_m\,x^m$ y $g=b_0+b_1\,x+\,\cdots\,+b_n\,x^n$ y sea
$c_k$ el coeficiente en grado $k$ del producto $f\,g$, dado por
\eqref{eq:ejer:capi:02:coeficienteproducto}. Si $f\,g$ es primitivo y
$\sum_k\,c_k\,\gamma_k=1$, entonces
\begin{math}
	\sum_i\,a_i\Big(\sum_j\,b_j\,\gamma_{i+j}\Big)=1
\end{math}, lo que implica que $f$ es primitivo. Si, en ca,bio, el producto
$f\,g$ n es primitivo y el conjunto $\{c_k\}_k$ est\'a contenido en alg\'un
ideal maximal $\frak m\subset A$, entonces $f\,g=0$ en $(A/\frak m)[x]$. Pero
$A/\frak m$ es cuerpo y, en particular, el anillo de polinomios
$(A/\frak m)[x]$ es un dominio \'{\i}ntegro, con lo cual, o bien $f=0$, o bien
$g=0$, all\'{\i}; es decir, o bien $f$ no es primitivo (y sus coeficientes
est\'an contenidos en $\frak m$), o bien $g$ no lo es.

