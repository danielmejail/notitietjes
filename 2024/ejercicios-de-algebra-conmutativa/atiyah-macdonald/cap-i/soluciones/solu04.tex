En un anillo conmutativo $A$, un elemento $x\in A$ pertenece al radical de
Jacobson $\jacrad(A)$, si y s\'olo si $1-y\,x\in A^\times$, cualquiera sea
$y\in A$. Tambi\'en, $x$ pertenece al nilradical $\nilrad(A)$, si y s\'olo si
es nilpotente. Pasando al anillo de polinomios, si $f\in\jacrad(A[x])$,
entonces $1+f$ es una unidad y, como en la demostraci\'on del Ejercicio~%
\ref{ejer:capi:02} \eqref{item:ejer:capi:02:ii}, los coeficientes de $f$ son
nilpotentes, lo que implica que $f$ es nilpotente.

