Si $\frak a=\generado 1$, existir\'{\i}an $\lista f{r}\in\Sigma$ y
$\lista h{r}\in A$ tales que
\begin{equation}
	\label{eq:ejer:13:idealpropio}
	1 \,=\,h_1\,f_1(x_{f_1})+\,\cdots\,+h_r\,f_r(x_{f_r})
	\text{ .}
\end{equation}
%
Esta igualdad es v\'alida, en particular, en el subanillo $B\subset A$
compuesto por todos aquellos polinomios con coeficientes en $K$ que involucran,
\'unicamente, las variables $x_{f_i}$ y las $x_f$ que aparecen en los $h_i$;
el anillo $B$ es una anillo de polinomios en una cantidad finita de
indeterminadas. Por otro lado, existe una extensi\'on finita $E/K$ que
contiene, para cada $f_i$, una ra\'{\i}z $\alpha_i$. Por propiedad universal
del anillo de polinomios $B$, existe un \'unico morfismo de $K$-\'algebras
$\mu:\,B\rightarrow E$ que verifica
\begin{align*}
	\mu(x_f) & \,=\,
		\begin{cases}
			\alpha_i & \text{ si } f=f_i \text{ ,} \\
			0 & \text{ en otro caso.}
		\end{cases}
\end{align*}
%
(Es suficiente pedir que $\mu(x_{f_i})=\alpha_i$, aunque se pierda la unicidad
de $\mu$). Aplicando $\mu$ a la identidad \eqref{eq:ejer:13:idealpropio}, se
deduce que $1=0$ en $E$. Pero esto es absurdo, pues $E$ es un cuerpo.

El cuerpo $L$ es algebraicamente cerrado, pues todo polinomio $f\in L[x]$
tiene coeficientes pertenecientes a alg\'un subcuerpo $K_n$ y, por lo tanto,
$f$ posee alguna ra\'{\i}z en $K_{n+1}\subset L$.

