Existen ideales maximales. Los ideales maximales son primos. Si $\cal C$ es una
cadena de ideales primos ordenada por inclusi\'on ($\frak p<\frak p'$, si
$\frak p\supset\frak p'$), entonces $\frak q=\bigcap\,\cal C$ es un ideal
primo: si $x\,y\in\frak q$, entonces, para cada $\frak p\in\cal C$, o bien
$x\in\frak p$, o bien $y\in\frak p$. Si ni $x$, ni $y$ pertenecieran a,
$\frak q$, existir\'{\i}an primos $\frak p$ y $\frak p'$ pertenecientes a la
cadena tales que $x\not\in\frak p$ e $y\not\in\frak p'$. Sin p\'erdida de
generalidad, $\frak p<\frak p'$, de lo que se deducir\'{\i}a que ni $x$, ni $y$
pertenecen a $\frak p'$.

