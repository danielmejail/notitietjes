Sea $D\subset A$ el conjunto de todos los divisores de cero en $A$ y sea
$\cal M$ el conjunto de los elementos maximales de la familia $\Sigma$. En
primer lugar, asumiendo que $\Sigma$ posee elementos maximales, si $x\in D$, el
ideal $\generado x$ pertenece a $\Sigma$ y est\'a contenido en alg\'un
$I\in\cal M$. Entonces, $D\subset\bigcup\,\cal M$ y, como los ideales
pertenecientes a $\Sigma$ est\'an incluidos, por definici\'on, en $D$,
$\bigcup\,\cal M\subset D$. Es decir, $D$ es la uni\'on de los elementos
maximales de $\Sigma$. M\'as aun, si dichos elementos son ideales primos,
entonces $D$ es uni\'on de ideales primos. Esto prueba la \'ultima
afirmaci\'on.

La familia $\Sigma$ es no vac\'{\i}a: $0\in\Sigma$. Si $\cal C$ es una cadena
de elementos de la familia ordenada por inclusi\'on, entonces $\bigcup\,\cal C$
es un ideal y todos sus elementos son divisores de cero. En conclusi\'on,
$\Sigma$ posee elementos maximales.

Sea $I\in\Sigma$ un elemento maximal y sean $x,y\in A$ tales que $x\,y\in I$.
Esto implica que $x\,y$ es divisor de cero y existe $a\in A$ no nulo, tal que
$x\,y\,a=0$. Si $y\,a=0$, $y$ es divisor de cero, si no, $x$ es divisor de
cero. En el primer caso, $\generado y + I\in\Sigma$, con lo cual, por
maximalidad de $I$, $y\in I$. En el segundo caso, $x\in I$. En definitiva, $I$
es primo.

