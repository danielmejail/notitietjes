Todo abierto es de la forma $\espectro(A)\setmin\ceros{\frak a}$ para alg\'un
ideal $\frak a\subset A$. Dados, entonces $\frak a$ y
$x\in\espectro(A)\setmin\ceros{A}$, existe $f\in\frak a$ tal que
$f\not\in\frak p_x$. Dado que
% \begin{equation}
	% \label{eq:ejer:capi:17:abiertos}
	% x\not\in\ceros{\frak a}
		% % \text{(\phantom)}
		% % \frak p_x\not\supset\frak a\text{\phantom()}
		% \,\Leftrightarrow\,\big(\exists\,f\in\frak a\big)\,
			% \big(f\not\in\frak p_x\big)
	% \text{ .}
% \end{equation}
% %
% Es decir, si $\principal f=\espectro(A)\setmin\ceros{f}$, entonces
\begin{equation}
\label{eq:ejer:capi:17:abiertosprincipales}
	\principal f \,=\,\big\{y\in\espectro(A)\,:\,f\not\in\frak p_y\big\}
	\text{ ,}
\end{equation}
%
se concluye que $x\in\principal f\subset\espectro(A)\setmin\ceros{\frak a}$. En
definitiva, los conjuntos $\principal f$ constituyen una base de abiertos para
la topolog\'{\i}a de Zariski.

Para probar que $\espectro(A)$ es compacto, ser\'a suficiente demostrar que
todo cubrimiento por abiertos principales, es decir, de la forma
$\principal f$, admite un subcubrimiento finito, puesto que todo abierto es
uni\'on de abiertos b\'asicos. Supongamos, entonces, que $\{f_i\}_i$ es una
familia de elementos de $A$ tal que
\begin{align*}
	\espectro(A) & \,=\,\bigcup_i\,\principal{f_i}
	\text{ .}
\end{align*}
%
El ideal $\frak a=\generado{f_i\,:\,i}$ no est\'a contenido en ning\'un ideal
primo, ya que, si $\frak p\supset\frak a$, entonces $f_i\in\frak p$ para todo
$i$, lo que equivale a que el punto correspondiente a $\frak p$ no pertenezca a
ninguno de los $\principal{f_i}$. Expresado de otra manera, $\frak a=1$, es
decir que existen un subconjunto finito de \'{\i}ndices $j$ y elementos
correspondientes $g_j\in A$ tales que
\begin{align*}
	1 & \,=\,\sum_j\,g_j\,f_j
	\text{ .}
\end{align*}
%
As\'{\i}, si $x\in\espectro(A)$, $1\not\in\frak p_x$ y existe $j$ tal que
$f_j\not\in\frak p_x$, lo que quiere decir que $x\in\principal{f_j}$.

Los abiertos $\principal f$ tambi\'en son compactos, pues la localizaci\'on
$A\rightarrow\localizacion{f}{A}$ induce un homeomorfismo
$\principal f=\espectro(\localizacion{f}{A})$.

