% Sea $M$ un $A$-m\'odulo f.g. y sea $\{\lista x{r}\}$ un conjunto finito de
% generadores. Existe un morfismo sobreyectivo $A^r\rightarrow M$ dado por
% $(\lista a{r})\mapsto \sum_i\,a_i\,x_i$. Dado un endomorfismo
% $\phi\in\Endo[A](M)$, tambi\'en existe un morfismo (de $A$-m\'odulos)
% \begin{align*}
	% & A[T]\,x_1\,\oplus\,\cdots\,\oplus\,A[T]\,x_r \,\rightarrow\, M
% \end{align*}
% %
% dado por
% \begin{align*}
	% & \sum_i\,p_i(T)\,x_i\,\mapsto\,\sum_i\,p_i(\phi)\,x_i
	% \text{ .}
% \end{align*}
% %
% Si $s:\,A[T]\rightarrow\Endo[A](M)$ denota el morfismo de $A$-\'algebras
% determinado por $s(T)=\phi$, entonces, para $p\in A[T]$, $p(\phi)=s(p)$. Dado
% que $\{\lista x{r}\}$ es un conjunto generador, existen coeficientes
% $a_{ij}\in A$ tales que
% \begin{align*}
	% \phi(x_i) & \,=\,\sum_j\,a_{ij}\,x_j
	% \text{ .}
% \end{align*}
% %
% En particular, a $\phi$ le podemos asociar una matriz
% \begin{align*}
	% \alpha & \,\in\,\MM[n\times n](A[T])\,=\,
		% \Endo[{A[T]}](A[T]\,x_1\oplus\,\cdots\,\oplus A[T]\,x_r)
% \end{align*}
% %
% dada por
% \begin{align*}
	% \alpha & \,=\,
		% \begin{bmatrix}
			% T-a_{11} & -a_{12} & \cdots & -a_{1r} \\
			% -a_{21} & T-a_{22} & \cdots & -a_{2r} \\
			% \vdots & \vdots & & \vdots \\
			% -a_{r1} & -a_{r2} & \cdots & T-a_{rr}
		% \end{bmatrix}
	% \text{ .}
% \end{align*}
% %
% 
% \begin{center}
	% \texttt{---X---}
% \end{center}

?`Es cierto que $\frak a\subset\jacrad$, si y s\'olo si los elementos de
$1+\frak a$ son unidades?

