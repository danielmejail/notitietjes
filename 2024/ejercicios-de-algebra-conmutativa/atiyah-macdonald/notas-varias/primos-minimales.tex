\begin{propoNotas}\label{propo:notas:primos-minimales}
	Los ideales primos minimales est\'an contenidos en el conjunto de
	divisores de cero.
\end{propoNotas}

Sea $A$ un anillo conmutativo con unidad y sea $D$ el \emph{conjunto}%
\footnote{
	$2+3=5$ en $\enterosmod[6]$. Si bien $xy=x'y'=0$ implica
	$(x+x')\,yy'=0$, aunque $y,y'\neq 0$, no se deduce que $yy'\neq 0$;
	justamente, $y$ e $y'$ tambi\'en son divisores de cero. Con lo cual,
	no se puede concluir que $x+x'$ sea divisor de cero, si $x$ y $x'$ lo
	son.
}
de divisores de cero de $A$. El radical de un subconjunto $E\subset A$ es el
conjunto $\rad(E)$ compuesto por todos los elementos del anillo que, elevados a
alguna potencia, pertenecen a $E$. Vale que
\begin{displaymath}
	D \,=\, \rad(D)\,=\,\bigcup_{x\neq 0}\,\rad(\Ann(x))
	\text{ .}
\end{displaymath}
%

\begin{lemaNotas}\label{lema:notas:primos-minimales:existen}
	Existen primos minimales.
\end{lemaNotas}

\begin{lemaNotas}\label{lema:notas:primos-minimales:union-de-ideales}
	El conjunto $D$ de divisores de cero es uni\'on de ideales primos.
\end{lemaNotas}

\begin{proof}
	Si $\Sigma$ denota el conjunto de ideales contenidos en $D$, entonces
	$\Sigma$ posee elementos maximales. Todo elemento maximal de $\Sigma$
	es un ideal primo. Veamos esta \'ultima afirmaci\'on. Si
	$\frak a\in\Sigma$ es un elemento maximal e $y\in A\setmin\frak a$,
	entonces $\generado y+\frak a$ es un ideal que contiene estrictamente
	a $\frak a$ y, por lo tanto, no pertenece a $\Sigma$. En particular,
	$\generado y+\frak a\not\subset D$, es decir, existen $t\in A$ y
	$b\in\frak a$ tales que
	\begin{displaymath}
		yt+b\,\not\in\,D
		\text{ .}
	\end{displaymath}
	%
	Si $x\in A\setmin\frak a$, an\'alogamente, existen $s\in A$ y
	$a\in\frak a$ tales que $xs+a\not\in D$. Por lo tanto, si
	$xy\in\frak a$, pero $x,y\not\in\frak a$,
	\begin{displaymath}
		(xs+a)\,(yt+b)\,\in\,(xy)\,st+\frak a\,\subset\,\frak a
			\,\subset\,D
		\text{ .}
	\end{displaymath}
	%
	Pero esto es absurdo, pues el conjunto $A\setmin D$ de los elementos
	que no son divisores de cero es multiplicativamente cerrado.

	Como $D$ es uni\'on de ideales, debe ser uni\'on de los ideales
	maximales que contiene. En particular, por lo anterior, es uni\'on de
	ideales primos.
\end{proof}

\begin{obsNotas}\label{obs:notas:primos-minimales:multiplicativamente-cerrados}
	Un subconjunto $S\subset A$ es multiplicativamente cerrado, si
	$s,t\in S$ implica $st\in S$. En particular, $S$ es multiplicativamente
	cerrado, si y s\'olo si su complemento cumple que $xy\in A\setmin S$
	implica $x\in A\setmin S$, o bien $y\in A\setmin S$.
\end{obsNotas}

\begin{obsNotas}\label{obs:notas:primos-minimales:anillo-cero}
	Supondremos, en lo que queda, que los conjuntos multiplicativamente
	cerrados verifican $1\in S$ y $0\not\in S$.%
	\footnote{
		Excluimos los anillos en donde $0=1$.
	}
\end{obsNotas}

\begin{lemaNotas}%
	\label{lema:notas:primos-minimales:multiplicativamente-cerrados}
	Si $S\subset A$ es un subconjunto multiplicativamente cerrado y
	maximal (con respecto a $0\not\in S$ y a $1\in S$), entonces
	$A\setmin S$ es un ideal primo minimal.
\end{lemaNotas}

\begin{proof}
	Veamos que $A\setmin S$ es ideal. Si $y\in A\setmin S$, entonces
	\begin{displaymath}
		T\,=\,\Big\{sy^r\,:\,s\in S,\,r\geq 0\Big\}
	\end{displaymath}
	%
	es un conjunto multiplicativamente cerrado que contiene a $S$ y a $1$.
	Por maximalidad de $S$, se deduce que $0\in T$, es decir, existen
	$s\in S$ y $r\geq 0$ tales que
	\begin{displaymath}
		sy^r\,=\,0
		\text{ ;}
	\end{displaymath}
	%
	como $0\not\in S$, $r\geq 1$. Sea $x\in A$ un elemento arbitrario;
	queremos ver que $xy\in A\setmin S$. Si $xy\in S$, entonces
	\begin{displaymath}
		0\,=\,xsy^r\,=\,(s\,(xy))\,y^{r-1}
		\text{ ,}
	\end{displaymath}
	%
	con $s\,(xy)\in S$, por ser multiplicativamente cerrado. Pero entonces
	es posible tomar un valor m\'as peque\~no para la potencia $r$. Como
	$r=0$ es imposible, se llega a una contradicci\'on de suponer que
	$xy$ podr\'{\i}a no pertenecer a $A\setmin S$. As\'{\i}, $A\setmin S$
	es cerrado por multiplicar por elementos arbitrarios de $A$. Veamos que
	es cerrado por sumas: sean $y,y_1\not\in S$. Argumentando de manera
	an\'aloga, existen $s,s_1\in S$ y $r,r_1\geq 1$ tales que
	\begin{displaymath}
		sy^r\,=\,s_1y_1^{r_1}\,=\,0
		\text{ .}
	\end{displaymath}
	%
	Pero, entonces, por la f\'ormula del binomio,
	\begin{displaymath}
		ss_1\,(y+y_1)^{r+r_1-1}\,=\,0
		\text{ ,}
	\end{displaymath}
	%
	con $r+r_1-1\geq 1$. Como $0\not\in S$ y $S$ es multiplicativamente
	cerrado, $y+y_1\not\in S$.

	Que $A\setmin S$ es un ideal primo, se deduce de la Observaci\'on~%
	\ref{obs:notas:primos-minimales:multiplicativamente-cerrados}. La
	minimalidad de $A\setmin S$ es inmediata de la maximalidad de $S$.
\end{proof}
