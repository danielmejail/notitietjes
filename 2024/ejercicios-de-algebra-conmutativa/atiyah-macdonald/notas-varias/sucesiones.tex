Sea $A$ un anillo (conmutativo con $1$) y sean $M$, $M'$ y $M''$ tres $A$-%
m\'odulos y sean $f:\,M'\rightarrow M$ y $g:\,M\rightarrow M''$ morfismos. Si
la sucesi\'on \emph{de $A$-m\'odulos}
\begin{center}
	\begin{tikzcd}
		0 \arrow[r] & \Hom[A](M'',N)\arrow[r,"{\pull g}"] &
			\Hom[A](M,N)\arrow[r,"{\pull f}"] &
			\Hom[A](M',N)
	\end{tikzcd}
\end{center}
es exacta para todo $A$-m\'odulo $N$, entonces la sucesi\'on
\begin{center}
	\begin{tikzcd}
		M'\arrow[r,"f"] & M\arrow[r,"g"] & M''\arrow[r] & 0
	\end{tikzcd}
\end{center}
es exacta.

\begin{obsNotas}\label{obs:notas:sucesiones}
	Esta es la versi\'on contravariante. La versi\'on covariante se deduce
	del isomorfismo $\Hom[A](A,N)\simeq N$.
\end{obsNotas}

Para demostrar la afirmaci\'on, consideramos, en primer lugar, el m\'odulo
$N=\coker(g)$ y el diagrama conmutativo:
\begin{center}
	\begin{tikzcd}
		M\arrow[r,"g"] \arrow[dr] & M''\arrow[d] \\
		& \coker(g)
	\end{tikzcd}
	,
\end{center}
donde la flecha vertical es el cociente por el subm\'odulo $\img(g)$ y la
flecha diagonal es la composici\'on de $g$ con el mismo. Notamos que
$g$ es sobre, si \emph{y s\'olo si} el cociente es el morfismo nulo. En
particular, la inyectividad de $\pull g$ y el hecho de que la composici\'on de
$g$ con el cociente es cero implican que $g$ debe ser sobre.

Por otro lado, $g\circ f=\pull f(g)=\pull f\circ\pull g(\id[M''])$. Tomando
$N=M''$ en la sucesi\'on exacta, se deduce que $g\circ f=0$ y, por lo tanto,
que $\img(f)\subset\ker(g)$.

Finalmente, tomamos $N=\coker(f)$ y consideramos el diagrama
\begin{center}
	\begin{tikzcd}
		M \arrow[d,"g"'] \arrow[r] & \coker(f) \\
		M'' \arrow[ur, dashed] &
	\end{tikzcd}
	,
\end{center}
donde la flecha horizontal es el cociente. Dado que $\pull f$ aplicada al
morfismo cociente da cero, el mismo debe, por exactitud, pertenecer a la imagen
de $\pull g$, es decir, debe existir una flecha diagonal que haga conmutar el
diagrama anterior. De la existencia de tal morfismo y de que el n\'ucleo del
cociente es $\img(f)$, se deduce que $\ker(g)\subset\img(f)$.

