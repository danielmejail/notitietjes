El bifuntor $\Hom$ posee las siguientes propiedades:
\begin{propoNotas}\label{propo:notas:exactitud:hom}
	Sea $A$ un anillo conmutativo con unidad y sean $f:\,M'\rightarrow M$ y
	$g:\,M\rightarrow M''$ morfismos de $A$-m\'odulos. Entonces,
	\begin{enumerate}[(i)]
		\item la sucesi\'on
			\begin{center}
				\begin{tikzcd}
					M'\arrow[r,"f"] & M\arrow[r,"g"] &
						M''\arrow[r] & 0
				\end{tikzcd}
			\end{center}
			es exacta, si y s\'olo si, para todo $A$-m\'odulo $N$,
			la sucesi\'on
			\begin{center}
				\begin{tikzcd}
					0\arrow[r] & \Hom[A](M'',N)
							\arrow[r,"{\pull g}"] &
						\Hom[A](M,N)
							\arrow[r,"{\pull f}"] &
						\Hom[A](M',N)
				\end{tikzcd}
			\end{center}
			lo es;
		\item la sucesi\'on
			\begin{center}
				\begin{tikzcd}
					0 \arrow[r] & M' \arrow[r,"f"] &
						M \arrow[r,"g"] & M''
				\end{tikzcd}
			\end{center}
			es exacta, si y s\'olo si, para todo $A$-m\'odulo $P$,
			la sucesi\'on
			\begin{center}
				\begin{tikzcd}
					0\arrow[r] & \Hom[A](P,M')
							\arrow[r,"{\push f}"] &
						\Hom[A](P,M)
							\arrow[r,"{\push g}"] &
						\Hom[A](P,M'')
				\end{tikzcd}
			\end{center}
			lo es.
	\end{enumerate}
	%
\end{propoNotas}

Sean $T$ y $U$ funtores adjuntos, es decir, que verifican
\begin{align*}
	\Hom[A](T(M),N) & \,\simeq\,\Hom[A](M,U(N))
\end{align*}
%
naturalmente. Entonces, dada una sucesi\'on exacta corta
\begin{center}
	\begin{tikzcd}
		M'\arrow[r,"f"] & M\arrow[r,"g"] & M''\arrow[r] & 0
	\end{tikzcd}
	,
\end{center}
para todo $N$, la siguiente es una sucesi\'on exacta corta:
\begin{center}
	\begin{tikzcd}
		0\arrow[r] & \Hom[A](M'',U(N)) \arrow[r,"{\pull g}"] &
			\Hom[A](M,U(N)) \arrow[r,"{\pull f}"] &
			\Hom[A](M',U(N))
	\end{tikzcd}
	.
\end{center}
Pero esta sucesi\'on es \emph{equivalente} a
\begin{center}
	\begin{tikzcd}
		0\arrow[r] & \Hom[A](T(M''),N) \arrow[r,"{\pull{T(g)}}"] &
			\Hom[A](T(M),N) \arrow[r,"{\pull{T(f)}}"] &
			\Hom[A](T(M'),N)
	\end{tikzcd}
	.
\end{center}
Como la exactitud de esta sucesi\'on es v\'alida para todo $A$-m\'odulo $N$, la
sucesi\'on
\begin{center}
	\begin{tikzcd}
		T(M') \arrow[r,"T(f)"] & T(M) \arrow[r,"T(g)"] &
			T(M'') \arrow[r] & 0
	\end{tikzcd}
\end{center}
es exacta. An\'alogamente, se demuestra que, si la sucesi\'on
\begin{center}
	\begin{tikzcd}
		0 \arrow[r] & M' \arrow[r,"f"] & M \arrow[r,"g"] & M''
	\end{tikzcd}
\end{center}
es exacta, entonces la siguiente es una sucesi\'on exacta corta, tambi\'en:
\begin{center}
	\begin{tikzcd}
		0\arrow[r] & U(M') \arrow[r,"U(f)"] & U(M) \arrow[r,"U(g)"] &
			U(M'')
	\end{tikzcd}
	.
\end{center}

