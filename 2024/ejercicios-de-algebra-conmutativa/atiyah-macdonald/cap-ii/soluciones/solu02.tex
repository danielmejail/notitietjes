Si bien $M$ podr\'{\i}a no ser playo, tensorizar con $M$ preserva la
inyectividad, en este caso. Por exactitud del producto tensorial, la sucesi\'on
\begin{equation}
	\label{eq:capii:ejer:02:tensorizada}
	\begin{tikzcd}
		\frak a\tensor[A] M\arrow[r] & A\tensor[A] M\arrow[r] &
			(A/\frak a)\tensor[A] M\arrow[r] & 0
	\end{tikzcd}
\end{equation}
%
es exacta. Resta ver que el n\'ucleo del primer morfismo de la izquierda es
cero. Ahora bien,
\begin{equation}
	\label{eq:capii:ejer:02:productocontraideal}
	\frak a\tensor[A] M \,\simeq\, A\tensor[A]\frak a\,M \,\simeq\,
		\frak a\,M
	\text{ ,}
\end{equation}
%
con lo cual, la exactitud de
\begin{align*}
	\begin{tikzcd}[ampersand replacement=\&]
		0\arrow[r] \& \frak a\,M\arrow[r] \& M
	\end{tikzcd}
\end{align*}
%
implica la exactitud de
\begin{equation*}
	\begin{tikzcd}
		0\arrow[r] & A\tensor[A]\frak a\,M\arrow[r] & A\tensor[A] M
	\end{tikzcd}
	\text{ ,}
\end{equation*}
%
pues $A$ es playo y, as\'{\i},
\begin{center}
	\begin{tikzcd}
		0\arrow[r] & \frak a\tensor[A] M\arrow[r] & M
	\end{tikzcd}
\end{center}
es exacta.

Otra manera de probar la inyectividad de $\frak a\tensor[A] M\rightarrow M$ es
la siguiente: si $\sum\,a\tensor x$ es tal que $\sum\,a\,x=0$ en $M$, entonces
\begin{align*}
	\sum\,a\tensor x & \,=\,\sum\,1\tensor\cdots\qquad!
\end{align*}
%
!`Ah! Entonces, en la ecuaci\'on \eqref{eq:capii:ejer:02:productocontraideal},
el morfismo natural $\frak a\tensor[A]M\rightarrow M$ dado por
$a\tensor x\mapsto a\,x$ no es un isomorfismo, en general. Por ejemplo, si
$A=\bb Z$, $\frak a=2\,\bb Z$ y $M=\enterosmod[2]$, entonces
\begin{align*}
	2\,\bb Z\tensor[\bb Z](Z/2\,\bb Z) \,\not\simeq\,
		(2\,\bb Z)\,(\enterosmod[2])
	\text{ ,}
\end{align*}
%
pues el lado izquierdo no es el m\'odulo trivial, mientras que el de la derecha
s\'{\i} lo es.

De esto se puede observar, en primer lugar, que el isomorfismo
\eqref{eq:capii:ejer:02:productocontraideal} es equivalente a la inyectividad
de $\frak a\tensor[A] M\rightarrow M$. En segundo lugar, si $M$ es playo,
entonces el isomorfismo \eqref{eq:capii:ejer:02:productocontraideal} es
v\'alido. Si $A$ es un dominio de Dedekind y $M$ es un ideal de $A$, entonces
\eqref{eq:capii:ejer:02:productocontraideal} se cumple.

En el contraejemplo a la afirmaci\'on, la sucesi\'on exacta corta
\begin{center}
	\begin{tikzcd}
		0\arrow[r] & 2\,\bb Z\arrow[r] & \bb Z\arrow[r] &
			\enterosmod[2]\arrow[r] & 0
	\end{tikzcd}
\end{center}
da lugar a la sucesi\'on exacta
\begin{center}
	\begin{tikzcd}
		2\,\bb Z\tensor[\bb Z](\enterosmod[2])\arrow[r,"0"] &
			\bb Z\tensor[\bb Z](\enterosmod[2])\arrow[r] &
			(\enterosmod[2])\tensor[\bb Z](\enterosmod[2])
				\arrow[r] & 0
	\end{tikzcd}
	.
\end{center}
Dado que, en este caso, $\frak a\,M=0$, se deduce que el isomorfismo
$(A/\frak a)\tensor[A]M\simeq M/\frak a\,M$ es v\'alido, aunque
\eqref{eq:capii:ejer:02:productocontraideal} no lo sea.

En el caso general, la sucesi\'on \eqref{eq:capii:ejer:02:tensorizada} es
exacta. Dado que la imagen del primer morfismo es el subm\'odulo $\frak a\,M$
(identificando $M$ con $A\tensor[A]M$) y dado que el segundo morfismo es
sobreyectivo, se deduce que
\begin{align*}
	(A/\frak a)\tensor[A] M & \,\simeq\, M/\frak a\,M
	\text{ .}
\end{align*}
%

