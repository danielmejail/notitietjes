\theoremstyle{plain}
\newtheorem{teoExtRec}{\teoname}[section]
\newtheorem{coroExtRec}[teoExtRec]{\coroname}
\newtheorem{lemaExtRec}[teoExtRec]{\lemaname}
\newtheorem{propoExtRec}[teoExtRec]{\proponame}

\theoremstyle{definition}
\newtheorem{defExtRec}[teoExtRec]{\defname}
\newtheorem{obsExtRec}[teoExtRec]{\obsname}
\newtheorem{ejemExtRec}[teoExtRec]{\ejemname}
\newtheorem{ejerExtRec}[teoExtRec]{\ejername}

%-------------

\subsection{Espacios bilineales}\label{subsec:extension:recap:bilineales}
De acuerdo con la \defname~\ref{def:definiciones:bilineal}, un \emph{espacio %
bilineal} es un par $(V,B)$, donde $V$ es un espacio vectorial sobre un cuerpo,
$F$, y $B:\,V\times V\rightarrow F$ es una forma bilineal.

\begin{defExtRec}\label{def:recap:morfismo:bilineal}
	Un \emph{morfismo de espacios bilineales},
	\begin{displaymath}
		(V,B_V)\,\rightarrow\,(W,B_W)
	\end{displaymath}
	%
	es una t.l. $A:\,V\rightarrow W$ tal que
	\begin{displaymath}
		B_W(A\,v,A\,v_1)\,=\,B_V(v,v_1)
		\text{ ,}
	\end{displaymath}
	%
	para todo $v,v_1\in V$.
\end{defExtRec}

Es decir, un morfismo entre espacios bilineales es una t.l. entre los e.v.
subyacentes que preserva las formas bilineales. Los morfismos de espacios
bilineales preservan la relaci\'on de ortogonalidad.%
\footnote{
	C.f. la \defname~\ref{def:nodegeneradas:perpendicular}.
}
En t\'erminos de la \defname~\ref{def:recap:morfismo:bilineal}, la
\obsname~\ref{obs:nodegeneradas:perpendicularidad} se reinterpreta de la
siguiente manera: todo morfismo de espacios bilineales \emph{no degenerados} es
inyectivo.

\begin{defExtRec}\label{def:recap:morfismo:embedding}
	Un \emph{embedding isom\'etrico} es un morfismo inyectivo de espacios
	bilineales. Una \emph{isometr\'{\i}a} es un embedding isom\'etrico que
	es sobreyectivo.
\end{defExtRec}

\begin{ejemExtRec}\label{ejem:recap:morfismo:embedding}
	Si $(V,B)$ es un espacio bilineal y $W\subset V$ es un subespacio, la
	inclusi\'on de $W\hookrightarrow V$ es un embedding isom\'etrico de
	$B|_W$ en $B$.
\end{ejemExtRec}

\begin{obsExtRec}\label{obs:recap:categoria:bilineales}
	Los espacios bilineales, junto con los morfismos de espacios bilineales
	conforman una categor\'{\i}a, la \emph{categor\'{\i}a de espacios %
	bilineales}. Las equivalencias de espacios bilineales son los
	isomorfismos de la categor\'{\i}a. Los espacios bilineales con los
	embeddings isom\'etricos como morfismos constituyen una
	subcategor\'{\i}a. Esta subcategor\'{\i}a no es \emph{plena}; los
	espacios bilineales no degenerados con los embeddings isom\'etricos
	como morfismos constituyen una subcategor\'{\i}a plena.
\end{obsExtRec}

El espacio vectorial nulo sobre un cuerpo $F$ admite, salvo isomorfismo, una
\'unica forma bilineal, la trivial. El par compuesto por este espacio y esta
forma bilineal es un espacio bilineal en la categor\'{\i}a de espacios
bilineales sobre el cuerpo $F$. No es cualquier objeto de la categor\'{\i}a; es
un \emph{objeto inicial}. Este objeto tambi\'en es un \emph{objeto final} en la
categor\'{\i}a, pues, dado $(V,B)$, existe una \'unica t.l. $V\rightarrow 0$ y
la misma es un morfismo de $(V,B)$ en $0$. Pero, si nos restringimos a la
subcategor\'{\i}a de espacios bilineales no degenerados, $0$ deja de ser un
objeto final. No hay objeto final en la categor\'{\i}a de espacios bilineales
no degenerados, aunque s\'{\i} existe objeto inicial.

Algo similar sucede con las construcciones de suma directa y producto directo.
Si $(V,B_V)$ y $(W,B_W)$ son espacios bilineales, $(V\oplus W,B_{V\oplus W})$,
donde
\begin{equation}
	\label{eq:recap:suma-directa:bilineales}
	B_{V\oplus W}(v+w,v_1+w_1)\,=\,B_V(v,v_1)\,+\,B_W(w,w_1)
	\text{ ,}
\end{equation}
%
es un espacio bilineal y es un \emph{coproducto} en la categor\'{\i}a de
espacios bilineales; tambi\'en es un \emph{producto} en la misma
categor\'{\i}a. Las inclusiones can\'onicas en $V\oplus W$ son embeddings
isom\'etricos, con lo cual la suma directa tambi\'en es un coproducto en la
subcategor\'{\i}a de espacios no degenerados; pero, al igual que el espacio
nulo, deja de ser un producto en la subcategor\'{\i}a. La suma directa la
llamamos \emph{suma directa ortogonal}.

Adem\'as de la construcci\'on de suma directa, el \emph{producto tensorial} de
dos espacios bileneales $(V,B_V)$ y $(W,B_W)$ es el par
$(V\tensor W,B_{V\tensor W})$, cuya forma bilineal est\'a dada, en tensores
elementales, por
\begin{equation}
	\label{eq:recap:producto-tensorial:bilineales}
	B_{V\tensor W}(v\tensor w,v_1\tensor w_1)\,=\,
		B_V(v,v_1)\,B_W(w,w_1)
	\text{ .}
\end{equation}
%
La matriz asociada al producto tensorial es el \emph{producto de Kronecker} de
las matrices asociadas a cada uno de los factores.

\begin{defExtRec}\label{def:recap:ortogonal:bilineales}
	El \emph{grupo ortogonal} de $(V,B)$ es el subgrupo
	\begin{displaymath}
		\OO(V,B)\,=\,\Big\{A\in\GL(V)\,:\,B(A\,v,A\,v_1)=B(v,v_1)
			\text{ para todo } v,v_1\in V\Big\}
	\end{displaymath}
	%
	de $\GL(V)$. Tambi\'en denotamos este subgrupo por $\OO(V)$ o por
	$\OO(B)$.
\end{defExtRec}

Si $A\in\OO(B)$, tomando bases,
\begin{equation}
	\label{eq:recap:ortogonal:bilineales:determinante}
	(\det\,A)^2\,\discriminante\,B\,=\,\discriminante\,B
	\text{ .}
\end{equation}
%
En particular, \emph{si $B$ es no degenerada}, $\det(A)\in\{\pm 1\}$.%
\footnote{
	Si $B$ es degenerada, diagonalizando, se ve que existen
	endomorfismos que preservan $B$ y que no son invertibles,
	necesariamente, o que, aunque sean invertibles, no tengan
	determinante a $\pm 1$.
}

Si $B$ es no degenerada, podemos hablar de la adjunta $\adjnt A$ de
$A\in\OO(V)$. Si $V=F^n$, identificamos $F^n$ con su dual y
$A\in\MM[n\times n](F)$, entonces $A$ preserva la forma bilineal $B$,
si y s\'olo si $\trnsp A$ la preserva:
\begin{displaymath}
	\begin{aligned}
		& B(A\,v,A\,v_1)\,=\,B(v,v_1)
			\quad\text{para todo }v,v_1
		\quad\Rightarrow\quad \adjnt A\,A=I \\
		& \qquad\quad\Rightarrow\quad A\,\adjnt A=I
		\quad\Rightarrow\quad\trnsp{\adjnt A}\,\trnsp A=I \\
		& \qquad\quad\Rightarrow\quad
			B(\trnsp A\,v,\trnsp A\,v_1)\,=\,B(v,v_1)
			\quad\text{para todo }v,v_1
		\text{ .}
	\end{aligned}
	%
\end{displaymath}
%

\begin{defExtRec}\label{def:recap:ortogonal:bilineales:rotaciones}
	Una \emph{rotaci\'on} en $V$ (con respecto a $B$)
	es un elemento $A\in\OO(B)$ tal que $\det(A)=1$. Una \emph{reflexi\'on}
	es un elemento $A\in\OO(B)$ tal que $\det(A)=-1$.
\end{defExtRec}

\begin{lemaExtRec}\label{lema:recap:ortogonal:bilineales:complemento}
	Si $v\in V$ es tal que $B(v,v)\neq 0$, entonces
	\begin{displaymath}
		V\,=\,\generado v\,\oplus\,v^\lperp
			\,=\,\generado v\,\oplus\,v^\rperp
		\text{ .}
	\end{displaymath}
	%
	Si la relaci\'on de perpendicularidad es sim\'etrica, entonces
	$v^\lperp=v^\rperp$ y la suma $V=v\oplus v^\perp$ es ortogonal.
	Adem\'as, en este caso, si $V$ es no degenerado, entonces $v^\perp$
	es no degenerado.
\end{lemaExtRec}

\begin{proof}
	Ver el \lemaname~\ref{lema:ortogonales:complemento}.
\end{proof}

\begin{ejemExtRec}\label{ejem:recap:ortogonal:bilineales:complemento}
	Sea $B=\sbmatrix{ 1 & 1 \\ & 1 }$. Si la caracter\'{\i}stica del
	cuerpo es impar, el vector $v=\sbmatrix{ 1 \\ 0 }$ cumple
	$B(v,v)\neq 0$. Los subespacios ortogonales son
	$v^\lperp=\generado{\sbmatrix{ 0 \\ 1 }}$ y
	$v^\rperp=\generado{\sbmatrix{ 1 \\ -1 }}$ (si la caracter\'{\i}stica
	del cuerpo es $2$, $v^\rperp=\generado v$).
\end{ejemExtRec}

% \begin{lemaExtRec}\label{lema:recap:ortogonal:bilineales:diagonal}
	% Sea $(V,B)$ un espacio bilineal tal que la relaci\'on de
	% perpendicularidad es sim\'etrica. Si $B$ no es alternada,
	% entonces existe alg\'un $v\in V$ tal que $B(v,v)\neq 0$.
% 
	% Sea $(V,B)$ un espacio bilineal sim\'etrico. Si $B$ no es
	% id\'enticamente cero, entonces existe alg\'un $v\in V$ tal que
	% $B(v,v)\neq 0$.
% \end{lemaExtRec}
% 
% \begin{proof}
	% Si la caracter\'{\i}stica del cuerpo es $2$, entonces
	% $v\mapsto B(v,v)$ es lineal. Si la caracter\'{\i}stica es impar, ver
	% el \lemaname~\ref{lema:ortogonales:diagonal}.
% \end{proof}

\begin{propoExtRec}\label{propo:recap:ortogonal:bilineales:determinante}
	Sea $(V,B)$ un espacio bilineal tal que la relaci\'on de
	perpendicularidad es sim\'etrica. Si $B$ no es alternada,
	el determinante $\det:\,\OO(B)\rightarrow\{\pm 1\}$ es un morfismo
	sobreyectivo.
	Si la caracter\'{\i}stica del cuerpo es $2$, el morfismo es
	sobreyectivo trivialmente.
\end{propoExtRec}

\begin{proof}
	Si, para todo $x,y\in V$, $B(x,y)=0$ implica $B(y,x)=0$, se verifica
	que, para toda terna $u,v,w\in V$, se verifica%
	\footnote{
		Ver \eqref{eq:definiciones:perpendicular}.
	}
	\begin{equation}
		\label{eq:recap:ortogonal:bilineales:determinante}
		B(v,u)\,B(u,w)\,=\,B(u,v)\,B(w,u)
		\text{ .}
	\end{equation}
	%
	La idea es construir un elemento de determinante $-1$.
	% Si $B$ est\'a diagonalizada, entonces
	% $\diagonal{1,\,\cdots,\,1,\,-1}$ preserva $B$ y tiene determinante
	% $-1$. Esta transformaci\'on es la reflexi\'on con respecto al
	% hiperplano $(0,\,\cdots,\,0,\,1)^\perp$. En general, si
	Si $u\in V$ es un vector tal que $B(u,u)\neq 0$, definimos
	$\refl[u]:\,V\rightarrow V$ por
	\begin{equation}
		\label{eq:recap:ortogonal:bilineales:reflexion}
		\refl[u](v)\,=\,v\,-\,\frac{B(v,u)+B(u,v)}{B(u,u)}\,u
		\text{ .}
	\end{equation}
	%
	La transformaci\'on \eqref{eq:recap:ortogonal:bilineales:reflexion}
	cumple:
	\begin{itemize}
		\item $B(\refl[u](v),\refl[u](w))=B(v,w)$ para todo $v,w\in V$
			(por
			\eqref{eq:recap:ortogonal:bilineales:determinante}),
		\item $\refl[u](u)=-u$ y
		\item $\refl[u](v)=v$, si $v\in u^\perp$.
	\end{itemize}
	%
	Por el \lemaname~\ref{lema:recap:ortogonal:bilineales:complemento},
	% Por el \lemaname~\ref{lema:ortogonales:complemento},
	$V=\generado u\oplus u^\perp$. Eligiendo una base adecuada (empezando
	con $u$), la transformaci\'on $\refl[u]$ est\'a representada por
	la matriz
	\begin{displaymath}
		\begin{bmatrix}
			-1 & & & \\
			& 1 & & \\
			& & \ddots & \\
			& & & 1
		\end{bmatrix}
		\text{ .}
	\end{displaymath}
	%
	En particular, $\det(\refl[u])=-1$.
\end{proof}

% Si $(V,B)$ es no degenerado, entonces el subespacio $u^\perp$ es no
% degenerado, tambi\'en. La simetr\'{\i}a de $B$ garantiza que $\refl[u]$
% preserve $B$. En general, si $v\in u^\lperp$, $\refl[u](v)=v$ y
% $B(\refl[u](v),\refl[u](w))=B(v,w)$.

\subsection{Espacios cuadr\'aticos}\label{subsec:extension:recap:cuadraticos}
De acuerdo con la \defname~\ref{def:impar:espacio}, un \emph{espacio %
cuadr\'atico} es un par $(V,Q)$, donde $V$ es un espacio vectorial sobre un
cuerpo, $F$, y $Q:\,V\rightarrow F$ es una forma cuadr\'atica.

\begin{defExtRec}\label{def:recap:morfismo}
	Un \emph{morfismo de espacios cuadr\'aticos},
	\begin{displaymath}
		(V,Q_V)\,\rightarrow\,(W,Q_W)
		\text{ ,}
	\end{displaymath}
	%
	es una t.l. $A:\,V\rightarrow W$ tal que
	\begin{displaymath}
		Q_W(A\,v)\,=\,Q_V(v)
		\text{ ,}
	\end{displaymath}
	%
	para todo $v\in V$.
\end{defExtRec}

Un morfismo entre espacios cuadr\'aticos es una t.l. entre los e.v. subyacentes
que preserva las formas cuadr\'aticas. Dado que
\eqref{eq:impar:definicion:perpendicular} y que
\eqref{eq:parias:definicion:perpendicular}, los morfismos de espacios
cuadr\'aticos preservan la relaci\'on de perpendicularidad (expresada tanto en
t\'erminos de formas cuadr\'aticas, como en t\'erminos de formas bilineales).
Los espacios cuadr\'aticos, junto con los morfismos de espacios cuadr\'aticos
conforman una categor\'{\i}a, la \emph{categor\'{\i}a de espacios %
cuadr\'aticos} ?`Qu\'e podemos decir de los morfismos entre espacios no
degenerados? El razonamiento depender\'a de la caracter\'{\i}stica.

Supongamos que $(V,Q_V)$ y $(W,Q_W)$ son espacios cuadr\'aticos no degenerados
y sea $A:\,V\rightarrow W$ un morfismo. Si $\car F\neq 2$, entonces que las
formas cuadr\'aticas sean no degeneradas equivale a que las formas bilineales
asociadas lo sean. En particular, en este caso, $A$ debe ser inyectiva. Otro
argumento es el siguiente: si $A\,v=0$, entonces, para todo $v_1\in V$,
\begin{equation}
	\label{eq:recap:nodegeneradas:morfismo:bilineales}
	0\,=\,B_W(0,A\,v_1)\,=\,B_W(A\,v,A\,v_1)\,=\,B_V(v,v_1)
	\text{ ,}
\end{equation}
%
con lo cual $v\in V^\perp$. Aqu\'{\i}, $B_V$ y $B_W$ son las formas bilineales
correspondientes a $Q_V$ y a $Q_W$. Si $B_V$ es no degenerada, $V^\perp=0$,
$v=0$ y $A$ es inyectiva. Notemos que en
\eqref{eq:recap:nodegeneradas:morfismo:bilineales} \emph{\'unicamente %
usamos que $A$ es un morfismo de espacios bilineales}. Si $\car F=2$,
$V^\perp$ podr\'{\i}a no ser cero, pero $Q:\,V^\perp\rightarrow F$ es
inyectiva. Como $A$ es morfismo \emph{de espacios cuadr\'aticos}, si
$v\in V^\perp$ y $A\,v=0$, entonces
\begin{equation}
	\label{eq:recap:nodegeneradas:morfismo:cuadraticas}
	0\,=\,Q_W(0)\,=\,Q_W(A\,v)\,=\,Q_V(v)
	\text{ ,}
\end{equation}
%
con lo cual $v\in V^\perp\cap\{Q_V=0\}$ y $v=0$.%
\footnote{
	Notemos que alcanza con asumir que el dominio es no degenerado, tanto
	si pensamos en formas bilineales, como si pensamos en formas
	cuadr\'aticas.
}

\begin{defExtRec}\label{def:recap:embedding}
	Un \emph{embedding isom\'etrico} es un morfismo inyectivo de espacios
	cuadr\'aticos. Una \emph{isometr\'{\i}a} es un embedding isom\'etrico
	que es sobreyectivo.
\end{defExtRec}

\begin{teoExtRec}\label{teo:recap:embedding}
	Todo morfismo entre espacios cuadr\'aticos no degenerados es un
	embedding isom\'etrico.
\end{teoExtRec}

% El argumento anterior muestra que, en caracter\'{\i}stica $2$, las nociones de
% forma cuadr\'atica y de forma bilineal son fundamentalmente distintas. No
% podemos recuperar la forma cuadr\'atica de la forma bilineal asociada, en
% general y existen morfismos de espacios bilineales que no son morfismos de los
% espacios cuadr\'aticos de los que provienen.
% 
Toda forma cuadr\'atica tiene asociada una forma bilineal.%
\footnote{
	La definici\'on depende de la caracter\'{\i}stica.
}
De esta manera, podemos definir una aplicaci\'on que a cada espacio
cuadr\'atico le asigna su espacio bilineal correpondiente:
\begin{equation}
	\label{eq:recap:funtor}
	(V,Q)\,\mapsto\,(V,B_Q)
	\text{ .}
\end{equation}
%
Si $A:\,V\rightarrow W$ es morfismo de espacios cuadr\'aticos $(V,Q_V)$ y
$(W,Q_W)$, entonces
\begin{displaymath}
	\begin{aligned}
		\ast\,B_W(A\,v,A\,v_1) & \,=\,
			Q_W(A\,v+A\,v_1)\,-\,Q_W(A\,v)\,-\,Q_W(A\,v_1) \\
			& \,=\,Q_V(v+v_1)\,-\,Q_V(v)\,-\,Q_V(v_1) \\
			& \,=\,\ast\,B_V(v,v_1)
		\text{ ,}
	\end{aligned}
	%
\end{displaymath}
%
donde $B_V$ y $B_W$ son las formas bilineales asociadas y $\ast\in\{1,2\}$. En
particular, $A$ es morfismo entre los espacios bilineales correspondientes. De
esta manera, la asignaci\'on~\ref{eq:recap:funtor} determina un
\emph{funtor} de la categor\'{\i}a de espacios cuadr\'aticos en la
categor\'{\i}a de espacios bilineales.

\begin{obsExtRec}\label{obs:recap:nodegeneradas:morfismo}
	El funtor de la categor\'{\i}a de espacios cuadr\'aticos en la
	categor\'{\i}a de espacios bilineales se restringe a un funtor en la
	subcategor\'{\i}a plena de espacios bilineales no degenerados desde la
	subcategor\'{\i}a tambi\'en plena de espacios cuadr\'aticos no
	degenerados.
\end{obsExtRec}

\begin{ejerExtRec}\label{ejer:recap:nodegeneradas:cuadraticas}
	% Probar que, si $v\in V$ es tal que $Q(v+w)-Q(w)$ es constante en
	% funci\'on de $w\in V$, entonces $Q(v+w)=Q(v)+Q(w)$ para todo $w\in V$,
	% es decir, $v\in V^\perp$.
	Probar que
	\begin{displaymath}
		v\,\in\,V^\perp\quad\Leftrightarrow\quad
			Q(v+w)\,-\,Q(w)
			\text{ es constante.}
	\end{displaymath}
	%
	Probar que $Q$ es no degenerada, si y s\'olo si,%
	\footnote{
		Si $Q$ es no degenerada y $v\in V^\perp$, entonces
		$Q(v+w)\neq Q(w)$ para todo $w$.
	}
	\begin{displaymath}
		v\,\neq\,0\quad\Rightarrow\quad
			Q(v+w)\neq Q(w)
			\text{ para alg\'un }w\in V
			\text{ .}
	\end{displaymath}
	%
\end{ejerExtRec}

\begin{defExtRec}\label{def:recap:ortogonal:cuadraticos}
	El \emph{grupo ortogonal} de $(V,Q)$ es el subgrupo
	\begin{displaymath}
		\OO(V,Q)\,=\,\Big\{A\in\GL(V)\,:\,Q(A\,v)=Q(v)
			\text{ para todo } v\in V\Big\}
	\end{displaymath}
	%
	de $\GL(V)$. Tambi\'en denotamos este subgrupo por $\OO(Q)$ o por
	$\OO(V)$.
\end{defExtRec}

Si la caracter\'{\i}stica del cuerpo es impar, entonces $\OO(Q)=\OO(B_Q)$. En
general, que $A$ preserve $Q$ implica que $A$ preserva $B_Q$.

\subsection{Geometr\'{\i}a}\label{subsec:extension:recap:geometria}
Empecemos reformulando algunos resultados de las secciones
\S\S~\ref{sec:intro:nodegeneradas}, \ref{sec:cuadraticas:impar} y
\ref{sec:cuadraticas:parias}.

\begin{teoExtRec}\label{teo:recap:radical}
	Sea $(V,B)$ un espacio bilineal tal que $v\perp w$ es una relaci\'on
	sim\'etrica. Existe un subespacio $W\subset V$ tal que $B|_W$ es no
	degenerada y
	\begin{displaymath}
		V\,=\,V^\perp\,\oplus\,W
		\text{ .}
	\end{displaymath}
	%
\end{teoExtRec}

\begin{proof}
	Si $v\perp w$ es una relaci\'on sim\'etrica, entonces
	$V^\lperp=V^\rperp=V^\perp$. Basta, entonces, con elegir un complemento
	de $V^\perp$ en $V$.
	% Si $V=V^\lperp+W$, entonces
	% \begin{displaymath}
		% \begin{aligned}
			% w\in W^\rperp & \quad\Rightarrow\quad
			% B(w_1,w)\,=\,0\quad\text{para todo }w_1\in W \\
			% & \quad\Rightarrow\quad
			% B(v,w)\,=\,0\quad\text{para todo } v\in V \\
			% & \quad\Rightarrow\quad
				% w\in V^\rperp
		% \end{aligned}
		% %
	% \end{displaymath}
	% %
	% Con lo cual $W^\rperp=V^\rperp$. Por otro lado,
	% \begin{displaymath}
		% W\,\cap\,W^\lperp\,=\,0
		% \quad\Leftrightarrow\quad B|_W \text{ es no degenerada}
			% \quad\Leftrightarrow\quad W\,\cap\,W^\rperp\,=\,0
		% \text{ .}
	% \end{displaymath}
	% %
	% De esta manera, si $V^\lperp\cap W=0$, entonces $W^\rperp\cap W=0$ y
	% $W$ es no degenerado.
\end{proof}

\begin{ejemExtRec}\label{ejem:recap:radical}
	Si $B=\sbmatrix{ 0 & 1 \\ 0 & 0 }$, entonces
	$V^\lperp=\generado{\sbmatrix{ 0 \\ 1 }}$ y
	$V^\rperp=\generado{\sbmatrix{ 1 \\ 0 }}$. En particular,
	$V=V^\lperp\oplus V^\rperp$. No es cierto en general que el mismo
	subespacio se complemento directo (ortogonal) tanto de $V^\lperp$ como
	de $V^\rperp$.
%
	Si $B=\sbmatrix{ & 1 & \\ & & 1 \\ \phantom{0} & & }$, entonces
	$V^\lperp=\generado{\sbmatrix{ 0 \\ 0 \\ 1 }}$ y
	$V^\rperp=\generado{\sbmatrix{ 1 \\ 0 \\ 0 }}$. Si
	$W=\generado{\sbmatrix{ 1 \\ 0 \\ 0 },\sbmatrix{ 0 \\ 1 \\ 0 }}$,
	entonces $W$ es complemento directo de $V^\lperp$, pero
	$V^\rperp\cap W\neq 0$. En particular, $B|_W$ no es no degenerada. De
	hecho, $B|_W=\sbmatrix{ & 1 \\ \phantom{0} & }$, en la base de los
	generadores.
%
	?`Qu\'e condiciones garantizan que exista alg\'un complemento com\'un?
\end{ejemExtRec}

\begin{obsExtRec}\label{obs:recap:radical}
	Si $V=V^\lperp+W$, entonces $V^\rperp=W^\rperp$. Si, adem\'as,
	$V^\rperp\cap W=0$, entonces $B|_W$ es no singular y
	$W^\lperp\cap W=0$. Pero, en particular, esto implica
	$V^\lperp\cap W=0$ y $W$ es complemento directo de $V^\lperp$ en $V$.
	Esto fuerza que las dimensiones cumplan
	$\dim\,W+\dim\,V^\lperp=\dim\,V$ y, en consecuencia,
	$\dim\,W+\dim\,V^\rperp=\dim\,V$. Pero, entonces, $W$ es complemento de
	$V^\rperp$, tambi\'en ?`Existe, en general, $W$ tal que $V=V^\lperp+W$
	y $V^\rperp\cap W=0$?
\end{obsExtRec}

% El Teorema~\ref{teo:impar:isotropico} garantiza que, en un espacio
% cuadr\'atico no degenerado $(V,Q)$ sobre un cuerpo de caracter\'{\i}stica
% impar, todo vector isotr\'opico ($Q(u)=0$) se completa a un plano
% hiperb\'olico, existe $v\in V$ tal que $\generado{u,v}$ es isomorfo
% (isom\'etrico) a un plano hiperb\'olico. M\'as aun, $\generado{u,v}^\perp$ es
% no degenerado y el espacio admite una descomposici\'on de la forma
% $V=\generado{u,v}\oplus\generado{u,v}^\perp$.
% 
% El comentario posterior a la demostraci\'on del resultado muestra que se puede
% proceder de manera inductiva hasta que resulte imposible encontrar vectores
% isotr\'opicos, de manera que $V=\hiperbolico^{\perp m}\perp W$, donde
% $W\subset V$ es anisotr\'opico. Pero quedaba pendiente la cuesti\'on de la
% invarianza del valor de $m$, de la dimensi\'on de \emph{la parte isotr\'opica}.

\begin{defExtRec}\label{def:recap:isotropico}
	Un vector $v$ en un espacio cuadr\'atico $(V,Q)$ se dice
	\emph{isotr\'opico}, si $v\neq0$ y $Q(v)=0$.%
	\footnote{
		En \cite[p.~114]{GroveClassical}, estos vectores se denominan
		\emph{singulares}.
	}
	El espacio cuadr\'atico $(V,Q)$ se dice \emph{isotr\'opico}, si es no
	degenerado y admite un vector isotr\'opico. Un subespacio $W\subset V$
	se dice \emph{totalmente isotr\'opico}, si $Q(w)=0$, para todo
	$w\in W$.%
	\footnote{
		En caracter\'{\i}stca impar, esto equivale a que la forma
		bilineal asociada sea id\'enticamente cero en el subespacio.
		En particular, $(W,B|_W)$ es degenerado, con lo que, de acuerdo
		con la definici\'on, \emph{no es un espacio isotr\'opico}.
	}
\end{defExtRec}

En un espacio cuadr\'atico $(V,Q)$, deber\'{\i}amos hacer una distinci\'on
entre aquellos vectores que cumplen $Q(v)=0$ y aquellos que cumplen
$B_Q(v,v)=0$. En caracter\'{\i}stica distinta de $2$, son \emph{los mismos %
vectores}, pero en caracter\'{\i}stica $2$, $B_Q(v,v)=0$ para todo $v$,
mientras que la condici\'on $Q(v)=0$ es restrictiva. La confuci\'on aparece
cuando introducimos la noci\'on de subespacios isotr\'opicos, totalmente
isotr\'opicos, anisotr\'opicos, degenerados y no degenerados. De nuevo, en
caracter\'{\i}stica impar, no hay distinci\'on, pero, en caracter\'{\i}stica
$2$, s\'{\i} la hay: $Q=0$ implica $B_Q=0$, pero no siempre es cierta la
vuelta; si $A$ preserva $Q$, entonces preserva $B_Q$, pero no es cierto en
general que si preserva $B_Q$, entonces preserve $Q$; si $B_Q$ es no
degenerada, entonces $V^\perp=0$ y $Q$ es no degenerada, pero $Q$ puede ser
no degenerada con $V^\perp\neq 0$. En particular, en caracter\'{\i}stica $2$,
no es lo mismo un subespacio totalmente isotr\'opico ($Q$ id\'enticamente cero)
que un subespacio en donde $B_Q$ sea id\'enticamente cero.

Los espacios bilineales sim\'etricos se diagonalizan, los espacios bilineales
alternados son simpl\'ecticos. Los espacios cuadr\'aticos son h\'{\i}birdos de
estos dos tipos de geometr\'{\i}as. Algunos de los resultados v\'alidos para
los espacios en donde la relaci\'on de perpendicularidad es sim\'etrica son
v\'alidos para espacios cuadr\'aticos. Un poco m\'as espec\'{\i}ficamente,
resultados acerca del espacio bilineal son reemplazados por resultados acerca
del espacio cuadr\'atico. Esta distinci\'on tiene especial relevancia en
caracter\'{\i}stica $2$, pues el funtor $Q\mapsto B_Q$ no es una equivalencia.

\begin{teoExtRec}\label{teo:recap:isotropico:bilineales}
	Sea $(V,B)$ un espacio bilineal en donde la relaci\'on de
	perpendicularidad es sim\'etrica. Si la caracter\'{\i}stica del cuerpo
	es $2$, asumimos que $B$ es alternada. Si $u\in V$ es un vector no nulo
	tal que $B(u,u)=0$, entonces:
	\begin{itemize}
		\item existe un segundo vector $v\in V$ no nulo tal que
			$B(v,v)=0$ y $B(u,v)=1$,
		\item $B$ es no degenerada en $\generado{u,v}$ y
			\begin{displaymath}
				V\,=\,\generado{u,v}\,\oplus\,
					\generado{u,v}^\perp
				\text{ .}
			\end{displaymath}
			%
	\end{itemize}
	%
	En particular, $\generado{u,v}\simeq\hiperbolico$.
\end{teoExtRec}

\begin{proof}
	Ver el \teoname~\ref{teo:impar:isotropico}.
	Si $B$ es no degenerada, $V^\perp=0$. Entonces, dado que $u\neq 0$,
	existe $v\in V$ tal que $B(u,v)\neq 0$. Reescalando, podemos asumir que
	$B(u,v)=1$. Considerando vectores de la forma
	$c\,u+v\in\generado{u,v}$, $c\in F$,
	\begin{displaymath}
		B(c\,u+v,c\,u+v)\,=\,B(v,v)\,+\,2\,c
		\text{ .}
	\end{displaymath}
	%
	Si $2\neq 0$, \'esta es una funci\'on af\'{\i}n y existe un (\'unico)
	valor de $c\in F$ tal que $B(c\,u+v,c\,u+v)=0$. Con esta elecci\'on,
	\begin{displaymath}
		B(u,c\,u+v)\,=\,c\,B(u,u)\,+\,B(u,v)\,=\,B(u,v)\,=\,1
		\text{ ,}
	\end{displaymath}
	%
	pues $B(u,u)=0$. Reemplazamos $v$ por $c\,u+v$. En todo caso, $u$ y $v$
	son l.i.
\end{proof}

\begin{teoExtRec}\label{teo:recap:isotropico}
	Sea $(V,Q)$ un espacio cuadr\'atico no degenerado sobre un cuerpo
	arbitrario. Si $e\in V$ es un vector isotr\'opico, entonces:
	\begin{itemize}
		\item existe un segundo vector isotr\'opico $f\in V$ tal que
			$B(e,f)=1$,
		\item $B$ es no degenerada en $\generado{e,f}$ y
			\begin{displaymath}
				V\,=\,\generado{e,f}\,\oplus\,
					\generado{e,f}^\perp
				\text{ .}
			\end{displaymath}
			%
	\end{itemize}
	%
	En particular, $\generado{e,f}\simeq\hiperbolico$.
\end{teoExtRec}

\begin{proof}
	Ver el \teoname~\ref{teo:parias:isotropico}.
	Si $Q$ es no degenerada, $Q:\,V^\perp\rightarrow F$ es inyectiva.
	Entonces, dado que $Q(e)=0$, $e\not\in V^\perp$ y existe $f$ tal que
	$B_Q(e,f)\neq 0$. Reescalando, podemos asumir que $B_Q(e,f)=1$.
	Considerando vectores de la forma $c\,e+f\in\generado{e,f}$, $c\in F$,
	la funci\'on
	\begin{displaymath}
		Q(c\,e+f)\,=\,Q(f)\,+\,\ast\,c
		\text{ ,}
	\end{displaymath}
	%
	donde $\ast\in\{1,2\}$, es af\'{\i}n y existe un (\'unico) valor de
	$c\in F$ tal que $Q(c\,e+f)=0$. Con esta elecci\'on,
	\begin{displaymath}
		B_Q(e,c\,e+f)\,=\,c\,B_Q(e,e)\,+\,B_Q(e,f)\,=\,B_Q(e,f)\,=\,1
		\text{ ,}
	\end{displaymath}
	%
	pues $B_Q(e,e)=0$.%
	\footnote{
		Si la caracter\'{\i}stica es impar, esto se debe a que
		$Q(e)=B_Q(e,e)$. Si la caracter\'{\i}stica es $2$, $B_Q$ es
		alternada.
	}
	Reemplazamos $f$ por $c\,e+f$. En todo caso, $e$ y $f$ son l.i.
\end{proof}

\begin{obsExtRec}\label{obs:recap:isotropico:bilineales}
	Los planos hiperb\'olicos fueron definidos como espacios
	cuadr\'aticos de dimensi\'on $2$ que admiten una base $\{e,f\}$ tal
	que $Q(e)=Q(f)=0$ y $B(e,f)=1$. El subespacio $\generado{u,v}$ del
	\teoname~\ref{teo:recap:isotropico:bilineales} no es un plano
	hiperb\'olico: aunque sea $B=B_Q$, \textit{a priori} no se sabe si
	$Q(u)=0$. El isomorfismo $\generado{u,v}\simeq\hiperbolico$ es en
	tanto espacios bilineales. El \teoname~\ref{teo:recap:isotropico}
	garantiza que, si se sabe que, adem\'as, $e$ es isotr\'opico, entonces
	el vector $f$ (que se construye esencialmente de la misma manera) es
	isotr\'opico tambi\'en. En particular, en este caso, el isomorfismo
	$\generado{e,f}\simeq\hiperbolico$ es de espacios cuadr\'aticos.
\end{obsExtRec}

\begin{obsExtRec}\label{obs:recap:isotropico}
	Todo espacio isotr\'opico es universal, pues contiene una copia del
	plano hiperb\'olico, que es universal.
\end{obsExtRec}

% \begin{obsExtRec}\label{obs:recap:nodegenerada}
	% En cualquier caracter\'{\i}stica, si $(V,Q)$ es un espacio
	% cuadr\'atico tal que la forma bilineal asociada, $B_Q$, es no
	% degenerada, entonces $Q$ es no degenerada. En particular, las
	% condiciones del \teoname~\ref{teo:recap:isotropico:subespacio} son
	% m\'as fuertes que las del \teoname~\ref{teo:recap:isotropico}.
% \end{obsExtRec}

\begin{teoExtRec}\label{teo:recap:isotropico:subespacio:bilineales}
	Sea $(V,B)$ un espacio bilineal no degenerado tal que la relaci\'on de
	perpendicularidad es sim\'etrica. Si la caracter\'{\i}stica del cuerpo
	es $2$, asumimos que $B$ es alternada. Entonces,
	\begin{enumerate}[(i)]
		\item\label{item:recap:isotropico:subespacio:bilineales:i}
			si $U\subset V$ es un subespacio tal que $B|_U=0$
			% ?```totalmente degenerado''?
			y de dimensi\'on $m\geq 1$, existe un subespacio
			$U'\subset V$ tal que $B|_{U'}=0$, de
			dimensi\'on $m$ y tal que $U\cap U'=0$ y
			$U+U'\simeq\hiperbolico^{\perp m}$;
		\item\label{item:recap:isotropico:subespacio:bilineales:ii}
			si $Z\subset V$ es un subespacio tal que
			$U\perp Z$ y $U\cap Z=0$, entonces podemos elegir
			$U'$ de manera que $U'\perp Z$ y $U'\cap Z=0$.
			% \footnote{
				% En particular, lo mismo es cierto para
				% $U+U'$, por linealidad.
			% }
	\end{enumerate}
	%
\end{teoExtRec}

\begin{teoExtRec}\label{teo:recap:isotropico:subespacio}
	Sea $(V,Q)$ un espacio cuadr\'atico tal que la forma bilineal asociada,
	$B_Q$, es no degenerada. Entonces,
	\begin{enumerate}[(i)]
		\item\label{item:recap:isotropico:subespacio:i}
			si $U\subset V$ es un subespacio totalmente
			isotr\'opico de dimensi\'on $m\geq 1$, existe un
			subespacio $U'\subset V$ totalmente isotr\'opico de
			dimensi\'on $m$ tal que $U\cap U'=0$ y
			$U+U'\simeq\hiperbolico^{\perp m}$;
		\item\label{item:recap:isotropico:subespacio:ii}
			si $Z\subset V$ es un subespacio tal que
			$U\perp Z$ y $U\cap Z=0$, entonces podemos elegir
			$U'$ de manera que $U'\perp Z$ y $U'\cap Z=0$.
			% \footnote{
				% En particular, lo mismo es cierto para
				% $U+U'$, por linealidad.
			% }
	\end{enumerate}
	%
\end{teoExtRec}

\begin{proof}
	% Ver la Observaci\'on~\ref{obs:simplecticas:dimension}.
	La forma bilineal $B_Q$ es no degenerada. Entonces, podemos proceder de
	la siguiente manera. Sea $W=\generado{\lista[2] e{m}}\perp Z$. Si
	$W^\perp\subset e_1^\perp$, entonces
	\begin{displaymath}
		\generado{e_1}\,=\,e_1^\pperp\,\subset\,
			W^\pperp\,=\,W
		\text{ .}
	\end{displaymath}
	%
	Las igualdades son consecuencia de que \emph{el espacio bilineal}
	$(V,B_Q)$ es no degenerado.%
	\footnote{
		\teoname~\ref{teo:nodegeneradas:perpendicular}.
	}
	Pero $\generado{e_1}\cap W=0$, con lo que existe $w\in W^\perp$ tal que
	$B_Q(e_1,w)\neq 0$. El argumento de la demostraci\'on del Teorema~%
	\ref{teo:recap:isotropico}, muestra que existe un (\'unico)
	$f_1\in\generado{e_1,w}$ tal que $Q(f_1)=0$ y $B_Q(e_1,f_1)=1$. El
	subespacio $H_1=\generado{e_1,w}=\generado{e_1,f_1}$ es un plano
	hiperb\'olico y
	\begin{displaymath}
		H_1\,\subset\,W^\perp
		\text{ ,}
	\end{displaymath}
	%
	por definici\'on. De nuevo, como $B_Q$ es no degnerada, 
	$W=W^\pperp\subset H_1^\perp$. Como $H_1$ es no degenerado,
	% \footnote{
		% Esto es cierto en cualquier caracter\'{\i}stica.
	% }
	$H_1^\perp$ tambi\'en lo es.
	% \footnote{
		% Teorema~\ref{teo:nodegeneradas:perpendicular}, usando que
		% $B_Q$ es no degenrada.
	% }
	Inductivamente en $m$,
	% ?`En $\dim\,W$? Conceptualmente, no. Pero, talvez, funciona.
	podemos concluir, con $H_1^\perp$ jugando el
	papel del espacio ambiente no degenerado, que existen
	$\lista[2] f{m}\in H_1^\perp$ que cumplen con las condiciones
	siguientes:
	\begin{itemize}
		\item $Q(f_i)=0$ y $B_Q(e_i,f_i)=1$,
		\item si $H_i=\generado{e_i,f_i}$, entonces
			$H_i\simeq\hiperbolico$ y son ortogonales y linealmente
			disjuntos.
	\end{itemize}
	%
	El subespacio $U':=\generado{\lista f{m}}$ cumple lo pedido.
\end{proof}

\begin{obsExtRec}\label{obs:recap:isotropico:subespacio}
	De la demostraci\'on, si $\{\lista e{m}\}$ es una base de $U$, existen
	$\lista f{m}$ tales que $B(e_i,f_i)=1$ y $U'=\generado{\lista f{m}}$.
	Adem\'as, si $H_i=\generado{e_i,f_i}$, entonces $H_i\simeq\hiperbolico$
	y son ortogonales y linealmente disjuntos.
\end{obsExtRec}

% \begin{obsExtRec}\label{obs:recap:isotropico:subespacio:parias}
	% Sobre un cuerpo de caracter\'{\i}stica impar, esta condici\'on es
	% equivalente a la primera. En caracter\'{\i}stica $2$, debemos asumir
	% expl\'{\i}citamente que $V^\perp=0$, o bien que
	% $(U+Z)\cap V^\perp=0$. Aunque $V^\perp$ puede ser no trivial, la
	% funci\'on $Q:\,V^\perp\rightarrow F$ es inyectiva y
	% $U\cap V^\perp=0$. Consideremos $V'\subset V$ tal que $U\subset V'$ y
	% $V=V'\oplus V^\perp$. Entonces, $B_Q|_{V'}$ es no degenerada%
	% \footnote{
		% !`En particular, $Q|_{V'}$ es no degenerada!
	% }
	% y podemos aplicar el argumento anterior, aunque la
	% caracter\'{\i}stica sea $2$, con $V'$ en lugar de $V$.%
	% \footnote{
		% Si $B_Q$ es id\'enticamente cero, entonces $V^\perp=V$ (y,
		% rec\'{\i}procamente, $V^\perp=V$ implica $B_Q=0$). En este
		% caso (y s\'olo en \'este), $Q$ es diagonal
		% (caracter\'{\i}stica $2$). Pero, si $Q$ es no degenerada, y
		% $V^\perp=V$, entonces $U=0$, con lo cual, el enunciado del
		% Teorema es trivialmente cierto, en este caso.
	% }
	% En general, deber\'{\i}amos asumir que $(U+Z)\cap V^\perp=0$ o,
	% directamente, que $V^\perp=0$, es decir, que el espacio bilineal
	% asociado es no degenerado.
% \end{obsExtRec}

\begin{coroExtRec}\label{coro:recap:isotropico:maximal}
	Si $W$ es un subespacio totalmente isotr\'opico maximal de un espacio
	cuadr\'atico no degenerado $V$, entonces $\dim\,W\leq(\dim\,V)/2$.
\end{coroExtRec}

\begin{teoExtRec}\label{teo:recap:isotropico:extension:bilineales}
	Sean $(V,B_V)$ y $(X,B_X)$ espacios bilineales no degenerados tales que
	las relaciones de perpendicularidad son sim\'etricas. Si la
	caracter\'{\i}stica del cuerpo es $2$, asumimos que $B_V$ y $B_X$ son
	alternadas.
	Sea $U\subset V$ un subespacio tal que $B_V|_U=0$ y sea $Z\subset V$
	un subespacio tal que $U\perp Z$ y $U\cap Z=0$. Si
	$A:\,U\perp Z\rightarrow X$ es un embedding isom\'etrico, entonces
	existe un embedding isom\'etrico
	$\tilde A:\,(U+U')\perp Z\rightarrow X$ tal que
	$\tilde A|_{U\perp Z}=A$.
\end{teoExtRec}

\begin{teoExtRec}\label{teo:recap:isotropico:extension}
	Sean $(V,Q_V)$ y $(X,Q_X)$ espacios cuadr\'aticos tales que sus formas
	bilineales asociadas son no degeneradas. Sea $U\subset V$ un subespacio
	totalmente isotr\'opico y sea $Z\subset V$ un subespacio tal que
	$U\perp Z$ y $U\cap Z=0$. Si $A:\,U\perp Z\rightarrow X$ es un
	embedding isom\'etrico, entonces existe un embedding isom\'etrico
	$\tilde A:\,(U+U')\perp Z\rightarrow X$ tal que
	$\tilde A|_{U\perp Z}=A$.%
	\footnote{
		Si $A$ es s\'olo morfismo de espacios cuadr\'aticos, podemos
		recurrir al Teorema~\ref{teo:recap:isotropico}. Si $Z=0$,
		alcanza con suponer que las formas cuadr\'aticas $Q_V$ y $Q_X$
		son no degeneradas ?`Talvez se puede generalizar un poco y
		suponer que $A$ s\'olo preserva la relaci\'on de ortogonalidad?
		No, porque es necesario que $A(e_i)$ sean isotr\'opicos.
	}
\end{teoExtRec}

\begin{proof}
	% Si $A:\,U\rightarrow X$ es un morfismo de espacios cuadr\'aticos, es
	% decir, $Q_X(A\,u)=Q_V(u)$ para todo $u\in U$, entonces los vectores
	% $A(e_i)$ son $Q_X$-isotr\'opicos. Por lo tanto, por el Teorema~%
	% \ref{teo:recap:isotropico}, existen vectores $Q_X$-isotr\'opicos
	% $\lista g{m}\in X$ tales que $B_X(A\,e_i,g_i)=1$. Definimos
	% $\tilde A(f_i):=g_i$ y extendemos linealmente.
	Si $A$ es un embedding, entonces $\{A(e_1),\,\dots,\,A(e_m)\}$ es un
	subconjunto l.i. conformado por vectores isotr\'opicos y ortogonales.
	Por lo tanto, el subespacio generado, $A(U)$, es totalmente
	isotr\'opico y est\'a en las condiciones del Teorema~%
	\ref{teo:recap:isotropico:subespacio}. Elegimos un subespacio de
	$X$ totalmente isotr\'opico que lo complemente, generado por vectores
	$\lista g{m}$, definimos $\tilde A(f_i):=g_i$ y extendemos linealmente.
\end{proof}

% El \teoname~\ref{teo:recap:isotropico:extension} ser\'a particularmente
% \'util en el caso de caracter\'{\i}stica impar. Si $V_1\subset V$ es un
% subespacio (arbitrario) de un espacio no degenerado, entonces por el
% \teoname~\ref{teo:recap:radical}, podemos descomponer $V_1=U_1\oplus W_1$,
% donde $U_1=V_1^\perp\cap V_1$ y $W_1\subset V_1$ es un subespacio no
% degenerado. Asumiendo que la caracter\'{\i}stica del cuerpo es impar, $U_1$
% es totalmente isotr\'opico. El \teoname~\ref{teo:recap:isotropico:subespacio}
% garantiza que existe $U_1'$ tal que $U_1+U_1'$ es isom\'etrico con
% $\hiperbolico^{\perp m}$ ($m=\dim\,U_1$), ortogonal con $W_1$ e
% independiente de \'este. Si $A:\,V_1\rightarrow X$ es un embedding
% isom\'etrico, el \teoname~\ref{teo:recap:isotropico:extension} garantiza que
% $A$ se extiende a $(U_1+U_1')\perp W_1$, al menos. El subespacio $W_1$ es no
% degenerado.
% % $W_1^\perp\cap W_1=0$.
% El subespacio $U_1+U_1'\subset V$, por ser isom\'etrico con una suma de
% planos hiperb\'olicos, es no degenerado. La suma directa ortogonal
% $(U_1+U_1')\perp W_1$ tambi\'en es no degenerada, es decir,
% \begin{displaymath}
	% \Big((U_1+U_1')\perp W_1\Big)^\perp\,\cap\,
		% \Big((U_1+U_1')\perp W_1\Big)\,=\,0
	% \text{ .}
% \end{displaymath}
% %
% % Si $u_0\in U_1+U_1'$ y $w_0\in W_1$ son tales que
% % $u_0+w_0\in (U_1+U_1')\perp W_1$, entonces $B(u_0+w_0,w)=0$ para todo
% % $w\in W_1$ implica $w_0=0$ y $B(u_0+w_0,u)=0$ para todo $u\in U_1+U_1'$
% % implica $u_0=0$.
% 
% \begin{defExtRec}\label{def:recap:clausura}
	% Si $U\perp Z\subset V$ es un subespacio de un espacio cuadr\'atico
	% en las condiciones del \teoname~%
	% \ref{teo:recap:isotropico:subespacio}, llamamos \emph{clausura %
	% hiperb\'olica} de $U$ en $V$ a cualquier subespacio no degenerado
	% $(U+U')\perp Z$ construido como en el p\'arrafo anterior.
	% Espec\'{\i}ficamente,
	% \begin{itemize}
		% \item $U_1:=V_1^\perp\cap V_1$,
		% \item $W_1\subset V_1$ es no degenerado y $V_1=U_1\perp W_1$,
		% \item $U_1'\subset V$ cumple:
			% \begin{itemize}
				% \item $U_1'\cap U_1=0$,
				% \item $U_1+U_1'\simeq\hiperbolico^{\perp m}$,
					% donde $m=\dim\,U_1$, y 
				% \item $W_1\cap(U_1+U_1')=0$ y
					% $W_1\perp(U_1+U_1')$.
			% \end{itemize}
			% %
	% \end{itemize}
	% %
	% Si $f:\,V_1\rightarrow X$ es un embedding isom\'etrico en las
	% condiciones del \teoname~\ref{teo:recap:isotropico:extension},
	% llamamos \emph{clausura hiperb\'olica de $f$} a cualquier
	% extensi\'on de $f$ a una clausura hiperb\'olica de $V_1$ en $V$.
% \end{defExtRec}
% 
% Si $(V,Q)$ es un espacio cuadr\'atico no degenerado y $U\subset V$ es un
% subespacio totalmente isotr\'opico, entonces no existe un complemento
% ortogonal de $U$ en $V$: si $V=U\perp W$, entonces $U\subset U^\perp$, pues
% $Q|_U=0$, y $U\subset W^\perp$, con lo cual $U\subset V^\perp$, pero
% $Q:\,V^\perp\rightarrow F$ es inyectiva, por hip\'otesis.

