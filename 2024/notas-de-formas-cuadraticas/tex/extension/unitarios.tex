\theoremstyle{plain}
\newtheorem{teoExtUni}{\teoname}[section]

\theoremstyle{definition}
\newtheorem{defExtUni}[teoExtUni]{\defname}

%-------------

\subsection{Formas hermitianas}\label{subsec:extension:unitarios:hermitianas}
Sea $K$ un anillo de divisi\'on, al que vamos a llamar ``cuerpo no
necesariamente conmutativo''. Supongamos, adem\'as, que contamos con un
``antiautomorfismo'' $\tau:\,K\rightarrow K$. Sea $V/K$ un m\'odulo a derecha.

\begin{defExtUni}\label{def:unitarios:pairing}
	Un \emph{pairing generalizado} (o forma $\tau$-sesquilineal) en $V$ es
	una funci\'on biaditiva $B:\,V\times V\,\rightarrow K$ que verifica%
	\footnote{
		C.f. \cite[p.~102]{ArtinGeometric}. Definimos una estructura
		\emph{a izquierda} en $V$ por
		\begin{displaymath}
			a\,v\,=\,v\,a^\tau
			\text{ .}
		\end{displaymath}
		%
		Con esta estructura,
		\begin{math}
			B(v,a\,w)=B(v,w\,a^\tau)=B(v,w)\,a^\tau
		\end{math}. Si $a$ pertenece al centro de $K$, entonces
		$a^\tau$ tambi\'en es central. Para elementos centrales, se
		verifica que $B(v,a\,w)=B(v\,a,w)$.
	}
	\begin{displaymath}
		B(v,w\,a)\,=\,B(v,w)\,a
		\quad\text{y}\quad
		B(v\,a,w)\,=\,a^\tau\,B(v,w)
		\text{ .}
	\end{displaymath}
	%
\end{defExtUni}

El siguiente resultado generaliza el \teoname~%
\ref{teo:definiciones:perpendicular}

\begin{teoExtUni}\label{teo:unitarios:perpendicular}
	La relaci\'on de perpendicularidad en $(V,B)$ es sim\'etrica, si y
	s\'olo si $B$ es sim\'etrica, alternada o \emph{hermitiana}.
\end{teoExtUni}

\subsection{Caracter\'{\i}stica $2$}\label{subsec:extension:unitarios:parias}
Los espacios cuadr\'aticos sobre cuerpos de caracter\'{\i}stica $2$ tienen
cierta semejanza con los espacios \emph{unitarios}.


