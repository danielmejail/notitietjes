\theoremstyle{plain}
\newtheorem{teoWittDef}{\teoname}[section]

\theoremstyle{definition}
\newtheorem{defWittDef}[teoWittDef]{\defname}
\newtheorem{obsWittDef}[teoWittDef]{\obsname}
\newtheorem{ejemWittDef}[teoWittDef]{\ejemname}
\newtheorem{ejerWittDef}[teoWittDef]{\ejername}

%-------------

\subsection{Acciones en el espacio de formas cuadr\'aticas}%
	\label{subsec:witt:definiciones:acciones}
Dadas una forma cuadr\'atica y una matriz, podemos definir una nueva forma
cuadr\'atica componiendo las funciones correspondientes. Precisamente, si
$Q:\,F^n\rightarrow F$ es una forma cuadr\'atica sibre un cuerpo $F$ y
$M\in\MM[n\times n](F)$ es una matriz de tama\~no $n\times n$ con coeficientes
en el cuerpo $F$, entonces la expresi\'on
\begin{equation}
	\label{eq:definiciones:accion}
	M\cdot q(x)\,=\,q(\trnsp M\,x)
	\text{ ,}
\end{equation}
%
v\'alida para $x\in F^n$, define una nueva forma cuadr\'atica en $F^n$.

\begin{defWittDef}\label{def:definiciones:accion}
	Un poco m\'as en general y de manera un poco m\'as abstracta, si
	$Q:\,V\rightarrow F$ es una forma cuadr\'atica en un espacio vectorial
	$V$ sobre un cuerpo $F$ y $M\in\Endo[F](V)$ es un endomorfismo del
	e.v., entonces la expresi\'on \eqref{eq:definiciones:accion} define
	una forma cuadr\'atica en $V$, que denotamos $M\cdot q$.
	% Esto tiene todo el sentido, pero para hablar de la acci\'on, hay
	% que introducir el grupo de automorfismos; mejor tener grupos
	% concretos (y algebraicos) como $\GL[n]$.
\end{defWittDef}

Si $Q:\,F^n\rightarrow F$ es una forma cuadr\'atica y
$M_1,M_2\in\MM[n\times n](F)$, entonces
\begin{displaymath}
	M_2\cdot M_1\cdot q \,=\,(M_2\,M_1)\cdot q
	\text{ .}
\end{displaymath}
%
Adem\'as, $\Id[n]\cdot q=q$. En particular, si nos restringimos a matrices
invertibles, queda definida una acci\'on en formas cuadr\'aticas.

\begin{defWittDef}\label{def:definiciones:equivalencia}
	Si $G\subset\GL[n](F)$ es un subgrupo del grupo de matrices
	invertibles, $G$ \emph{act\'ua} en el conjunto de formas cuadr\'aticas
	en $F^n$ v\'{\i}a \eqref{eq:definiciones:accion}. Dadas dos formas $Q$
	y $Q'$, decimos que son \emph{$G$-equivalentes}, si existe $M\in G$ tal
	que $Q=M\cdot Q'$.
\end{defWittDef}

Escribimos $Q\simeq Q'$ para indicar que $Q$ y $Q'$ son equivalentes.

\begin{ejemWittDef}\label{ejem:definiciones:accion}
	Las formas $Q=x^2-y^2$ y $Q'=x\,y$ son $\GL[2](\bb Q)$-equivalentes,
	pero no son $\GL[2](\bb Z)$-equivalentes. En t\'erminos de las matrices
	asociadas,
	\begin{displaymath}
		\begin{bmatrix} 1 & 1 \\ 1 & -1 \end{bmatrix}\,
		\begin{bmatrix} & 1/2 \\ 1/2 & \end{bmatrix}\,
		\begin{bmatrix} 1 & 1 \\ 1 & -1 \end{bmatrix}\,=\,
		\begin{bmatrix} 1 & \\ & -1 \end{bmatrix}
		\text{ .}
	\end{displaymath}
	%
	Sin embargo, si fuesen equivalentes con respecto a $\GL[2](\bb Z)$,
	entonces los subconjuntos $Q(\bb Z^2)$ y $Q'(\bb Z^2)$ ser\'{\i}an
	iguales, pero no lo son.
\end{ejemWittDef}

\begin{obsWittDef}\label{obs:definiciones:invariantes}
	El subconjunto $Q(V)\setmin\{0\}\subset F$ es un invariante de la
	clase de equivalencia de la forma cuadr\'atica $Q$ en $V$.
\end{obsWittDef}

\begin{defWittDef}\label{def:definiciones:isotropia}
	El \emph{grupo de isotrop\'{\i}a} de una forma cuadr\'atica $Q$ en
	$F^n$ es el subgrupo%
	\footnote{
		Dado que $M\cdot Q(v)=Q(\trnsp M\,v)$, esta definici\'on no es,
		tal vez, la definici\'on deseada. Es el problema de c\'omo fue
		definida la acci\'on, o, mejor dicho de cu\'al es el grupo
		que act\'ua (ver la \obsname~%
		\ref{obs:definiciones:isotropia:accion:identificacion}).
	}
	\begin{displaymath}
		\Big\{M\in\GL[n](F)\,:\,M\cdot Q=Q\Big\}
	\end{displaymath}
	%
	de $\GL[n](F)$. Este subgrupo se denotar\'a por $\GL[n](F)_Q$. Si
	$G\subset\GL[n](F)$ es un subgrupo, el grupo de isotrop\'{\i}a de $Q$
	para la acci\'on de $G$ es $G_Q:=\GL[n](F)_Q\cap G$.
\end{defWittDef}

\begin{obsWittDef}\label{obs:definiciones:isotropia:identificacion}
	Al definir la acci\'on de $\GL[n](F)$ en formas cuadr\'aticas
	$Q:\,F^n\rightarrow F$ por \eqref{eq:definiciones:accion},
	impl\'{\i}citamente estamos identificando $F^n$ con $\dual{(F^n)}$.
	Espec\'{\i}ficamente, la identificaci\'on es a trav\'es de la forma
	bilineal no degenerada dada por el producto escalar
	$(x,y)\mapsto x\cdot y$. Si $V$ es un e.v. sobre $F$ y
	$Q:\,\dual V\rightarrow F$ es una forma cuadr\'atica \emph{definida %
	en el espacio dual}, entonces, para $A\in\Endo[F](V)$ podemos definir
	\begin{displaymath}
		\big(A\cdot Q\big)(\varphi)\,=\,Q(\dual A\,\varphi)
		\text{ .}
	\end{displaymath}
	%
	Esta definici\'on tiene sentido sin necesidad de hacer
	identificaciones. Si, ahora, $Q:\,V\rightarrow F$ es una forma
	cuadr\'atica en $V$ y $A\in\Endo[F](V)$ podemos definir 
	\begin{displaymath}
		\big(Q\cdot A\big)(v)\,=\,Q(A\,v)
		\text{ ,}
	\end{displaymath}
	%
	sin necesidad de identificar espacios. Estas dos definiciones son, por
	suerte, compatibles; si tomamos base $\repr\cdot:\,V\rightarrow F^n$ en
	$V$, la base dual ${\repr\cdot}':\,\dual V\rightarrow F^n$ en $\dual V$
	e identificamos $\dual{(F^n)}$ con $F^n$ v\'{\i}a el producto escalar,
	entonces $A\cdot Q=Q\cdot \trnsp A$.
\end{obsWittDef}

Adem\'as de la acci\'on de matrices, hay otra acci\'on definida en el conjunto
de formas cuadr\'aticas. Si $\alpha\in F$,
\begin{equation}
	\label{eq:definiciones:accion:similitud}
	x\,\mapsto\,\alpha\,Q(x)
\end{equation}
%
define una forma cuadr\'atica.

\begin{defWittDef}\label{def:definiciones:similitud}
	El grupo $F^\times$ act\'ua en el conjunto de formas cuadr\'aticas por
	\emph{similitudes}. Dadas dos formas $Q$ y $Q'$, decimos que son
	\emph{similares}, si existe $\alpha\in F^\times$ tal que
	$Q=\alpha\,Q'$.
\end{defWittDef}

\begin{obsWittDef}\label{obs:definiciones:similitud}
	El grupo $F^\times$ de escalares no nulos puede verse como un subgrupo
	de $\GL[n](F)$ v\'{\i}a inmersi\'on diagonal. Por lo tanto, hay dos
	acciones de $F^\times$ en formas cuadr\'aticas. \'Estas est\'an
	relacionadas por:
	\begin{displaymath}
		\alpha\cdot Q\,=\,\alpha^2\,Q
		\text{ .}
	\end{displaymath}
	%
	En particular, como la acci\'on \eqref{eq:definiciones:accion}
	preserva (define) la relaci\'on de equivalencia de formas
	cuadr\'aticas, la acci\'on por similitudes induce una acci\'on del
	cociente $\modcuadrados F$ en el conjunto de clases de equivalencia.
\end{obsWittDef}

\subsection{Otros comentarios}\label{subsec:definiciones:comentarios}
Vamos a ver dos demostraciones de los Teoremas de Witt. Empezamos con algunos
comentarios generales.

Sea $F$ un cuerpo de caracter\'{\i}stica distinta de $2$ y sea $(V,Q)$ un
espacio cuadr\'atico. Introducimos la siguiente notaci\'on para formas
cuadr\'aticas diagonalizadas: si $\lista* a{n}\in F$, definimos
\begin{displaymath}
	\diagonal{\lista* a{n}}(\lista x{n})\,=\,
		a^1\,x_1^2\,+\,\cdots\,+\,a^n\,x_n^2
	\text{ .}
\end{displaymath}
%

\begin{teoWittCan}\label{teo:cancelacion:diagonal:binaria}
	Sea $Q$ una forma cuadr\'atica binaria, no degenerada, definida sobre
	un cuerpo $F$ de caracter\'{\i}stica distinta de $2$. Si
	$d:=\discriminante\,Q$ y $Q$ representa $\alpha\in F^\times$, entonces
	\begin{displaymath}
		Q\,\simeq\,\diagonal{\alpha,\alpha\,d}
		\text{ .}
	\end{displaymath}
	%
	En particular, dos formas cuadr\'aticas binarias no degeneradas son
	equivalentes, si y s\'olo si tienen el mismo discriminante y
	representan alg\'un valor com\'un en $F^\times$.
\end{teoWittCan}

Las isometr\'{\i}as del espacio constituyen un subgrupo del grupo de
transformaciones lineales invertibles $\GL(V)$.

\begin{defWittCan}\label{def:cancelacion:impar:ortogonal}
	El \emph{grupo ortogonal} de $(V,Q)$ es el subgrupo
	\begin{displaymath}
		\OO(V)\,=\,\Big\{A\in\GL(V)\,:\,Q(A\,v)=Q(v)
			\text{ para todo } v\in V\Big\}
	\end{displaymath}
	%
	de $\GL(V)$. Tambi\'en denotamos este subgrupo por $\OO(Q)$ o por
	$\OO(V,Q)$.
\end{defWittCan}

El grupo ortogonal de $(V,Q)$ es el subgrupo de transformaciones lineales
$A:\,V\rightarrow V$ invertibles tales que $Q(A\,v)=Q(v)$, para todo
$v\in V$; no es el subgrupo de transformaciones que cumplen
$A\cdot Q=Q$.%
\footnote{
	Sin identificaciones, $A\cdot Q$ no tendr\'{\i}a sentido. El grupo
	ortogonal $\OO(V)$ es el subgrupo de isotrop\'{\i}a \emph{con %
	respecto a la acci\'on a derecha} $Q\cdot A$.
	%
	Talvez haya que cambiar la perspectiva y pensar en formas cuadr\'aticas
	del tipo $Q:\,\dual V\rightarrow F$.
	%
	O, talvez, lo mejor sea \emph{elegir} de entrada una identificaci\'on
	$V\rightarrow\dual V$ (o $\dual V\rightarrow V$).
}

% Si bien las definiciones y los enunciados son para espacios cuadr\'aticos (y,
% en caracter\'{\i}stica impar, es lo mismo un espacio cuadr\'atico que un
% espacio bilineal sim\'etrico), los argumentos parecen tener m\'as que ver con
% la forma bilineal y con preservar la forma bilineal que con la forma
% cuadr\'atica. Por ejemplo, en caracter\'{\i}stica $2$ ser\'a necesario asumir
% que la forma bilineal asociada es no degenerada; esta condici\'on es m\'as
% restrictiva que asumir que la forma cuadr\'atica es no degenerada.

\begin{obsWittCan}\label{obs:cancelacion:impar:ortogonal}
	Si $B=B_Q$ es la forma bilineal asociada, entonces%
	\footnote{
		Que $A$ preserve $Q$ implica que $A$ preserva $B$; la
		rec\'{\i}proca es cierta en caracter\'{\i}stica impar, pues
		podemos recuperar la forma cuadr\'atica de la forma bilineal
		sim\'etrica asocida.
	}
	\begin{displaymath}
		\OO(V)\,=\,\Big\{A\in\GL(V)\,:\,B(A\,v,A\,v_1)=B(v,v_1)
			\text{ para todo }v,v_1\in V\Big\}
		\text{ .}
	\end{displaymath}
	%
	Si $A\in\OO(V)$, tomando bases,
	\begin{equation}
		\label{eq:cancelacion:impar:ortogonal:determinante}
		(\det\,A)^2\,\discriminante\,B\,=\,\discriminante\,B
		\text{ .}
	\end{equation}
	%
	En particular, \emph{si $B$ es no degenerada}, $\det(A)\in\{\pm 1\}$.%
	\footnote{
		Si $B$ es degenerada, diagonalizando, se ve que existen
		endomorfismos que preservan $B$ y que no son invertibles,
		necesariamente, o que, aunque sean invertibles, no tengan
		determinante a $\pm 1$.
	}
\end{obsWittCan}

\begin{obsWittCan}\label{obs:cancelacion:impar:ortogonal:transpuesta}
	Si $B$ es no degenerada, podemos hablar de la adjunta $\adjnt A$ de
	$A\in\OO(V)$. Si $V=F^n$, identificamos $F^n$ con su dual y
	$A\in\MM[n\times n](F)$, entonces $A$ preserva la forma bilineal $B$,
	si y s\'olo si $\trnsp A$ la preserva:
	\begin{displaymath}
		\begin{aligned}
			& B(A\,v,A\,v_1)\,=\,B(v,v_1)
				\quad\text{para todo }v,v_1
			\quad\Rightarrow\quad \adjnt A\,A=I \\
			& \qquad\quad\Rightarrow\quad A\,\adjnt A=I
			\quad\Rightarrow\quad\trnsp{\adjnt A}\,\trnsp A=I \\
			& \qquad\quad\Rightarrow\quad
				B(\trnsp A\,v,\trnsp A\,v_1)\,=\,B(v,v_1)
				\quad\text{para todo }v,v_1
			\text{ .}
		\end{aligned}
		%
	\end{displaymath}
	%
\end{obsWittCan}

\begin{defWittCan}\label{def:cancelacion:impar:ortogonal:rotaciones}
	Una \emph{rotaci\'on} en $V$ (con respecto a $Q$)%
	\footnote{
		O a $B$, mejor.
	}
	es un elemento $A\in\OO(V)$ tal que $\det(A)=1$. Una \emph{reflexi\'on}
	es un elemento $A\in\OO(V)$ tal que $\det(A)=-1$.
\end{defWittCan}

\begin{propoWittCan}\label{propo:cancelacion:impar:ortogonal:determinante}
	El determinante $\det:\,\OO(V)\rightarrow\{\pm 1\}$ es un morfismo
	sobreyectivo.
\end{propoWittCan}

\begin{proof}
	La idea es construir un elemento de determinante $-1$. Si $B$ est\'a
	diagonalizada, entonces $\diagonal{1,\,\cdots,\,1,\,-1}$ preserva $B$ y
	tiene determinante $-1$. Esta transformaci\'on es la reflexi\'on con
	respecto al hiperplano $(0,\,\cdots,\,0,\,1)^\perp$. En general, si
	$u\in V$ es un vector anisotr\'opico, definimos
	$\refl[u]:\,V\rightarrow V$ por
	\begin{equation}
		\label{eq:cancelacion:impar:ortogonal:reflexion}
		\refl[u](v)\,=\,v\,-\,2\,\frac{B(v,u)}{Q(u)}\,u
			% \,=\,\Big(\id-2\,\frac{R_B(u)}{Q(u)}\Big)(v)
		\text{ .}
	\end{equation}
	%
	La transformaci\'on \eqref{eq:cancelacion:impar:ortogonal:reflexion}
	cumple:
	\begin{itemize}
		\item $B(\refl[u](v),\refl[u](w))=B(v,w)$ para todo $v,w\in V$,
		\item $\refl[u](u)=-u$ y
		\item $\refl[u](v)=v$, si $v\in u^\perp$.
	\end{itemize}
	%
	Por el \lemaname~\ref{lema:ortogonales:complemento},
	$V=\generado u\oplus u^\perp$. Eligiendo una base adecuada (empezando
	con $u$), la transformaci\'on $\refl[u]$ est\'a representada por
	la matriz
	\begin{displaymath}
		\begin{bmatrix}
			-1 & & & \\
			& 1 & & \\
			& & \ddots & \\
			& & & 1
		\end{bmatrix}
		\text{ .}
	\end{displaymath}
	%
	En particular, $\det(\refl[u])=-1$.
\end{proof}

Si $(V,B)$ es no degenerado, entonces el subespacio $u^\perp$ es no degenerado,
tambi\'en.

