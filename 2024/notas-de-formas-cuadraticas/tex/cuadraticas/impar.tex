\theoremstyle{plain}
\newtheorem{teoCuadImp}{\teoname}[section]
\newtheorem{coroCuadImp}[teoCuadImp]{\coroname}
\newtheorem{lemaCuadImp}[teoCuadImp]{\lemaname}

\theoremstyle{definition}
\newtheorem{defCuadImp}[teoCuadImp]{\defname}
\newtheorem{obsCuadImp}[teoCuadImp]{\obsname}
\newtheorem{ejemCuadImp}[teoCuadImp]{\ejemname}
\newtheorem{ejerCuadImp}[teoCuadImp]{\ejername}

%-------------

En su versi\'on m\'as concreta, las formas cuadr\'aticas son polinomios
homog\'eneos de grado $2$. Por ejemplo,
\begin{enumerate}[(I)]
	\item\label{item:impar:ejemplos:i}
		$x^2+y^2+z^2$
	\item\label{item:impar:ejemplos:ii}
		$x^2+5xy-y^2$
	\item\label{item:impar:ejemplos:iii}
		$2x^2+3y^2$
	\item\label{item:impar:ejemplos:iv}
		$x_1^2+\,\cdots\,+x_n^2$
	\item\label{item:impar:ejemplos:v}
		$x_1^2+\,\cdots\,+x_p^2-x_{p+1}^2-\,\cdots\,-x_n^2$
\end{enumerate}
%
Los ejemplos anteriores, salvo \eqref{item:impar:ejemplos:ii}, son
ejemplos de formas cuadr\'aticas \emph{diagonales}, dado que no aparecen
t\'erminos cruzados. La forma \eqref{item:impar:ejemplos:ii} se puede
\emph{diagonalizar} completando cuadrados, mediante un cambio de variables: 
\begin{displaymath}
	x^2\,+\,5\,xy\,-\,y^2\,=\,\big(x+\tfrac 5{2}\,y\big)^2\,-\,
					\tfrac{29} 4\,y^2
	\text{ .}
\end{displaymath}
%
Las formas \eqref{item:impar:ejemplos:iv} y
\eqref{item:impar:ejemplos:v} provienen de formas bilineales sim\'etricas
que conocemos:
\begin{displaymath}
	\begin{aligned}
		x_1^2\,+\,\cdots\,+\,x_n & \,=\,v\,\cdot\,v \quad\text{y} \\
		x_1^2\,+\,\cdots\,+x_p^2\,-\,x_{p+1}^2\,-\,\cdots\,-\,x_n^2
			& \,=\,\langle v,v\rangle_{p,q}
			\text{ ,}
	\end{aligned}
	%
\end{displaymath}
%
donde $v=\trnsp{(\lista x{n})}\in\bb R^n$. La forma
\eqref{item:impar:ejemplos:iii} es diagonal y, sobre $\bb R$, es
\emph{equivalente} a $x^2+y^2$; sobre $\bb Q$, no son equivalentes. Una
raz\'on suficiente que explica esta diferencia es que la ecuaci\'on
\begin{displaymath}
	2\,x^2\,+\,3\,y^2\,=\,1
\end{displaymath}
%
tiene soluci\'on en variable real, pero no tiene soluci\'on si $x$ e $y$
s\'olo pueden tomar valores racionales.

Sobre un cuerpo de caracter\'{\i}stica distinta de $2$, toda forma
cuadr\'atica es diagonalizable. Esto deja de ser cierto sobre un cuerpo de
caracter\'{\i}stica $2$. Si, adem\'as, todo elemento no nulo del cuerpo de
base es un cuadrado --en el cuerpo--, entonces toda forma es equivalente a
\eqref{item:impar:ejemplos:iv}.
% En lo que resta de esta secci\'on, tratamos el caso de
% formas cuadr\'aticas sobre cuerpos de caracter\'{\i}stica impar.

\subsection{Definiciones y ejemplos}%
	\label{subsec:cuadraticas:impar:definiciones}
Sea $F$ un cuerpo de caracter\'{\i}stica impar y sea $V$ un espacio vectorial
sobre $F$.

\begin{defCuadImp}\label{def:impar:definicion}
	Una \emph{forma cuadr\'atica} en $V$ es una funci\'on
	$Q:\,V\rightarrow F$ que cumple:
	\begin{enumerate}[(i)]
		\item\label{item:impar:definicion:homogenea}
			$Q(c\,v)=c^2\,Q(v)$, para todo $v\in V$ y toda
			$c\in F$, y
		\item\label{item:impar:definicion:bilineal}
			la funci\'on
			\begin{math}
				B(v,w):=\tfrac 1{2}\,\big(Q(v+w)-Q(v)-Q(w)\big)
			\end{math} es bilineal.
	\end{enumerate}
	%
	La funci\'on $B=B_Q$ definida en
	\eqref{item:impar:definicion:bilineal} se denomina \emph{forma %
	bilineal asociada} a la forma cuadr\'atica $Q$.
\end{defCuadImp}

\begin{obsCuadImp}\label{obs:impar:definicion}
	La relaci\'on entre la forma cuadr\'atica $Q$ y su forma bilineal
	asociada $B$ se puede expresar como:
	\begin{equation}
		\label{eq:impar:definicion:bilineal}
		Q(v+w)\,=\,Q(v)\,+\,Q(w)\,+\,2\,B(v,w)
		\text{ .}
	\end{equation}
	%
	La relaci\'on de perpendicularidad correspondiente a $B$ se puede
	expresar en t\'erminos de $Q$:
	\begin{equation}
		\label{eq:impar:definicion:perpendicular}
		B(v,w)\,=\,0\quad\Leftrightarrow\quad
			Q(v+w)\,=\,Q(v)\,+\,Q(w)
		\text{ .}
	\end{equation}
	%
\end{obsCuadImp}

\begin{obsCuadImp}\label{obs:impar:definicion:polinomios}
	Inductivamente, de la identidad
	\eqref{eq:impar:definicion:bilineal}, se deduce que
	\begin{displaymath}
		Q(v_1+\,\,\cdots\,+v_r)\,=\,Q(v_1)\,+\,\cdots\,+\,Q(v_r)\,+\,
			2\,\sum_{i<j}\,B(v_i,v_j)
		\text{ ,}
	\end{displaymath}
	%
	para todo $\lista v{r}\in V$, $r\geq 2$. En particular, fijando una
	base y usando \eqref{item:impar:definicion:homogenea},
	\begin{equation}
		\label{eq:impar:definicion:polinomios}
		f(\lista x{n}) \,=\,Q(x_1\,e^1+\,\cdots\,+x_n\,e^n)
			\,=\,\sum_{i=1}^n\,a^i\,x_i^2\,+\,
				\sum_{i<j}\,a^{ij}\,x_i\,x_j \\
			% \,=\, a^i\,x_i^2\,+\,\tfrac 1{2}\,a^{ij}\,x_i\,x_j
		\text{ ,}
	\end{equation}
	%
	donde $a^i=Q(e^i)$ y $a^{ij}=2\,B(e^i,e^j)$. Es decir, en coordenadas,
	$Q$ est\'a representada por un polinomio homog\'eneo de grado $2$.

	Rec\'{\i}procamente, si $f(\lista x{n})$ es un polinomio homog\'eneo
	de grado $2$ con coeficientes $a^i$ y $a^{ij}$ ($i<j$), como en
	\eqref{eq:impar:definicion:polinomios}, la funci\'on
	$Q:\,V\rightarrow F$ definida por
	\begin{math}
		Q(x_1\,e^1+\,\cdots\,+x_n\,e^n):=f(\lista x{n})
	\end{math} --una vez fijada una base-- verifica
	\eqref{item:impar:definicion:homogenea} y la funci\'on asociada
	$B=B_Q$ est\'a dada por
	\begin{equation}
		\label{eq:impar:definicion:polinomio:bilineal}
		B(v,w) % \,=\,\tfrac 1{2}\,\big(Q(v+w)\,-\,Q(v)\,-\,Q(w)\big)
			\,=\,\sum_{i=1}^n\,a^i\,x_i\,y_i\,+\,\tfrac 1{2}\,
				\sum_{i<j}\,a^{ij}\,\big(x_i\,y_j+x_j\,y_i\big)
			\,=\,\repr v\,\cdot\,M\,\repr w
		\text{ ,}
	\end{equation}
	%
	si $v=x_i\,e^i$, $w=y_i\,e^i$, donde $M$ denota la matriz
	\begin{displaymath}
		M\,=\,
		\begin{bmatrix}
			a^1 & a^{12}/2 & \cdots & a^{1n}/2 \\
			a^{12}/2 & a^2 & \cdots & a^{2n}/2 \\
			\vdots & \vdots & \ddots & \vdots \\
			a^{1n}/2 & a^{2n}/2 & \cdots & a^n
		\end{bmatrix}
		\text{ .}
	\end{displaymath}
	%
	En particular, $B$ es bilineal y $Q$ es una forma cuadr\'atica.
\end{obsCuadImp}

De acuerdo con la \obsname~\ref{obs:impar:definicion:polinomios}, la
elecci\'on de una base determina una correspondencia entre formas cuadr\'aticas y polinomios homog\'eneos de grado $2$, v\'{\i}a
\eqref{eq:impar:definicion:polinomios}.
\begin{center}
	\begin{tikzcd}
		\left\{\begin{matrix}
			\text{formas} \\
			\text{cuadr\'aticas}
		\end{matrix}\right\} \arrow[r,leftrightarrow] &
		\left\{\begin{matrix}
			\text{polinomios homog\'eneos} \\
			\text{de grado } 2
		\end{matrix}\right\}
	\end{tikzcd}
\end{center}
Al mismo tiempo, la elecci\'on de
una base determina una correspondencia entre formas bilineales y matrices.%
\footnote{
	\teoname~\ref{teo:matrices:matrices}.
}
A trav\'es de esta biyecci\'on, formas sim\'etricas se corresponden con
matrices sim\'etricas, formas no degeneradas se corresponden con matrices no
singulares, etc.%
\footnote{
	\teoname~\ref{teo:matrices:simetria}, \teoname~%
	\ref{teo:nodegeneradas:nodegeneradas}.
}
\begin{center}
	\begin{tikzcd}
		\left\{\begin{matrix}
			\text{formas bilineales} \\
			\text{sim\'etricas}
		\end{matrix}\right\} \arrow[r,leftrightarrow] &
		\left\{\begin{matrix}
			\text{matrices} \\
			\text{sim\'etricas}
		\end{matrix}\right\}
	\end{tikzcd}
\end{center}

\begin{obsCuadImp}\label{obs:impar:definicion:bilineal}
	Toda forma cuadr\'atica se puede recuperar de su forma bilineal
	asociada evaluando en la diagonal:
	\begin{equation}
		\label{eq:impar:definicion:bilineal:diagonal}
		Q(v)\,=\,B(v,v)
		\text{ .}
	\end{equation}
	%
	Rec\'{\i}procamente, dada una forma bilineal $B$, la funci\'on $Q$
	definida por \eqref{eq:impar:definicion:bilineal:diagonal} es una
	forma cuadr\'atica y, si $B$ es sim\'etrica, la forma bilineal asociada
	a $Q$ es $B_Q=B$, por polarizaci\'on.%
	\footnote{
		\teoname~\ref{teo:definiciones:polarizacion}.
	}
\end{obsCuadImp}

La \obsname~\ref{obs:impar:definicion:bilineal} muestra que existe
una correspondencia entre formas bilineales sim\'etricas y formas
cuadr\'aticas.
\begin{center}
	\begin{tikzcd}
		\left\{\begin{matrix}
			\text{formas bilineales} \\
			\text{sim\'etricas}
		\end{matrix}\right\} \arrow[r,leftrightarrow] &
		\left\{\begin{matrix}
			\text{formas} \\
			\text{cuadr\'aticas}
		\end{matrix}\right\}
	\end{tikzcd}
\end{center}
A trav\'es de esta correspondencia, $B(v,w)=0$ para todo
$v,w\in V$, si y s\'olo si $Q(v)=0$ para todo $v\in V$. Adem\'as, la
noci\'on de perpendicularidad se puede expresar tanto en t\'erminos de $B$ como
en t\'erminos de $Q$, por \eqref{eq:impar:definicion:perpendicular}.
De esta manera, podemos asociar a toda forma cuadr\'atica una matriz
sim\'etrica eligiendo una base del espacio. Llamaremos a esta matriz la
\emph{matriz asociada} a la forma cuadr\'atica.

\begin{ejemCuadImp}\label{ejem:impar:traza}
	La forma traza $Q(L)=\Traza(L^2)$ en $\Endo[F](V)$ es una forma
	cuadr\'atica. Si $V=F^2$ y $L$ est\'a representada por la matriz
	$L=\sbmatrix{ x & y \\ z & t }$, entonces
	\begin{displaymath}
		Q(L)\,=\,x^2\,+\,2\,y\,z\,+\,t^2
		\text{ .}
	\end{displaymath}
	%
\end{ejemCuadImp}

\begin{ejemCuadImp}\label{ejem:impar:binaria}
	Si $f(x,y)=a\,x^2+b\,xy+c\,y^2$, entonces
	\begin{math}
		Q\big(\sbmatrix{ x \\ y }\big)=f(x,y)
	\end{math} es una forma cuadr\'atica \emph{binaria}, en $F^2$. La forma
	$Q$ est\'a representada por la matriz sim\'etrica
	$\sbmatrix{ a & b/2 \\ b/2 & c }$ en la base can\'onica:
	\begin{displaymath}
		Q\Big(\begin{bmatrix} x \\ y \end{bmatrix}\Big)\,=\,
		\begin{bmatrix} x \\ y \end{bmatrix}\,\cdot\,
			\begin{bmatrix} a & b/2 \\ b/2 & c \end{bmatrix}\,
		\begin{bmatrix} x \\ y \end{bmatrix}
		\text{ .}
	\end{displaymath}
	%
	La forma $Q$ tambi\'en se puede expresar como
	\begin{math}
		Q\big(\sbmatrix{ x \\ y }\big)=
			\sbmatrix{ x \\ y }\,\cdot\,\sbmatrix{ a & b \\ & c }\,
				\sbmatrix{ x \\ y }
	\end{math}, pero $\sbmatrix{ a & b \\ & c }$ no es una matriz
	sim\'etrica.
\end{ejemCuadImp}

\begin{defCuadImp}\label{def:impar:discriminante}
	El \emph{discriminante} de una forma cuadr\'atica se define como
	el discriminante de su forma bilineal asociada. Equivalentemente, se
	define como el determinante de cualquiera de sus matrices asociadas.
	Este valor est\'a bien definido m\'odulo cuadrados del cuerpo y lo
	denotamos $\discriminante\,Q$.
\end{defCuadImp}

\begin{ejemCuadImp}\label{ejem:impar:hiperbolico}
	El discriminante de la forma $x^2-y^2$ en $F^2$ es $-1$.
\end{ejemCuadImp}

\begin{defCuadImp}\label{def:impar:espacio}
	Un \emph{espacio cuadr\'atico} es un par $(V,Q)$, donde $V$ es un
	espacio vectorial sobre un cuerpo $F$ y $Q:\,V\rightarrow F$ es una
	forma cuadr\'atica en $V$.
\end{defCuadImp}

\begin{obsCuadImp}\label{obs:impar:subespacio}
	Si $(V,Q)$ es un espacio cuadr\'atico y $W\subset V$ es un subespacio
	vectorial, el par $(W,Q|_W)$ es un espacio cuadr\'atico.
\end{obsCuadImp}

\subsection{Primeros resultados}\label{subsec:cuadraticas:impar:primeros}
Sea $(V,Q)$ un espacio cuadr\'atico de dimensi\'on finita $n\geq 1$, sobre un
cuerpo de caracter\'{\i}stica distinta de $2$.

\begin{teoCuadImp}\label{teo:impar:diagonalizacion}
	Existe una base de $V$ con respecto a la cual $Q$ se diagonaliza. Es
	decir,
	\begin{displaymath}
		Q\big(x_i\,e^i\big)\,=\,a^i\,x_i^2
		\text{ ,}
	\end{displaymath}
	%
	donde $a^i=Q(e^i)$. El discriminante de $Q$ es igual al producto
	$a^1\,\cdots\,a^n$.
\end{teoCuadImp}

\begin{teoCuadImp}\label{teo:impar:ortogonal:discriminante}
	Si $V=W\oplus U$ con $W\perp U$, entonces
	$\discriminante\,V=\discriminante\,W\,\discriminante\,U$.
\end{teoCuadImp}

Para diagonalizar una forma cuadr\'atica, usamos los \lemaname~%
\ref{lema:ortogonales:diagonal} y \ref{lema:ortogonales:complemento}.%
\footnote{
	Sea $(V,B)$ un espacio bilineal no necesariamente sim\'etrico, sobre un
	cuerpo arbitrario $F$ no necesariamente de caracter\'{\i}stica impar.
	Si $\lista v{r},\,w\in V$ cumplen
	\begin{itemize}
		\item $B(v_i,v_i)\neq 0$ para todo $i$,
		\item $B(v_i,v_j)=0$ ($v_j\in v_i^\rperp$), si $j>i$, y
		\item $0\neq w\in v_1^\rperp\cap\,\cdots\,\cap v_r^\rperp$,
	\end{itemize}
	%
	entonces el subconjunto $\{\lista v{r},\,w\}$ es l.i. Si
	\begin{displaymath}
		b^i\,v_i\,+\,c\,w\,=\,0
		\text{ ,}
	\end{displaymath}
	%
	se deduce, inductivamente, que $b^i=0$:
	\begin{displaymath}
		0 \,=\,B(v_i,b^j\,v_j+c\,w)\,=\,
			b^1\,B(v_i,v_1)\,+\,\cdots\,b^i\,B(v_i,v_i)
		\text{ .}
	\end{displaymath}
	%
	El \'unico t\'ermino restante en la combinaci\'on es $c\,w=0$. Como
	$w\neq 0$, $c=0$ y el subconjunto era l.i. El problema es hallar $w$
	para proceder inductivamente.
}
El proceso eventualmente termina cuando se llega a un subespacio en donde la
forma cuadr\'atica es id\'enticamente cero (equivalentemente, la forma bilineal
asociada es id\'enticamente cero; la condici\'on $B(v,v)=0$ es suficiente). Si
dicho subespacio no es el subespacio nulo, entonces los vectores hallados,
juntos con cualquier base del subespacio, conforman una base con respecto a la
cual la forma es diagonal.

\begin{ejerCuadImp}\label{ejer:impar:isotropico}
	Diagonalizar $Q(x,y,z)=xy+yz+zx$ ?`Cu\'al es su discriminante? Hallar
	un vector $v$ tal que $Q(v)=0$. Hallar otro vector $w$ tal que $Q(w)=0$
	y $B(v,w)=1$. Si $U=\generado{v,w}$, calcular el discriminante de
	$Q|_U$ y de $Q|_{U^\perp}$ ?`Se cumple $U\oplus U^\perp$ para los
	vectores elegidos?
\end{ejerCuadImp}

\begin{coroCuadImp}\label{coro:impar:diagonalizacion}
	Si $a\in F^\times$, entonces $a$ es un coeficiente en alguna
	diagonalizaci\'on de $Q$, si y s\'olo si $a\in Q(V)$.%
	\footnote{
		C.f. el \teoname~\ref{teo:ortogonales:base}.
	}
\end{coroCuadImp}

\begin{proof}
	Si $Q=a\,x^2+\,\cdots$, entonces $Q(1,0,\,\dots,\,0)=a$. Si
	$Q(v)=a\neq 0$, entonces, aplicando el \lemaname~%
	\ref{lema:ortogonales:complemento}, $V=\generado v\oplus v^\perp$.
	Eligiendo una base de $v^\perp$ y complet\'andola con $v$,
	$Q=a\,x^2+\,\cdots$. Diagonalizando $Q|_{v^\perp}$ se llega a una
	diagonalizaci\'on de $Q$ en la que $a$ es uno de los coeficientes (el
	primero).
\end{proof}

\begin{defCuadImp}\label{def:impar:equivalencia}
	Dos espacios cuadr\'aticos $(V,Q_V)$ y $(W,Q_W)$ se dicen
	\emph{isomorfos} y las formas cuadr\'aticas se dicen
	\emph{equivalentes}, si existe un isomorfismo $A:\,V\rightarrow W$ tal
	que $Q_W(A\,v)=Q_V(v)$ para todo $v\in V$.
\end{defCuadImp}

\begin{ejemCuadImp}\label{ejem:impar:equivalencia:hiperbolico}
	Las formas cuadr\'aticas $x^2-y^2$ y $xy$ son equivalentes sobre
	cualquier cuerpo (de caracter\'{\i}stica distinta de $2$).
\end{ejemCuadImp}

\begin{ejemCuadImp}\label{ejem:impar:equivalencia:racional}
	Las formas cuadr\'aticas $2\,x^2+3\,y^2$ y $x^2+y^2$ son equivalentes
	sobre $\bb R$, pero no lo son sobre $\bb Q$.
\end{ejemCuadImp}

\begin{teoCuadImp}\label{teo:impar:equivalencia:bilineal}
	Dos formas cuadr\'aticas son equivalentes, si y s\'olo si las formas
	bilineales asociadas lo son.
\end{teoCuadImp}

\begin{defCuadImp}\label{def:impar:nodegeneradas}
	Una forma cuadr\'atica se dice \emph{no degenerada}, si su forma
	bilineal asociada es no degenerada. Respectivamente, si la forma
	bilineal es degenerada, se dice que la forma cuadr\'atica de la que
	proviene es \emph{degenerada}.
\end{defCuadImp}

El siguiente resultado ser\'a relevante m\'as adelante, pero hace uso de las
nociones reci\'en definidas.

\begin{defCuadImp}\label{def:impar:representa}
	Decimos que una forma cuadr\'atica $Q$ \emph{representa} un valor $a$,
	si existe $v\in V$ tal que $Q(v)=a$. Decimos que $Q$ \emph{representa %
	$0$} o, m\'as precisamente, que \emph{representa $0$ de manera no %
	trivial}, si existe $v\neq 0$ tal que $Q(v)=0$.
\end{defCuadImp}

\begin{teoCuadImp}\label{teo:impar:hiperbolico}
	Sea $(V,Q)$ un espacio cuadr\'atico de dimensi\'on $2$ sobre un cuerpo
	de caracter\'{\i}stica distinta de $2$. Las siguientes afirmaciones son
	equivalentes:
	\begin{enumerate}[(i)]
		\item\label{item:hiperbolico:forma}
			en alguna base, $Q=x^2-y^2$;
		\item\label{item:hiperbolico:discriminante}
			m\'odulo cuadrados, $\discriminante\,Q=-1$;
		\item\label{item:hiperbolico:isotropico}
			la forma $Q$ es no degenerada y representa $0$ de
			manera no trivial.
	\end{enumerate}
	%
\end{teoCuadImp}

\begin{proof}
	Si diagonalizamos $Q$, en alguna base $a\,x^2+b\,y^2$. Si $Q$ es no
	degenerada, $ab\neq 0$. Y, si $Q(x_0,y_0)=0$, o bien $(x_0,y_0)=(0,0)$,
	o bien $x_0y_0\neq 0$. En tal caso, $b=-a\,\tfrac{x_0^2}{y_0^2}$ y
	$\discriminante\,Q=-\big(a\,\tfrac{x_0}{y_0}\big)^2$.

	Si $\discriminante\,Q=-1$, en alguna base $Q=a\,x^2+b\,y^2$ con
	$ab=-1\tmodulo[\cuadrados F]$. Pero
	\begin{displaymath}
		a\,x^2\,-\,\tfrac 1{a}\,y^2\,=\,\big(a\,x+y\big)\,
			\big(x-\tfrac 1{a}\,y\big)
		\text{ ,}
	\end{displaymath}
	%
	que, despu\'es de un cambio de variables (en $F$), es equivalente a la
	forma del \ejemname~\ref{ejem:impar:equivalencia:hiperbolico}.

	Si $Q=x^2-y^2$, entonces $\discriminante\,Q=-1$.
\end{proof}

\subsection{Clasificaci\'on de formas cuadr\'aticas}%
	\label{subsec:cuadraticas:impar:clasificacion}
\begin{teoCuadImp}\label{teo:impar:arquimedianas}
	Toda forma cuadr\'atica definida en un espacio vectorial complejo de
	dimensi\'on $n$ no degenerada es equivalente, sobre $\bb C$, a
	$x_1^2+\,\cdots\,+x_n^2$. Toda forma cuadr\'atica definida en un
	espacio vectorial real de dimensi\'on $n$ no degenerada es equivalente,
	sobre $\bb R$, a $x_1^2+\,\cdots\,+x_p^2-x_{p+1}^2-\,\cdots\,-x_n^2$
	para un \'unico $p$ entre $0$ y $n$.
\end{teoCuadImp}

En el caso real, el valor de $p$, o bien el par $(p,q)$, se denomina
\emph{signatura} de la forma cuadr\'atica real.

\begin{proof}
	Que toda forma compleja o real es equivalente a una de las formas
	mencionadas es consecuencia de que $\modcuadrados{\bb C}=1$ y
	de que $\modcuadrados{\bb R}=\{\pm1\}$.

	Dado $0\leq p\leq n$, sea
	$Q_p=x_1^2+\,\cdots\,+x_p^2-x_{p+1}^2-\,\cdots\,-x_n^2$. Veamos que
	$Q_p\sim Q_{p'}$ implica $p=p'$. La equivalencia de estas dos formas
	definidas en $\bb R^n$ significa que existen bases
	$\{\lista* e{n}\}$ y $\{\lista* f{n}\}$ con respecto a las cuales una
	\'unica forma $Q$ se representa por $Q_p$ y por $Q_{p'}$,
	respectivamente. En ese caso, si $W=\generado{\lista* e{p}}$ y
	$W'=\generado{\lista*[{p'+1}] f{n}}$, entonces $Q>0$ en $W$ y $Q<0$ en
	$W'$. Esto implica que $W\cap W'=0$ y que
	$\dim(W+W')=\dim\,W+\dim\,W'=p+(n-p')$. Pero $\dim(W+W')\leq n$, con lo
	que $p-p'\leq 0$.%
	\footnote{
		La demostraci\'on depende \'unicamente de
		$\modcuadrados{\bb R}=\{\pm1\}$ y de la existencia de
		un orden en $\bb R$ compatible con la estructura algebraica.
	}
\end{proof}

\begin{ejerCuadImp}\label{ejer:impar:gauss}
	Determinar el valor de la integral
	$\int_{\bb R^n}\,e^{-\pi\,Q(x)}\,\de x$, donde
	$Q:\,\bb R^n\rightarrow\bb R$ es una forma cuadr\'atica definida
	positiva.%
	\hint{
		Todas las formas de grado $n$ definidas positivas son
		equivalentes.
	}
\end{ejerCuadImp}

\begin{teoCuadImp}\label{teo:impar:finitos:representa}%
	\footnote{
		C.f. el \lemaname~\ref{lema:parias:finitos:representa}.
	}
	Sea $\bb F$ un cuerpo finito de caracter\'{\i}stica impar y sea $(V,Q)$
	un espacio cuadr\'atico no degenerado sobre $\bb F$. Si
	$\dim\,V\geq 3$, entonces $Q$ representa $0$ de manera no trivial.
\end{teoCuadImp}

\begin{proof}
	Si $\dim\,V=3$, diagonalizando, podemos suponer que
	$Q=a\,x^2+b\,y^2+c\,z^2$. Como asumimos que $Q$ es no degenerada,
	$abc\neq 0$. En esta situaci\'on, podemos hallar una soluci\'on
	a $Q(x,y,z)=0$ con $z=1$.
\end{proof}

\begin{lemaCuadImp}\label{lema:impar:finitos:representa}
	Si $a,b,c\in\bb F^\times$, la ecuaci\'on
	\begin{equation}
		\label{eq:impar:finitos:representa}
		a\,x^2\,+\,b\,y^2\,+\,c\,=\,0
	\end{equation}
	%
	tiene soluci\'on en $\bb F$.
\end{lemaCuadImp}

\begin{proof}
	Elevar al cuadrado,
	$(x\mapsto x^2):\,\bb F^\times\rightarrow\bb F^\times$, es un morfismo
	cuyo n\'ucleo tiene orden $2$. Si $q=|\bb F|$, entonces
	$\cuadrados{\bb F}=(q-1)/2$. Si $ab\neq 0$, los subconjuntos
	$\{a\,x^2\,:\,x\in\bb F\}$ y $\{-b\,y^2-c\,:\,y\in\bb F\}$ est\'an en
	correspondencia con los cuadrados, $\cuadrados{\bb F}\cup\{0\}$. En
	particular, tienen el mismo cardinal, que es $(q+1)/2$, se solapan y
	la ecuaci\'on \eqref{eq:impar:finitos:representa} tiene
	soluci\'on.
\end{proof}

\begin{teoCuadImp}\label{teo:impar:universal}%
	\footnote{
		C.f. el \lemaname~\ref{teo:parias:universal}.
	}
	Sea $F$ un cuerpo de caracter\'{\i}stica distinta de $2$. Si
	$Q:\,V\rightarrow F$ es una forma cuadr\'atica no degenerada que
	representa $0$ de manera no trivial, entonces $Q(V)=F$.
\end{teoCuadImp}

\begin{proof}
	Sea $v\neq 0$ tal que $Q(v)=0$ y sea $B=B_Q$ la forma bilineal
	asociada.%
	\footnote{
		Notar que, como $Q$ (es decir, $B$) es no degenerada y
		$Q|_{\generado v}=0$, $\dim\,V\geq 2$.
	}
	Como $B$ es no degenerada y $v\neq 0$, existe $w\in V$ tal que
	$B(v,w)\neq 0$.%
	\footnote{
		Necesariamente, $\{v,w\}$ es l.i.
	}
	La expresi\'on
	\begin{displaymath}
		Q(c\,w+v)\,=\,Q(w)\,+\,2\,B(v,w)\,c
	\end{displaymath}
	%
	define una funci\'on af\'{\i}n en la variable $c\in F$. Como
	$2\,B(v,w)\neq 0$, la funci\'on es sobreyectiva.
\end{proof}

\begin{defCuadImp}\label{def:impar:universal}
	Una forma cuadr\'atica $Q:\,V\rightarrow F$ tal que $Q(V)=F$ se dice
	que es \emph{universal}.
\end{defCuadImp}

\begin{coroCuadImp}\label{coro:impar:finitos:universal}%
	\footnote{
		C.f. el \lemaname~\ref{lema:parias:finitos:universal}.
	}
	Sea $\bb F$ un cuerpo finito de caracter\'{\i}stica impar y sea $(V,Q)$
	un espacio cuadr\'atico no degenerado sobre $\bb F$. Si
	$\dim\,V\geq 2$, entonces $Q$ es universal.
\end{coroCuadImp}

% Representa $0$ de manera no trivial en dimensi\'on $3$, si y s\'olo si
% es universal en dimensi\'on $2$.

\begin{teoCuadImp}\label{teo:impar:finitos}
	Sea $\bb F$ un cuerpo finito de caracter\'{\i}stica impar y sea
	$d\in\nocuadrados{\bb F}$. Toda forma cuadr\'atica
	definida en un espacio vectorial sobre $\bb F$ de dimensi\'on $n$ no
	degenerada es equivalente, sobre $\bb F$, a
	\begin{displaymath}
		x_1^2\,+\,\cdots\,+\,x_{n-1}^2\,+\,x_n^2
		\quad\text{o a}\quad
		x_1^2\,+\,\cdots\,+\,x_{n-1}^2\,+\,d\,x_n^2
		\text{ .}
	\end{displaymath}
	%
	Estas dos formas no son equivalentes. En particular, la clase de
	equivalencia de una forma cuadr\'atica no degenerada sobre $\bb F$
	est\'a determinada por su dimensi\'on y su discriminante.
\end{teoCuadImp}

\begin{proof}
	El discriminante de la primera es $1$, mientras que el de la segunda es
	$d$, que no es cuadrado en $\bb F$. En particular, estas formas no son
	equivalentes.

	Sea $n=\dim\,Q$. Si $n=1$, $Q(x)=a\,x^2$ (en alguna base) y la clase de
	equivalencia de $Q$ depende \'unicamente de $a$, m\'odulo cuadrados.
	Como $\modcuadrados{\bb F}=\{1,d\}$, $Q$ es equivalente a
	$x^2$, si $a\in\cuadrados{\bb F}$, o bien a $d\,x^2$, si no.

	Si $n\geq 2$, $Q$ es universal y existe $v\in V$ tal que $Q(v)=1$.
	Eligiendo una base $\{\lista* e{n}\}$ con respecto a la que $Q$ es
	diagonal y $e^1=v$, podemos asumir que
	\begin{displaymath}
		Q(x_i\,e^i)\,=\,x_1^2\,+\,a^2\,x_2^2\,+\,\cdots\,+\,a^n\,x_n^2
		\text{ .}
	\end{displaymath}
	%
	La forma $Q|_{v^\perp}$ es no degenerada y de dimensi\'on $n-1$.
	Inductivamente, podemos asumir que
	$Q|_{v^\perp}=x_2^2+\,\cdots\,+x_{n-1}^2+a\,x_n^2$, donde
	$a=1$, o bien $a=d$.%
	\footnote{
		La demostraci\'on depende de que $1\in Q(V)$ y de que
		$|\modcuadrados{\bb F}|=2$.
	}
\end{proof}

\subsection{Un argumento geom\'etrico}%
	\label{subsec:cuadraticas:impar:geometria}
A continuaci\'on, introducimos algunas nociones ``geom\'etricas'' y las
aplicamos a la clasificaci\'on de formas cuadr\'aticas sobre un cuerpo finito
de caracter\'{\i}stica impar.

\begin{defCuadImp}\label{def:impar:suma}
	La \emph{suma directa ortogonal} de espacios cuadr\'aticos
	$(V_1,Q_1)$ y $(V_2,Q_2)$ es el $(V,Q)$, donde $V=V_1\oplus V_2$ y
	$Q(v_1+v_2)=Q_1(v_1)+Q_2(v_2)$.
\end{defCuadImp}

\begin{defCuadImp}\label{def:impar:isotropico}
	Un \emph{vector isotr\'opico} para una forma cuadr\'atica $Q$ es un
	vector $v\neq 0$ tal que $Q(v)=0$. Un \emph{espacio isotr\'opico} es un
	espacio cuadr\'atico que contiene vectores isotr\'opicos, es decir, un
	espacio cuadr\'atico cuya forma representa $0$ de manera no trivial.
\end{defCuadImp}

El \teoname~\ref{teo:impar:universal} implica que, en caracter\'{\i}stica
impar, toda forma isotr\'opica es universal.

\begin{defCuadImp}\label{def:impar:hiperbolico}
	Un \emph{plano hiperb\'olico} es un espacio cuadr\'atico de dimensi\'on
	$2$ cuya forma cuadr\'atica es equivalente a $x^2-y^2$.%
	\footnote{
		En caracter\'{\i}stica impar, esto es lo mismo que decir
		equivalente a $x\,y$.
	}
\end{defCuadImp}

Escribimos $\hiperbolico$ para denotar un plano hiperb\'olico. Un plano
hiperb\'olico es lo mismo que un espacio cuadr\'atico de dimensi\'on $2$, no
degenerado e isotr\'opico (\teoname~\ref{teo:impar:hiperbolico}).

\begin{ejemCuadImp}\label{ejem:impar:pseudoeuclideos}
	El espacio $\bb R^{2,1}$ es isomorfo a $\hiperbolico\perp\bb R$
	?`Qu\'e pasa con $\bb R^{p,q}$?%
	\footnote{
		Separar los casos $p>q$ y $p<q$.
	}
\end{ejemCuadImp}

\begin{teoCuadImp}\label{teo:impar:isotropico}%
	\footnote{
		C.f. el \teoname~\ref{teo:parias:isotropico}.
	}
	Sea $F$ un cuerpo de caracter\'{\i}stica distinta de $2$ y sea $(V,Q)$
	un espacio cuadr\'atico no degenerado. Si $Q$ admite un vector
	isotr\'opico, entonces $V\simeq\hiperbolico\perp W$ y $W\subset V$ es
	un subespacio no degenerado.
\end{teoCuadImp}

\begin{proof}
	Sea $v\neq 0$ tal que $Q(v)=0$. Veamos que existe $u\in V$ tal que
	$B(u,v)\neq 0$ y $Q(u)=0$.%
	\footnote{
		C.f. la demostraci\'on del \teoname~%
		\ref{teo:simplecticas:dimension}.
	}
	Si $Q(u)=0$, listo. Si no, consideramos vectores de la forma $u+c\,v$,
	$c\in F$. Por un lado, $B(u+c\,v,v)=B(u,v)\neq 0$. Por otro,%
	\footnote{
		C.f. la demostraci\'on del \teoname~%
		\ref{teo:impar:universal}.
	}
	\begin{displaymath}
		Q(u+c\,v)\,=\,Q(u)\,+\,2\,B(u,v)\,c
		\text{ .}
	\end{displaymath}
	%
	Como esta expresi\'on es af\'{\i}n en $c\in F$, existe $c$ de manera
	que se cumpla lo pedido con $u+c\,v$ en lugar de $u$.

	Con estas elecciones, $Q(x\,u+y\,v)=2\,x\,y\,B(u,v)$. Si
	$U=\generado{u,v}$, reescalando, $Q|_U=x\,y$. Entonces, $U$ es un
	plano hiperb\'olico contenido en $V$ y, en consecuencia, es un
	subespacio no degenerado. Por el \teoname~%
	\ref{teo:nodegeneradas:perpendicular}~%
	\eqref{item:perpendicular:complemento}, si $W=U^\perp$, $V=U\oplus W$
	y, como $V$ es no degenerado, por la parte
	\eqref{item:perpendicular:dimension}, $W$ es no degenerado.
\end{proof}

Si $(V,Q)$ es un espacio cuadr\'atico no degenerado y $\dim\,V\geq 1$,
inductivamente, $V$ es isomorfo a $\hiperbolico^{\perp\,m}\perp V'$, donde
$V'\subset V$ es un subespacio \emph{anisotr\'opico}, es decir, tal que no
existe $v'\in V'$ no nulo que satisface $Q(v')=0$. La forma $Q$ se puede
expresar, en alguna base, de la siguiente manera:
\begin{equation}
	\label{eq:impar:isotropico:equivalente}
	x_1\,x_2\,+\,\cdots\,+\,x_{2m-1}\,x_{2m}\,+\,Q'
	\text{ ,}
\end{equation}
%
donde $Q':\,V'\rightarrow F$ es una forma anisotr\'opica de grado $n-2m$
?`Qu\'e se puede decir al respecto de $m$ y de $Q'$? ?`El valor de $m\geq 0$ y
la clase de equivalencia de $Q'$ est\'an un\'{\i}vocamente determinados por (la
clase de) $Q$? Una cosa m\'as o menos inmediata es la siguiente f\'ormula para
el discriminante:
\begin{equation}
	\label{eq:impar:isotropico:discriminante}
	\discriminante\,V\,=\,(-1)^m\,\discriminante\,V'
	\text{ .}
\end{equation}
%

Volvamos al caso en que $F=\bb F$ es un cuerpo finito de caracter\'{\i}stica
impar. Fijamos un no cuadrado $d\in\nocuadrados{\bb F}$ y un espacio
cuadr\'atico $(V,Q)$ no degenerado de dimensi\'on $n$ sobre $\bb F$.

\begin{teoCuadImp}\label{teo:impar:finitos:geometria}
	La forma cuadr\'atica $Q$ es equivalente, sobre $\bb F$, a
	\begin{displaymath}
		x_1\,x_2\,+\,\cdots\,+\,x_{n-3}\,x_{n-2}\,+\,
			\begin{cases}
				x_{n-1}\,x_n & \quad\text{o a} \\
				x_{n-1}^2\,-\,d\,x_n^2 & \text{ ,}
			\end{cases}
	\end{displaymath}
	%
	si $n$ es par, y a
	\begin{displaymath}
		x_1\,x_2\,+\,\cdots\,+\,x_{n-2}\,x_{n-1}\,+\,
			\begin{cases}
				x_n^2 & \quad\text{o a} \\
				d\,x_n^2 & \text{ ,}
			\end{cases}
	\end{displaymath}
	%
	si $n$ es impar.
\end{teoCuadImp}

\begin{proof}
	Si $n=\dim\,V\geq 3$, $V$ posee un vector isotr\'opico y
	$V\simeq\hiperbolico\perp W$. Si $\dim\,W\geq 3$, $W$ posee un vector
	isotr\'opico. Repitiendo esto una cantidad finita de veces, hasta
	conseguimos un subespacio $V'\subset V$ de dimensi\'on $\dim\,V'\leq 2$
	tal que $V\sim\hiperbolico^m\perp V'$. Si $n$ es par,
	$\dim\,V'\in\{0,2\}$. Si $n$ es impar, $\dim\,V'=1$. En ambos casos,
	$V'$ es no degenerado. El resultado se reduce a clasificar las formas
	no degeneradas de grados $0$, $1$ y $2$, sobre $\bb F$.
\end{proof}

\begin{lemaCuadImp}\label{lema:impar:finitos:bajas}
	Si $\dim\,V=1$, entonces $Q$ es equivalente a
	\begin{displaymath}
		x^2 \quad\text{o a}\quad d\,x^2
		\text{ .}
	\end{displaymath}
	%
	Si $\dim\,V=2$, entonces $Q$ es equivalente a
	\begin{displaymath}
		x\,y \quad\text{o a}\quad x^2\,-\,d\,y^2
		\text{ .}
	\end{displaymath}
	%
	En cada caso, las formas no son equivalentes.
\end{lemaCuadImp}

\begin{proof}
	Si $\dim\,V=1$, en alguna base $Q=a\,x^2$, $a\neq 0$. Si
	$a\in\cuadrados{\bb F}$, entonces $Q$ es equivalente a $x^2$. Si
	$a\not\in\cuadrados{\bb F}$, entonces
	$a\equiv d\tmodulo[\cuadrados{\bb F}]$ y $Q$ es equivalente a $d\,x^2$.

	Si $\dim\,V=2$, entonces $Q$ es universal, por el \coroname~%
	\ref{coro:impar:finitos:universal}. En particular,
	$Q$ representa $1$ y, diagonalizando, es equivalente a $x^2-a\,y^2$,
	$a\neq 0$. Si $a\in\cuadrados{\bb F}$, entonces $Q$ es equivalente a
	$x^2-y^2$, que, en caracter\'{\i}stica impar, es equivalente a $x\,y$.
	Si $a\not\in\cuadrados{\bb F}$, entonces $Q$ es equivalente a
	$x^2-d\,y^2$.

	En ambos casos, $\dim\,V=1$ y $\dim\,V=2$, las dos representantes
	pertenecen a clases distintas, como se puede ver calculando sus
	discriminantes.%
	\footnote{
		Otra manera de distinguirlas es por sus im\'agenes, por los
		valores que representan. La forma $x^2$ s\'olo representa
		cuadrados (todos), mientras que $d\,x^2$ s\'olo representa no
		cuadrados (todos). La forma $x\,y$ es isotr\'opica, mientras
		que $x^2-d\,y^2$ no lo es.
	}
\end{proof}

?`C\'omo termina la demostraci\'on del \teoname~%
\ref{teo:impar:finitos:geometria}? Hay que probar que las formas que
aparecen en el enunciado no son equivalentes. Hemos clasificado las formas de
dimensiones bajas --que corresponden a los t\'erminos que aparecen en la
separaci\'on en casos. Intuitivamente, ser\'{\i}a correcto ``cancelar'' las
partes id\'enticas y comparar lo que queda.%
\footnote{
	C.f. el \teoname~\ref{teo:teoremas:cancelacion:bilineales:general}
}
Otra manera es apelar al discriminante. Si $n$ es par, la primera forma tiene
discriminante $(-1)^{n/2}$, mientras que la segunda tiene discriminante
$(-1)^{n/2-1}(-d)$. Como $d\not\equiv 1\tmodulo[\cuadrados{\bb F}]$, las formas
no son equivalentes. El caso con $n$ impar es an\'alogo.

En el siguiente resultado, volvemos al caso general de un cuerpo de
caracter\'{\i}stica impar y un espacio cuadr\'atico $(V,Q)$ sobre el mismo.

\begin{teoCuadImp}\label{teo:impar:nodegeneradas}
	Las siguientes afirmaciones son equivalentes:
	\begin{enumerate}[(i)]
		\item\label{item:impar:nodegeneradas:discriminante}
			la forma $Q$ es no degenerada, es decir,
			$\discriminante\,Q\neq 0$;
		\item\label{item:impar:nodegeneradas:variables}
			si $n=\dim\,V$, en ninguna base se puede expresar $Q$
			como un polinomio homog\'eneo de grado $2$ en menos de
			$n$ variables;
		\item\label{item:impar:nodegeneradas:derivadas}
			con respecto a cualquier base, expresando a $Q$ como
			polinomio homog\'eneo de grado $2$, la \'unica
			soluci\'on com\'un en $V$ a las ecuaciones
			$\partial Q/\partial x_i=0$ es $v=0$.
	\end{enumerate}
	%
\end{teoCuadImp}

\begin{proof}
	La afirmaci\'on \eqref{item:impar:nodegeneradas:discriminante} y
	la afirmaci\'on \eqref{item:impar:nodegeneradas:variables} son
	equivalentes: para un lado, diagonalizar; para el otro, diagonalizar en
	bloques (separar las variables superfluas). La afirmaci\'on
	\eqref{item:impar:nodegeneradas:discriminante} y la afirmaci\'on
	\eqref{item:impar:nodegeneradas:derivadas} son equivalentes: el
	sistema de ecuaciones $\partial Q/\partial x_i=0$ est\'a representado
	por la matriz
	\begin{displaymath}
		\begin{bmatrix}
			2\,a^1 & a^{12} & \cdots & a^{1n} \\
			a^{12} & 2\,a^2 & \cdots & a^{2n} \\
			\vdots & \vdots & \ddots & \vdots \\
			a^{1n} & a^{2n} & \cdots & 2\,a^n
		\end{bmatrix}\,=\,2\,M
	\end{displaymath}
	%
	igual a dos veces la matriz $M$ asociada a la forma bilineal asociada a
	$Q$.
\end{proof}

Sea $V$ un espacio vectorial de dimensi\'on $n\geq 1$ sobre un cuerpo $F$ de
caracter\'{\i}stica distinta de $2$ y sea $Q$ una forma cuadr\'atica en $V$ no
nula. Fijando una base de $V$, la forma $Q$ est\'a representada por un
polinomio homog\'eneo de grado $2$ y determina una hipersuperficie proyectiva
definida sobre $F$.

\begin{coroCuadImp}\label{coro:impar:cuadricas}
	Si $n\geq 3$, entonces $Q$ es no degenerada, si y s\'olo si la
	hipersuperficie $\{Q=0\}$ en $\bb P^{n-1}(\algclos F)$ es irreducible
	y suave. Cuando $n=2$, $\{Q=0\}$ contiene dos puntos en
	$\bb P^1(\algclos F)$, si $Q$ es no degenerada, y contiene un punto, si
	$Q$ es degenerada.
\end{coroCuadImp}

% \subsection{Algunos invariantes}\label{subsec:cuadraticas:impar:invariantes}
% La dimensi\'on de un espacio cuadr\'atico $(V,Q)$ es un invariante. El
% discriminante $\discriminante\,Q$ y la imagen $Q(V)\subset F$ tambi\'en lo
% son.

\begin{ejerCuadImp}\label{ejer:impar:norma}
	Sea $K/F$ una extensi\'on cuadr\'atica. Probar que la funci\'on norma
	$\Norma[K/F]$ es una forma cuadr\'atica no degenerada, sin vectores
	isotr\'opicos.
\end{ejerCuadImp}

\begin{ejerCuadImp}\label{ejer:impar:traza}
	Sea $K=\bb Q(\theta)$, donde $\theta$ es ra\'{\i}z de un polinomio
	irreducible c\'ubico de la forma $f=X^3+a\,X+b$. Sea
	$Q(\alpha)=\Traza[K/\bb Q](\alpha^2)$. Probar que $Q$ es una forma
	cuadr\'atica sobre $\bb Q$ cuya clase de equivalencia est\'a
	determinada por el valor $4\,a^3+27\,b^2$.%
	\footnote{
		Interpretar.
	}
\end{ejerCuadImp}

