\theoremstyle{plain}
\newtheorem{teoCuadPar}{\teoname}[section]
\newtheorem{lemaCuadPar}[teoCuadPar]{\lemaname}

\theoremstyle{definition}
\newtheorem{defCuadPar}[teoCuadPar]{\defname}
\newtheorem{obsCuadPar}[teoCuadPar]{\obsname}
\newtheorem{ejemCuadPar}[teoCuadPar]{\ejemname}
\newtheorem{ejerCuadPar}[teoCuadPar]{\ejername}

%-------------

% La definici\'on de forma cuadr\'atica sobre cuerpos de caracter\'{\i}stica
% $2$ es similar a la definici\'on en caracter\'{\i}stica impar.
Sea $F$ un cuerpo de caracter\'{\i}stica $2$ y sea $V$ un espacio vectorial
sobre $F$.

\begin{defCuadPar}\label{def:parias:definicion}
	Una \emph{forma cuadr\'atica} en $V$ es una funci\'on
	$Q:\,V\rightarrow F$ que cumple:
	\begin{enumerate}[(i)]
		\item\label{item:parias:definicion:homogenea}
			$Q(c\,v)=c^2\,Q(v)$, para todo $v\in V$ y toda
			$c\in F$, y
		\item\label{item:parias:definicion:bilineal}
			la funci\'on
			\begin{math}
				B(v,w):=Q(v+w)-Q(v)-Q(w)
			\end{math} es bilineal.
	\end{enumerate}
	%
	La funci\'on $B=B_Q$ definida en
	\eqref{item:parias:definicion:bilineal} se denomina \emph{forma %
	bilineal asociada} a la forma cuadr\'atica $Q$.
\end{defCuadPar}

\begin{obsCuadPar}\label{obs:parias:definicion}
	La relaci\'on entre la forma cuadr\'atica $Q$ y su forma bilineal
	asociada $B$ se puede expresar como:
	\begin{equation}
		\label{eq:parias:definicion:bilineal}
		Q(v+w)\,=\,Q(v)\,+\,Q(w)\,+\,B(v,w)
		\text{ .}
	\end{equation}
	%
	La relaci\'on de perpendicularidad correspondiente a $B$ se puede
	expresar en t\'erminos de $Q$:
	\begin{equation}
		\label{eq:parias:definicion:perpendicular}
		B(v,w)\,=\,0\quad\Leftrightarrow\quad
			Q(v+w)\,=\,Q(v)\,+\,Q(w)
		\text{ .}
	\end{equation}
	%
	De \eqref{eq:parias:definicion:bilineal}, la forma $B$ es sim\'etrica.
	M\'as aun, como $2=0$,
	\begin{displaymath}
		B(v,v)\,=\,0
		\text{ ,}
	\end{displaymath}
	%
	para todo $v\in V$. Es decir, la forma bilineal asociada a una forma
	cuadr\'atica es siempre \emph{alternada}.
\end{obsCuadPar}

\begin{obsCuadPar}\label{obs:parias:definicion:polinomios}
	Inductivamente, de la identidad
	\eqref{eq:parias:definicion:bilineal}, se deduce que
	\begin{displaymath}
		Q(v_1+\,\,\cdots\,+v_r)\,=\,Q(v_1)\,+\,\cdots\,+\,Q(v_r)\,+\,
			\sum_{i<j}\,B(v_i,v_j)
		\text{ ,}
	\end{displaymath}
	%
	para todo $\lista v{r}\in V$, $r\geq 2$. En particular, fijando una
	base y usando \eqref{item:parias:definicion:homogenea},
	\begin{equation}
		\label{eq:parias:definicion:polinomios}
		f(\lista x{n}) \,=\,Q(x_1\,e^1+\,\cdots\,+x_n\,e^n)
			\,=\,\sum_{i=1}^n\,a^i\,x_i^2\,+\,
				\sum_{i<j}\,a^{ij}\,x_i\,x_j \\
			% \,=\, a^i\,x_i^2\,+\,\tfrac 1{2}\,a^{ij}\,x_i\,x_j
		\text{ ,}
	\end{equation}
	%
	donde $a^i=Q(e^i)$ y $a^{ij}=B(e^i,e^j)$. Es decir, en coordenadas,
	$Q$ est\'a representada por un polinomio homog\'eneo de grado $2$.

	Rec\'{\i}procamente, si $f(\lista x{n})$ es un polinomio homog\'eneo
	de grado $2$ con coeficientes $a^i$ y $a^{ij}$ ($i<j$), como en
	\eqref{eq:parias:definicion:polinomios}, la funci\'on
	$Q:\,V\rightarrow F$ definida por
	\begin{math}
		Q(x_1\,e^1+\,\cdots\,+x_n\,e^n):=f(\lista x{n})
	\end{math} --una vez fijada una base-- verifica
	\eqref{item:parias:definicion:homogenea} y la funci\'on asociada
	$B=B_Q$ est\'a dada por
	\begin{equation}
		\label{eq:parias:definicion:polinomio:bilineal}
		B(v,w)\,=\,\sum_{i<j}\,a^{ij}\,\big(x_i\,y_j+x_j\,y_i\big)
			\,=\,\repr v\,\cdot\,M\,\repr w
		\text{ ,}
	\end{equation}
	%
	si $v=x_i\,e^i$, $w=y_i\,e^i$, donde $M$ denota la matriz
	\begin{displaymath}
		M\,=\,
		\begin{bmatrix}
			0 & a^{12} & \cdots & a^{1n} \\
			a^{12} & 0 & \cdots & a^{2n} \\
			\vdots & \vdots & \ddots & \vdots \\
			a^{1n} & a^{2n} & \cdots & 0
		\end{bmatrix}
		\text{ .}
	\end{displaymath}
	%
	En particular, $B$ es bilineal y $Q$ es una forma cuadr\'atica.
\end{obsCuadPar}

\begin{obsCuadPar}\label{obs:parias:coeficientes}
	Los coeficientes en la diagonal son todos cero. Esto no quiere
	decir que $B=0$. De hecho, $B=0$, si y s\'olo si los coeficientes de
	los t\'erminos cruzados $a^{ij}$ son todos cero, es decir, cuando la
	forma $Q$ se diagonaliza. En caracter\'{\i}stica distinta de $2$,
	siempre es posible hallar una base con respecto a la cual los
	coeficientes $a^{ij}$ sean cero. En caracter\'{\i}stica $2$, es
	necesario (y suficiente) que estos coeficientes no sean cero para que
	la forma bilineal asociada (no la forma cuadr\'atica) no sea
	id\'enticamente cero:
	\begin{displaymath}
		B_Q\,=\,0 \quad\Leftrightarrow\quad
			a^{ij}\,=\,0\text{ para todo }i<j
		\text{ .}
	\end{displaymath}
	%
\end{obsCuadPar}

\begin{obsCuadPar}\label{obs:parias:matriz}
	Una forma cuadr\'atica se puede expresar, fijada una base del espacio
	vectorial subyacente, como un polinomio homog\'eneo de grado $2$.
	Tambi\'en es posible expresar una forma cuadr\'atica en t\'erminos de
	productos de matrices:
	\begin{equation}
		\label{eq:parias:matriz}
		Q(v) \,=\,\repr v\,\cdot\,N\,\repr v
		\text{ ,}
	\end{equation}
	%
	donde $N$ denota la matriz \emph{triangular superior}
	\begin{displaymath}
		N\,=\,
		\begin{bmatrix}
			a^1 & a^{12} & \cdots & a^{1n} \\
			0 & a^2 & \cdots & a^{2n} \\
			\vdots & \vdots & \ddots & \vdots \\
			0 & 0 & \cdots & a^n
		\end{bmatrix}
		\text{ .}
	\end{displaymath}
	%
	A diferencia de lo que ocurre con la matriz $M$ de la forma bilineal
	asociada, en esta matriz aparecen todos los t\'erminos: los
	$a^i=Q(e^i)$ y tambi\'en los $a^{ij}=B(e^i,e^j)$. La matriz de la
	forma bilineal asociada es $M=N+\trnsp N$.%
	\footnote{
		Podr\'{\i}amos haber definido esta matriz al introducir formas
		cuadr\'aticas sobre un cuerpo de caracter\'{\i}stica impar,
		pero, en aquella situaci\'on, la matriz contendr\'{\i}a la
		misma informaci\'on que la matriz de la forma bilineal
		asociada. La relaci\'on entre ambas matrices ser\'{\i}a
		$M=\tfrac 1{2}\,(N+\trnsp N)$.
	}
\end{obsCuadPar}

\begin{ejemCuadPar}\label{ejem:parias:binaria}
	En $F^2$, la matriz de la forma $Q(x,y)=a\,x^2+b\,x\,y+c\,y^2$ es
	$\sbmatrix{ a & b \\ & c }$. La forma bilineal asociada es
	\begin{displaymath}
		B((x,y),(x_1,y_1))\,=\,Q(x+x_1,y+y_1)\,-\,Q(x,y)\,-\,Q(x_1,y_1)
			\,=\,b\,(xy_1+yx)
		\text{ .}
	\end{displaymath}
	%
	La matriz asociada a la forma bilineal es $\sbmatrix{ & b \\ b & }$ y
	$B$ es no degenerada, si y s\'olo si $b\neq 0$.
\end{ejemCuadPar}

\begin{ejemCuadPar}\label{ejem:parias:falla}
	Las formas cuadr\'aticas $x^2+x\,y$ y $x\,y$ est\'an asociadas a la
	misma forma bilineal sim\'etrica v\'{\i}a
	\eqref{eq:parias:definicion:bilineal}. La forma bilineal sim\'etrica
	$B((x,y),(x_1,y_1))=xx_1+yy_1$ no es la forma bilineal asociada a
	ninguna forma cuadr\'atica en caracter\'{\i}stica $2$.
\end{ejemCuadPar}

Como se ve en el \ejemname~\ref{ejem:parias:falla}, la aplicaci\'on
$Q\mapsto B_Q$, que a cada forma cuadr\'atica le asigna su forma bilineal
asociada, no es ni inyectiva, ni sobreyectiva en el espacio de formas
bilineales sim\'etricas. Ahora, una forma bilineal $B=B_Q$, proveniente de una
forma cuadr\'atica, no s\'olo es sim\'etrica, sino que, m\'as aun, es
alternada. Si \emph{correstringimos} la aplicaci\'on $Q\mapsto B_Q$ al espacio
de formas alternadas, se ve que, entonces, es sobreyectiva, pues toda forma
alternada est\'a representada en una base por una matriz (anti) sim\'etrica con
ceros en la diagonal. Conocer la forma bilineal asociada no nos permite
recuperar la forma cuadr\'atica.

\subsection{Un sustituto de la diagonalizaci\'on}
	\label{subsec:cuadraticas:parias:simplecticas}
Sea $(V,Q)$ un espacio cuadr\'atico%
\footnote{
	La definici\'on es an\'aloga a la dada para cuerpos de
	caracter\'{\i}stica impar.
}
y sea $B=B_Q$ la forma bilineal asociada. Si $B$ es no degenerada, por el
\coroname~\ref{coro:simplecticas:dimension}, $\dim\,V=2m$ y existe una base
simpl\'ectica $\{e^1,\,f^1,\,\dots,\,e^m,\,f^m\}$ para $B$. Es decir,
$B(e^i,e^i)=B(f^i,f^i)=0$, $B(e^i,f^i)=1$ y los subespacios
$\generado{e^i,f^i}$ son ortogonales entre s\'{\i}. Con respecto a esta base,
\begin{equation}
	\label{eq:parias:base:simplecticas}
	\begin{aligned}
		& Q(x_1\,e^1+y_1\,f^1+\,\cdots\,+x_m\,e^m+y_m\,f^m)\,=\,
			\sum_{i=1}^m\,Q(x_i\,e^i+y_i\,f^i) \\
		& \qquad\,=\, \sum_{i=1}^m\,(a^i\,x_i^2+x_iy_i+b^i\,y_i^2)
		\text{ ,}
	\end{aligned}
	%
\end{equation}
%
donde $a^i=Q(e^i)$ y $b^i=Q(f^i)$.%
\footnote{
	Los coeficientes $a^i$ y $b^i$ podr\'{\i}an ser cero. Eso no
	cambiar\'{\i}a el hecho de que $B_Q$ es no degenerada.
}
La matriz de $Q$ en esta base es
\begin{displaymath}
	N\,=\,
	\begin{bmatrix}
		a^1 & 1 & & & & & \\
		& b^1 & & & & & \\
		& & \ddots & & & & \\
		& & & & & a^m & 1 \\
		& & & & & & b^m
	\end{bmatrix}
	\text{ .}
\end{displaymath}
%
Rec\'{\i}procamente, si $Q$ en cierta base se expresa como en
\eqref{eq:parias:base:simplecticas}, la matriz de $B_Q$ en la misma base es la
matriz simpl\'ectica \eqref{item:simplecticas:base:numerica}.

En caracter\'{\i}stica distinta de $2$, toda forma cuadr\'atica se diagonaliza,
incluso aquellas cuyas formas bilineales asociadas son degeneradas; el
resultado es una descomposici\'on
\begin{equation}
	\label{eq:parias:diagonalizacion:impar}
	V\,=\,W_1\perp\,\cdots\,\perp W_r\,\perp\,V^\perp
	\text{ ,}
\end{equation}
%
donde $\dim\,W_i=1$ y $Q|_{V^\perp}=0$, que se obtiene buscando sucesivamente
vectores ortogonales y anisotr\'opicos. En caracter\'{\i}stica $2$, si $B=B_Q$
no fuese no degenerada,%
\footnote{
	Aun no definimos lo que significa que $Q$ sea no degenerada.
}
eso significa que $V^\perp\neq 0$. Asumiendo que $B$ no es id\'enticamente
cero,%
\footnote{
	Que alguno de los coeficientes fuera de la diagonal no son cero;
	en caracter\'{\i}stica $2$, una forma cuadr\'atica es diagonal, si
	y s\'olo si su forma bilineal es id\'enticamente cero.
}
$V^\perp\neq V$ y existe alg\'un subespacio $0\neq W\subset V$ tal que
$V=W\oplus V^\perp$ (suma directa de espacios vectoriales). Ahora,
independientemente de la elecci\'on de $W$, $B|_W$ es no degenerada.%
\footnote{
	Si $w_0\in W^\perp$, entonces $B(w_0,w+v')=B(w_0,w)+B(w,v')=0$ y
	$w_0\in V^\perp$.
}
Aplicando lo que sabemos del caso no degenerado, existe una base simpl\'ectica
para $W$. Esta base la completamos con una base de $V^\perp$ (arbitraria). En
una base de este tipo, la forma $Q$ tiene la expresi\'on siguiente:
\begin{displaymath}
	\sum_{i=1}^m\,(a^i\,x_i^2+x_iy_i+b^i\,y_i^2)\,+\,
		\sum_{k=1}^r\,c^k\,z_k^2
	\text{ ,}
\end{displaymath}
%
donde $\dim\,W=2m$ y $\dim\,V^\perp=r$.

\subsection{Formas cuadr\'aticas no degeneradas}%
	\label{subsec:cuadraticas:parias:nodegeneradas}
El \teoname~\ref{teo:impar:nodegeneradas} da condiciones equivalentes a la
propiedad de una forma cuadr\'atica de ser no degenerada, en
caracter\'{\i}stica impar. En caracter\'{\i}stica $2$, un resultado an\'alogo
nos permitir\'a \emph{definir} la noci\'on correspondiente. Sea $(V,Q)$ un
espacio cuadr\'atico sobre un cuerpo de caracter\'{\i}stica $2$ y sea $B=B_Q$
la forma bilineal asociada.

\begin{teoCuadPar}\label{teo:parias:nodegeneradas}
	Las siguientes afirmaciones son equivalentes:
	\begin{enumerate}[(i)]
		\item\label{item:parias:nodegeneradas:nulo}
			el \'unico vector de $V$ que cumple $Q(v)=0$ y
			$B(v,w)=0$ para todo $w\in V$ es $v=0$;
		\item\label{item:parias:nodegeneradas:radical}
			la funci\'on $Q:\,V^\perp\rightarrow F$ es inyectiva;%
			\footnote{
				En t\'erminos geom\'etricos,
				$V^\perp\cap\{Q=0\}=\{0\}$.
			}
		\item\label{item:parias:nodegeneradas:variables}
			si $n=\dim\,V$, en ninguna base se puede expresar $Q$
			como un polinomio homog\'eneo de grado $2$ en menos de
			$n$ variables;
		\item\label{item:parias:nodegeneradas:derivadas}
			con respecto a cualquier base, expresando a $Q$ como
			polinomio homog\'eneo de grado $2$, la \'unica
			soluci\'on com\'un en $V$ a las ecuaciones $Q(v)=0$ y
			$\partial Q/\partial x_i=0$ es $v=0$.
	\end{enumerate}
	%
\end{teoCuadPar}

\begin{defCuadPar}\label{def:parias:nodegeneradas}
	La forma cuadr\'atica $Q$ se dice \emph{no degenerada}, si verifica las
	condiciones del \teoname~\ref{teo:parias:nodegeneradas}. Un
	\emph{vector isotr\'opico} para $Q$ es un vector $v\neq 0$ tal que
	$Q(v)=0$. Decimos que $Q$ es \emph{universal}, si $Q(V)=F$. Dos formas
	$Q_1:\,V_1\rightarrow F$ y $Q_2:\,V_2\rightarrow F$ son
	\emph{equivalentes}, si existe un isomorfismo $A:\,V_1\rightarrow V_2$
	tal que $Q_2(A\,v)=Q_1(v)$ para todo $v\in V_1$.
\end{defCuadPar}

Las definiciones anteriores son, esencialmente, las mismas que para
caracter\'{\i}stica impar. Sin embargo, no todas se pueden interpretar en
t\'erminos de la forma bilineal asociada.

\subsection{Ejemplos}\label{subsec:cuadraticas:parias:ejemplos}

\begin{ejemCuadPar}\label{ejem:parias:bajas}
	En dimensi\'on $1$, todo espacio no nulo es no degenerado. En
	dimensi\'on $2$, la forma $a\,x^2+b\,xy+c\,y^2$ es no degenerada, si
	\begin{itemize}
		\item $b\neq 0$, o bien
		\item $b=0$ y $ac\not\in\cuadrados F$.
	\end{itemize}
	%
	Por ejemplo, $x\,y$ es no degenerada, pero $x^2-y^2=x^2+y^2$ es
	degenerada.
\end{ejemCuadPar}

\begin{ejemCuadPar}\label{ejem:parias:tres}
	En $F^3$, definimos $Q(x,y,z)=x^2+x\,y+y^2+z^2$. La forma bilineal
	asociada es $B=x\,y_1+x_1\,y$. En particular, $B$ es degenerada, pero
	$Q$ no lo es: $V^\perp=\generado{(0,0,1)}$, pero
	$Q(0,0,\gamma)=\gamma^2$.
\end{ejemCuadPar}

\begin{ejemCuadPar}\label{ejem:parias:extensiones}
	Sea $Q(x,y)=x^2+c\,y^2$ en $F^2$, donde $c\not\in\cuadrados F$. Como
	$Q$ es diagonal, $V^\perp=V$, pero $Q(x,y)=0$ implica $x=y=0$. Sin
	embargo, sobre $\algclos F$,%
	\footnote{
		O en $F(\sqrt c)$.
	}
	$c$ es un cuadrado y $Q(x,y)=0$ tiene soluciones no triviales. En
	consecuencia, \emph{$Q$ es degenerada sobre $\algclos F$}.
\end{ejemCuadPar}

\begin{ejemCuadPar}\label{ejem:parias:extensiones:bis}
	En $F^4$, definimos $Q(x,y,z,w)=x\,y+z^2+c\,w^2$, donde
	$c\not\in\cuadrados F$. Esta forma no es degenerada sobre $F$, pero,
	en una clausura algebraica, podemos escribir $Q$ como un polinomio en
	menos de cuatro variables, pues
	$z^2+c\,w^2=(z+\sqrt c\,w)^2$.
\end{ejemCuadPar}

\begin{obsCuadPar}\label{obs:parias:extensiones}
	Si $F$ es un cuerpo de caracter\'{\i}stica impar, una forma
	$Q:\,V\rightarrow F$ es no degenerada, si y s\'olo si
	(\teoname~\ref{teo:impar:nodegeneradas}) su discriminante es
	distinto de cero, es decir, si su matriz asociada (en cualquier base)
	es invertible. Esta propiedad no se ve afectada por extensiones de
	cuerpos. Es decir, si $Q$ es no degenerada y $K/F$ es una extensi\'on
	de cuerpos, la forma cuadr\'atica $Q_K:\,K\tensor[F] V\rightarrow K$
	definida por%
	\footnote{
		Extender de manera que se preserve la homogeneidad, o bien, la
		bilinealidad.
	}
	\begin{equation}
		\label{eq:impar:extensiones}
		Q_K(c\tensor v)\,=\,c^2\,Q(v)
	\end{equation}
	%
	es no degenerada, pues su discriminante es distinto de cero. En
	caracter\'{\i}stica $2$, este hecho es falso.
\end{obsCuadPar}

\subsection{Clasificaci\'on en caracter\'{\i}stica $2$}%
	\label{subsec:cuadraticas:parias:clasificacion}

\begin{teoCuadPar}\label{teo:parias:universal}%
	\footnote{
		C.f. el \teoname~\ref{teo:impar:universal}.
	}
	Sea $F$ un cuerpo de caracter\'{\i}stica $2$. Si $Q:\,V\rightarrow F$
	es una forma cuadr\'atica no degenerada que representa $0$ de manera no
	trivial, entonces $Q$ es universal.
\end{teoCuadPar}

\begin{proof}
	Empezamos con un vector isotr\'opico $v$. Como $Q$ no es
	id\'enticamente cero (es no degenerada) y $Q(c\,v)=c^2\,Q(v)=0$, la
	dimensi\'on del espacio debe ser, al menos, $2$. M\'as aun, como $Q$ es
	no degenerada, $v\not\in V^\perp$ y existe $w\in V$ tal que
	$B(v,w)\neq 0$. El argumento es, ahora, id\'entico al del \teoname~%
	\ref{teo:impar:universal}, con la diferencia del factor $2$.
\end{proof}

\begin{teoCuadPar}\label{teo:parias:isotropico}%
	\footnote{
		C.f. el \teoname~\ref{teo:impar:isotropico}.
	}
	Sea $F$ un cuerpo de caracter\'{\i}stica $2$ y sea $Q:\,V\rightarrow F$
	una forma cuadr\'atica no degenerada. Si $Q$ admite un vector
	isotr\'opico $e$, entonces existe un segundo vector isotr\'opico $f$
	tal que $B(e,f)=1$ y $B$ es no degenerada en el plano $\generado{e,f}$.
\end{teoCuadPar}

\begin{proof}
	Como en la demostraci\'on del \teoname~\ref{teo:parias:universal},
	dado que $Q$ es no degenerada y $Q(e)=0$, se deduce que
	$e\not\in V^\perp$ y existe $w\in V$ tal que $B(e,w)\neq 0$.
	Reescalando, podemos suponer que $B(e,w)=1$. Si $c=Q(w)$, entonces
	elegimos $f:=c\,e+w$.
	% Se cumple que $B(e,f)=1$ y que $Q(f)=0$.
\end{proof}

La condici\'on $Q(v)=0$ que caracteriza vectores isotr\'opicos es m\'as
restrictiva que la condici\'on $B(v,v)=0$ en la construcci\'on de una base
simpl\'ectica para $B$.

\begin{lemaCuadPar}\label{lema:parias:finitos:representa}%
	\footnote{
		C.f. el \teoname~\ref{teo:impar:finitos:representa}.
	}
	Sea $\bb F$ un cuerpo finito de caracter\'{\i}stica $2$ y sea $(V,Q)$
	un espacio cuadr\'atico.%
	\footnote{
		No necesariamente no degenerado.
	}
	Si $\dim\,V\geq 3$, entonces $Q$ representa $0$ de manera no trivial.
\end{lemaCuadPar}

\begin{proof}
	Supongamos que $Q(v)\neq 0$. Como $B(v,-):\,V\rightarrow \bb F$ es
	lineal, $\dim\,v^\perp\geq n-1\geq 2$. En particular, existe
	$w\in v^\perp\setmin\generado v$. Como $Q(v)\neq 0$, existe $a\in\bb F$
	tal que $Q(w)=a\,Q(v)$. Si $a=b^2$, $Q(w)=Q(b\,v)$ pero $w\neq b\,v$.
	Ahora, $w\perp v$ implica
	\begin{displaymath}
		Q(w+b\,v)\,=\,Q(w)\,+\,Q(b\,v)\,=\,0
	\end{displaymath}
	%
	y el vector no nulo $w+b\,v$ es isotr\'opico.%
	\footnote{
		La demostraci\'on usa que $\bb F$ es perfecto, no que es
		finito.
	}
\end{proof}

\begin{lemaCuadPar}\label{lema:parias:finitos:universal}%
	\footnote{
		C.f. el \coroname~\ref{coro:impar:finitos:universal}.
	}
	Sea $\bb F$ un cuerpo finito de caracter\'{\i}stica $2$ y sea $(V,Q)$
	un espacio cuadr\'atico.%
	\footnote{
		No necesariamente no degenerado.
	}
	Si $Q$ no es id\'enticamente $0$, entonces es universal.
\end{lemaCuadPar}

\begin{proof}
	Si $Q(v_0)\neq 0$, como todo elemento de $\bb F$ es un cuadrado,
	$\big\{Q(c\,v_0)\,:\,c\in\bb F\big\}=\bb F$.
\end{proof}

\begin{lemaCuadPar}\label{lema:parias:finitos:radical}
	Si $Q:\,V\rightarrow\bb F$ es no degenerada, entonces
	$\dim\,V^\perp\leq 1$. De hecho, $\dim\,V^\perp=0$, si $\dim\,V$ es
	par, y $\dim\,V^\perp=1$, si $\dim\,V$ es impar.
\end{lemaCuadPar}

\begin{proof}
	La forma bilineal alternada $B=B_Q$ induce una forma bilineal alternada
	no degenerada en $V/V^\perp$. Entonces $\dim\,V/V^\perp$ es par.%
	\footnote{
		Esto es cierto en general, sobre cualquier cuerpo de cualquier
		caracter\'{\i}stica.
	}
	Si $V^\perp\neq 0$, elegimos $v_0\neq 0$ en $V^\perp$. Como $Q$ es
	no degenerada, $Q(v_0)\neq 0$. Si $v\in V$, $Q(v)=a\,Q(v_0)=Q(b\,v_0)$,
	para cierto $b\in\bb F$. Si, m\'as aun, $v\in V^\perp$, entonces
	\begin{displaymath}
		Q(v+b\,v_0)\,=\,Q(v)\,+\,Q(b\,v_0)\,=\,0
		\text{ ,}
	\end{displaymath}
	%
	por perpendicularidad. Como $Q$ es no degenerada, $v+b\,v_0=0$. Como
	$v\in V^\perp$ era arbitrario, se deduce que $V^\perp=\generado{v_0}$.
	Es decir, en general, $\dim\,V^\perp\leq 1$. El resultado es
	consecuencia de esto y de la observaci\'on general anterior.
\end{proof}

\begin{obsCuadPar}\label{obs:parias:nodegeneradas:par}
	Una consecuencia del \lemaname~\ref{lema:parias:finitos:radical} es que, en
	caracter\'{\i}stica $2$, cuando $\dim\,V$ es par, $Q$ es no degenerada,
	si y s\'olo si $B$ es no degenerada ($V^\perp=0$).
\end{obsCuadPar}

\begin{obsCuadPar}\label{obs:parias:perfectos}
	Los Lemas~\ref{lema:parias:finitos:representa},
	\ref{lema:parias:finitos:universal} y
	\ref{lema:parias:finitos:radical}, si bien fueron enunciados para
	formas cuadr\'aticas defindas sobre un cuerpo finito de
	caracter\'{\i}stica $2$, son ciertos sobre un cuerpo \emph{perfecto} de
	caracter\'{\i}stica $2$.
\end{obsCuadPar}

Sobre un cuerpo $F$ de caracter\'{\i}stica impar, el discriminante y el grupo
$\modcuadrados{F}$ juegan un papel importante en la clasificaci\'on de formas
cuadr\'aticas. En caracter\'{\i}stica $2$, especialmente cuando el cuerpo $F$
es perfecto, como es el caso de los cuerpos finitos, el lugar del discriminante
lo toma otra funci\'on; en un cuerpo finito de caracter\'{\i}stica $2$, todo
elemento es un cuadrado.

\begin{defCuadPar}\label{def:parias:valoresespeciales}
	Si $F$ es un cuerpo de caracter\'{\i}stica $2$, la \emph{funci\'on %
	$\valoresfuncion$} es la funci\'on $\valoresfuncion:\,F\rightarrow F$
	dada por $\valores a=a^2+a$.
\end{defCuadPar}

\begin{obsCuadPar}\label{obs:parias:valoresespeciales}
	Si $F$ es un cuerpo finito de caracter\'{\i}stica $2$, la funci\'on
	$\valoresfuncion$ es aditiva y su n\'ucleo es $\{0,1\}$. Si, adem\'as,
	$F=\bb F$ es un cuerpo finito, el cociente $\modvalores{\bb F}$ tiene
	orden $2$. En particular, la suma de dos elementos que no est\'an en la
	imagen de $\valoresfuncion$ pertenece a la imagen, mientras que la suma
	de un elemento que est\'a en la imagen con otro que no est\'a, no
	pertenece a la imagen.
\end{obsCuadPar}

Sea $\bb F$ un cuerpo finito de caracter\'{\i}stica $2$. Fijamos un elemento
$d\in\novalores{\bb F}$ y un espacio cuadr\'atico $(V,Q)$ no degenerado de
dimensi\'on $n$ sobre $\bb F$.

El siguiente resultado juega el rol del \lemaname~%
\ref{lema:impar:finitos:bajas}.

\begin{teoCuadPar}\label{teo:parias:finitos:bajas}
	La forma cuadr\'atica $Q$ es equivalente, sobre $\bb F$,
	\begin{enumerate}[(1)]
		\item\label{item:parias:finitos:bajas:i}
			a $x^2$, si $\dim\,V=1$,
		\item\label{item:parias:finitos:bajas:ii}
			a $x\,y$ o a $x^2+x\,y+d\,y^2$, si $\dim\,V=2$, y
		\item\label{item:parias:finitos:bajas:iii}
			a $x\,y+z^2$, si $\dim\,V=3$.
	\end{enumerate}
	%
	Las dos formas en \eqref{item:parias:finitos:bajas:ii} no son
	equivalentes.
\end{teoCuadPar}

\begin{proof}
	Si $\dim\,V=1$, $Q$ es de la forma $a\,x^2$ en alguna base. Usando que
	$\bb F$ es perfecto, se deduce que $a\,x^2$ es equivalente a $x^2$.

	Las formas $x\,y$ y $x^2+x\,y+d\,y^2$ en $\bb F^2$ no son equivalentes:
	la primera admite un vector isotr\'opico, pero la segunda no.%
	\footnote{
		Si $(x_0,y_0)$ fuese isotr\'opico, $y_0\neq 0$ y
		$d=(x_0/y_0)^2+x_0/y_0=\valores{x_0/y_0}$.
	}
	Para probar que toda forma no degenerada en un espacio de dimensi\'on
	$2$ es equivalente a una de estas formas, por el \lemaname~%
	\ref{lema:parias:finitos:universal}, $Q(v)=1$ para cierto $v\in V$.%
	\footnote{
		C.f. el argumento del \teoname~\ref{teo:impar:finitos}.
	}
	Por el \lemaname~\ref{lema:parias:finitos:radical}, $V^\perp=0$ y $B=B_Q$ es
	no degenerada. Elegimos $w\in V$ tal que $B(v,w)=1$. Entonces,
	$\{v,w\}$ es una base de $V$ y
	\begin{displaymath}
		\begin{aligned}
			Q(x\,v+y\,w) & \,=\,
				x^2\,Q(v)\,+\,y^2\,Q(w)\,+\,x\,y\,B(v,w) \\
			& x^2\,+\,x\,y\,+\,Q(w)\,y^2
			\text{ .}
		\end{aligned}
		%
	\end{displaymath}
	%
	Si $Q(w)=\valores a$ para cierto $a\in\bb F$, entonces,
	\begin{displaymath}
		x^2\,+\,x\,y\,+\,Q(w)\,y^2\,=\,(x+a\,y)\,(x+(a+1)\,y)
		\text{ .}
	\end{displaymath}
	%
	Si $Q(w)\not\in\valores{\bb F}$, entonces $Q(w)+d=a^2+a$ para alg\'un
	$a\in\bb F$ y
	\begin{displaymath}
		x^2\,+\,x\,y\,+\,Q(w)\,y^2\,=\,
			(x+a\,y)^2\,+\,(x+a\,y)\,y\,+\,d\,y^2
		\text{ .}
	\end{displaymath}
	%

	Si $\dim\,V=3$, por el \lemaname~\ref{lema:parias:finitos:representa},
	existe $e\neq 0$ tal que $Q(e)=0$ y, por el \teoname~%
	\ref{teo:parias:isotropico}, existe $f$ tal que $Q(f)=0$ y
	$B(e,f)=1$. Como $B$ es no degenerada en el plano $\generado{e,f}$,
	\begin{displaymath}
		V^\perp\,\cap\,\generado{e,f}\,=\,0
		\text{ .}
	\end{displaymath}
	%
	En particular, eligiendo cualquier vector $g\in V^\perp\setmin\{0\}$,
	$\{e,f,g\}$ constituye una base de $V$. Ahora, como $Q$ es no
	degenerada, $Q(g)\neq 0$. Como $\bb F$ es perfecto y
	$Q(b\,g)=b^2\,Q(g)$, podemos asumir que $Q(g)=1$. Entonces,
	\begin{displaymath}
		Q(x\,e+y\,f+z\,g)\,=\,Q(x\,e+y\,f)\,+\,z^2\,Q(g)
			\,=\,x\,y\,+\,z^2
		\text{ .}
	\end{displaymath}
	%
\end{proof}

\begin{obsCuadPar}\label{obs:parias:finitos:bajas}
	Si $U=\generado{e,f}$ es un plano como en el \teoname~%
	\ref{teo:parias:isotropico} con $F=\bb F$ finito, entonces, aplicando
	el \teoname~\ref{teo:parias:finitos:bajas}, dado que $U$ contiene
	vectores isotr\'opicos, la forma $Q|_U$ debe ser equivalente a $x\,y$.
\end{obsCuadPar}

\begin{teoCuadPar}\label{teo:parias:finitos:geometria}
	Si $n\geq 2$, entonces $Q$ es equivalente a
	\begin{displaymath}
		x_1\,x_2\,+\,\cdots\,+\,x_{n-3}\,x_{n-2}\,+\,
			\begin{cases}
				x_{n-1}\,x_n & \text{o a} \\
				x_{n-1}^2\,+\,x_{n-1}\,x_n\,+\,d\,x_n^2 &
					\text{ ,}
			\end{cases}
	\end{displaymath}
	%
	si $n$ es par, y a
	\begin{displaymath}
		x_1\,x_2\,+\,\cdots\,+\,x_{n-2}\,x_{n-1}\,+\,x_n^2
		\text{ ,}
	\end{displaymath}
	%
	si $n$ es impar.
\end{teoCuadPar}

\begin{proof}
	Podemos asumir $n=\dim\,V\geq 4$. Entonces, por el \teoname~%
	\ref{lema:parias:finitos:representa}, existe $v\neq 0$ tal que
	$Q(v)=0$. Por el \teoname~\ref{teo:parias:isotropico}, existe $w$ tal
	que $B(v,w)=1$, $Q(w)=0$ y $U:=\generado{v,w}$ es no degenerado. Por
	la \obsname~\ref{obs:parias:finitos:bajas}, $Q|_U$ es equivalente
	a $x\,y$.

	Si $n$ es par, por el \lemaname~\ref{lema:parias:finitos:radical},
	$V^\perp=0$ y $B$ es no degenerada. Entonces, por el \teoname~%
	\ref{teo:nodegeneradas:perpendicular}, $V=U\oplus U^\perp$ y
	$U^\perp$ es no degenerado. Inducci\'on.

	Si $n$ es impar, $V^\perp$ es unidimensional y $n\geq 5$. Como
	$Q|_{V^\perp}$ no es id\'enticamente nula, es no degenerada y,
	aplicando el \teoname~\ref{teo:parias:finitos:bajas}, est\'a
	representada por $x^2$, en alguna base (eligiendo un generador).
	Si elegimos cualquier complemento (lineal), $V=V^\perp\oplus W$, vale
	que $\dim\,W$ es par y $B|_W$ es no degenerada (pues
	$W\cap V^\perp=0$). Por la \obsname~%
	\ref{obs:parias:nodegeneradas:par}, $Q|_W$ es no degenerada.
	Ahora, aplicamos el caso de dimensi\'on par a $Q|_W$ y descomponemos
	$Q=Q|_W+Q|_{V^\perp}$. Lo que resta notar es que la forma
	\begin{displaymath}
		x^2\,+\,x\,y\,+\,d\,y^2\,+\,z^2
		\text{ ,}
	\end{displaymath}
	%
	en $\bb F^3$ es no degenerada y, por el \teoname~%
	\ref{teo:parias:finitos:bajas}, equivalente a $x\,y+z^2$.
\end{proof}

El enunciado del \teoname~\ref{teo:parias:finitos:geometria} no garantiza que
las formas cuadr\'aticas del caso $n$ par no sean no equivalentes. A
continuaci\'on, demostramos que esto es as\'{\i}. Recordemos que en el caso de
caracter\'{\i}stica impar, \teoname~\ref{teo:impar:finitos:geometria},
tampoco es inmediato que las distintas formas que aparecen en el enunciado no
son equivalente. El argumento involucraba el discriminante de cada una de las
formas \emph{can\'onicas} y sus clases m\'odulo cuadrados. Pero sus im\'agenes
tambi\'en pod\'{\i}an ser \'utiles.

\begin{defCuadPar}\label{def:parias:ceros}
	Dada una forma cuadr\'atica $Q:\,V\rightarrow F$, el \emph{conjunto %
	de ceros} es el conjunto $\big\{v\in V\,:\,Q(v)=0\big\}$. Cuando el
	cuerpo de base $F$ es un cuerpo finito, denotamos el cardinal de este
	conjunto por $\nceros Q$.
\end{defCuadPar}

El conjunto de ceros est\'a compuesto por los vectores isotr\'opicos y el
vector cero. Es un invariante para la relaci\'on de equivalencia de formas
cuadr\'aticas.

\begin{ejemCuadPar}\label{ejem:parias:ceros}
	Si $\bb F$ es un cuerpo finito de caracter\'{\i}stica $2$,
	$V=\bb F^2$ y $Q:\,V\rightarrow\bb F$ es una forma no degenerada,
	entonces $Q$ es equivalente a $x\,y$ o a $x^2+x\,y+d\,y^2$. Como vimos,
	estas formas no son equivalentes. Se puede comprobar que:
	\begin{itemize}
		\item $\nceros{x\,y}=2q-1$ y que
		\item $\nceros{x^2+x\,y+d\,y^2}=1$,
	\end{itemize}
	%
	donde $q=|\bb F|$.
\end{ejemCuadPar}

\begin{obsCuadPar}\label{obs:parias:ceros}
	Si $a\in\bb F^\times$, entonces $a=b^2$ para cierto $b\in\bb F^\times$.
	Los subconjuntos $Q^{-1}(a)$ y $Q^{-1}(1)$ de $V$ est\'an en
	biyecci\'on v\'{\i}a $v\mapsto(1/b)\,v$. En particular,
	\begin{equation}
		\label{eq:parias:finitos:ceros:representa}
		|Q^{-1}(a)|\,=\,|Q^{-1}(1)|
		\text{ ,}
	\end{equation}
	%
	para todo $a\neq 0$.
\end{obsCuadPar}

\begin{ejerCuadPar}\label{ejer:parias:ceros}
	Calcular $|Q^{-1}(a)|$ para cada $a\neq 0$ para la forma $x\,y$.%
	\hint{
		Probar que $|Q^{-1}(1)|\geq 1$ y apelar a la \obsname~%
		\ref{obs:parias:ceros} y al \ejemname~\ref{ejem:parias:ceros}
	}
	Hacer lo mismo para la forma $x^2+x\,y+d\,y^2$.
\end{ejerCuadPar}

\begin{lemaCuadPar}\label{lema:parias:finitos:suma}
	Sea $Q:\,V\rightarrow\bb F$ una forma cuadr\'atica y sea $h$ la forma
	$x\,y$ en $\bb F^2$. Entonces,
	\begin{displaymath}
		\nceros{h\perp Q}\,=\,q\,\nceros Q\,+\,(q-1)\,|V|
		\text{ ,}
	\end{displaymath}
	%
	donde $q=|\bb F|$.
\end{lemaCuadPar}

\begin{proof}
	En primer lugar, $(h\perp Q)(u,v)=0$, si y s\'olo si $Q(v)=h(u)$.
	Entonces,
	\begin{displaymath}
		\nceros{h\perp Q}\,=\,\sum_u\,|Q^{-1}(h(u))|
		\text{ .}
	\end{displaymath}
	%
	Separar en casos $h(u)=0$ y $h(u)\neq 0$ y usar la \obsname~%
	\ref{obs:parias:ceros} y el \ejemname~\ref{ejem:parias:ceros}.
\end{proof}

\begin{teoCuadPar}\label{teo:parias:finitos:ceros}
	Sea $n=2m$, $m\geq 1$. Entonces,
	\begin{displaymath}
		\begin{aligned}
			\nceros{x_1\,x_2+\,\cdots\,+x_{n-3}\,x_{n-2} %
				+x_{n-1}\,x_n}
				& \,=\,q^{2m-1}+q^m-q^{m-1}
			\quad\text{y} \\
			\nceros{x_1\,x_2+\,\cdots\,x_{n-3}\,x_{n-2} %
				+x_{n-1}^2+x_{n-1}\,x_n+d\,x_n^2}
				& \,=\,q^{2m-1}-q^m+q^{m-1}
			\text{ .}
		\end{aligned}
		%
	\end{displaymath}
	%
	En particular, las dos formas de dimensiones pares del \teoname~%
	\ref{teo:parias:finitos:geometria} no son equivalentes.
\end{teoCuadPar}

En la clasificaci\'on de formas cuadr\'aticas sobre cuerpos finitos de
caracter\'{\i}stica $2$, la dificultad al final result\'o ser distinguir las
clases de formas no degeneradas en dimensiones pares. Sobre un cuerpo $F$ de
caracter\'{\i}stica $2$ y perfecto, hay una \'unica clase de equivalencia de
formas cuadr\'aticas no degeneradas por cada dimensi\'on impar y
$|\modvalores F|$ tantas clases en cada dimensi\'on par. Para poder
distinguirlas, introducimos el \emph{invariante de Arf} de una forma.

Sea $F$ un cuerpo de caracter\'{\i}stica $2$ (no necesariamente finito, no
necesariamente perfecto) y sea $Q:\,V\rightarrow F$ una forma cuadr\'atica tal
que $B=B_Q$ sea no degenerada.%
\footnote{
	En dimensiones pares, esto equivale a que $Q$ sea no degenerada, y es
	en esos casos en los que necesitaremos distinguir clases.
}

\begin{defCuadPar}\label{def:parias:arf}
	El \emph{invariante de Arf} de $Q$ es la clase
	\begin{displaymath}
		\sum_{i=1}^m\,a^i\,b^i\tmodulo[\valores F]
	\end{displaymath}
	%
	en $\modvalores F$, donde $a^i$ y $b^i$ son los coeficientes diagonales
	de la matriz de $Q$ en una base simpl\'ectica.
\end{defCuadPar}

\begin{ejerCuadPar}\label{ejer:parias:arf}
	Probar que, cambiando la base simpl\'ectica, la suma $\sum_i\,a^i\,b^i$
	se ve modificada por sumar un elemento en $\valores F$. En particular,
	el invariante de Arf de una forma cuadr\'atica cuya forma bilineal es
	no degenerada est\'a bien definido y es un invariante para la
	relaci\'on de equivalencia de formas cuadr\'aticas.
\end{ejerCuadPar}

\subsection{Extras}\label{subsec:cuadraticas:parias:extras}
En caracter\'{\i}stica $2$ las formas $x\,y$ y $x^2-y^2$ no son equivalentes:
una es degenerada y la otra no.
\begin{defCuadPar}\label{def:parias:hiperbolico}
	Un \emph{plano hiperb\'olico} es un espcio cuadr\'atico de dimensi\'on
	$2$ equivalente a $x\,y$.
\end{defCuadPar}

\begin{teoCuadPar}\label{teo:parias:hiperbolico}
	Sea $(V,Q)$ un espacio cuadr\'atico sobre un cuerpo de
	caracter\'{\i}stica $2$. Las siguientes afirmaciones son equivalentes:
	\begin{enumerate}[(i)]
		\item\label{item:parias:hiperbolico}
			$(V,Q)$ es un plano hiperb\'olico;
		\item\label{item:parias:nodegenerado-e-isotropico}
			$Q$ es no degenerada y admite un vector isotr\'opico.
	\end{enumerate}
	%
\end{teoCuadPar}

\begin{ejerCuadPar}\label{ejer:parias:norma}
	Sea $K/F$ una extensi\'on cuadr\'atica. Probar que la norma es no
	degenerada, si y s\'olo si $K/F$ es una extensi\'on separable. Si
	$F=\bb F$ es un cuerpo finito, $\Norma[K/F]$ es igual a
	$x^2+c\,x\,y+y^2$ en alguna base; el polinomio $T^2+c\,T+1$ es
	irrecucible.
\end{ejerCuadPar}

\begin{ejerCuadPar}\label{ejer:parias:binarias}
	Si $a\,x^2+x\,y+b\,y^2$ y $a'\,x^2+x\,y+b'\,y^2$ son equivalentes,
	mostrar (expl\'{\i}citamente) que $ab\equiv a'b'\tmodulo[\valores F]$.
	No asumir que el cuerpo $F$ es perfecto ?`Vale la vuelta?
\end{ejerCuadPar}

\begin{ejerCuadPar}\label{ejer:parias:democracia}
	Sea $n\geq 2$ un entero par y sea $\bb F$ un cuerpo finito de
	caracter\'{\i}stica $2$. Sea $Q$ una forma cuadr\'atica en un espacio
	de $V$ dimensi\'on $n$ sobre $\bb F$. Definimos:
	\begin{displaymath}
		n_+(Q) \,=\,\big|\big\{v\in V\,:\,Q(v)\in\valores F\big\}\big|
			\quad\text{y}\quad
		n_-(Q) \,=\,\big|\big\{v\in V\,:\,
				Q(v)\not\in\valores F\big\}\big|
		\text{ .}
	\end{displaymath}
	%
	Supongamos que $Q$ es no degenerada. Probar que los valores de $n_+(Q)$
	y de $n_-(Q)$ son iguales a $q^n\,(q^n+1)/2$ o a $q^n\,(q^n-1)/2$;
	probar que $n_+(Q)>n_-(Q)$, si el invariante de Arf de $Q$ pertence a
	$\valores{\bb F}$ y que $n_-(Q)>n_+(Q)$, si no.%
	\footnote{
		Sobre un cuerpo finito, hay s\'olo dos clases en el cociente
		$\modvalores{\bb F}$.
	}
	Es decir, el invariante de Arf de $Q$ es la clase en el cociente
	$\modvalores{\bb F}$ que contiene una mayor\'{\i}a de valores de
	$Q(v)$.
\end{ejerCuadPar}

