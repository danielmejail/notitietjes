\theoremstyle{plain}
\newtheorem{teoIntroOrto}{Teorema}[section]
\newtheorem{lemaIntroOrto}[teoIntroOrto]{Lema}

\theoremstyle{definition}
\newtheorem{defIntroOrto}[teoIntroOrto]{Definici\'on}
\newtheorem{ejemIntroOrto}[teoIntroOrto]{Ejemplo}

%-------------

Fijamos un espacio bilineal \emph{sim\'etrico} $(V,B)$ sobre un cuerpo $F$.

\begin{defIntroOrto}\label{def:ortogonales:base}
	Decimos que un subconjunto $S\subset V$ es \emph{ortogonal} con
	respecto a $B$, si $v\perp w$ para todo $v,w\in S$, $v\neq w$. En
	particular, una \emph{base ortogonal} de $V$ es una base
	$\{\lista* e{n}\}$ de $V$ tal que $e^i\perp e^j$ si $i\neq j$.
\end{defIntroOrto}

\begin{ejemIntroOrto}\label{ejem:ortogonales:escalar}
	La base can\'onica de $F^n$ es una base ortogonal para la forma
	bilineal sim\'etrica dada por el producto escalar. En esta base, la
	matriz de la forma bilineal es la matriz identidad, que es diagonal.
\end{ejemIntroOrto}

\begin{ejemIntroOrto}\label{ejem:ortogonales:alternada}
	La forma $B(v,w)=v\cdot\sbmatrix{ & 1 \\ 1 & }\,w$ en $\bb R^2$ es
	sim\'etrica. La base $\{\sbmatrix{ 1 \\ 0 }, \sbmatrix{ 0 \\ 1 }\}$ no
	es ortogonal. De hecho, no hay bases ortogonales para esta forma
	bilineal que contengan al vector $\sbmatrix{ 1 \\ 0 }$. La base
	$\{\sbmatrix{ 1 \\ 1 },\sbmatrix{ 1 \\ -1 }\}$ es ortogonal con
	respecto a $B$.

	Estas mismas observaciones son ciertas, si se reemplaza $\bb R$ por un
	cuerpo de caracter\'{\i}stica distinta de $2$. Si el cuerpo de base es
	de caracter\'{\i}stica $2$, no hay base ortogonal para la forma $B$. En
	este \'ultimo caso, $B$ no s\'olo es sim\'etrica, sino tambi\'en es
	alternada.
\end{ejemIntroOrto}

\begin{lemaIntroOrto}\label{lema:ortogonales:diagonal}
	Si la caracter\'{\i}stica de $F$ es distinta de $2$ y $B$ no es
	id\'enticamente cero, entonces existe \emph{alg\'un} $v\in V$ tal que
	$B(v,v)\neq 0$.
\end{lemaIntroOrto}

\begin{proof}
	Ejercicio.
\end{proof}

\begin{lemaIntroOrto}\label{lema:ortogonales:complemento}
	Si $v\in V$ es tal que $B(v,v)\neq 0$, entonces
	$V=\generado v\oplus v^\perp$ y la suma es ortogonal. Si $V$ es no
	degenerado, entonces $v^\perp$ es no degenerado.
\end{lemaIntroOrto}

\begin{proof}
	Ejercicio.
\end{proof}

Notar que, si $B(v,v)\neq 0$ y $v_1\in V$, entonces $B(v_1,v)=c\,B(v,v)$ para
cierta constante $c\in F$ y $v_1-c\,v\in v^\perp$. El Lema~%
\ref{lema:ortogonales:complemento} es v\'alido en cualquier
caracter\'{\i}stica.

\begin{teoIntroOrto}\label{teo:ortogonales:base}
	Si la caracter\'{\i}stica de $F$ es distinta de $2$, entonces existe
	una base ortogonal para $(V,B)$.
\end{teoIntroOrto}

\begin{proof}
	Si $a=B(v,v)\neq 0$, entonces podemos elegir $v$ como primer elemento
	de la base y buscar una base ortogonal para el complemento
	$v^\perp$. Con respecto a esta base,
	\begin{displaymath}
		\repr B \,=\,
			\begin{bmatrix}
				a & \\
				& \repr{B|_{v^\perp}}
			\end{bmatrix}
		\text{ .}
	\end{displaymath}
	%
	El valor $a$ aparece como uno (el primero) de los coeficientes y el
	vector $v$ hallado es el primer elemento de la base.
\end{proof}

El Teorema~\ref{teo:ortogonales:base} es v\'alido incluso si el espacio no es
no degenerado. Una base ortogonal es, esencialmente, una descomposici\'on de
$V$ en suma ortogonal de subespacios de dimensi\'on $1$:
\begin{displaymath}
	V\,=\,W_1\,\oplus\,\cdots\,\oplus\,W_n
	\text{ ,}
\end{displaymath}
donde $W_i\perp W_j$, si $i\neq j$. La matriz asociada a una forma bilineal en
una base ortogonal es diagonal (en particular, es sim\'etrica y $B$ es,
\emph{a fortiori}, sim\'etrica). Si $\{\lista* e{n}\}$ es una base ortogonal,
entonces $V$ es no degenerado, si y s\'olo si $e^i\not\perp e^i$ para todo $i$.
