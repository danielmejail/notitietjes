\theoremstyle{plain}
\newtheorem{teoIntroNodeg}{Teorema}[section]
\newtheorem{lemaIntroNodeg}[teoIntroNodeg]{Lema}

\theoremstyle{definition}
\newtheorem{defIntroNodeg}[teoIntroNodeg]{Definici\'on}
\newtheorem{obsIntroNodeg}[teoIntroNodeg]{Observaci\'on}
\newtheorem{ejemIntroNodeg}[teoIntroNodeg]{Ejemplo}
\newtheorem{ejerIntroNodeg}[teoIntroNodeg]{Ejercicio}

%-------------

\begin{teoIntroNodeg}\label{teo:nodegeneradas:nodegeneradas}
	Sea $(V,B)$ un espacio bilineal. Las siguientes condiciones son
	equivalentes:
	\begin{enumerate}[(i)]
		\item\label{item:nodegeneradas:matriz}
			con respecto a alguna base $\{\lista* e{n}\}$ de $V$,
			la matriz asociada $B(e^i,e^j)$ es invertible;
		\item\label{item:nodegeneradas:inyectiva}
			si $v\neq 0$, entonces $B(v_1,v)\neq 0$ para alg\'un
			$v_1\in V$;
		\item\label{item:nodegeneradas:sobre}
			todo elemento de $\dual V$ es de la forma $B(-,v)$
			para alg\'un $v\in V$;
		\item\label{item:nodegeneradas:biyectiva}
			todo elemeneto de $\dual V$ es de la forma $B(-,v)$
			para un \'unico $v\in V$.
	\end{enumerate}
	%
	En tal caso, \emph{toda} representaci\'on matricial de $B$ es
	invertible.
\end{teoIntroNodeg}

\begin{proof}
	% Ejercicio.
	Las condiciones \eqref{item:nodegeneradas:inyectiva},
	\eqref{item:nodegeneradas:sobre} y \eqref{item:nodegeneradas:biyectiva}
	hacen referencia a la inyectividad, sobreyectividad y biyectividad,
	respectivamente, de la transformaci\'on lineal
	$R_B:\,V\rightarrow\dual V$. Como $\dim\,V<\infty$, \'estas son
	equivalentes. La condici\'on \eqref{item:nodegeneradas:matriz}
	significa que la matriz asociada a $B$ en alguna base es invertible.
	Pero esto equivale a que $R_B$ sea un isomorfismo.
\end{proof}

\begin{defIntroNodeg}\label{def:nodegeneradas:nodegeneradas}
	Una forma bilineal $B$ en un espacio $V$ se dice \emph{no degenerada},
	si cumple cualquiera de las condiciones equivalentes enunciadas en el
	Teorema~\ref{teo:nodegeneradas:nodegeneradas}. En caso contrario, se
	dice que $B$ es \emph{degenerada}.
\end{defIntroNodeg}

\begin{obsIntroNodeg}\label{obs:nodegeneradas:nodegeneradas}
	El Teorema~\ref{teo:nodegeneradas:nodegeneradas} caracteriza las
	formas no degeneradas \emph{a derecha}, aquellas tales que $R_B$ es
	inyectiva (sobre o, equivalentemente, biyectiva). Ahora, si $M$ es una
	matriz, entonces $M$ es invertible, si y s\'olo si $\trnsp M$ lo es. Si
	$M$ es la matriz asociada a $B$ en alguna base, $\cal B$, entonces $M$
	es la matriz de la transformaci\'on lineal $R_B:\,V\rightarrow\dual V$,
	con respecto a dicha base en $V$ y su base dual en $\dual V$,
	$\dual{\cal B}$. Por otro lado, si $J:\,V\rightarrow\ddual V$ es el
	isomorfismo can\'onico
	\begin{equation}
		\label{eq:nodegeneradas:dobledual}
		J(v)(\varphi)\,=\,\varphi(v)
		\text{ ,}
	\end{equation}
	%
	la matriz $\trnsp M$ es la matriz de la transformaci\'on transpuesta
	$\dual{R_B}:\,\ddual V\rightarrow\dual V$, con respecto a
	$\dual{\cal B}$ y a la base $J(\cal B)$. Pero, v\'{\i}a la
	identificaci\'on \eqref{eq:nodegeneradas:dobledual}, $\dual{R_B}=L_B$.
	La codici\'on \eqref{item:nodegeneradas:matriz} significa que $R_B$ es
	biyectiva. Pero $R_B$ es biyectiva, si y s\'olo si $\dual{R_B}=L_B$ lo
	es. En definitiva, $B$ es no degenerada a derecha, si y s\'olo si lo es
	a izquierda. El Teorema~\ref{teo:nodegeneradas:nodegeneradas} garantiza
	que $V$ parametriza $\dual V$ v\'{\i}a $R_B$, si $B$ es no degenerada
	a derecha; todo elemento de $\dual V$ es de la forma
	$B(-,v)$. Pero, entonces, $L_B$ tambi\'en es un isomorfismo y todo
	elemento de $\dual V$ se puede escribir de la forma $B(v,-)$ para
	alg\'un $v\in V$.
\end{obsIntroNodeg}

Si $V$ es de dimensi\'on finita, $\ddual V=V$ (naturalmente isomorfos).
Adem\'as, por un argumento de dimensi\'on, $V\simeq\dual V$, es decir,
\emph{existe} un isomorfismo. Pero dicho isomorfismo no es can\'onico. Una
forma bilineal es, esencialmente, una t.l. $V\rightarrow\dual V$. Una forma
bilineal no degenerada es una elecci\'on de isomorfismo entre estos espacios.

Si $V=\{0\}$, entonces $\dual V\simeq V$, de manera \'unica. Paralelamente, hay
una \'unica forma bilineal en $V$. Por esta raz\'on, se considera que el
espacio cero con su \'unica forma bilineal es un espacio bilineal no
degenerado. Sin embargo, dicha forma no admite una matriz que la represente,
pues \'unica base en $\{0\}$ es vac\'{\i}a.

\begin{ejemIntroNodeg}\label{ejem:nodegeneradas:pseudoeuclideos}
	El producto escalar en $\bb R^n$, al igual que las formas
	$\langle\cdot,\cdot\rangle_{p,q}$ definidas en el Ejemplo~%
	\ref{ejem:matrices:pseudoeuclideos}, son no degeneradas; est\'an
	representadas por matrices invertibles en la base can\'onica.%
	\footnote{
		Aunque podr\'{\i}a ocurrir $\langle v,v\rangle_{p,q}=0$,
		para alg\'un $v\neq 0$.
	}
	Todas ellas parametrizan el dual del espacio vectorial
	$\bb R^n$, el mismo espacio, pero de distintas maneras. Si
	\begin{math}
		M=
		\left[\begin{smallmatrix}
			I_p & 0 \\ 0 & -I_q
		\end{smallmatrix}\right]
	\end{math}, entonces, usando que $M^{-1}=\trnsp M=M$,
	\begin{displaymath}
		\begin{aligned}
			\langle v,w\rangle_{p,q} & \,=\,v\,\cdot\,(M\,w)
				\,=\,(M\,v)\,\cdot\,w \quad\text{y} \\
			v\,\cdot\,w & \,=\,\langle v,M^{-1}\,w\rangle_{p,q}
				\,=\,\langle v,M\,w\rangle_{p,q}
				\,=\,\langle M\,v,w\rangle_{p,q}
			\text{ .}
		\end{aligned}
		%
	\end{displaymath}
	%
\end{ejemIntroNodeg}

\begin{ejemIntroNodeg}\label{ejem:nodegeneradas:alternadas}
	La forma bilineal alternada del Ejemplo~%
	\ref{ejem:definiciones:alternadas} es no degenerada, pues
	$\varphi(v)=0$, para todo $v\in V$ si y s\'olo si $\varphi=0$,
	y para todo $\varphi\in\dual V$, si y s\'olo si $v=0$.
\end{ejemIntroNodeg}

\begin{ejemIntroNodeg}\label{ejem:nodegeneradas:killing}
	Si $\frak g$ es un \'algebra de Lie de dimensi\'on finita y sea
	\begin{equation}
		\label{eq:nodegeneradas:killing}
		\killing(x,y)\,=\,\Traza(\adjoint[x]\,\adjoint[y])
	\end{equation}
	%
	su \emph{forma de Killing}. Es una forma bilineal sim\'etrica. Si el
	cuerpo de base es de caracter\'{\i}stica $0$, $\frak g$ es semisimple,
	si y s\'olo si $\killing$ es no degenerada.
\end{ejemIntroNodeg}

\begin{ejemIntroNodeg}\label{ejem:nodegeneradas:subespaciodegenerado}
	El espacio bilineal $\bb R^{2,1}$, por ejemplo, es no degenerado, como
	se mencion\'o en el Ejemplo~\ref{ejem:nodegeneradas:pseudoeuclideos}.
	Sin embargo, el plano $W$ generado por $v_1=(1,0,1)$ y por
	$v_2=(0,1,0)$ es degenerado con respecto a la restricci\'on de la forma
	$\langle\cdot,\cdot\rangle_{2,1}$. Por ejemplo, $v_1\in W^\perp$.
\end{ejemIntroNodeg}

\begin{obsIntroNodeg}\label{obs:nodegeneradas:subespaciodegenerado}
	Un espacio bilineal no degenerado, puede contener subespacios tales que
	la restricci\'on de la misma forma bilineal a dicho subespacio es
	degenerada, como ocurre en el Ejemplo~%
	\ref{ejem:nodegeneradas:subespaciodegenerado}. Esto no ocurre, si el
	cuerpo de base es $\bb R$ y la forma es definida positiva. La propiedad
	de una forma bilineal real de ser (semi) definida positiva es
	hereditaria.
\end{obsIntroNodeg}

\begin{teoIntroNodeg}\label{teo:nodegeneradas:perpendicular}
	Sea $(V,B)$ un espacio bilineal sim\'etrico o alternado.%
	\footnote{
		Es decir, un espacio en donde la relaci\'on de
		perpendicularidad es sim\'etrica (c.f. el Teorema~%
		\ref{teo:definiciones:perpendicular}).
	}
	\footnote{
		Si $B$ es arbitraria, no necesariamente sim\'etrica ni
		alternada, la relaci\'on $\perp$ deja de ser sim\'etrica y
		distinguimos entre $W^\lperp$ y $W^\rperp$ en el \'{\i}tem~%
		\eqref{item:perpendicular:complemento}. Esto da lugar a
		versiones a izquierda y a derecha de
		\eqref{item:perpendicular:complemento:b} y de
		\eqref{item:perpendicular:complemento:c}. Asumiendo dimensi\'on
		finita, se puede probar que
		\begin{displaymath}
			\begin{aligned}
			\text{\eqref{item:perpendicular:complemento:a}}
				& \,\Rightarrow\,\big(
			\text{\eqref{item:perpendicular:complemento:b}}_L
				\wedge
			\text{\eqref{item:perpendicular:complemento:b}}_R
				\big) \quad\text{que} \\
			\text{\eqref{item:perpendicular:complemento:b}}_L
				& \,\Rightarrow\,
			\text{\eqref{item:perpendicular:complemento:c}}_L
				\,\Rightarrow\,
			\text{\eqref{item:perpendicular:complemento:a}}
				\quad\text{e, ir\'onicamente, %
					sim\'etricamente, que} \\
			\text{\eqref{item:perpendicular:complemento:b}}_R
				& \,\Rightarrow\,
			\text{\eqref{item:perpendicular:complemento:c}}_R
				\,\Rightarrow\,
			\text{\eqref{item:perpendicular:complemento:a}}
			\end{aligned}
			%
		\end{displaymath}
		%
		La raz\'on es que no hay una distinci\'on entre degeneraci\'on
		a derecha y degeneraci\'on a izquierda.
	}
	\begin{enumerate}[(1)]
		\item\label{item:perpendicular:complemento}
			Si $W\subset V$ es un subespacio, las siguientes
			condiciones son equivalentes:
			\begin{enumerate}[(a)]
				\item\label{item:perpendicular:complemento:a}
					el subespacio $W$ es no degenerado;
				\item\label{item:perpendicular:complemento:b}
					$W\cap W^\perp=0$;
				\item\label{item:perpendicular:complemento:c}
					$V=W\oplus W^\perp$.
			\end{enumerate}
			%
		\item\label{item:perpendicular:dimension}
			Si $V$ es no degenerado, entonces
			$\dim\,W+\dim\,W^\perp=\dim\,V$ y $(W^\perp)^\perp=W$.
	\end{enumerate}
	%
	En particular, si $V$ es no degenerado, un subespacio es no degenerado,
	si y s\'olo si $W^\perp$ lo es.
\end{teoIntroNodeg}

\begin{proof}
	Que $B$ sea no degenerada a izquierda significa
	$w\neq 0\Rightarrow B(w,w_1)$ para alg\'un $w_1$. Esto implica
	$W\cap W^\lperp=0$ (notar que $W^\lperp$ es un subespacio de $V$, no
	de $W$).

	Ahora, vemamos que $W\cap W^\lperp=0$ implica $W\oplus W^\lperp=V$.
	Dado que la intersecci\'on de los subespacios es cero, s\'olo resta
	probar que $W+W^\lperp=V$, es decir que todo elemento de $V$ es suma
	de un elementon de $W$ m\'as uno en $W^\lperp$. Sea
	$L:\,W\rightarrow\dual W$ la t.l. $L(w)=B(w,-)|_W$. Su n\'ucleo,
	$W\cap W^\lperp$ es, por hip\'otesis, nulo. Por lo tanto, puesto que
	$\dim\,W<\infty$,%
	\footnote{
		Aqu\'{\i} usamos que $W$ tiene dimensi\'on finita, y no $V$.
	}
	vale que $\dim\,\dual W=\dim\,W$ y, entonces, $L$ es sobreyectiva.
	Dado $v\in V$, $B(v,-)|_W=B(w,-)|_W$, para alg\'un $w\in W$, y
	$v-w\in W^\lperp$.

	Finalmente, si $V=W\oplus W^\lperp$, entonces $B(w,w_1)=0$ para todo
	$w_1\in W$ implica que $w\in W^\lperp$ y $w=0$. En consecuencia, $B$ es
	no degenerada a izquierda.

	Para probar \eqref{item:perpendicular:dimension}, supongamos que
	$V$ es no degenerado, es decir $L_B:\,V\rightarrow\dual V$ es un
	isomorfismo. Por Hahn-Banach (?), la restricci\'on
	$\dual V\rightarrow\dual W$ es sobre. El n\'ucleo de la composici\'on
	es $W^\lperp$. As\'{\i},
	\begin{displaymath}
		V/W^\lperp\,\simeq\,\dual W
		\text{ .}
	\end{displaymath}
	%
	Calculando dimensiones, $\dim\,V-\dim\,W^\lperp=\dim\,W$.%
	\footnote{
		En particular, $\dim\,W^\lperp=\dim\,W^\rperp$.
	}
	Ahora, $W\subset (W^\lperp)^\rperp$. Por un argumento de dimensi\'on,
	$W=(W^\lperp)^\rperp$.%
	\footnote{
		En particular, $(W^\lperp)^\rperp=(W^\rperp)^\lperp$.
	}
	Por \eqref{item:perpendicular:complemento:b} (su versi\'on a izquierda
	y su versi\'on a derecha), cuando $V$ es no degenerado,
	$W=(W^\lperp)^\rperp=(W^\rperp)^\lperp$ y $W$ es no degenerado, si y
	s\'olo si $W^\lperp$ lo es, si y s\'olo si $W^\rperp$ lo es.
\end{proof}

\begin{ejemIntroNodeg}\label{ejem:nodegeneradas:subespaciodegenerado:bis}
	Siguiendo con el Ejemplo~\ref{ejem:nodegeneradas:subespaciodegenerado},
	si bien $W=\generado{(1,0,1),(0,1,0)}$ es degenerado con respecto a la
	restricci\'on de $\langle\cdot,\cdot\rangle_{2,1}$, se puede ver que
	\begin{displaymath}
		W^\perp\,=\,\generado{(1,0,1)}\,\subset\,W
		\text{ .}
	\end{displaymath}
	%
	Entonces $\bb R^{2,1}$ no es suma directa de $W$ y $W^\perp$. Sin
	embargo, $\dim\,W+\dim\,W^\perp=3$. Esto es consistente con el hecho de
	que $\bb R^{2,1}$ es no degenerado pero la restricci\'on de la forma al
	subespacio $W$ s\'{\i} lo es.
\end{ejemIntroNodeg}

\begin{ejemIntroNodeg}\label{ejem:nodegeneradas:complemento}
	En $V=\bb R^2$ con la forma $B((x,y),(x_1,y_1))=xx_1$, el vector
	$(0,1)$ es perpendicular a todo el espacio. Es decir, $V^\perp\neq 0$ y
	la forma es degenerada. Sin embargo, el subespacio $W=\generado{(1,0)}$
	es no degenerado con respecto a la restricci\'on $B|_W$. Por lo tanto,
	$\bb R^2=W\oplus W^\perp$. Efectivamente, $W^\perp=\generado{(0,1)}$.
	Adem\'as, se verifica que $(W^\perp)^\perp=\bb R^2\neq W$, que $W$ es
	no degenerado, pero $W^\perp$ es degenerado.
\end{ejemIntroNodeg}

\begin{teoIntroNodeg}\label{teo:nodegeneradas:propiedades}
	Sea $(V,B)$ un espacio bilineal no degenerado. Entonces,
	\begin{enumerate}[(i)]
		\item\label{item:propiedades:hiperplanos}
			todo hiperplano en $V$ es de la forma
			$\{w\,:\,w\perp v\}$ para alg\'un $v\neq 0$ y de la
			forma $\{w\,:\,v_1\perp w\}$ para alg\'un $v_1\neq 0$;
		\item\label{item:propiedades:vectores}
			si $B(v,w)=B(v,w_1)$ para todo $v\in V$, entonces
			$w=w_1$;
		\item\label{item:propiedades:transformaciones}
			si $B(v,A\,w)=B(v,A_1\,w)$ para todo $v,w\in V$,
			entonces $A=A_1$;
		\item\label{item:propiedades:bilineales}
			toda forma bilineal en $V$ es de la forma
			$B(v,A\,w)$ para alguna t.l. $A:\,V\rightarrow V$.%
			\footnote{
				El hecho de que toda forma bilineal admite
				una representaci\'on matricial y que, tomando
				una base, se puede expresar en t\'erminos del
				producto escalar es un caso particular de este
				resultado.
			}
	\end{enumerate}
	%
\end{teoIntroNodeg}

\begin{proof}
	Para \ref{item:propiedades:hiperplanos}, usar que los hiperplanos son
	n\'ucleos de funcionales lineales.
\end{proof}

Si $(V,B)$ es un espacio bilineal no degenerado y $A:\,V\rightarrow V$ es una
transformaci\'on lineal, la funci\'on
\begin{displaymath}
	\tilde B(v,w)\,=\,B(A\,v,w)
\end{displaymath}
%
es una forma bilineal en $V$. Existe, por el Teorema~%
\ref{teo:nodegeneradas:propiedades}, una \emph{\'unica} transformaci\'on lineal
$\adjnt A:\,V\rightarrow V$ tal que
\begin{equation}
	\label{eq:nodegeneradas:adjunta}
	B(A\,v,w)\,=\,B(v,\adjnt A\,w)
	\text{ ,}
\end{equation}
%
para todo $v,w\in V$.

\begin{defIntroNodeg}\label{def:nodegeneradas:adjunta}
	Si $(V,B)$ es un espacio bilineal no degenerado y $A:\,V\rightarrow V$
	es una t.l., la \'unica t.l. $\adjnt A:\,V\rightarrow V$ que verifica
	\eqref{eq:nodegeneradas:adjunta} se denomina \emph{adjunta} de $A$, con
	respecto a $B$.
\end{defIntroNodeg}

\begin{ejemIntroNodeg}\label{ejem:nodegeneradas:adjunta:escalar}
	En $F^n$ con el producto escalar, la adjunta de una transformaci\'on
	lineal representada por una matriz $A\in\MM[n\times n](F)$ en la base
	can\'onica es la transformaci\'on representada por $\trnsp A$ en la
	misma base.
\end{ejemIntroNodeg}

\begin{ejemIntroNodeg}\label{ejem:nodegeneradas:adjunta}
	En $\bb R^2$ con la forma bilineal
	\begin{displaymath}
		B\Big( \begin{bmatrix} x \\ y \end{bmatrix},
			\begin{bmatrix} x_1 \\ y_1 \end{bmatrix}\Big)\,=\,
			\begin{bmatrix} x \\ y \end{bmatrix}\,\cdot\,
			\begin{bmatrix} 3 & \\ & -2 \end{bmatrix}\,
				\begin{bmatrix} x_1 \\ y_1 \end{bmatrix}
		\text{ ,}
	\end{displaymath}
	%
	la adjunta de una matriz est\'a dada por:
	\begin{displaymath}
		\adjnt{\begin{bmatrix} a & b \\ c & d \end{bmatrix}} \,=\,
			\begin{bmatrix}
				a & -(2/3)\,c \\ -(3/2)\,b & d
			\end{bmatrix}
		\text{ .}
	\end{displaymath}
	%
\end{ejemIntroNodeg}

\begin{ejemIntroNodeg}\label{ejem:nodegeneradas:adjunta:degenerada}
	La forma del Ejemplo~\ref{ejem:nodegeneradas:complemento} es
	degenerada. Se verifica que, si
	\begin{math}
		A=\sbmatrix{ a & b \\ c & d }
	\end{math},
	\begin{displaymath}
		B\Big(A\,\begin{bmatrix} 1 \\ 0 \end{bmatrix},
			\begin{bmatrix} 0 \\ 1 \end{bmatrix}\Big) \,=\,b
			\quad\text{y que}\quad
		B\Big(\begin{bmatrix} 1 \\ 0 \end{bmatrix},A'\,
			\begin{bmatrix} 0 \\ 1 \end{bmatrix}\Big) \,=\,0
		\text{ ,}
	\end{displaymath}
	%
	para cualquier matriz $A'$. En particular, si $b\neq 0$, $A$ no posee
	adjunta con respecto a $B$.
\end{ejemIntroNodeg}

\begin{teoIntroNodeg}\label{teo:nodegeneradas:adjunta:matriz}
	Sea $(V,B)$ un espacio bilineal no degenerado y sea
	$A:\,V\rightarrow V$ una t.l. Fijemos una base
	$\repr{\cdot}:\,V\rightarrow F^n$ de $V$. Sea $M$ la matriz asociada a
	$B$ con respecto a esta base y sean $\repr A$ y $\repr{\adjnt A}$ las
	matrices de $A$ y de $\adjnt A$, respectivamente, en la base elegida.
	Entonces,
	\begin{displaymath}
		\repr{\adjnt A}\,=\,M^{-1}\,\trnsp{\repr A}\,M
		\text{ .}
	\end{displaymath}
	%
\end{teoIntroNodeg}

\begin{proof}
	Comprobar que se cumple $R_B\,\adjnt A=\dual A\,R_B$ y pasar a la
	representaci\'on matricial. Se puede dar otra demostraci\'on, usando la
	identidad que define la adjunta, \eqref{eq:nodegeneradas:adjunta},
	pasando a la representaci\'on matricial de dicha identidad y usando que
	el producto escalar en $F^n$ es una forma bilineal no degenerada.
\end{proof}

La noci\'on de transformaci\'on dual-transpuesta tiene sentido incluso para
transformaciones lineales que no son endomorfismos (por ejemplo, en el caso
$V\rightarrow\dual V$). Si $A:\,V\rightarrow W$ es una transformaci\'on lineal
entre espacios vectoriales y cada uno de ellos tiene asociada una forma
bilineal no degenerada, deber\'{\i}a ser posible dar una noci\'on de
transformaci\'on adjunta $\adjnt A:\,W\rightarrow V$, relacion\'andola con la
transpuesta $\dual A:\,\dual W\rightarrow\dual V$.
\begin{center}
	\begin{tikzcd}
		V \arrow[d, "A"'] \\ W
	\end{tikzcd}
	\qquad
	\begin{tikzcd}
		\dual V & V \arrow[l, "R"'] \\
		\dual W\arrow[u,"\dual A"] &
			W \arrow[l,"R"] \arrow[u, dashed, "\adjnt A"']
	\end{tikzcd}
\end{center}

\begin{defIntroNodeg}\label{def:nodegeneradas:perpendicular}
	Si $A:\,V\rightarrow W$ es una transformaci\'on lineal entre espacios
	bilineales, decimos que $A$ \emph{preserva la relaci\'on de %
	ortogonalidad}, si
	\begin{displaymath}
		v\,\perp\,v_1\,\Rightarrow\,A\,v\,\perp\,A\,v_1
		\text{ ,}
	\end{displaymath}
	%
	para todo $v,v_1\in V$.%
	\footnote{
		No asumimos que $\dim\,V=\dim\,W$.
	}
\end{defIntroNodeg}

\begin{teoIntroNodeg}\label{teo:nodegeneradas:perpendicularidad}
	Dados espacios bilineales no degenerados, $(V,B_V)$ y $(W,B_W)$, y una
	t.l. $A:\,V\rightarrow W$, las siguientes propiedades son equivalentes:
	\begin{enumerate}[(i)]
		\item\label{item:perpendicularidad:preserva}
			$A$ preserva la relaci\'on de ortogonalidad;
		\item\label{item:perpendicularidad:constante}
			$B_W(A\,v,A\,v_1)=c\,B_V(v,v_1)$, para todo
			$v,v_1\in V$, para cierta constante $c\in F$;
		\item\label{item:perpendicularidad:adjunta}
			$\adjnt A\,A=c\,\id[V]$, para cierta constante
			$c\in F$.
	\end{enumerate}
	%
\end{teoIntroNodeg}

\begin{obsIntroNodeg}\label{obs:nodegeneradas:perpendicularidad}
	En particular, del Teorema~\ref{teo:nodegeneradas:perpendicularidad} se
	deduce que, si $A:\,V\rightarrow W$ preserva las formas bilineales, es
	decir, si $B_W(A\,v,A\,v_1)=B_V(v,v_1)$, para todo $v,v_1\in V$,
	entonces $A$ tiene inversa a izquierda; la inversa a izquierda est\'a
	dada pro $\adjnt A$. Adem\'as, si $W=V$, $A^{-1}$ existe y es igual a
	$\adjnt A$ y, por el \teoname~\ref{teo:nodegeneradas:adjunta:matriz},
	si $M$ es la matriz asociada a $B$ en una base,
	$M=\trnsp{\repr A}\,M\,\repr A$, es decir, $\repr A$, como matriz de
	cambio de base, no afecta a $M$.
\end{obsIntroNodeg}

\begin{lemaIntroNodeg}\label{lema:nodegeneradas:perpendicularidad}
	Sea $V$ un $F$-e.v. y sea $T:\,V\rightarrow V$ una t.l. Entonces,
	\begin{enumerate}[(1)]
		\item\label{item:perpendicularidad:preserva:rectas}
			si, para toda recta $L\subset V$ que pasa por el
			origen, $T(L)\subset L$, entonces $T$ es una homotecia;
		\item\label{item:perpendicularidad:preserva:hiperplanos}
			si, para todo hiperplano $H\subset V$ que pasa por el
			origen, $T(H)\subset H$, entonces $T$ es una homotecia.
	\end{enumerate}
	%
\end{lemaIntroNodeg}

\begin{proof}[Demostraci\'on del Teorema~%
	\ref{teo:nodegeneradas:perpendicularidad}]
	Asumiendo \eqref{item:perpendicularidad:preserva}, podemos afirmar que,
	para todo $v,v_1\in V$, si $B_V(v,v_1)=0$, entonces
	$B_W(A\,v,A\,v_1)=0$ y $B_V(v,\adjnt A\,A\,v_1)=0$. En particular,
	la t.l. $\adjnt A\,A:\,V\rightarrow V$ preserva todos los hiperplanos.
	Apelando al Lema~\ref{lema:nodegeneradas:perpendicularidad}, deducimos
	\eqref{item:perpendicularidad:constante}.

	Las afirmaciones \eqref{item:perpendicularidad:constante} y
	\eqref{item:perpendicularidad:adjunta} son equivalentes porque $B_V$ es
	no degenerada.
\end{proof}

\begin{ejerIntroNodeg}\label{ejer:nodegeneradas:adjunta:doble}
	?`Para qu\'e t.l. $A$ se cumple que $\aadjnt A=A$?
\end{ejerIntroNodeg}

\begin{ejerIntroNodeg}\label{ejer:nodegeneradas:dimension}
	Sea $V$ un espacio sim\'etrico o alternado y sea $W\subset V$ un
	subespacio tal que $\dim\,W+\dim\,W^\perp=\dim\,V$. Entonces, si
	$U\subset W^\perp$ es tal que $U+W=V$, $U=W^\perp$.
\end{ejerIntroNodeg}

\begin{ejerIntroNodeg}\label{ejer:nodegeneradas:anulador}
	Si $W\subset V$ es un subespacio ($\dim\,V<\infty$) y
	$W'=\{\varphi\in\dual V\,:\,\varphi(W)=0\}$, entonces
	$\dim\,W'+\dim\,W=\dim V$.
\end{ejerIntroNodeg}

\begin{ejerIntroNodeg}\label{ejer:nodegeneradas:perfecto}
	Una funci\'on bilineal $B:\,V\times W\rightarrow F$ se dice
	\emph{perfecta},%
	\footnote{
		O \emph{pairing perfecto}.
	}
	si
	\begin{displaymath}
		\begin{aligned}
			\big(v\,\mapsto\,B(v,-)\big) & \,:\,
				V\,\rightarrow\,\dual W
			\quad\text{y} \\
			\big(w\,\mapsto\,B(-,w)\big) & \,:\,
				W\,\rightarrow\,\dual V
		\end{aligned}
		%
	\end{displaymath}
	%
	son isomorfismos. Dada una funci\'on bilineal $B$ en $V\times W$ y
	dado un subespacio $U\subset V$, se define
	\begin{math}
		U^\perp=\{w\in W\,:\,B(U,w)=0\}
	\end{math}. Si $B$ es perfecta, entonces la funci\'on inducida
	\begin{displaymath}
		U\,\times\,\big(W/U^\perp\big)\,\rightarrow\,F
	\end{displaymath}
	%
	es perfecta.
\end{ejerIntroNodeg}

\begin{ejerIntroNodeg}\label{ejer:nodegeneradas:adjunta:propiedades}
	Si $B$ es una forma bilineal no degenerada en $V$, entonces
	\begin{itemize}
		\item $\adjnt{(A_1+A_2)}=\adjnt{A_1}+\adjnt{A_2}$,
		\item $\adjnt{(c\,A)}=c\,\adjnt A$,
		\item $\adjnt{\id[V]}=\id[V]$,
		\item $\adjnt{(A_1\,A_2)}=\adjnt{A_2}\,\adjnt{A_1}$,
		\item $\adjnt{(A^{-1})}=(\adjnt A)^{-1}$,
		\item $\det\,\adjnt A=\det\,A$, $\Traza(\adjnt A)=\Traza(A)$ y
			$\caracteristico[{\adjnt A}]=\caracteristico[A]$.
	\end{itemize}
	%
\end{ejerIntroNodeg}

