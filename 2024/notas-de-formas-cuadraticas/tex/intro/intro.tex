% Teor\'{\i}a elemental de formas bilineales sobre cuerpos; formas
% cuadr\'aticas sobre un cuerpo de caracter\'{\i}stica $2$ y de
% caracter\'{\i}stica distinta de $2$. Resumen de las notas
% \cite{ConradBilinear}.

\section{Definiciones y ejemplos}\label{sec:intro:definiciones}
\theoremstyle{plain}
\newtheorem{teoDefiniciones}{\teoname}[section]
\newtheorem{coroDefiniciones}[teoDefiniciones]{\coroname}

\theoremstyle{definition}
\newtheorem{defDefiniciones}[teoDefiniciones]{\defname}
\newtheorem{obsDefiniciones}[teoDefiniciones]{\obsname}

%-------------

\subsection{Un grupo}\label{subsec:grupo}
El grupo lineal (especial) de rango $2$ jugar\'a un rol fundamental
en toda la discusi\'on. El grupo con coeficientes reales est\'a
definido de la siguiente manera:
\begin{displaymath}
	\SL(2,\Reales)\,=\,\bigg\{
		\begin{bmatrix} a & b \\ c & d \end{bmatrix}\,:\,
			a,b,c,d\in\Reales,\,ad-bc=1
		\bigg\}
	\dispstop
\end{displaymath}
%
Como transformaciones del espacio eucl\'{\i}deo $\Reales^2$,
estas matrices preservan volumen y orientaci\'on, aunque no todas
preservan las distancias entre puntos del plano.
Dentro del plano, tenemos el ret\'{\i}culo $\Enteros^2$ de puntos
con coordenadas enteras. El \emph{grupo modular} es el subgrupo
del grupo lineal especial que preserva este ret\'{\i}culo:
\begin{displaymath}
	\modulgruppe\,=\,\SL(2,\Enteros)\,=\,
		% \SL(2,\Reales)\,\cap\, \Mat(2\times 2,\Enteros)
		\bigg\{
		\begin{bmatrix} a & b \\ c & d \end{bmatrix}\,:\,
			a,b,c,d\in\Enteros,\,ad-bc=1
		\bigg\}
	\dispstop
\end{displaymath}
%
Algunas matrices pertenecientes a $\SL(2,\Enteros)$ son:
\begin{displaymath}
	\begin{bmatrix} 1 & h \\ & 1 \end{bmatrix}
	\dispcomma\quad
	\begin{bmatrix} & -1 \\ 1 & \end{bmatrix}
	\dispand
	\pm\,\begin{bmatrix} 1 & \\ & 1 \end{bmatrix}
	\dispstop
\end{displaymath}
%
Si $\gamma\in\Gamma$, entonces $\trnsp\gamma\in\Gamma$ tambi\'en.



\subsection{Funciones en el semiplano de Poincar\'e}\label{subsec:funciones}
El \emph{semiplano complejo superior} est\'a conformado por los
n\'umeros complejos con parte imaginaria definida positiva:
\begin{displaymath}
	\semiplano\,=\,\big\{z\in\Complejos\,:\,\Imag(z)>0\big\}
	\dispstop
\end{displaymath}
%
El grupo $\SL(2,\Reales)$ act\'ua en $\semiplano$ por
\emph{transformaciones de M\"obius}: si $z\in\semiplano$ y
$\gamma\in\SL(2,\Reales)$, definimos un nuevo punto de la siguiente
manera:
\begin{equation}
	\label{eq:definiciones:moebius}
	\gamma\accion z\,=\,\frac{az+b}{cz+d}
	\dispstop
\end{equation}
%
Dado que $\Imag(z)>0$ y $c,d\in\Reales$ no ambos nulos, se deduce
que $cz+d\neq 0$. Adem\'as,
\begin{equation}
	\label{eq:definiciones:imaginaria}
	\Imag(\gamma\accion z)\,=\,\frac{\Imag(z)}{|cz+d|^2}
	\dispcomma
\end{equation}
%
de lo que se deduce que $\gamma\accion z\in\semiplano$.
Que \eqref{eq:definiciones:moebius} defina una \emph{acci\'on}
quiere decir que, si $\gamma,\gamma'\in\modulgruppe$ y $z\in\semiplano$,
entonces $(\gamma\gamma')\accion z=\gamma\accion{\gamma'\accion z}$.
Las matrices escalares, $\pm\,\sbmatrix{ 1 & \\ & 1 }$, act\'uan
trivialmente, es decir, no mueven ning\'un punto.
% Con lo cual, la acci\'on \eqref{eq:definiciones:moebius} es, en verdad,
% una acci\'on del grupo
% \begin{displaymath}
	% \PSL(2,\Reales)\,=\,\SL(2,\Reales)/\{\pm\Id\}
	% \dispstop
% \end{displaymath}
% %

% La acci\'on \eqref{eq:definiciones:moebius} es transitiva: si
% $z=x+\raizcuarta y$, $y>0$, entonces
% \begin{equation}
	% \label{eq:definiciones:transitividad}
	% z\,=\,
	% \begin{bmatrix}
		% \sqrt y & x\sqrt y^{-1} \\ & \sqrt y^{-1}
	% \end{bmatrix}\accion\raizcuarta
	% \,=\,
	% \bigg(
	% \begin{bmatrix} 1 & x \\ & 1 \end{bmatrix}\,
	% \begin{bmatrix} \sqrt y & \\ & \sqrt y^{-1} \end{bmatrix}
	% \bigg)\accion\raizcuarta
	% \dispcomma
% \end{equation}
% %
% una dilaci\'on seguida de una traslaci\'on. Las matrices que
% dejan fijo el punto $\raizcuarta\in\semiplano$ forman un subgrupo,
% el \emph{grupo ortogonal especial}:
% \begin{displaymath}
	% \SO(2,\Reales)\,=\,\bigg\{
		% \begin{bmatrix} a & b \\ -b & a \end{bmatrix}\,:\,
			% a,b\in\Reales,\,a^2+b^2=1
		% \bigg\}
	% \dispstop
% \end{displaymath}
% %
% Esto permite reinterpretar el espacio $\semiplano$ como un cociente:
% \begin{displaymath}
	% \semiplano\,=\,\SL(2,\Reales)/\SO(2,\Reales)
	% \dispstop
% \end{displaymath}
% %

\begin{defDefiniciones}\label{def:definiciones:modular}
	Una \emph{forma modular} es una funci\'on
	$f:\,\semiplano\rightarrow\Complejos$ que
	\begin{enumerate}[label=(M\arabic*)]
		\item\label{item:modular:holomorfia}
			es holomrfa,
		\item\label{item:modular:transformaciones}
			existe $k$ tal que, para toda
			$\gamma=\sbmatrix{ * & * \\ c & d }\in\modulgruppe$
			y $z\in\semiplano$, cumple
			\begin{equation}
				\label{eq:definiciones:transformacion}
				f(\gamma\accion z)\,=\,(cz+d)^k\,f(z)
				\dispand
			\end{equation}
			%
		\item\label{item:modular:crecimiento}
			est\'a acotada en regiones de la forma
			$\{y\geq \delta\}$, para todo $\delta>0$.
	\end{enumerate}
	%
	El n\'umero $k$ es el \emph{peso} de la forma.
\end{defDefiniciones}

\begin{obsDefiniciones}\label{obs:definiciones:modular:desarrollo}
	Si $f$ es una forma modular, por \ref{item:modular:transformaciones},
	la funci\'on es, en particular, peri\'odica de per\'{\i}odo $1$:
	\begin{math}
		f(z+1)=f(z)
	\end{math},
	si $z\in\semiplano$. Esto implica que existe una funci\'on
	$g$ tal que
	\begin{displaymath}
		f(z)\,=\,g(\varexp^{2\pi\raizcuarta z})
		\dispstop
	\end{displaymath}
	%
	El dominio de definici\'on de $g$ es $D\setmin\{0\}$, donde
	\begin{displaymath}
		D\,=\,\big\{ q\in\Complejos\,:\,|q|<1 \big\}
		\dispstop
	\end{displaymath}
	%
	Como, por \ref{item:modular:holomorfia}, $f$ es holomorfa en
	$\semiplano$, la funci\'on $g$ es holomorfa en el disco pinchado.
	La transformaci\'on $z\mapsto q=\varexp^{2\pi\raizcuarta z}$
	es holomorfa y la imagen de un subconjunto $\{y\geq\delta\}$,
	$\delta>0$, es un disco (pinchado y con borde) de radio
	$\varexp^{-2\pi\delta}$. Ahora, por \ref{item:modular:crecimiento},
	$f(z)$ est\'a acotada en $\{y\geq\delta\}$. En consecuencia,
	$g$ est\'a acotada en un entorno de $q=0$. Pero esto quiere decir
	que $g$ se extiende a una funci\'on holomorfa definida en todo
	el disco $D$. El desarrollo de Taylor de $g$ en $q=0$ da lugar a
	un desarrollo de $f$ ``en $\infty$'':
	\begin{equation}
		\label{eq:definiciones:desarrollo}
		f(z)\,=\,\sum_{n\geq 0}\,a_n\,\varexp^{2\pi\raizcuarta n z}
		\dispstop
	\end{equation}
	%
	Se dice que la serie en \eqref{eq:definiciones:desarrollo} es
	el \emph{desarrollo de Fourier de $f$ en el infinito}.
	La serie comienza en $n=0$, porque la funci\'on $g$ es holomorfa.
	Por esta raz\'on, se suele decir que
	$f$ es ``holomorfa en $\infty$''.
\end{obsDefiniciones}

A veces se escribe $q$, en lugar de $\varexp^{2\pi\raizcuarta z}$.
Es la existencia de estos desarrollos lo que hace relevantes a las
formas modulares en el contexto de Teor\'{\i}a de n\'umeros.



\subsection{El dominio fundamental}\label{subsec:fundamental}
Si una funci\'on $f$ satisface \ref{item:modular:transformaciones}
con $k=0$, entonces $f$ define una funci\'on en las \'orbitas
$\modulgruppe\backslash\semiplano$.
Si $k\neq 0$, esto no es cierto (salvo talvez en el caso de la
funci\'on constante $0$).
Sin embargo, entender el comportamiento de $f$ en un conjunto de
representantes adecuado es esencialmente todo lo que se necesita
para algunas aplicaciones.

Ahora bien, el grupo modular $\modulgruppe$ contiene las matrices
$\pm\Id$, que son los \'unicos elementos de $\SL(2,\Reales)$ que
act\'uan trivialmente en $\semiplano$.
En particular, toda forma modular de peso entero impar debe ser
id\'enticamente cero. Si $\gamma\in\SL(2,\Reales)$, ser\'a conveniente
pensar en las matrices $\pm\gamma$ como la misma.

\begin{obsDefiniciones}\label{obs:definiciones:generadores}
	M\'odulo $\pm\Id$, el subgrupo
	$\modulgruppe\subgrpeq\SL(2,\Reales)$ est\'a generado
	por las matrices
	\begin{displaymath}
		T\,=\,\begin{bmatrix} 1 & 1 \\ & 1 \end{bmatrix}
		\dispand
		S\,=\,\begin{bmatrix} & -1 \\ 1 & \end{bmatrix}
		\dispstop
	\end{displaymath}
	%
	Estas matrices act\'uan de la siguiente manera:
	\begin{displaymath}
		z\,\mapsto\,z+1\dispand
		z\,\mapsto\,-1/z
		\dispstop
	\end{displaymath}
	%
	En particular, para verificar si una funci\'on $f$ cumple
	con la propipedad \ref{item:modular:transformaciones},
	alcanza con verificar que se cumplan las siguientes
	igualdades:
	\begin{displaymath}
		f(z+1)\,=\,f(z)\dispand
		f(-1/z)\,=\,z^k\,f(z)
		\dispcomma
	\end{displaymath}
	%
	donde $k$ ser\'{\i}a el peso de $f$.
\end{obsDefiniciones}

\begin{defDefiniciones}\label{def:definiciones:fundamental}
	Un \emph{dominio fundamental} es un subconjunto
	$\cal F\subseteq\semiplano$ que
	\begin{enumerate}[label=(F\arabic*)]
		\item\label{item:fundamental:abierto}
			es abierto,
		\item\label{item:fundamental:representantes}
			no contiene dos puntos de la misma \'orbita
			y toda \'orbita interseca su clausura y
		\item\label{item:fundamental:conexo}
			es conexo.
	\end{enumerate}
	%
\end{defDefiniciones}

\newcommand{\aFundamental}[1][]{\ensuremath{\mathcal{F}_{#1}}}
\newcommand{\sFundamental}[1][]{\ensuremath{\widetilde{\mathcal F}_{#1}}}
\newcommand{\cFundamental}[1][]{\ensuremath{\overline{\mathcal F}_{#1}}}
\begin{teoDefiniciones}\label{teo:definiciones:fundamental}
	El conjunto
	\begin{displaymath}
		\aFundamental\,=\,\big\{z\in\semiplano\,:\,
			|z|>1,\,|\Real(z)|<1/2\big\}
	\end{displaymath}
	%
	es un dominio fundamental para $\modulgruppe$.
\end{teoDefiniciones}

\begin{proof}
	Si $z\in\semiplano$, el subconjunto
	$L=\{mz+n\,:\,m,n\in\Enteros\}\subset\Complejos$ es un
	ret\'{\i}culo y posee un punto de m\'odulo m\'{\i}nimo
	$cz+d$ (minimiza $|mz+n|$ con $m,n\in\Enteros$ no ambos nulos).
	Deber ser $\mcd{c,d}=1$, por minimalidad. Entonces, existe
	$\gamma_1\in\modulgruppe$ tal que
	$\gamma_1=\sbmatrix{ * & * \\ c & d }$. Por la f\'ormula
	\eqref{eq:definiciones:imaginaria}, el valor
	$\Imag(\gamma_1\accion z)$ es m\'aximo entre
	$\Imag(\gamma\accion z)$, $\gamma\in\modulgruppe$.
	Sea $h\in\Enteros$ tal que $|\Real(\gamma_1\accion z + h)|\leq 1/2$
	y sea $z^*=T^h\gamma_1\accion z=\gamma_1\accion z + h$.
	El punto $z^*\in\cFundamental$: si fuese $|z^*|<1$, entonces
	$\Imag(S\accion{z^*})=\Imag(z^*)/|z^*|^2>\Imag(z^*)$
	contradir\'{\i}a la maximalidad de $\Imag(z^*)$.

	Supongamos, ahora, que $z_1,z_2\in\aFundamental$ y que
	$\gamma\in\modulgruppe$ satisfacen $\gamma\accion{z_1}=z_2$.
	Como $|\Real(z_1)|<1/2$ y tambi\'en $|\Real(z_2)|<1/2$,
	debe ser $c\neq 0$, o bien $c=b=0$ (o sea $\gamma=\pm\Id$;
	hay una f\'ormula para la parte real, tambi\'en, pero
	oscurecer\'{\i}a; talvez en otros contextos tal f\'ormula
	se vuelve imprescindible, por falta de intuici\'on o
	visi\'on/visibilidad). Si fuese $c\neq 0$, notando que
	$\Imag(z)>\sqrt 3/2$ para $z\in\aFundamental$, se ve que
	\begin{displaymath}
		\frac{\sqrt 3} 2\,<\,\Imag(z_2)\,=\,
			\frac{\Imag(z_1)}{|cz_1+d|^2}\,\leq\,
			\frac{\Imag(z_1)}{c^2\Imag(z_1)^2}\,<\,
			\frac 2{c^2\sqrt 3}
		\dispstop
	\end{displaymath}
	%
	Pero, entonces, deber\'{\i}a ser $c=\pm 1$. Esto es absurdo,
	pues $|\pm z_1+d|\geq |z_1|>1$ y, por lo tanto, suponiendo
	sin p\'erdida de generalidad que era $\Imag(z_1)\leq\Imag(z_2)$,
	\begin{displaymath}
		\Imag(z_1)\,\leq\,\Imag(z_2)\,=\,
			\frac{\Imag(z_1)}{|\pm z_1+d|^2}\,<\,\Imag(z_1)
		\dispstop
	\end{displaymath}
	%
\end{proof}

% \paragraph{Aplicaci\'on: finitud del n\'umero de clases}
% \newcommand{\red}{\ensuremath{\mathsf{red}}}
% Una forma cuadr\'atica binaria es una expresi\'on del tipo
% \begin{displaymath}
	% Q(x,y)\,=\,Ax^2\,+\,Bxy\,+\,Cy^2
	% \dispstop
% \end{displaymath}
% %
% su discriminante es $B^2-4AC$. Si $A,B,C\in\Enteros$, entonces
% $D=B^2-4AC\in\Enteros$ y, de hecho, $D\equiv 0,1\tmodulo[4]$.
% Si el discriminante de una forma cuadr\'atica es negativo,
% entonces la forma es definida. En particular, toma s\'olo
% valores positivos o s\'olo valores negativos. En el primer caso,
% decimos que la forma es definida positiva. Esto es lo mismo que
% pedir que $A>0$.
% 
% El grupo $\modulgruppe$ act\'ua en el conjunto $\cal Q$ de formas
% cuadr\'aticas (binarias, con coeficientes enteros):
% si $Q\in\cal Q$ y $\gamma=\sbmatrix{ a & b \\ c & d }\in\modulgruppe$,
% entonces podemos definir una nueva forma cuadr\'atica por
% \begin{displaymath}
	% (Q\cdot\gamma)(x,y)\,=\,Q(ax+by,cx+dy)
	% \dispstop
% \end{displaymath}
% %
% Se verifica que $Q\cdot\gamma\in\cal Q$.
% Dos formas $Q$ y $Q'$ son (propiamente) equivalentes, si existe
% $\gamma\in\modulgruppe$ tal que $Q'=Q\cdot\gamma$.
% 
% Dado $D\equiv 0,1\tmodulo[4]$, sea $\cal Q_D\subseteq\cal Q$
% el subconjunto de formas de discriminante $D$.
% Si $Q$ y $Q'$ son equivalentes entonces:
% \begin{itemize}
	% \item $Q$ y $Q'$ tienen igual discriminante y
	% \item $Q$ y $Q'$ toman los mismos valores (la misma cantidad de veces).
% \end{itemize}
% %
% En particular, $\modulgruppe$ act\'ua en $\cal Q$ preservando cada
% $\cal Q_D$. M\'as aun, si $D<0$ y $\cal Q_D^+\subseteq\cal Q_D$ es el
% subconjunto de formas de discriminante $D$ y definidas positivas, entonces
% $\modulgruppe$ preserva $\cal Q_D^+$.
% 
% \begin{teoDefiniciones}\label{teo:definiciones:numero}
	% La contidad de clases de equivalencia propia de formas
	% cuadr\'aticas binarias, con coeficientes enteros, de
	% discriminante $D<0$ y definidas positivas es finito.
% \end{teoDefiniciones}
% 
% \begin{proof}
	% Dada $Q=\binaria{A,B,C}\in\cal Q_D^+$,
	% definimos $z_Q\in\semiplano$ por
	% \begin{displaymath}
		% z_Q\,=\,\frac{-B+\sqrt D}{2A}
	% \end{displaymath}
	% %
	% (donde $\sqrt D$ es la ra\'{\i}z de $D$ perteneciente a $\semiplano$;
	% o sea, podemos elegir consistentemente una ra\'{\i}z
	% $z_Q$ de cada polinomio $Q(z,1)$, de manera que $z_Q\in\semiplano$).
	% Se puede comprobar que, si $\gamma\in\modulgruppe$ y $Q\in\cal Q_D^+$,
	% entonces
	% \begin{displaymath}
		% z_{Q\cdot\gamma}\,=\,\gamma^{-1}\accion{z_Q}
		% \dispstop
	% \end{displaymath}
	% %
	% Adem\'as, $z_Q\in\sFundamental$, si y s\'olo si $Q$
	% pertenece al subconjunto de formas reducidas:
	% \begin{displaymath}
		% \cal Q_D^{+,\red}\,=\,
			% \big\{\binaria{A,B,C}\in\cal Q_D^+\,:\,
				% -A<B\leq A<C\dispor
				% 0\leq B\leq A=C\big\}
		% \dispstop
	% \end{displaymath}
	% %
	% Pero el subconjunto $\cal Q_D^{+,\red}$ es finito y
	% toda $Q\in\cal Q_D^+$ es propiamente equivalente a una
	% (\'unica) $Q'\in\cal Q_D^{+,\red}$, pues
	% $z_Q$ es equivalente v\'{\i}a $\modulgruppe$ a un
	% (\'unico) $z'\in\sFundamental$.
% \end{proof}
% 


\subsection{La dimensi\'on de los espacios de formas}\label{subsec:dimension}
Si bien una forma modular no define una funci\'on en el espacio
de \'orbitas $\modulgruppe\backslash\semiplano$, s\'{\i} tiene
sentido hablar de los ceros de una forma modular en este cociente:
el factor de automorf\'{\i}a $cz+d$ no tiene ni ceros ni polos
en $\semiplano$, si $\sbmatrix{ * & * \\ c & d }\in\modulgruppe$.
M\'as precisamente, si $z\in\semiplano$, $\gamma\in\modulgruppe$ y
$f$ es una forma modular, entonces
\begin{displaymath}
	\orden[{\gamma\accion z}](f)\,=\,\orden[z](f)
	\dispstop
\end{displaymath}
%
El hecho de que los espacios de formas modulares $\modulformen[k]$
son de dimensi\'on finita, es consecuencia de que la cantidad de ceros
(con multiplicidad) de $f\in\modulformen[k]$ que pertenecen a distintas
\'orbitas (es decir, los ceros en $\modulgruppe\backslash\semiplano$)
depende s\'olo de $k$ (y de $\modulgruppe$).
T\'ecnicamente, tambi\'en es necesario tener en cuenta la contribuci\'on
del orden de anulaci\'on de $f$ en $\infty$ y contar adecuadamente la
contribuci\'on de ciertos puntos especiales, los puntos el\'{\i}pticos.

\begin{defDefiniciones}\label{def:definiciones:elipticos}
	Un \emph{punto el\'{\i}ptico} en $\semiplano$ es un punto
	que queda fijo por alguna matriz $\gamma\in\modulgruppe$,
	$\gamma\neq\pm\Id$.
\end{defDefiniciones}

\begin{obsDefiniciones}\label{obs:definiciones:elipticos}
	Los \'unicos puntos el\'{\i}pticos en $\cFundamental$ son
	$\raizcuarta$, $\raizcubica$ y $\raizcubica+1$. Con respecto
	a estos puntos, los estabilizadores (en $\modulgruppe$)
	tienen \'ordenes $4$ y $6$, respectivamente ($\raizcubica$ y
	$\raizcubica+1$ pertenecen a la misma \'orbita, con lo que sus
	estabilizadores son conjugados).
	El resto de los puntos tiene estabilizador $\pm\Id$.
	Para cada punto $z\in\semiplano$, definimos un entero $n_z$
	de la siguiente manera:
	\begin{displaymath}
		n_z\,=\,
			\begin{cases}
				2 \dispcomma & \dispif
					z\in\modulgruppe\accion\raizcuarta
					\dispcomma \\
				3 \dispcomma & \dispif
					z\in\modulgruppe\accion\raizcubica
					\dispand \\
				1 & \text{en otro caso.}
			\end{cases}
			%
	\end{displaymath}
	%
\end{obsDefiniciones}

\begin{defDefiniciones}\label{def:definiciones:orden-en-infinito}
	El \emph{orden de anulaci\'on en $\infty$} de una forma
	modular es el menor $n\geq 0$ tal que el coeficiente $a_n$
	en su desarrollo de Fourier es distinto de cero.
\end{defDefiniciones}

\begin{teoDefiniciones}\label{teo:definiciones:cota}
	Sea $f\in\modulformen[k]$, $f\neq 0$. Entonces,
	\begin{displaymath}
		\sum_{z\in\modulgruppe\backslash\semiplano}\,
			\frac 1{n_z}\,\orden[z](f)\,+\,
			\orden[\infty](f)\,=\,\frac k{12}
		\dispstop
	\end{displaymath}
	%
\end{teoDefiniciones}

La sumatoria en el \teoname~\ref{teo:definiciones:cota} se realiza
sobre cualquier sistema de representantes de la \'orbitas de
$\modulgruppe$ en $\semiplano$.

\begin{proof}
	Sea $D=D(\epsilon)$ el cerrado que queda despu\'es de
	eliminar de $\cFundamental$ discos (abiertos) de radio
	$\epsilon>0$ (suficientemente chico) centrados en cada
	cero de $f$, en cada punto el\'{\i}ptico, as\'{\i} como
	el abierto $\{y>\epsilon^{-1}\}$. Integrando a lo largo
	del borde de este cerrado $f'(z)/f(z)$, el resultado es $0$.
	Ahora calculamos la integral en distintos segmentos, sumamos
	y comparamos con el resultado anterior.

	El contorno de los entornos removidos de los ceros y de
	los puntos el\'{\i}pticos son:
	\begin{itemize}
		\item un c\'{\i}rculo entero, si es alrededor
			de un punto no el\'{\i}ptico,
		\item medio c\'{\i}rculo, si es alrededor de
			$\raizcuarta$ y
		\item un tercio de c\'{\i}rculo, si es alrededor
			de $\raizcubica$.
	\end{itemize}
	%
	La integral a lo largo de un borde vertical se compensa con
	la integral a lo largo del borde opuesto ($f(z+1)=f(z)$ y
	la orientaci\'on se invierte). La integral a lo largo
	del borde $\{y=\epsilon^{-1}\}$ aporta
	$2\pi\raizcuarta\orden[\infty](f)$, pues, v\'{\i}a el cambio
	de variables $z\mapsto q$, integramos alrededor de un
	c\'{\i}rculo peque\~no centrado en $q=0$ (recorrido una vez,
	en sentido antihorario). La contribuci\'on de la integral a
	lo largo de un c\'{\i}rculo centrado en un punto $z$ es
	$2\pi\raizcuarta\orden[z](f)/n_z$. Por \'ultimo, la integral
	a lo largo del arco inferior de $\cFundamental$ da como
	resultado $\pi\raizcuarta k/6$, pues las dos mitades del arco
	se identifican v\'{\i}a $S$ y
	\begin{displaymath}
		% \frac{f'(S\accion z)}{f(S\accion z)}\,=\,
			% \frac{f'(z)}{f(z)}\,+\,\frac k z
		\big(\log\,f(S\accion z)\big)'\,=\,
			\big(\log\,f(z)\big)\,+\,\frac k z
		\dispstop
	\end{displaymath}
	%
\end{proof}

\begin{coroDefiniciones}\label{coro:definiciones:cota}
	La dimensi\'on de $\modulformen[k]$ es $0$ si $k<0$ o
	impar. Si $k\geq 0$ es par, entonces
	\begin{displaymath}
		\dim\,\modulformen[k]\,\leq\,
			\begin{cases}
				\piso{k/12}+1\dispcomma & \dispif
					k\not\equiv 2\tmodulo[12]
					\dispand \\
				\piso{k/12}\dispcomma & \dispif
					k\equiv 2\tmodulo[12]
					\dispstop
			\end{cases}
			%
	\end{displaymath}
	%
\end{coroDefiniciones}

\begin{proof}
	Tomar $m=\piso{k/12}+1$ puntos no el\'{\i}pticos y
	$m+1$ formas $\lista f{m+1}\in\modulformen[k]$.
	Alguna combinaci\'on lineal de ellas, $f$, se anula en los
	$m$ puntos. Pero como $m>k/12$, por el \teoname~%
	\ref{teo:definiciones:cota}, $f=0$.
	Si $k\equiv 2\tmodulo[12]$, se puede mejorar la cota
	teniendo en cuenta la f\'ormula del \teoname~%
	\ref{teo:definiciones:cota}.
\end{proof}

\begin{coroDefiniciones}\label{coro:definiciones:parametrizacion}
	El espacio $\modulformen[12]$ tiene dimensi\'on $\leq 2$ y,
	si $f,g\in\modulformen[12]$ son linealmente independientes,
	entonces $z\mapsto f(z)/g(z)$ es un isomorfismo entre
	$\modulgruppe\backslash\semiplano\cup\{\infty\}$ y
	$\bb P^1(\Complejos)$.
\end{coroDefiniciones}

\begin{proof}
	La cota se deduce del \coroname~\ref{coro:definiciones:cota}.
	Sean $f,g\in\modulformen[12]$ dos formas linealmente
	independientes. Si $\lambda,\mu\in\Complejos$, no ambos nulos,
	entonces la cantidad de ceros en
	$\modulgruppe\backslash\semiplano\cup\{\infty\}$ de la diferencia
	$\lambda f-\mu g$ es exactamente $1$ (tiene peso $12$).
	En particular, el cociente $\psi(z)=f(z)/g(z)$ toma cada valor
	de $\Complejos\cup\{\infty\}$ exactamente una vez.
\end{proof}

\begin{obsDefiniciones}\label{obs:definiciones:cota}
	Toda terna de formas modulares es algebraicamente
	dependiente: si $f$, $g$ y $h$ fuesen formas modulares
	algebraicamente independientes, entonces, para $k$
	suficientemente grande, la dimensi\'on de $\modulformen[k]$
	ser\'{\i}a, al menos, la cantidad de monomios en las
	variables $f$, $g$ y $h$ de peso total $k$, que es cuadr\'atica
	en $k$.
\end{obsDefiniciones}

Una consecuencia importante es que, para comparar dos formas modulares
de igual peso (y grupo), basta con comparar una cantidad finita de
coeficientes.





\section{La matriz asociada}\label{sec:intro:matrices}
\theoremstyle{plain}
\newtheorem{teoIntroMat}{Teorema}[section]

\theoremstyle{definition}
\newtheorem{defIntroMat}[teoIntroMat]{Definici\'on}
\newtheorem{ejemIntroMat}[teoIntroMat]{Ejemplo}
\newtheorem{obsIntroMat}[teoIntroMat]{Observaci\'on}

%-------------

Generalizando el Ejemplo~\ref{ejem:definiciones:escalar}, sobre un cuerpo
arbitrario $F$, si $n\geq 1$ es entero, definimos el \emph{producto escalar}
en $F^n$ por:%
\footnote{
	Como producto matricial, $x\cdot y=\trnsp x\,y$.
}
\begin{displaymath}
	x\,\cdot\,y\,=\,\sum_{i=1}^n\,x_iy_i
	\text{ ,}
\end{displaymath}
%
si $\trnsp x=(\lista x{n})$ e $\trnsp y=(\lista y{n})$. Si
$M\in\MM[n\times n](F)$ es una matriz con coeficientes en el cuerpo $F$,
\begin{displaymath}
	B(x,y) \,=\,x\,\cdot\,M\,y
\end{displaymath}
%
tambi\'en define una forma bilineal. Toda forma bilineal es de este tipo.

Sea $(V,B)$ un espacio bilineal de dimensi\'on $n\geq 1$ y sea
$\{\lista* e{n}\}$ una base del espacio vectorial. Usando la bilinealidad de
$B$,
\begin{equation}
	\label{eq:matrices:base}
	B(v,w)\,=\,\sum_{i,j=1}^n\,x_iy_i\,B(e^i,e^j)
	\text{ ,}
\end{equation}
%
si $v=x_ie^i$ y $w=y_je^j$.

\begin{defIntroMat}\label{def:matrices:matrices}
	La \emph{matriz} de $B$ en la base $\{\lista* e{n}\}$ es
	\begin{displaymath}
		M_{ij}\,=\,B(e^i,e^j)
		\text{ .}
	\end{displaymath}
	%
\end{defIntroMat}

Si $v\in V$, escribimos $\repr v$ para referirnos a la representaci\'on del
vector $v$ en una base. Con esta notaci\'on, se verifica que
\begin{displaymath}
	B(v,w)\,=\,\repr v\,\cdot\,M\,\repr w
	\text{ .}
\end{displaymath}
%

\begin{ejemIntroMat}\label{ejem:matrices:matrices}
	Determinar las matrices asociadas a los espacios de los ejemplos de la
	secci\'on \S~\ref{sec:intro:definiciones} en alguna base.
\end{ejemIntroMat}

\begin{teoIntroMat}\label{teo:matrices:matrices}
	La elecci\'on de una base determina una correspondencia (isomorfismo)
	entre (los espacios de) formas bilineales en $V$ y matrices en
	$\MM[n\times n](F)$.
\end{teoIntroMat}

\begin{ejemIntroMat}\label{ejem:matrices:pseudoeuclideos}
	En $\bb R^n$, si $p,q\geq 0$ y $p+q=n$, definimos
	\begin{displaymath}
		\langle x,y\rangle_{p,q}\,=\,x_1y_1\,+\,\cdots\,+\,x_py_p\,-\,
			x_{p+1}y_{p+1}\,-\,\cdots\,-\,x_ny_n
		\text{ .}
	\end{displaymath}
	%
	La matriz asociada a $\langle\cdot,\cdot,\rangle_{p,q}$ en la base
	can\'onica de $\bb R^n$ es
	\begin{displaymath}
		\begin{bmatrix} I_p & 0 \\ 0 & -I_q \end{bmatrix}
		\text{ .}
	\end{displaymath}
	%
	La forma, al igual que la matriz, es sim\'etrica. Denotamos estos
	espacios por $\bb R^{p,q}$. El espacio del Ejemplo~%
	\ref{ejem:definiciones:hiperbolico} coincide con $\bb R^{1,1}$.
	El espacio $\bb R^{3,1}$ se denomina \emph{espacio de Minkowski}.
	El espacio $\bb R^{n,0}$ es $\bb R^n$ con el producto escalar,
	usualmente llamado \emph{espacio euclideo}.
\end{ejemIntroMat}

\begin{teoIntroMat}\label{teo:matrices:dual}
	Sea $(V,B)$ un espacio bilineal y sea $\cal B=\{\lista* e{n}\}$ una
	base de $V$. La matriz asociada a $B$ con respecto a $\cal B$ coincide
	con la matriz asociada a la t.l. $R_B:\,V\rightarrow\dual V$ con
	respecto a $\cal B$ y su base dual.
\end{teoIntroMat}

\begin{proof}
	Sea $\repr\cdot:\,V\rightarrow F^n$ el isomorfismo dado por elegir la
	base $\cal B$ de $V$ y sea $\repr\cdot':\,\dual V\rightarrow F^n$ el
	isomorfismo correspondiente a elegir la base dual a $\cal B$ en
	$\dual V$. Sea $\dual{\cal B}=\{\lista\varepsilon{n}\}$ dicha base. La
	matriz de $R_B$ con respecto a estas bases tiene \emph{columnas}
	$\repr{R_B(e^j)}'$. Si
	\begin{displaymath}
		R_B(e^j)\,=\,c^i\varepsilon_i
		\text{ ,}
	\end{displaymath}
	%
	evaluando en $e^i$ recuperamos los coeficientes:
	\begin{displaymath}
		c^i\,=\,R_B(e^j)(e^i)\,=\,B(e^i,e^j)\,=\,M_{ij}
		\text{ .}
	\end{displaymath}
	%
\end{proof}

\begin{obsIntroMat}\label{obs:matrices:dual}
	Podr\'{\i}amos haber definido la matriz de $B$ por $N\,v\cdot w$. En
	ese caso, hubi\'esemos deducido que esta matriz coincide con la
	matriz de $L_B$.
\end{obsIntroMat}

\begin{teoIntroMat}\label{teo:matrices:simetria}
	Sea $(V,B)$ un espacio bilineal y sea $M$ la matriz asociada a $B$ en
	alguna base. Entonces,
	\begin{itemize}
		\item $B$ es sim\'etrica, si y s\'olo si $\trnsp M=M$;
		\item $B$ es antisim\'etrica, si y s\'olo si $\trnsp M=-M$;
		\item $M$ es alternada, si y s\'olo si $\trnsp M=-M$ y las
			coordenadas en la diagonal de $M$ son nulas.
	\end{itemize}
	%
\end{teoIntroMat}

\begin{teoIntroMat}\label{teo:matrices:cambio}
	Sea $(V,B)$ un espacio bilineal y sean $\repr[1]\cdot$ y
	$\repr[2]\cdot$ dos bases en $V$. Sea $C$ la matriz de cambio de base
	cuyas columnas representan los vectores de la segunda base en
	t\'erminos de la primera.%
	\footnote{
		$\repr[1] v\,=\,C\,\repr[2] v$, para $v\in V$.
	}
	Si $M$ es la matriz asociada a $B$ en la base $\repr[1]\cdot$, entonces
	la matriz en la base $\repr[2]\cdot$ es $\trnsp CMC$.
\end{teoIntroMat}

\begin{proof}
	La demostraci\'on depende de la identificaci\'on $\dual{F^n}=F^n$
	v\'{\i}a el producto escalar.
\end{proof}

\begin{defIntroMat}\label{def:matrices:equivalentes}
	Dos espacios bilineales $(V,B_V)$ y $(W,B_W)$ se dicen
	\emph{equivalentes}, si existe un isomorfismo $A:\,V\rightarrow W$ tal
	que
	\begin{displaymath}
		B_W(A\,v,A\,v_1)\,=\,B_V(v,v_1)
		\text{ ,}
	\end{displaymath}
	%
	para todo $v,v_1\in V$.%
	\footnote{
		En t\'erminos de las matrices asociadas, dos formas son
		equivalentes, si y s\'olo si existen bases de $V$ y de $W$ con
		respecto a las cuales las matrices correspondientes a $B_V$ y a
		$B_W$, $M$ y $N$, son equivalentes, es decir, existe $C$
		invertible tal que $M=\trnsp C\,N\,C$.
	}
\end{defIntroMat}

\begin{ejemIntroMat}\label{ejem:matrices:equivalentes}
	Las formas
	\begin{displaymath}
		v\,\cdot\,\begin{bmatrix} 0 & 1 \\ -1 & 0 \end{bmatrix}\,w
		\quad\text{y}\quad
		v\,\cdot\,\begin{bmatrix} 1/2 & 0 \\ 0 & 1/2 \end{bmatrix}\,w
	\end{displaymath}
	%
	en $\bb R^2$ son equivalentes v\'{\i}a el cambio de base
	\begin{displaymath}
		\begin{bmatrix} 1 & 1 \\ 1 & -1 \end{bmatrix}
		\text{ .}
	\end{displaymath}
	%
\end{ejemIntroMat}

\begin{defIntroMat}\label{def:matrices:discriminante}
	El \emph{discriminante} de una forma bilineal $B$ se define como
	el determinante de cualquier representaci\'on de $B$ como matriz.
\end{defIntroMat}

\begin{obsIntroMat}\label{obs:matrices:discriminante}
	Este valor est\'a determinado a menos de cuadrados. Si $M$ es la
	matriz de $B$ en alguna base, entonces $\discriminante(B)$ es la clase
	de $\det(M)$ en $F/{F^\times}^2$.%
	\footnote{
		$\discriminante(B):=0$, si $\det(M)=0$ en alguna (y, por lo
		tanto, en toda) base, o bien $\discriminante(B)$ es la clase
		de $\det(M)$ en el grupo $F^\times/{F^\times}^2$.
	}
\end{obsIntroMat}



\section{Formas bilineales no degeneradas}\label{sec:intro:nodegeneradas}
\theoremstyle{plain}
\newtheorem{teoIntroNodeg}{Teorema}[section]
\newtheorem{lemaIntroNodeg}[teoIntroNodeg]{Lema}

\theoremstyle{definition}
\newtheorem{defIntroNodeg}[teoIntroNodeg]{Definici\'on}
\newtheorem{obsIntroNodeg}[teoIntroNodeg]{Observaci\'on}
\newtheorem{ejemIntroNodeg}[teoIntroNodeg]{Ejemplo}
\newtheorem{ejerIntroNodeg}[teoIntroNodeg]{Ejercicio}

%-------------

\begin{teoIntroNodeg}\label{teo:nodegeneradas:nodegeneradas}
	Sea $(V,B)$ un espacio bilineal. Las siguientes condiciones son
	equivalentes:
	\begin{enumerate}[(i)]
		\item\label{item:nodegeneradas:matriz}
			con respecto a alguna base $\{\lista* e{n}\}$ de $V$,
			la matriz asociada $B(e^i,e^j)$ es invertible;
		\item\label{item:nodegeneradas:inyectiva}
			si $v\neq 0$, entonces $B(v_1,v)\neq 0$ para alg\'un
			$v_1\in V$;
		\item\label{item:nodegeneradas:sobre}
			todo elemento de $\dual V$ es de la forma $B(-,v)$
			para alg\'un $v\in V$;
		\item\label{item:nodegeneradas:biyectiva}
			todo elemeneto de $\dual V$ es de la forma $B(-,v)$
			para un \'unico $v\in V$.
	\end{enumerate}
	%
	En tal caso, \emph{toda} representaci\'on matricial de $B$ es
	invertible.
\end{teoIntroNodeg}

\begin{proof}
	% Ejercicio.
	Las condiciones \eqref{item:nodegeneradas:inyectiva},
	\eqref{item:nodegeneradas:sobre} y \eqref{item:nodegeneradas:biyectiva}
	hacen referencia a la inyectividad, sobreyectividad y biyectividad,
	respectivamente, de la transformaci\'on lineal
	$R_B:\,V\rightarrow\dual V$. Como $\dim\,V<\infty$, \'estas son
	equivalentes. La condici\'on \eqref{item:nodegeneradas:matriz}
	significa que la matriz asociada a $B$ en alguna base es invertible.
	Pero esto equivale a que $R_B$ sea un isomorfismo.
\end{proof}

\begin{defIntroNodeg}\label{def:nodegeneradas:nodegeneradas}
	Una forma bilineal $B$ en un espacio $V$ se dice \emph{no degenerada},
	si cumple cualquiera de las condiciones equivalentes enunciadas en el
	Teorema~\ref{teo:nodegeneradas:nodegeneradas}. En caso contrario, se
	dice que $B$ es \emph{degenerada}.
\end{defIntroNodeg}

\begin{obsIntroNodeg}\label{obs:nodegeneradas:nodegeneradas}
	El Teorema~\ref{teo:nodegeneradas:nodegeneradas} caracteriza las
	formas no degeneradas \emph{a derecha}, aquellas tales que $R_B$ es
	inyectiva (sobre o, equivalentemente, biyectiva). Ahora, si $M$ es una
	matriz, entonces $M$ es invertible, si y s\'olo si $\trnsp M$ lo es. Si
	$M$ es la matriz asociada a $B$ en alguna base, $\cal B$, entonces $M$
	es la matriz de la transformaci\'on lineal $R_B:\,V\rightarrow\dual V$,
	con respecto a dicha base en $V$ y su base dual en $\dual V$,
	$\dual{\cal B}$. Por otro lado, si $J:\,V\rightarrow\ddual V$ es el
	isomorfismo can\'onico
	\begin{equation}
		\label{eq:nodegeneradas:dobledual}
		J(v)(\varphi)\,=\,\varphi(v)
		\text{ ,}
	\end{equation}
	%
	la matriz $\trnsp M$ es la matriz de la transformaci\'on transpuesta
	$\dual{R_B}:\,\ddual V\rightarrow\dual V$, con respecto a
	$\dual{\cal B}$ y a la base $J(\cal B)$. Pero, v\'{\i}a la
	identificaci\'on \eqref{eq:nodegeneradas:dobledual}, $\dual{R_B}=L_B$.
	La codici\'on \eqref{item:nodegeneradas:matriz} significa que $R_B$ es
	biyectiva. Pero $R_B$ es biyectiva, si y s\'olo si $\dual{R_B}=L_B$ lo
	es. En definitiva, $B$ es no degenerada a derecha, si y s\'olo si lo es
	a izquierda. El Teorema~\ref{teo:nodegeneradas:nodegeneradas} garantiza
	que $V$ parametriza $\dual V$ v\'{\i}a $R_B$, si $B$ es no degenerada
	a derecha; todo elemento de $\dual V$ es de la forma
	$B(-,v)$. Pero, entonces, $L_B$ tambi\'en es un isomorfismo y todo
	elemento de $\dual V$ se puede escribir de la forma $B(v,-)$ para
	alg\'un $v\in V$.
\end{obsIntroNodeg}

Si $V$ es de dimensi\'on finita, $\ddual V=V$ (naturalmente isomorfos).
Adem\'as, por un argumento de dimensi\'on, $V\simeq\dual V$, es decir,
\emph{existe} un isomorfismo. Pero dicho isomorfismo no es can\'onico. Una
forma bilineal es, esencialmente, una t.l. $V\rightarrow\dual V$. Una forma
bilineal no degenerada es una elecci\'on de isomorfismo entre estos espacios.

Si $V=\{0\}$, entonces $\dual V\simeq V$, de manera \'unica. Paralelamente, hay
una \'unica forma bilineal en $V$. Por esta raz\'on, se considera que el
espacio cero con su \'unica forma bilineal es un espacio bilineal no
degenerado. Sin embargo, dicha forma no admite una matriz que la represente,
pues \'unica base en $\{0\}$ es vac\'{\i}a.

\begin{ejemIntroNodeg}\label{ejem:nodegeneradas:pseudoeuclideos}
	El producto escalar en $\bb R^n$, al igual que las formas
	$\langle\cdot,\cdot\rangle_{p,q}$ definidas en el Ejemplo~%
	\ref{ejem:matrices:pseudoeuclideos}, son no degeneradas; est\'an
	representadas por matrices invertibles en la base can\'onica.%
	\footnote{
		Aunque podr\'{\i}a ocurrir $\langle v,v\rangle_{p,q}=0$,
		para alg\'un $v\neq 0$.
	}
	Todas ellas parametrizan el dual del espacio vectorial
	$\bb R^n$, el mismo espacio, pero de distintas maneras. Si
	\begin{math}
		M=
		\left[\begin{smallmatrix}
			I_p & 0 \\ 0 & -I_q
		\end{smallmatrix}\right]
	\end{math}, entonces, usando que $M^{-1}=\trnsp M=M$,
	\begin{displaymath}
		\begin{aligned}
			\langle v,w\rangle_{p,q} & \,=\,v\,\cdot\,(M\,w)
				\,=\,(M\,v)\,\cdot\,w \quad\text{y} \\
			v\,\cdot\,w & \,=\,\langle v,M^{-1}\,w\rangle_{p,q}
				\,=\,\langle v,M\,w\rangle_{p,q}
				\,=\,\langle M\,v,w\rangle_{p,q}
			\text{ .}
		\end{aligned}
		%
	\end{displaymath}
	%
\end{ejemIntroNodeg}

\begin{ejemIntroNodeg}\label{ejem:nodegeneradas:alternadas}
	La forma bilineal alternada del Ejemplo~%
	\ref{ejem:definiciones:alternadas} es no degenerada, pues
	$\varphi(v)=0$, para todo $v\in V$ si y s\'olo si $\varphi=0$,
	y para todo $\varphi\in\dual V$, si y s\'olo si $v=0$.
\end{ejemIntroNodeg}

\begin{ejemIntroNodeg}\label{ejem:nodegeneradas:killing}
	Si $\frak g$ es un \'algebra de Lie de dimensi\'on finita y sea
	\begin{equation}
		\label{eq:nodegeneradas:killing}
		\killing(x,y)\,=\,\Traza(\adjoint[x]\,\adjoint[y])
	\end{equation}
	%
	su \emph{forma de Killing}. Es una forma bilineal sim\'etrica. Si el
	cuerpo de base es de caracter\'{\i}stica $0$, $\frak g$ es semisimple,
	si y s\'olo si $\killing$ es no degenerada.
\end{ejemIntroNodeg}

\begin{ejemIntroNodeg}\label{ejem:nodegeneradas:subespaciodegenerado}
	El espacio bilineal $\bb R^{2,1}$, por ejemplo, es no degenerado, como
	se mencion\'o en el Ejemplo~\ref{ejem:nodegeneradas:pseudoeuclideos}.
	Sin embargo, el plano $W$ generado por $v_1=(1,0,1)$ y por
	$v_2=(0,1,0)$ es degenerado con respecto a la restricci\'on de la forma
	$\langle\cdot,\cdot\rangle_{2,1}$. Por ejemplo, $v_1\in W^\perp$.
\end{ejemIntroNodeg}

\begin{obsIntroNodeg}\label{obs:nodegeneradas:subespaciodegenerado}
	Un espacio bilineal no degenerado, puede contener subespacios tales que
	la restricci\'on de la misma forma bilineal a dicho subespacio es
	degenerada, como ocurre en el Ejemplo~%
	\ref{ejem:nodegeneradas:subespaciodegenerado}. Esto no ocurre, si el
	cuerpo de base es $\bb R$ y la forma es definida positiva. La propiedad
	de una forma bilineal real de ser (semi) definida positiva es
	hereditaria.
\end{obsIntroNodeg}

\begin{teoIntroNodeg}\label{teo:nodegeneradas:perpendicular}
	Sea $(V,B)$ un espacio bilineal sim\'etrico o alternado.%
	\footnote{
		Es decir, un espacio en donde la relaci\'on de
		perpendicularidad es sim\'etrica (c.f. el Teorema~%
		\ref{teo:definiciones:perpendicular}).
	}
	\footnote{
		Si $B$ es arbitraria, no necesariamente sim\'etrica ni
		alternada, la relaci\'on $\perp$ deja de ser sim\'etrica y
		distinguimos entre $W^\lperp$ y $W^\rperp$ en el \'{\i}tem~%
		\eqref{item:perpendicular:complemento}. Esto da lugar a
		versiones a izquierda y a derecha de
		\eqref{item:perpendicular:complemento:b} y de
		\eqref{item:perpendicular:complemento:c}. Asumiendo dimensi\'on
		finita, se puede probar que
		\begin{displaymath}
			\begin{aligned}
			\text{\eqref{item:perpendicular:complemento:a}}
				& \,\Rightarrow\,\big(
			\text{\eqref{item:perpendicular:complemento:b}}_L
				\wedge
			\text{\eqref{item:perpendicular:complemento:b}}_R
				\big) \quad\text{que} \\
			\text{\eqref{item:perpendicular:complemento:b}}_L
				& \,\Rightarrow\,
			\text{\eqref{item:perpendicular:complemento:c}}_L
				\,\Rightarrow\,
			\text{\eqref{item:perpendicular:complemento:a}}
				\quad\text{e, ir\'onicamente, %
					sim\'etricamente, que} \\
			\text{\eqref{item:perpendicular:complemento:b}}_R
				& \,\Rightarrow\,
			\text{\eqref{item:perpendicular:complemento:c}}_R
				\,\Rightarrow\,
			\text{\eqref{item:perpendicular:complemento:a}}
			\end{aligned}
			%
		\end{displaymath}
		%
		La raz\'on es que no hay una distinci\'on entre degeneraci\'on
		a derecha y degeneraci\'on a izquierda.
	}
	\begin{enumerate}[(1)]
		\item\label{item:perpendicular:complemento}
			Si $W\subset V$ es un subespacio, las siguientes
			condiciones son equivalentes:
			\begin{enumerate}[(a)]
				\item\label{item:perpendicular:complemento:a}
					el subespacio $W$ es no degenerado;
				\item\label{item:perpendicular:complemento:b}
					$W\cap W^\perp=0$;
				\item\label{item:perpendicular:complemento:c}
					$V=W\oplus W^\perp$.
			\end{enumerate}
			%
		\item\label{item:perpendicular:dimension}
			Si $V$ es no degenerado, entonces
			$\dim\,W+\dim\,W^\perp=\dim\,V$ y $(W^\perp)^\perp=W$.
	\end{enumerate}
	%
	En particular, si $V$ es no degenerado, un subespacio es no degenerado,
	si y s\'olo si $W^\perp$ lo es.
\end{teoIntroNodeg}

\begin{proof}
	Que $B$ sea no degenerada a izquierda significa
	$w\neq 0\Rightarrow B(w,w_1)$ para alg\'un $w_1$. Esto implica
	$W\cap W^\lperp=0$ (notar que $W^\lperp$ es un subespacio de $V$, no
	de $W$).

	Ahora, vemamos que $W\cap W^\lperp=0$ implica $W\oplus W^\lperp=V$.
	Dado que la intersecci\'on de los subespacios es cero, s\'olo resta
	probar que $W+W^\lperp=V$, es decir que todo elemento de $V$ es suma
	de un elementon de $W$ m\'as uno en $W^\lperp$. Sea
	$L:\,W\rightarrow\dual W$ la t.l. $L(w)=B(w,-)|_W$. Su n\'ucleo,
	$W\cap W^\lperp$ es, por hip\'otesis, nulo. Por lo tanto, puesto que
	$\dim\,W<\infty$,%
	\footnote{
		Aqu\'{\i} usamos que $W$ tiene dimensi\'on finita, y no $V$.
	}
	vale que $\dim\,\dual W=\dim\,W$ y, entonces, $L$ es sobreyectiva.
	Dado $v\in V$, $B(v,-)|_W=B(w,-)|_W$, para alg\'un $w\in W$, y
	$v-w\in W^\lperp$.

	Finalmente, si $V=W\oplus W^\lperp$, entonces $B(w,w_1)=0$ para todo
	$w_1\in W$ implica que $w\in W^\lperp$ y $w=0$. En consecuencia, $B$ es
	no degenerada a izquierda.

	Para probar \eqref{item:perpendicular:dimension}, supongamos que
	$V$ es no degenerado, es decir $L_B:\,V\rightarrow\dual V$ es un
	isomorfismo. Por Hahn-Banach (?), la restricci\'on
	$\dual V\rightarrow\dual W$ es sobre. El n\'ucleo de la composici\'on
	es $W^\lperp$. As\'{\i},
	\begin{displaymath}
		V/W^\lperp\,\simeq\,\dual W
		\text{ .}
	\end{displaymath}
	%
	Calculando dimensiones, $\dim\,V-\dim\,W^\lperp=\dim\,W$.%
	\footnote{
		En particular, $\dim\,W^\lperp=\dim\,W^\rperp$.
	}
	Ahora, $W\subset (W^\lperp)^\rperp$. Por un argumento de dimensi\'on,
	$W=(W^\lperp)^\rperp$.%
	\footnote{
		En particular, $(W^\lperp)^\rperp=(W^\rperp)^\lperp$.
	}
	Por \eqref{item:perpendicular:complemento:b} (su versi\'on a izquierda
	y su versi\'on a derecha), cuando $V$ es no degenerado,
	$W=(W^\lperp)^\rperp=(W^\rperp)^\lperp$ y $W$ es no degenerado, si y
	s\'olo si $W^\lperp$ lo es, si y s\'olo si $W^\rperp$ lo es.
\end{proof}

\begin{ejemIntroNodeg}\label{ejem:nodegeneradas:subespaciodegenerado:bis}
	Siguiendo con el Ejemplo~\ref{ejem:nodegeneradas:subespaciodegenerado},
	si bien $W=\generado{(1,0,1),(0,1,0)}$ es degenerado con respecto a la
	restricci\'on de $\langle\cdot,\cdot\rangle_{2,1}$, se puede ver que
	\begin{displaymath}
		W^\perp\,=\,\generado{(1,0,1)}\,\subset\,W
		\text{ .}
	\end{displaymath}
	%
	Entonces $\bb R^{2,1}$ no es suma directa de $W$ y $W^\perp$. Sin
	embargo, $\dim\,W+\dim\,W^\perp=3$. Esto es consistente con el hecho de
	que $\bb R^{2,1}$ es no degenerado pero la restricci\'on de la forma al
	subespacio $W$ s\'{\i} lo es.
\end{ejemIntroNodeg}

\begin{ejemIntroNodeg}\label{ejem:nodegeneradas:complemento}
	En $V=\bb R^2$ con la forma $B((x,y),(x_1,y_1))=xx_1$, el vector
	$(0,1)$ es perpendicular a todo el espacio. Es decir, $V^\perp\neq 0$ y
	la forma es degenerada. Sin embargo, el subespacio $W=\generado{(1,0)}$
	es no degenerado con respecto a la restricci\'on $B|_W$. Por lo tanto,
	$\bb R^2=W\oplus W^\perp$. Efectivamente, $W^\perp=\generado{(0,1)}$.
	Adem\'as, se verifica que $(W^\perp)^\perp=\bb R^2\neq W$, que $W$ es
	no degenerado, pero $W^\perp$ es degenerado.
\end{ejemIntroNodeg}

\begin{teoIntroNodeg}\label{teo:nodegeneradas:propiedades}
	Sea $(V,B)$ un espacio bilineal no degenerado. Entonces,
	\begin{enumerate}[(i)]
		\item\label{item:propiedades:hiperplanos}
			todo hiperplano en $V$ es de la forma
			$\{w\,:\,w\perp v\}$ para alg\'un $v\neq 0$ y de la
			forma $\{w\,:\,v_1\perp w\}$ para alg\'un $v_1\neq 0$;
		\item\label{item:propiedades:vectores}
			si $B(v,w)=B(v,w_1)$ para todo $v\in V$, entonces
			$w=w_1$;
		\item\label{item:propiedades:transformaciones}
			si $B(v,A\,w)=B(v,A_1\,w)$ para todo $v,w\in V$,
			entonces $A=A_1$;
		\item\label{item:propiedades:bilineales}
			toda forma bilineal en $V$ es de la forma
			$B(v,A\,w)$ para alguna t.l. $A:\,V\rightarrow V$.%
			\footnote{
				El hecho de que toda forma bilineal admite
				una representaci\'on matricial y que, tomando
				una base, se puede expresar en t\'erminos del
				producto escalar es un caso particular de este
				resultado.
			}
	\end{enumerate}
	%
\end{teoIntroNodeg}

\begin{proof}
	Para \ref{item:propiedades:hiperplanos}, usar que los hiperplanos son
	n\'ucleos de funcionales lineales.
\end{proof}

Si $(V,B)$ es un espacio bilineal no degenerado y $A:\,V\rightarrow V$ es una
transformaci\'on lineal, la funci\'on
\begin{displaymath}
	\tilde B(v,w)\,=\,B(A\,v,w)
\end{displaymath}
%
es una forma bilineal en $V$. Existe, por el Teorema~%
\ref{teo:nodegeneradas:propiedades}, una \emph{\'unica} transformaci\'on lineal
$\adjnt A:\,V\rightarrow V$ tal que
\begin{equation}
	\label{eq:nodegeneradas:adjunta}
	B(A\,v,w)\,=\,B(v,\adjnt A\,w)
	\text{ ,}
\end{equation}
%
para todo $v,w\in V$.

\begin{defIntroNodeg}\label{def:nodegeneradas:adjunta}
	Si $(V,B)$ es un espacio bilineal no degenerado y $A:\,V\rightarrow V$
	es una t.l., la \'unica t.l. $\adjnt A:\,V\rightarrow V$ que verifica
	\eqref{eq:nodegeneradas:adjunta} se denomina \emph{adjunta} de $A$, con
	respecto a $B$.
\end{defIntroNodeg}

\begin{ejemIntroNodeg}\label{ejem:nodegeneradas:adjunta:escalar}
	En $F^n$ con el producto escalar, la adjunta de una transformaci\'on
	lineal representada por una matriz $A\in\MM[n\times n](F)$ en la base
	can\'onica es la transformaci\'on representada por $\trnsp A$ en la
	misma base.
\end{ejemIntroNodeg}

\begin{ejemIntroNodeg}\label{ejem:nodegeneradas:adjunta}
	En $\bb R^2$ con la forma bilineal
	\begin{displaymath}
		B\Big( \begin{bmatrix} x \\ y \end{bmatrix},
			\begin{bmatrix} x_1 \\ y_1 \end{bmatrix}\Big)\,=\,
			\begin{bmatrix} x \\ y \end{bmatrix}\,\cdot\,
			\begin{bmatrix} 3 & \\ & -2 \end{bmatrix}\,
				\begin{bmatrix} x_1 \\ y_1 \end{bmatrix}
		\text{ ,}
	\end{displaymath}
	%
	la adjunta de una matriz est\'a dada por:
	\begin{displaymath}
		\adjnt{\begin{bmatrix} a & b \\ c & d \end{bmatrix}} \,=\,
			\begin{bmatrix}
				a & -(2/3)\,c \\ -(3/2)\,b & d
			\end{bmatrix}
		\text{ .}
	\end{displaymath}
	%
\end{ejemIntroNodeg}

\begin{ejemIntroNodeg}\label{ejem:nodegeneradas:adjunta:degenerada}
	La forma del Ejemplo~\ref{ejem:nodegeneradas:complemento} es
	degenerada. Se verifica que, si
	\begin{math}
		A=\sbmatrix{ a & b \\ c & d }
	\end{math},
	\begin{displaymath}
		B\Big(A\,\begin{bmatrix} 1 \\ 0 \end{bmatrix},
			\begin{bmatrix} 0 \\ 1 \end{bmatrix}\Big) \,=\,b
			\quad\text{y que}\quad
		B\Big(\begin{bmatrix} 1 \\ 0 \end{bmatrix},A'\,
			\begin{bmatrix} 0 \\ 1 \end{bmatrix}\Big) \,=\,0
		\text{ ,}
	\end{displaymath}
	%
	para cualquier matriz $A'$. En particular, si $b\neq 0$, $A$ no posee
	adjunta con respecto a $B$.
\end{ejemIntroNodeg}

\begin{teoIntroNodeg}\label{teo:nodegeneradas:adjunta:matriz}
	Sea $(V,B)$ un espacio bilineal no degenerado y sea
	$A:\,V\rightarrow V$ una t.l. Fijemos una base
	$\repr{\cdot}:\,V\rightarrow F^n$ de $V$. Sea $M$ la matriz asociada a
	$B$ con respecto a esta base y sean $\repr A$ y $\repr{\adjnt A}$ las
	matrices de $A$ y de $\adjnt A$, respectivamente, en la base elegida.
	Entonces,
	\begin{displaymath}
		\repr{\adjnt A}\,=\,M^{-1}\,\trnsp{\repr A}\,M
		\text{ .}
	\end{displaymath}
	%
\end{teoIntroNodeg}

\begin{proof}
	Comprobar que se cumple $R_B\,\adjnt A=\dual A\,R_B$ y pasar a la
	representaci\'on matricial. Se puede dar otra demostraci\'on, usando la
	identidad que define la adjunta, \eqref{eq:nodegeneradas:adjunta},
	pasando a la representaci\'on matricial de dicha identidad y usando que
	el producto escalar en $F^n$ es una forma bilineal no degenerada.
\end{proof}

La noci\'on de transformaci\'on dual-transpuesta tiene sentido incluso para
transformaciones lineales que no son endomorfismos (por ejemplo, en el caso
$V\rightarrow\dual V$). Si $A:\,V\rightarrow W$ es una transformaci\'on lineal
entre espacios vectoriales y cada uno de ellos tiene asociada una forma
bilineal no degenerada, deber\'{\i}a ser posible dar una noci\'on de
transformaci\'on adjunta $\adjnt A:\,W\rightarrow V$, relacion\'andola con la
transpuesta $\dual A:\,\dual W\rightarrow\dual V$.
\begin{center}
	\begin{tikzcd}
		V \arrow[d, "A"'] \\ W
	\end{tikzcd}
	\qquad
	\begin{tikzcd}
		\dual V & V \arrow[l, "R"'] \\
		\dual W\arrow[u,"\dual A"] &
			W \arrow[l,"R"] \arrow[u, dashed, "\adjnt A"']
	\end{tikzcd}
\end{center}

\begin{defIntroNodeg}\label{def:nodegeneradas:perpendicular}
	Si $A:\,V\rightarrow W$ es una transformaci\'on lineal entre espacios
	bilineales, decimos que $A$ \emph{preserva la relaci\'on de %
	ortogonalidad}, si
	\begin{displaymath}
		v\,\perp\,v_1\,\Rightarrow\,A\,v\,\perp\,A\,v_1
		\text{ ,}
	\end{displaymath}
	%
	para todo $v,v_1\in V$.%
	\footnote{
		No asumimos que $\dim\,V=\dim\,W$.
	}
\end{defIntroNodeg}

\begin{teoIntroNodeg}\label{teo:nodegeneradas:perpendicularidad}
	Dados espacios bilineales no degenerados, $(V,B_V)$ y $(W,B_W)$, y una
	t.l. $A:\,V\rightarrow W$, las siguientes propiedades son equivalentes:
	\begin{enumerate}[(i)]
		\item\label{item:perpendicularidad:preserva}
			$A$ preserva la relaci\'on de ortogonalidad;
		\item\label{item:perpendicularidad:constante}
			$B_W(A\,v,A\,v_1)=c\,B_V(v,v_1)$, para todo
			$v,v_1\in V$, para cierta constante $c\in F$;
		\item\label{item:perpendicularidad:adjunta}
			$\adjnt A\,A=c\,\id[V]$, para cierta constante
			$c\in F$.
	\end{enumerate}
	%
\end{teoIntroNodeg}

\begin{obsIntroNodeg}\label{obs:nodegeneradas:perpendicularidad}
	En particular, del Teorema~\ref{teo:nodegeneradas:perpendicularidad} se
	deduce que, si $A:\,V\rightarrow W$ preserva las formas bilineales, es
	decir, si $B_W(A\,v,A\,v_1)=B_V(v,v_1)$, para todo $v,v_1\in V$,
	entonces $A$ tiene inversa a izquierda; la inversa a izquierda est\'a
	dada pro $\adjnt A$. Adem\'as, si $W=V$, $A^{-1}$ existe y es igual a
	$\adjnt A$ y, por el \teoname~\ref{teo:nodegeneradas:adjunta:matriz},
	si $M$ es la matriz asociada a $B$ en una base,
	$M=\trnsp{\repr A}\,M\,\repr A$, es decir, $\repr A$, como matriz de
	cambio de base, no afecta a $M$.
\end{obsIntroNodeg}

\begin{lemaIntroNodeg}\label{lema:nodegeneradas:perpendicularidad}
	Sea $V$ un $F$-e.v. y sea $T:\,V\rightarrow V$ una t.l. Entonces,
	\begin{enumerate}[(1)]
		\item\label{item:perpendicularidad:preserva:rectas}
			si, para toda recta $L\subset V$ que pasa por el
			origen, $T(L)\subset L$, entonces $T$ es una homotecia;
		\item\label{item:perpendicularidad:preserva:hiperplanos}
			si, para todo hiperplano $H\subset V$ que pasa por el
			origen, $T(H)\subset H$, entonces $T$ es una homotecia.
	\end{enumerate}
	%
\end{lemaIntroNodeg}

\begin{proof}[Demostraci\'on del Teorema~%
	\ref{teo:nodegeneradas:perpendicularidad}]
	Asumiendo \eqref{item:perpendicularidad:preserva}, podemos afirmar que,
	para todo $v,v_1\in V$, si $B_V(v,v_1)=0$, entonces
	$B_W(A\,v,A\,v_1)=0$ y $B_V(v,\adjnt A\,A\,v_1)=0$. En particular,
	la t.l. $\adjnt A\,A:\,V\rightarrow V$ preserva todos los hiperplanos.
	Apelando al Lema~\ref{lema:nodegeneradas:perpendicularidad}, deducimos
	\eqref{item:perpendicularidad:constante}.

	Las afirmaciones \eqref{item:perpendicularidad:constante} y
	\eqref{item:perpendicularidad:adjunta} son equivalentes porque $B_V$ es
	no degenerada.
\end{proof}

\begin{ejerIntroNodeg}\label{ejer:nodegeneradas:adjunta:doble}
	?`Para qu\'e t.l. $A$ se cumple que $\aadjnt A=A$?
\end{ejerIntroNodeg}

\begin{ejerIntroNodeg}\label{ejer:nodegeneradas:dimension}
	Sea $V$ un espacio sim\'etrico o alternado y sea $W\subset V$ un
	subespacio tal que $\dim\,W+\dim\,W^\perp=\dim\,V$. Entonces, si
	$U\subset W^\perp$ es tal que $U+W=V$, $U=W^\perp$.
\end{ejerIntroNodeg}

\begin{ejerIntroNodeg}\label{ejer:nodegeneradas:anulador}
	Si $W\subset V$ es un subespacio ($\dim\,V<\infty$) y
	$W'=\{\varphi\in\dual V\,:\,\varphi(W)=0\}$, entonces
	$\dim\,W'+\dim\,W=\dim V$.
\end{ejerIntroNodeg}

\begin{ejerIntroNodeg}\label{ejer:nodegeneradas:perfecto}
	Una funci\'on bilineal $B:\,V\times W\rightarrow F$ se dice
	\emph{perfecta},%
	\footnote{
		O \emph{pairing perfecto}.
	}
	si
	\begin{displaymath}
		\begin{aligned}
			\big(v\,\mapsto\,B(v,-)\big) & \,:\,
				V\,\rightarrow\,\dual W
			\quad\text{y} \\
			\big(w\,\mapsto\,B(-,w)\big) & \,:\,
				W\,\rightarrow\,\dual V
		\end{aligned}
		%
	\end{displaymath}
	%
	son isomorfismos. Dada una funci\'on bilineal $B$ en $V\times W$ y
	dado un subespacio $U\subset V$, se define
	\begin{math}
		U^\perp=\{w\in W\,:\,B(U,w)=0\}
	\end{math}. Si $B$ es perfecta, entonces la funci\'on inducida
	\begin{displaymath}
		U\,\times\,\big(W/U^\perp\big)\,\rightarrow\,F
	\end{displaymath}
	%
	es perfecta.
\end{ejerIntroNodeg}

\begin{ejerIntroNodeg}\label{ejer:nodegeneradas:adjunta:propiedades}
	Si $B$ es una forma bilineal no degenerada en $V$, entonces
	\begin{itemize}
		\item $\adjnt{(A_1+A_2)}=\adjnt{A_1}+\adjnt{A_2}$,
		\item $\adjnt{(c\,A)}=c\,\adjnt A$,
		\item $\adjnt{\id[V]}=\id[V]$,
		\item $\adjnt{(A_1\,A_2)}=\adjnt{A_2}\,\adjnt{A_1}$,
		\item $\adjnt{(A^{-1})}=(\adjnt A)^{-1}$,
		\item $\det\,\adjnt A=\det\,A$, $\Traza(\adjnt A)=\Traza(A)$ y
			$\caracteristico[{\adjnt A}]=\caracteristico[A]$.
	\end{itemize}
	%
\end{ejerIntroNodeg}



\section{Bases ortogonales}\label{sec:intro:ortogonales}
\theoremstyle{plain}
\newtheorem{teoIntroOrto}{Teorema}[section]
\newtheorem{lemaIntroOrto}[teoIntroOrto]{Lema}

\theoremstyle{definition}
\newtheorem{defIntroOrto}[teoIntroOrto]{Definici\'on}
\newtheorem{ejemIntroOrto}[teoIntroOrto]{Ejemplo}

%-------------

Fijamos un espacio bilineal \emph{sim\'etrico} $(V,B)$ sobre un cuerpo $F$.

\begin{defIntroOrto}\label{def:ortogonales:base}
	Decimos que un subconjunto $S\subset V$ es \emph{ortogonal} con
	respecto a $B$, si $v\perp w$ para todo $v,w\in S$, $v\neq w$. En
	particular, una \emph{base ortogonal} de $V$ es una base
	$\{\lista* e{n}\}$ de $V$ tal que $e^i\perp e^j$ si $i\neq j$.
\end{defIntroOrto}

\begin{ejemIntroOrto}\label{ejem:ortogonales:escalar}
	La base can\'onica de $F^n$ es una base ortogonal para la forma
	bilineal sim\'etrica dada por el producto escalar. En esta base, la
	matriz de la forma bilineal es la matriz identidad, que es diagonal.
\end{ejemIntroOrto}

\begin{ejemIntroOrto}\label{ejem:ortogonales:alternada}
	La forma $B(v,w)=v\cdot\sbmatrix{ & 1 \\ 1 & }\,w$ en $\bb R^2$ es
	sim\'etrica. La base $\{\sbmatrix{ 1 \\ 0 }, \sbmatrix{ 0 \\ 1 }\}$ no
	es ortogonal. De hecho, no hay bases ortogonales para esta forma
	bilineal que contengan al vector $\sbmatrix{ 1 \\ 0 }$. La base
	$\{\sbmatrix{ 1 \\ 1 },\sbmatrix{ 1 \\ -1 }\}$ es ortogonal con
	respecto a $B$.

	Estas mismas observaciones son ciertas, si se reemplaza $\bb R$ por un
	cuerpo de caracter\'{\i}stica distinta de $2$. Si el cuerpo de base es
	de caracter\'{\i}stica $2$, no hay base ortogonal para la forma $B$. En
	este \'ultimo caso, $B$ no s\'olo es sim\'etrica, sino tambi\'en es
	alternada.
\end{ejemIntroOrto}

\begin{lemaIntroOrto}\label{lema:ortogonales:diagonal}
	Si la caracter\'{\i}stica de $F$ es distinta de $2$ y $B$ no es
	id\'enticamente cero, entonces existe \emph{alg\'un} $v\in V$ tal que
	$B(v,v)\neq 0$.
\end{lemaIntroOrto}

\begin{proof}
	Ejercicio.
\end{proof}

\begin{lemaIntroOrto}\label{lema:ortogonales:complemento}
	Si $v\in V$ es tal que $B(v,v)\neq 0$, entonces
	$V=\generado v\oplus v^\perp$ y la suma es ortogonal. Si $V$ es no
	degenerado, entonces $v^\perp$ es no degenerado.
\end{lemaIntroOrto}

\begin{proof}
	Ejercicio.
\end{proof}

Notar que, si $B(v,v)\neq 0$ y $v_1\in V$, entonces $B(v_1,v)=c\,B(v,v)$ para
cierta constante $c\in F$ y $v_1-c\,v\in v^\perp$. El Lema~%
\ref{lema:ortogonales:complemento} es v\'alido en cualquier
caracter\'{\i}stica.

\begin{teoIntroOrto}\label{teo:ortogonales:base}
	Si la caracter\'{\i}stica de $F$ es distinta de $2$, entonces existe
	una base ortogonal para $(V,B)$.
\end{teoIntroOrto}

\begin{proof}
	Si $a=B(v,v)\neq 0$, entonces podemos elegir $v$ como primer elemento
	de la base y buscar una base ortogonal para el complemento
	$v^\perp$. Con respecto a esta base,
	\begin{displaymath}
		\repr B \,=\,
			\begin{bmatrix}
				a & \\
				& \repr{B|_{v^\perp}}
			\end{bmatrix}
		\text{ .}
	\end{displaymath}
	%
	El valor $a$ aparece como uno (el primero) de los coeficientes y el
	vector $v$ hallado es el primer elemento de la base.
\end{proof}

El Teorema~\ref{teo:ortogonales:base} es v\'alido incluso si el espacio no es
no degenerado. Una base ortogonal es, esencialmente, una descomposici\'on de
$V$ en suma ortogonal de subespacios de dimensi\'on $1$:
\begin{displaymath}
	V\,=\,W_1\,\oplus\,\cdots\,\oplus\,W_n
	\text{ ,}
\end{displaymath}
donde $W_i\perp W_j$, si $i\neq j$. La matriz asociada a una forma bilineal en
una base ortogonal es diagonal (en particular, es sim\'etrica y $B$ es,
\emph{a fortiori}, sim\'etrica). Si $\{\lista* e{n}\}$ es una base ortogonal,
entonces $V$ es no degenerado, si y s\'olo si $e^i\not\perp e^i$ para todo $i$.


\section{Bases simpl\'ecticas}\label{sec:intro:simplecticas}
\theoremstyle{plain}
\newtheorem{teoIntroSimp}{Teorema}[section]
\newtheorem{coroIntroSimp}[teoIntroSimp]{Corolario}
\newtheorem{lemaIntroSimp}[teoIntroSimp]{Lema}

\theoremstyle{definition}
\newtheorem{defIntroSimp}[teoIntroSimp]{Definici\'on}
\newtheorem{obsIntroSimp}[teoIntroSimp]{Observaic\'on}
\newtheorem{ejemIntroSimp}[teoIntroSimp]{Ejemplo}
\newtheorem{ejerIntroSimp}[teoIntroSimp]{Ejercicio}

%-------------

Fijamos un espacio bilineal \emph{alternado} $(V,B)$ sobre un cuerpo $F$.

\begin{teoIntroSimp}\label{teo:simplecticas:dimension}
	Si $(V,B)$ es un espacio bilineal alternado no degenerado, entonces
	$\dim\,V$ es par.
\end{teoIntroSimp}

\begin{proof}
	Asumiendo que $B$ es alternada, si $M$ es la matriz asociada a $B$ en
	alguna base, por el Teorema~\ref{teo:matrices:simetria},
	$\trnsp M=-M$ (y las coordenadas de la diagonal son nulas). Tomando
	determinantes,
	\begin{math}
		\det\,M=(-1)^{\dim\,V}\,\det M
	\end{math}. Si $B$ es no degenerada, $\det\,M\neq 0$. Si
	$\car F\neq 2$, $-1\neq 1$. De esto se deduce el resultado en el caso
	en que la caracter\'{\i}stica del cuerpo de base es distinta de $2$.

	El siguiente argumento es v\'alido en cualquier caracter\'{\i}stica.
	Supongamos que $B$ es alternada. Si $\dim\,V=1$, entonces $B=0$, con lo
	cual no puede ser no degenerada. Supongamos, entonces, que
	$\dim\,V>2$ y que $B$ es no degenerada.

	Si $v\in V$ no es el vector cero, $B(v,-)\neq 0$ en $\dual V$.%
	\footnote{
		$L_B$ es inyectiva.
	}
	Entonces, $B(v,w)=1$ para cierto $w\in V$.%
	\footnote{
		Por linealidad. Toda funcional $F$-lineal $V\rightarrow F$ no
		nula es una funci\'on sobreyectiva.
	}
	Si $U=\generado{v,w}$, entonces $\dim\,U=2$%
	\footnote{
		El subconjunto $\{v,w\}$ es l.i. por alternancia de $B$.
	}
	y, con respecto a la base, $\{v,w\}$, $B|_U$ est\'a representada por la
	matriz $\sbmatrix{ & 1 \\ -1 & }$. Esta matriz tiene determinante $1$,
	es invertible y $B|_U$ es no degenerada. Por el Teorema~%
	\ref{teo:nodegeneradas:perpendicular}, $V=U\oplus U^\perp$. Adem\'as,
	como $V$ es no degenerado y $U$ es un subespacio no degenerado,
	$U^\perp$ tambi\'en es no degenerado. Inducci\'on.
\end{proof}

\begin{obsIntroSimp}\label{obs:simplecticas:dimension}
	Si $(V,B)$ es un espacio bilineal sim\'etrico no degenerado, el Lema~%
	\ref{lema:ortogonales:complemento} garantiza, dado cualquier $v\in V$
	tal que $B(v,v)\neq 0$, la descomposici\'on ortogonal
	$V=\generado v\oplus v^\perp$; los subespacios $\generado v$ y
	$v^\perp$ son no degenerados. An\'alogamente, si $(V,B)$ es bilineal
	alternado no degenerado, la demostraci\'on del Teorema~%
	\ref{teo:simplecticas:dimension}, muestra que, dado cualquier $v\in V$,
	existe otro vector $w\in V$ tal que
	\begin{itemize}
		\item $B(v,w)=1$,
		\item $V=U\oplus U^\perp$, si $U=\generado{v,w}$, y
		\item $U$ y $U^\perp$ son subespacios no degenerados.
	\end{itemize}
	%
\end{obsIntroSimp}

\begin{defIntroSimp}\label{def:simplecticas:base}
	Sea $(V,B)$ una espacio alternado no degenerado. Si $\dim\,V=2m\geq 2$,
	una \emph{base simpl\'ectica} es una base
	\begin{math}
		\big\{e_1,\,f_1,\,\dots,\,e_m,\,f_m\big\}
	\end{math} que cumple:
	\begin{itemize}
		\item $B(e_i,f_i)=1$ y
		\item $U_i=\generado{e_i,f_i}$ son perpendiculares entre
			s\'{\i}.
	\end{itemize}
	%
\end{defIntroSimp}

\begin{obsIntroSimp}\label{obs:simplecticas:bases}
	Tambi\'en se llama base simpl\'ectica a cualquier base cuyos vectores
	cumplen con las dos propiedades de la Definici\'on~%
	\ref{def:simplecticas:base}, aunque est\'en ordenados de otra manera.
	Hay dos o tres maneras est\'andar de ordenarlas:
	\begin{enumerate}[(I)]
		\item\label{item:simplecticas:base:numerica}
			con el orden $e_1,\,f_1,\,\dots,\,e_m,\,f_m$, la matriz
			asociada a $B$ es:
			\begin{displaymath}
				\begin{bmatrix}
					0 & 1 & & & & & \\
					-1 & 0 & & & & & \\
					& & 0 & 1 & & & \\
					& & -1 & 0 & & & \\
					& & & & \ddots & & \\
					& & & & & 0 & 1 \\
					& & & & & -1 & 0
				\end{bmatrix}
				\text{ ,}
			\end{displaymath}
			%
		\item\label{item:simplecticas:base:alfabetica}
			con el orden $e_1,\,\dots,\,e_m,\,f_1,\,\dots,\,f_m$,
			la matriz asociada es:
			\begin{displaymath}
				\begin{bmatrix}
					& & & 1 & & \\
					& & & & \ddots & \\
					& & & & & 1 \\
					-1 & & & & & \\
					& \ddots & & & & \\
					& & -1 & & &
				\end{bmatrix}
				\quad\text{y}
			\end{displaymath}
			%
		\item\label{item:simplecticas:base:invertida}
			con el orden $e_1,\,\dots,\,e_m,\,f_m,\,\dots,\,f_1$,
			la matriz asociada es:
			\begin{displaymath}
				\begin{bmatrix}
					& & & & & 1 \\
					& & & & \iddots & \\
					& & & 1 & & \\
					& & -1 & & & \\
					& \iddots & & & & \\
					-1 & & & & &
				\end{bmatrix}
				\text{ .}
			\end{displaymath}
			%
	\end{enumerate}
	%
\end{obsIntroSimp}

\begin{ejerIntroSimp}\label{ejer:simplecticas:base}
	Hallar una f\'ormula para $B$ en coordenadas en la base simpl\'ectica
	con el orden \eqref{item:simplecticas:base:alfabetica} y relacionarla
	con la forma del Ejemplo~\ref{ejem:definiciones:determinante}.
\end{ejerIntroSimp}

\begin{obsIntroSimp}\label{obs:simplecticas:base}
	En un espacio sim\'etrico no degenerado, cualquier vector $v\in V$ no
	nulo que cumpla $B(v,v)\neq 0$ es parte de una base ortogonal. En un
	espacio alternado no degenerado, cualquier vector no nulo es parte de
	una base simpl\'ectica.
\end{obsIntroSimp}

\begin{coroIntroSimp}\label{coro:simplecticas:dimension}
	Todo espacio alternado no degenerado admite una base simpl\'ectica. Dos
	espacios alternados no degenerados de la misma dimensi\'on son
	equivalentes.
\end{coroIntroSimp}

En general, si $V^\perp\neq 0$, elegimos un complemento directo arbitrario
$W\subset V$, de manera que $V=W\oplus V^\perp$. Como $W\cap V^\perp=0$, $B|_W$
es no degenerada. Eligiendo una base simpl\'ectica para $W$ y completando con
una base arbitraria de  $V^\perp$, la forma alternada $B$ tiene asociada una
matriz de la forma
\begin{displaymath}
	\begin{bmatrix}
		0 & I_r & 0 \\
		-I_r & 0 & 0 \\
		0 & 0 & 0
	\end{bmatrix}
	\text{ ,}
\end{displaymath}
%
donde $2r=\dim\,W$. El valor de $r$ no depende de la elecci\'on de complemento
$W$: $\dim\,W=\dim (V/V^\perp)$ En definitiva, la clase de equivalencia de un
espacio bilineal alternado est\'a determinada por:
\begin{itemize}
	\item la dimensi\'on del espacio, $\dim\,V=n$, y
	\item su ``grado de degeneraci\'on'', $\dim\,V^\perp=n-2r$.
\end{itemize}
%

\begin{ejerIntroSimp}\label{ejer:simplecticas:adjunta}
	Hallar la adjunta de una matriz
	\begin{math}
		\sbmatrix{ A & B \\ C & D }\in\MM[2m\times 2m](F)
	\end{math}
	con respecto a la forma bilineal alternada en $F^{2m}$ representada por
	\eqref{item:simplecticas:base:alfabetica}.
\end{ejerIntroSimp}

\begin{ejerIntroSimp}\label{ejer:simplecticas:dual}
	Sea $B$ la forma alternada del Ejemplo~%
	\ref{ejem:definiciones:alternadas} en $V\oplus\dual V$. Probar que, si
	$\{\lista* e{m}\}$ es una base de $V$ y $\{\lista\varepsilon{m}\}$ es
	la base dual en $\dual V$, entonces
	$\{e^1,\,\varepsilon_1,\,\dots,\,e^m,\,\varepsilon_m\}$ es una base
	simpl\'ectica para $B$ con respecto a la cual la matriz asociada es
	\eqref{item:simplecticas:base:numerica}.
\end{ejerIntroSimp}

\begin{ejerIntroSimp}\label{ejer:simplecticas:parias}
	Sea $(V,B)$ un espacio bilineal sim\'etrico no degenerado sobre un
	cuerpo de caracter\'{\i}stica $2$. Probar que $(V,B)$ admite una base
	ortogonal, si y s\'olo si $B$ \emph{no es} alternada.%
	\hint{
		Sin p\'erdida de generalidad, asumir que $\dim\,V\geq 2$. De
		acuerdo con el comentario despu\'es del Teorema~%
		\ref{teo:ortogonales:base}, la condici\'on es necesaria. Para
		ver que es suficiente con no ser alternada, elegir $v_0\in V$
		tal que $a=B(v_0,v_0)\neq 0$. Notar que, por el Lema~%
		\ref{lema:ortogonales:complemento}, $v_0^\perp$ es no
		degenerado. Si, en este subespacio, $B$ no es alternada, el
		resultado se deduce por un argumento inductivo. Si, en cambio,
		$B$ es alternada en $v_0^\perp$, existen $e,f\in v_0^\perp$
		tales que $B(e,f)=1$ y $B(e,e)=B(f,f)=0$ (parte de una base
		simpl\'ectica). Probar que $v_1:=v_0+e+f$ verifica
		$B(v_1,v_1)\neq 0$ y que $B$ no es alternada en $v_1^\perp$.
	}
\end{ejerIntroSimp}

\paragraph{El \emph{pfaffiano}}
Hay una manera gen\'erica de determinar una ra\'{\i}z cuadrada del determinante
de una matriz alternada.

\begin{lemaIntroSimp}\label{lema:simplecticas:pfaff}
	El determinante de una matriz alternada invertible con coeficientes en
	un cuerpo $F$ es un cuadrado perfecto no nulo.
\end{lemaIntroSimp}

\begin{proof}
	Si $M$ es la matriz y es de tama\~no $n\times n$, por el Corolario~%
	\ref{coro:simplecticas:dimension}, $n=2m$, $m\geq 1$, y
	\begin{math}
		\trnsp C\,M\,C=\sbmatrix{ & I_m \\ -I_m & }
	\end{math}, para cierta $C$ invertible.
\end{proof}

\begin{ejemIntroSimp}\label{ejem:simplecticas:pfaff}
	Si $n=2$,
	\begin{math}
		\svmatrix{ & x \\ -x & }=x^2
	\end{math}. Si $n=4$,
	\begin{displaymath}
		\begin{vmatrix}
			& x & y & z \\
			-x & & a & b \\
			-y & -a & & c \\
			-z & -b & -c &
		\end{vmatrix}
		\,=\,(xc-yb+az)^2
		\text{ .}
	\end{displaymath}
	%
\end{ejemIntroSimp}

Gen\'ericamente, Si $M(x_{ij})$ denota la matriz alternada gen\'erica de
tama\~no $n\times n$, $n=2m$ par, sobre el cuerpo $\bb Q(x_{ij})$, su
determinante es un polinomio no nulo con coeficientes en $\bb Z$.%
\footnote{
	Especializar en la matriz \eqref{item:simplecticas:base:numerica}, por
	ejemplo.
}
En particular, $M(x_{ij})\in\GL[n](\bb Q(x_{ij}))$ y, por el Lema~%
\ref{lema:simplecticas:pfaff}, su determinante es un cuadrado en
$\bb Q(x_{ij})$. M\'as aun, como $\bb Z[x_{ij}]$ es un DFU,
\begin{equation}
	\label{eq:simplecticas:pfaff}
	\det(M(x_{ij}))\,=\,(\pfaff(x_{ij}))^2
	\text{ ,}
\end{equation}
%
donde $\pfaff(x_{ij})\in\bb Z[x_{ij}]$. Este polinomio est\'a determinado a
menos de un signo. Especializando las variables $x_{ij}$, se obtienen
f\'ormulas para las ra\'{\i}ces cuadradas de los determinantes de las matrices
alternadas no degeneradas. Para fijar el signo de $\pfaff(x_{ij})$, nuevamente,
especializamos en una matriz conocida, por ejemplo la matriz
\eqref{item:simplecticas:base:numerica}.

\begin{defIntroSimp}\label{def:simplecticas:pfaff}
	El \emph{polinomio pfaffiano} es el polinomio con coeficientes
	enteros $\pfaff(x_{ij})\in\bb Z[x_{ij}]$ determinado por
	\eqref{eq:simplecticas:pfaff} y $\pfaff(\repr B)=1$, donde
	$\repr B$ denota la matriz \eqref{item:simplecticas:base:numerica}.
	Si $M\in\MM[n\times n](F)$, el \emph{pfaffiano} de $M$ es la
	especializaci\'on $\pfaff(M)$.
\end{defIntroSimp}

El pfaffiano est\'a definida para matrices alternadas no necesariamente
invertibles. El tama\~no tiene que ser $n\times n$ con $n\geq 2$ par.

\begin{ejerIntroSimp}\label{ejer:simplecticas:pfaff}
	Si $n=2m\geq 2$, dada una matriz alternada $M$ (no necesariamente
	invertible),
	\begin{enumerate}[(i)]
		\item\label{item:pfaff:cambio}
			$\pfaff(\trnsp C\,M\,C)=(\det\,C)\,\pfaff(M)$, para
			toda $C$;
		\item\label{item:pfaff:transpuesta}
			$\pfaff(\trnsp M)=(-1)^{n/2}\,\pfaff(M)$;
		\item\label{item:pfaff:singular}
			si $M$ no es invertible ($\det\,M=0$), entonces
			$\pfaff(M)=0$;
		\item\label{item:pfaff:nosingular}
			si $M$ no es invertible y, mediante un cambio de base,
			$\trnsp C\,M\,C$ es la matriz en
			\eqref{item:simplecticas:base:numerica}, entonces
			$\pfaff(M)=(\det\,C)^{-1}$.
	\end{enumerate}
	%
\end{ejerIntroSimp}



% \section{Producto tensorial}\label{sec:intro:tensorial}
% \input{./tex/intro/tensorial.tex}

\section{Formas sesquilineales}\label{sec:intro:sesquilineales}
\input{./tex/intro/sesquilineales.tex}

