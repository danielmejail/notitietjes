\theoremstyle{plain}
\newtheorem{teoIntroMat}{Teorema}[section]

\theoremstyle{definition}
\newtheorem{defIntroMat}[teoIntroMat]{Definici\'on}
\newtheorem{ejemIntroMat}[teoIntroMat]{Ejemplo}
\newtheorem{obsIntroMat}[teoIntroMat]{Observaci\'on}

%-------------

Generalizando el Ejemplo~\ref{ejem:definiciones:escalar}, sobre un cuerpo
arbitrario $F$, si $n\geq 1$ es entero, definimos el \emph{producto escalar}
en $F^n$ por:%
\footnote{
	Como producto matricial, $x\cdot y=\trnsp x\,y$.
}
\begin{displaymath}
	x\,\cdot\,y\,=\,\sum_{i=1}^n\,x_iy_i
	\text{ ,}
\end{displaymath}
%
si $\trnsp x=(\lista x{n})$ e $\trnsp y=(\lista y{n})$. Si
$M\in\MM[n\times n](F)$ es una matriz con coeficientes en el cuerpo $F$,
\begin{displaymath}
	B(x,y) \,=\,x\,\cdot\,M\,y
\end{displaymath}
%
tambi\'en define una forma bilineal. Toda forma bilineal es de este tipo.

Sea $(V,B)$ un espacio bilineal de dimensi\'on $n\geq 1$ y sea
$\{\lista* e{n}\}$ una base del espacio vectorial. Usando la bilinealidad de
$B$,
\begin{equation}
	\label{eq:matrices:base}
	B(v,w)\,=\,\sum_{i,j=1}^n\,x_iy_i\,B(e^i,e^j)
	\text{ ,}
\end{equation}
%
si $v=x_ie^i$ y $w=y_je^j$.

\begin{defIntroMat}\label{def:matrices:matrices}
	La \emph{matriz} de $B$ en la base $\{\lista* e{n}\}$ es
	\begin{displaymath}
		M_{ij}\,=\,B(e^i,e^j)
		\text{ .}
	\end{displaymath}
	%
\end{defIntroMat}

Si $v\in V$, escribimos $\repr v$ para referirnos a la representaci\'on del
vector $v$ en una base. Con esta notaci\'on, se verifica que
\begin{displaymath}
	B(v,w)\,=\,\repr v\,\cdot\,M\,\repr w
	\text{ .}
\end{displaymath}
%

\begin{ejemIntroMat}\label{ejem:matrices:matrices}
	Determinar las matrices asociadas a los espacios de los ejemplos de la
	secci\'on \S~\ref{sec:intro:definiciones} en alguna base.
\end{ejemIntroMat}

\begin{teoIntroMat}\label{teo:matrices:matrices}
	La elecci\'on de una base determina una correspondencia (isomorfismo)
	entre (los espacios de) formas bilineales en $V$ y matrices en
	$\MM[n\times n](F)$.
\end{teoIntroMat}

\begin{ejemIntroMat}\label{ejem:matrices:pseudoeuclideos}
	En $\bb R^n$, si $p,q\geq 0$ y $p+q=n$, definimos
	\begin{displaymath}
		\langle x,y\rangle_{p,q}\,=\,x_1y_1\,+\,\cdots\,+\,x_py_p\,-\,
			x_{p+1}y_{p+1}\,-\,\cdots\,-\,x_ny_n
		\text{ .}
	\end{displaymath}
	%
	La matriz asociada a $\langle\cdot,\cdot,\rangle_{p,q}$ en la base
	can\'onica de $\bb R^n$ es
	\begin{displaymath}
		\begin{bmatrix} I_p & 0 \\ 0 & -I_q \end{bmatrix}
		\text{ .}
	\end{displaymath}
	%
	La forma, al igual que la matriz, es sim\'etrica. Denotamos estos
	espacios por $\bb R^{p,q}$. El espacio del Ejemplo~%
	\ref{ejem:definiciones:hiperbolico} coincide con $\bb R^{1,1}$.
	El espacio $\bb R^{3,1}$ se denomina \emph{espacio de Minkowski}.
	El espacio $\bb R^{n,0}$ es $\bb R^n$ con el producto escalar,
	usualmente llamado \emph{espacio euclideo}.
\end{ejemIntroMat}

\begin{teoIntroMat}\label{teo:matrices:dual}
	Sea $(V,B)$ un espacio bilineal y sea $\cal B=\{\lista* e{n}\}$ una
	base de $V$. La matriz asociada a $B$ con respecto a $\cal B$ coincide
	con la matriz asociada a la t.l. $R_B:\,V\rightarrow\dual V$ con
	respecto a $\cal B$ y su base dual.
\end{teoIntroMat}

\begin{proof}
	Sea $\repr\cdot:\,V\rightarrow F^n$ el isomorfismo dado por elegir la
	base $\cal B$ de $V$ y sea $\repr\cdot':\,\dual V\rightarrow F^n$ el
	isomorfismo correspondiente a elegir la base dual a $\cal B$ en
	$\dual V$. Sea $\dual{\cal B}=\{\lista\varepsilon{n}\}$ dicha base. La
	matriz de $R_B$ con respecto a estas bases tiene \emph{columnas}
	$\repr{R_B(e^j)}'$. Si
	\begin{displaymath}
		R_B(e^j)\,=\,c^i\varepsilon_i
		\text{ ,}
	\end{displaymath}
	%
	evaluando en $e^i$ recuperamos los coeficientes:
	\begin{displaymath}
		c^i\,=\,R_B(e^j)(e^i)\,=\,B(e^i,e^j)\,=\,M_{ij}
		\text{ .}
	\end{displaymath}
	%
\end{proof}

\begin{obsIntroMat}\label{obs:matrices:dual}
	Podr\'{\i}amos haber definido la matriz de $B$ por $N\,v\cdot w$. En
	ese caso, hubi\'esemos deducido que esta matriz coincide con la
	matriz de $L_B$.
\end{obsIntroMat}

\begin{teoIntroMat}\label{teo:matrices:simetria}
	Sea $(V,B)$ un espacio bilineal y sea $M$ la matriz asociada a $B$ en
	alguna base. Entonces,
	\begin{itemize}
		\item $B$ es sim\'etrica, si y s\'olo si $\trnsp M=M$;
		\item $B$ es antisim\'etrica, si y s\'olo si $\trnsp M=-M$;
		\item $M$ es alternada, si y s\'olo si $\trnsp M=-M$ y las
			coordenadas en la diagonal de $M$ son nulas.
	\end{itemize}
	%
\end{teoIntroMat}

\begin{teoIntroMat}\label{teo:matrices:cambio}
	Sea $(V,B)$ un espacio bilineal y sean $\repr[1]\cdot$ y
	$\repr[2]\cdot$ dos bases en $V$. Sea $C$ la matriz de cambio de base
	cuyas columnas representan los vectores de la segunda base en
	t\'erminos de la primera.%
	\footnote{
		$\repr[1] v\,=\,C\,\repr[2] v$, para $v\in V$.
	}
	Si $M$ es la matriz asociada a $B$ en la base $\repr[1]\cdot$, entonces
	la matriz en la base $\repr[2]\cdot$ es $\trnsp CMC$.
\end{teoIntroMat}

\begin{proof}
	La demostraci\'on depende de la identificaci\'on $\dual{F^n}=F^n$
	v\'{\i}a el producto escalar.
\end{proof}

\begin{defIntroMat}\label{def:matrices:equivalentes}
	Dos espacios bilineales $(V,B_V)$ y $(W,B_W)$ se dicen
	\emph{equivalentes}, si existe un isomorfismo $A:\,V\rightarrow W$ tal
	que
	\begin{displaymath}
		B_W(A\,v,A\,v_1)\,=\,B_V(v,v_1)
		\text{ ,}
	\end{displaymath}
	%
	para todo $v,v_1\in V$.%
	\footnote{
		En t\'erminos de las matrices asociadas, dos formas son
		equivalentes, si y s\'olo si existen bases de $V$ y de $W$ con
		respecto a las cuales las matrices correspondientes a $B_V$ y a
		$B_W$, $M$ y $N$, son equivalentes, es decir, existe $C$
		invertible tal que $M=\trnsp C\,N\,C$.
	}
\end{defIntroMat}

\begin{ejemIntroMat}\label{ejem:matrices:equivalentes}
	Las formas
	\begin{displaymath}
		v\,\cdot\,\begin{bmatrix} 0 & 1 \\ -1 & 0 \end{bmatrix}\,w
		\quad\text{y}\quad
		v\,\cdot\,\begin{bmatrix} 1/2 & 0 \\ 0 & 1/2 \end{bmatrix}\,w
	\end{displaymath}
	%
	en $\bb R^2$ son equivalentes v\'{\i}a el cambio de base
	\begin{displaymath}
		\begin{bmatrix} 1 & 1 \\ 1 & -1 \end{bmatrix}
		\text{ .}
	\end{displaymath}
	%
\end{ejemIntroMat}

\begin{defIntroMat}\label{def:matrices:discriminante}
	El \emph{discriminante} de una forma bilineal $B$ se define como
	el determinante de cualquier representaci\'on de $B$ como matriz.
\end{defIntroMat}

\begin{obsIntroMat}\label{obs:matrices:discriminante}
	Este valor est\'a determinado a menos de cuadrados. Si $M$ es la
	matriz de $B$ en alguna base, entonces $\discriminante(B)$ es la clase
	de $\det(M)$ en $F/{F^\times}^2$.%
	\footnote{
		$\discriminante(B):=0$, si $\det(M)=0$ en alguna (y, por lo
		tanto, en toda) base, o bien $\discriminante(B)$ es la clase
		de $\det(M)$ en el grupo $F^\times/{F^\times}^2$.
	}
\end{obsIntroMat}

