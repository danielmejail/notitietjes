\theoremstyle{plain}
\newtheorem{teoIntroSimp}{Teorema}[section]
\newtheorem{coroIntroSimp}[teoIntroSimp]{Corolario}
\newtheorem{lemaIntroSimp}[teoIntroSimp]{Lema}

\theoremstyle{definition}
\newtheorem{defIntroSimp}[teoIntroSimp]{Definici\'on}
\newtheorem{obsIntroSimp}[teoIntroSimp]{Observaic\'on}
\newtheorem{ejemIntroSimp}[teoIntroSimp]{Ejemplo}
\newtheorem{ejerIntroSimp}[teoIntroSimp]{Ejercicio}

%-------------

Fijamos un espacio bilineal \emph{alternado} $(V,B)$ sobre un cuerpo $F$.

\begin{teoIntroSimp}\label{teo:simplecticas:dimension}
	Si $(V,B)$ es un espacio bilineal alternado no degenerado, entonces
	$\dim\,V$ es par.
\end{teoIntroSimp}

\begin{proof}
	Asumiendo que $B$ es alternada, si $M$ es la matriz asociada a $B$ en
	alguna base, por el Teorema~\ref{teo:matrices:simetria},
	$\trnsp M=-M$ (y las coordenadas de la diagonal son nulas). Tomando
	determinantes,
	\begin{math}
		\det\,M=(-1)^{\dim\,V}\,\det M
	\end{math}. Si $B$ es no degenerada, $\det\,M\neq 0$. Si
	$\car F\neq 2$, $-1\neq 1$. De esto se deduce el resultado en el caso
	en que la caracter\'{\i}stica del cuerpo de base es distinta de $2$.

	El siguiente argumento es v\'alido en cualquier caracter\'{\i}stica.
	Supongamos que $B$ es alternada. Si $\dim\,V=1$, entonces $B=0$, con lo
	cual no puede ser no degenerada. Supongamos, entonces, que
	$\dim\,V>2$ y que $B$ es no degenerada.

	Si $v\in V$ no es el vector cero, $B(v,-)\neq 0$ en $\dual V$.%
	\footnote{
		$L_B$ es inyectiva.
	}
	Entonces, $B(v,w)=1$ para cierto $w\in V$.%
	\footnote{
		Por linealidad. Toda funcional $F$-lineal $V\rightarrow F$ no
		nula es una funci\'on sobreyectiva.
	}
	Si $U=\generado{v,w}$, entonces $\dim\,U=2$%
	\footnote{
		El subconjunto $\{v,w\}$ es l.i. por alternancia de $B$.
	}
	y, con respecto a la base, $\{v,w\}$, $B|_U$ est\'a representada por la
	matriz $\sbmatrix{ & 1 \\ -1 & }$. Esta matriz tiene determinante $1$,
	es invertible y $B|_U$ es no degenerada. Por el Teorema~%
	\ref{teo:nodegeneradas:perpendicular}, $V=U\oplus U^\perp$. Adem\'as,
	como $V$ es no degenerado y $U$ es un subespacio no degenerado,
	$U^\perp$ tambi\'en es no degenerado. Inducci\'on.
\end{proof}

\begin{obsIntroSimp}\label{obs:simplecticas:dimension}
	Si $(V,B)$ es un espacio bilineal sim\'etrico no degenerado, el Lema~%
	\ref{lema:ortogonales:complemento} garantiza, dado cualquier $v\in V$
	tal que $B(v,v)\neq 0$, la descomposici\'on ortogonal
	$V=\generado v\oplus v^\perp$; los subespacios $\generado v$ y
	$v^\perp$ son no degenerados. An\'alogamente, si $(V,B)$ es bilineal
	alternado no degenerado, la demostraci\'on del Teorema~%
	\ref{teo:simplecticas:dimension}, muestra que, dado cualquier $v\in V$,
	existe otro vector $w\in V$ tal que
	\begin{itemize}
		\item $B(v,w)=1$,
		\item $V=U\oplus U^\perp$, si $U=\generado{v,w}$, y
		\item $U$ y $U^\perp$ son subespacios no degenerados.
	\end{itemize}
	%
\end{obsIntroSimp}

\begin{defIntroSimp}\label{def:simplecticas:base}
	Sea $(V,B)$ una espacio alternado no degenerado. Si $\dim\,V=2m\geq 2$,
	una \emph{base simpl\'ectica} es una base
	\begin{math}
		\big\{e_1,\,f_1,\,\dots,\,e_m,\,f_m\big\}
	\end{math} que cumple:
	\begin{itemize}
		\item $B(e_i,f_i)=1$ y
		\item $U_i=\generado{e_i,f_i}$ son perpendiculares entre
			s\'{\i}.
	\end{itemize}
	%
\end{defIntroSimp}

\begin{obsIntroSimp}\label{obs:simplecticas:bases}
	Tambi\'en se llama base simpl\'ectica a cualquier base cuyos vectores
	cumplen con las dos propiedades de la Definici\'on~%
	\ref{def:simplecticas:base}, aunque est\'en ordenados de otra manera.
	Hay dos o tres maneras est\'andar de ordenarlas:
	\begin{enumerate}[(I)]
		\item\label{item:simplecticas:base:numerica}
			con el orden $e_1,\,f_1,\,\dots,\,e_m,\,f_m$, la matriz
			asociada a $B$ es:
			\begin{displaymath}
				\begin{bmatrix}
					0 & 1 & & & & & \\
					-1 & 0 & & & & & \\
					& & 0 & 1 & & & \\
					& & -1 & 0 & & & \\
					& & & & \ddots & & \\
					& & & & & 0 & 1 \\
					& & & & & -1 & 0
				\end{bmatrix}
				\text{ ,}
			\end{displaymath}
			%
		\item\label{item:simplecticas:base:alfabetica}
			con el orden $e_1,\,\dots,\,e_m,\,f_1,\,\dots,\,f_m$,
			la matriz asociada es:
			\begin{displaymath}
				\begin{bmatrix}
					& & & 1 & & \\
					& & & & \ddots & \\
					& & & & & 1 \\
					-1 & & & & & \\
					& \ddots & & & & \\
					& & -1 & & &
				\end{bmatrix}
				\quad\text{y}
			\end{displaymath}
			%
		\item\label{item:simplecticas:base:invertida}
			con el orden $e_1,\,\dots,\,e_m,\,f_m,\,\dots,\,f_1$,
			la matriz asociada es:
			\begin{displaymath}
				\begin{bmatrix}
					& & & & & 1 \\
					& & & & \iddots & \\
					& & & 1 & & \\
					& & -1 & & & \\
					& \iddots & & & & \\
					-1 & & & & &
				\end{bmatrix}
				\text{ .}
			\end{displaymath}
			%
	\end{enumerate}
	%
\end{obsIntroSimp}

\begin{ejerIntroSimp}\label{ejer:simplecticas:base}
	Hallar una f\'ormula para $B$ en coordenadas en la base simpl\'ectica
	con el orden \eqref{item:simplecticas:base:alfabetica} y relacionarla
	con la forma del Ejemplo~\ref{ejem:definiciones:determinante}.
\end{ejerIntroSimp}

\begin{obsIntroSimp}\label{obs:simplecticas:base}
	En un espacio sim\'etrico no degenerado, cualquier vector $v\in V$ no
	nulo que cumpla $B(v,v)\neq 0$ es parte de una base ortogonal. En un
	espacio alternado no degenerado, cualquier vector no nulo es parte de
	una base simpl\'ectica.
\end{obsIntroSimp}

\begin{coroIntroSimp}\label{coro:simplecticas:dimension}
	Todo espacio alternado no degenerado admite una base simpl\'ectica. Dos
	espacios alternados no degenerados de la misma dimensi\'on son
	equivalentes.
\end{coroIntroSimp}

En general, si $V^\perp\neq 0$, elegimos un complemento directo arbitrario
$W\subset V$, de manera que $V=W\oplus V^\perp$. Como $W\cap V^\perp=0$, $B|_W$
es no degenerada. Eligiendo una base simpl\'ectica para $W$ y completando con
una base arbitraria de  $V^\perp$, la forma alternada $B$ tiene asociada una
matriz de la forma
\begin{displaymath}
	\begin{bmatrix}
		0 & I_r & 0 \\
		-I_r & 0 & 0 \\
		0 & 0 & 0
	\end{bmatrix}
	\text{ ,}
\end{displaymath}
%
donde $2r=\dim\,W$. El valor de $r$ no depende de la elecci\'on de complemento
$W$: $\dim\,W=\dim (V/V^\perp)$ En definitiva, la clase de equivalencia de un
espacio bilineal alternado est\'a determinada por:
\begin{itemize}
	\item la dimensi\'on del espacio, $\dim\,V=n$, y
	\item su ``grado de degeneraci\'on'', $\dim\,V^\perp=n-2r$.
\end{itemize}
%

\begin{ejerIntroSimp}\label{ejer:simplecticas:adjunta}
	Hallar la adjunta de una matriz
	\begin{math}
		\sbmatrix{ A & B \\ C & D }\in\MM[2m\times 2m](F)
	\end{math}
	con respecto a la forma bilineal alternada en $F^{2m}$ representada por
	\eqref{item:simplecticas:base:alfabetica}.
\end{ejerIntroSimp}

\begin{ejerIntroSimp}\label{ejer:simplecticas:dual}
	Sea $B$ la forma alternada del Ejemplo~%
	\ref{ejem:definiciones:alternadas} en $V\oplus\dual V$. Probar que, si
	$\{\lista* e{m}\}$ es una base de $V$ y $\{\lista\varepsilon{m}\}$ es
	la base dual en $\dual V$, entonces
	$\{e^1,\,\varepsilon_1,\,\dots,\,e^m,\,\varepsilon_m\}$ es una base
	simpl\'ectica para $B$ con respecto a la cual la matriz asociada es
	\eqref{item:simplecticas:base:numerica}.
\end{ejerIntroSimp}

\begin{ejerIntroSimp}\label{ejer:simplecticas:parias}
	Sea $(V,B)$ un espacio bilineal sim\'etrico no degenerado sobre un
	cuerpo de caracter\'{\i}stica $2$. Probar que $(V,B)$ admite una base
	ortogonal, si y s\'olo si $B$ \emph{no es} alternada.%
	\hint{
		Sin p\'erdida de generalidad, asumir que $\dim\,V\geq 2$. De
		acuerdo con el comentario despu\'es del Teorema~%
		\ref{teo:ortogonales:base}, la condici\'on es necesaria. Para
		ver que es suficiente con no ser alternada, elegir $v_0\in V$
		tal que $a=B(v_0,v_0)\neq 0$. Notar que, por el Lema~%
		\ref{lema:ortogonales:complemento}, $v_0^\perp$ es no
		degenerado. Si, en este subespacio, $B$ no es alternada, el
		resultado se deduce por un argumento inductivo. Si, en cambio,
		$B$ es alternada en $v_0^\perp$, existen $e,f\in v_0^\perp$
		tales que $B(e,f)=1$ y $B(e,e)=B(f,f)=0$ (parte de una base
		simpl\'ectica). Probar que $v_1:=v_0+e+f$ verifica
		$B(v_1,v_1)\neq 0$ y que $B$ no es alternada en $v_1^\perp$.
	}
\end{ejerIntroSimp}

\paragraph{El \emph{pfaffiano}}
Hay una manera gen\'erica de determinar una ra\'{\i}z cuadrada del determinante
de una matriz alternada.

\begin{lemaIntroSimp}\label{lema:simplecticas:pfaff}
	El determinante de una matriz alternada invertible con coeficientes en
	un cuerpo $F$ es un cuadrado perfecto no nulo.
\end{lemaIntroSimp}

\begin{proof}
	Si $M$ es la matriz y es de tama\~no $n\times n$, por el Corolario~%
	\ref{coro:simplecticas:dimension}, $n=2m$, $m\geq 1$, y
	\begin{math}
		\trnsp C\,M\,C=\sbmatrix{ & I_m \\ -I_m & }
	\end{math}, para cierta $C$ invertible.
\end{proof}

\begin{ejemIntroSimp}\label{ejem:simplecticas:pfaff}
	Si $n=2$,
	\begin{math}
		\svmatrix{ & x \\ -x & }=x^2
	\end{math}. Si $n=4$,
	\begin{displaymath}
		\begin{vmatrix}
			& x & y & z \\
			-x & & a & b \\
			-y & -a & & c \\
			-z & -b & -c &
		\end{vmatrix}
		\,=\,(xc-yb+az)^2
		\text{ .}
	\end{displaymath}
	%
\end{ejemIntroSimp}

Gen\'ericamente, Si $M(x_{ij})$ denota la matriz alternada gen\'erica de
tama\~no $n\times n$, $n=2m$ par, sobre el cuerpo $\bb Q(x_{ij})$, su
determinante es un polinomio no nulo con coeficientes en $\bb Z$.%
\footnote{
	Especializar en la matriz \eqref{item:simplecticas:base:numerica}, por
	ejemplo.
}
En particular, $M(x_{ij})\in\GL[n](\bb Q(x_{ij}))$ y, por el Lema~%
\ref{lema:simplecticas:pfaff}, su determinante es un cuadrado en
$\bb Q(x_{ij})$. M\'as aun, como $\bb Z[x_{ij}]$ es un DFU,
\begin{equation}
	\label{eq:simplecticas:pfaff}
	\det(M(x_{ij}))\,=\,(\pfaff(x_{ij}))^2
	\text{ ,}
\end{equation}
%
donde $\pfaff(x_{ij})\in\bb Z[x_{ij}]$. Este polinomio est\'a determinado a
menos de un signo. Especializando las variables $x_{ij}$, se obtienen
f\'ormulas para las ra\'{\i}ces cuadradas de los determinantes de las matrices
alternadas no degeneradas. Para fijar el signo de $\pfaff(x_{ij})$, nuevamente,
especializamos en una matriz conocida, por ejemplo la matriz
\eqref{item:simplecticas:base:numerica}.

\begin{defIntroSimp}\label{def:simplecticas:pfaff}
	El \emph{polinomio pfaffiano} es el polinomio con coeficientes
	enteros $\pfaff(x_{ij})\in\bb Z[x_{ij}]$ determinado por
	\eqref{eq:simplecticas:pfaff} y $\pfaff(\repr B)=1$, donde
	$\repr B$ denota la matriz \eqref{item:simplecticas:base:numerica}.
	Si $M\in\MM[n\times n](F)$, el \emph{pfaffiano} de $M$ es la
	especializaci\'on $\pfaff(M)$.
\end{defIntroSimp}

El pfaffiano est\'a definida para matrices alternadas no necesariamente
invertibles. El tama\~no tiene que ser $n\times n$ con $n\geq 2$ par.

\begin{ejerIntroSimp}\label{ejer:simplecticas:pfaff}
	Si $n=2m\geq 2$, dada una matriz alternada $M$ (no necesariamente
	invertible),
	\begin{enumerate}[(i)]
		\item\label{item:pfaff:cambio}
			$\pfaff(\trnsp C\,M\,C)=(\det\,C)\,\pfaff(M)$, para
			toda $C$;
		\item\label{item:pfaff:transpuesta}
			$\pfaff(\trnsp M)=(-1)^{n/2}\,\pfaff(M)$;
		\item\label{item:pfaff:singular}
			si $M$ no es invertible ($\det\,M=0$), entonces
			$\pfaff(M)=0$;
		\item\label{item:pfaff:nosingular}
			si $M$ no es invertible y, mediante un cambio de base,
			$\trnsp C\,M\,C$ es la matriz en
			\eqref{item:simplecticas:base:numerica}, entonces
			$\pfaff(M)=(\det\,C)^{-1}$.
	\end{enumerate}
	%
\end{ejerIntroSimp}

