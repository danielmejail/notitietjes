\theoremstyle{definition}
\newtheorem{ejerAcciones}{\ejername}[section]

\theoremstyle{plain}

%-------------

\begin{ejerAcciones}\label{ejer:acciones:dosclases}
	Sea $G$ un grupo con dos clases de conjugaci\'on, exactamente.
	Entonces,
	\begin{enumerate}[(i)]
		\item\label{item:dosclases:grupo-finito}
			si el orden $\ordenvert G$ del grupo es finito,
			$\ordenvert G=2$;
		\item\label{item:dosclases:elemento-de-orden-finito}
			si existe en $G$ un elemento de orden finito,
			$\ordenvert G=2$.%
			\hint{
				Existe un primo $p$ tal que $a^p=e$ para todo
				$a\in G$. Si $p>2$ es impar, entonces
				$a^2=xax^{-1}$ para cierto $x\in G$.
				Inductivamente, $a^{2^p}=x^pax^{-p}=a$, con lo
				cual $2^p\equiv 1\tmodulo[p]$.
			}
	\end{enumerate}
	%
\end{ejerAcciones}

\begin{solucion}
	Si $\ordenvert G<\infty$, podemos argumentar de la siguiente manera.
	El grupo $G$ se descompone como uni\'on disjunta de dos clases de
	conjugaci\'on
	\begin{equation}
		\label{eq:dosclases}
		G\,=\,\{e\}\,\sqcup\,a^G
		\text{ ,}
	\end{equation}
	%
	donde $e\in G$ es el elemento neutro y $a^G=\{xax^{-1}\,:\,x\in G\}$
	denota la clase de conjugaci\'on de alg\'un otro elemento $a\in G$.
	Necesariamente, una de las clases debe ser $\{e\}$, pues
	$e\in\centre G$. Pero, si $\centraliza[G](a)$ denota el centralizador
	de $a$ en $G$,
	\begin{displaymath}
		\ordenvert{a^G}\,=\,\indice{G:\centraliza[G](a)}
		\text{ ,}
	\end{displaymath}
	%
	que divide a $\ordenvert G$, pues el orden es finito. Por otro lado,
	\begin{math}
		\ordenvert G=1+\ordenvert{a^G}
	\end{math}, de lo que se deduce que $\ordenvert{a^G}=1$ y
	$\ordenvert G=2$.

	El argumento anterior no es aplicable sin la hip\'otesis de que
	$\ordenvert G$ sea finito. Sin embargo, la descomposici\'on
	\eqref{eq:dosclases} sigue siendo v\'alida y podemos elegir como $a$
	cualquier elemento distinto del neutro; en particular, podemos suponer
	que $\orden(a)=n$ es finito y mayor que $1$. Dado que hay s\'olo dos
	clases de conjugaci\'on, una de las cuales es la del neutro del grupo,
	todo elemento distinto del neutro debe ser conjugado de $a$ y, en
	particular, tener orden finito $n$.

	Si $n=2$, todo elemento salvo el neutro tiene orden $2$ y el grupo es
	abeliano.%
	\footnote{
		$(bb_1)^2=e$ implica $bb_1=b_1b$.
	}
	As\'{\i}, $a^G=\{a\}$ y $\ordenvert G=2$. Si $n>2$, entonces existe
	alg\'un primo impar $p>2$ que divide a $n$. En particular, reemplazando
	$a$ por $a^{n/p}\neq e$, podemos asumir que el orden de $a$ es $p$,
	usando, de nuevo, que $a$ y $a^{n/p}$ son conjugados. Como $a^2\neq e$,
	existe $x\in G$ tal que $a^2=xax^{-1}$. En consecuencia, dado
	cualquier elemento $y\in G$, $(yay^{-1})^2=yxa(yx)^{-1}$. En
	particular, eligiendo $y=a^{2^k}$, se deduce inductivamente que
	\begin{displaymath}
		a^{2^k}=x^kax^{-k}
		\text{ ,}
	\end{displaymath}
	%
	para todo $k\geq 1$. Si $k=p$, se deduce que $a^{2^p}=a$, pues
	$x$ es o bien el neutro, o bien conjugado de $a$ y, por lo tanto, de
	orden $p$. Como el orden de $a$ es $p$, se cumple que
	\begin{math}
		2^p\equiv 1\tmodulo[p]
	\end{math}, lo cual es absurdo, si $p$ es primo. La contradicci\'on
	viene de suponer que $p>2$.
\end{solucion}

\begin{ejerAcciones}\label{ejer:acciones:orden-primo-centralizador-propio}
	Sea $G$ un grupo finito y $H\triangleleft G$ un subgrupo normal de
	\'{\i}ndice primo. Si $x\in H$ verifica
	\begin{math}
		\centraliza[H](x)<\centraliza[G](x)
	\end{math}%
	\footnote{
		Es un subgrupo contenido propiamente.
	}
	y si $y\in H$ es conjugado de $x$ \emph{en $G$}, entonces $x$ e $y$ son
	conjugados \emph{en $H$}.
\end{ejerAcciones}

\begin{solucion}
	Empecemos con algunas observaciones. Primero, si $x,y\in G$ son
	conjugados, es decir, $y=gxg^{-1}$, $g\in G$, entonces
	\begin{displaymath}
		\centraliza[G](y)\,=\,g\,\centraliza[G](x)\,g^{-1}
		\text{ .}
	\end{displaymath}
	%
	Segundo, si $H\leq G$ es un subgrupo arbitrario y $x\in H$, entonces
	\begin{displaymath}
		\centraliza[H](x)\,=\,\centraliza[G](x)\,\cap\,H
		\text{ .}
	\end{displaymath}
	%
	Ahora, si $H\triangleleft G$ es normal, $x,y\in H$ y existe $g\in G$
	tal que $y=gxg^{-1}$, entonces
	\begin{displaymath}
		\begin{aligned}
			\centraliza[H](y) & \,=\,\centraliza[G](y)\,\cap\,H
				\,=\,\big(g\centraliza[G](x)g^{-1}\big)
					\,\cap\,H \\
			& \,=\,\big(g\centraliza[G](x)g^{-1}\big)
					\,\cap\,gHg^{-1}
				\,=\,g\,\big(\centraliza[G](x)\cap H\big)\,
					g^{-1} \\
			& \,=\,g\,\centraliza[H](x)\,g^{-1}
			\text{ .}
		\end{aligned}
		%
	\end{displaymath}
	%

	El objetivo es demostrar que $x^H=x^G\cap H$.%
	\footnote{
		En este caso, como $H$ es normal, $x^G\cap H=x^G$.
	}
	En general, $x^H\subset x^G$; en t\'erminos de los \'{\i}ndices de los
	centralizadores,
	\begin{displaymath}
		\indice{G:\centraliza[G]x}\,=\,\ordenvert{x^G}\,\geq\,
			\ordenvert{x^H}\,=\,\indice{H:\centraliza[H]x}
		\text{ .}
	\end{displaymath}
	%
	Por otra parte, dado que el \'{\i}ndice es multiplicativo,
	\begin{displaymath}
		\indice{G:H}\,\indice{H:\centraliza[H]x}\,=\,
			\indice{G:\centraliza[G]x}\,
			\indice{\centraliza[G]x:\centraliza[H]x}
		\text{ .}
	\end{displaymath}
	%
	Ser\'a suficiente ver que $p=\indice{G:H}$ divide a
	$\indice{\centraliza[G]x:\centraliza[H]x}$.%
	\footnote{
		No es suficiente saber que este \'{\i}ndice es poitivo.
	}
	De las observaciones iniciales, como $H$ es normal en $G$, si
	$g\in\centraliza[G](x)$ (un elemento arbitrario del centralizador, no
	el que da lugar a $y$), entonces $g$ normaliza a $\centraliza[H](x)$.
	En particular,
	\begin{displaymath}
		\centraliza[H](x)\,\triangleleft\,\centraliza[G](x)
		\text{ .}
	\end{displaymath}
	%
	Como $g^p\in H$ para todo $g\in G$, se deduce que
	$g^p\in\centraliza[H](x)$ para todo $g\in\centraliza[G](x)$. Como
	$\ordenvert{\centraliza[G]x/\centraliza[H]x}>1$, alg\'un elemento
	$g\in\centraliza[G](x)$ tiene orden $p$ m\'odulo el subgrupo normal
	$\centraliza[H](x)$. En particular, $p$ divide al orden del cociente,
	es decir, al \'{\i}ndice de los centralizadores de $x$.
\end{solucion}

\begin{ejerAcciones}\label{ejer:acciones:monomiales}
	Una matriz $m=[m_{ij}]\in\MM[m\times m](R)$ se dice \emph{monomial}, si
	existe una permutaci\'on $\alpha\in\simetrico[m]$ y elementos
	$\lista x{m}\in R$ tales que
	\begin{displaymath}
		m_{ij}\,=\,
			\begin{cases}
				x_i & \text{si }j=\alpha(i) \\
				0 & \text{si no}
			\end{cases}
		\text{ .}
	\end{displaymath}
	%
	Las matrices monomiales tienen a lo sumo una coordenada distinta de
	cero por fila y por columna. Una matriz de permutaci\'on es un caso
	especial de matriz monomial: para cada $i$, $x_i=1$. El producto de
	matrices monomiales es una matriz monomial.

	Sea $k$ un anillo no nulo (no necesariamente conmutativo) con la
	propiedad de que $ab=0$ implica $a=0$ o $b=0$. Sea $G=\GL[m](k)$,
	$m\geq 2$, y sea $T\leq G$ el subgrupo de matrices diagonales.
	\begin{enumerate}[(i)]
		\item\label{item:monomiales:normalizador}
			Demostrar que $\normaliza[G](T)$ es el subgrupo de las
			matrices monomiales en $G$.
		\item\label{item:monomiales:cociente}
			Demostrar que $\normaliza[G](T)/T\simeq\simetrico[m]$.
	\end{enumerate}
	%
\end{ejerAcciones}

\begin{solucion}
	Un elemento $g\in G$ pertenece a $\normaliza[G](T)$, si y s\'olo si
	para todo $t\in T$, existe $t'\in T$ tal que $gtg^{-1}=t'$. Si
	$g=[g_{ij}]$, $t=\diag(\lista t{m})$ y $t'=\diag(t'_1,\,\dots,\,t'_m)$,
	\begin{displaymath}
		({t'}^{-1}gt)_{ij}\,=\,{t'_i}^{-1}g_{ij}t_j
		\text{ .}
	\end{displaymath}
	%
	Entonces, podemos expresar la condici\'on de pertenencia al
	normalizador de $T$ de la siguiente manera: dados $\lista t{m}\in k$,
	existen $t'_1,\,\dots,\,t'_m\in k$ tales que
	\begin{displaymath}
		g_{ij}t_j\,=\,t'_ig_{ij}
		\text{ ,}
	\end{displaymath}
	%
	para todo $i,j$. Ahora, si $g_{i_0j_0}\neq 0$, elegimos $t\in T$ de
	manera que
	\begin{displaymath}
		t_j\,=\,
			\begin{cases}
				1 & \text{si }j=j_0 \\
				0 & \text{si no}
			\end{cases}
		\text{ .}
	\end{displaymath}
	%
	Si $g\in\normaliza[G](T)$, entonces los elementos $t'_i\in k$ verifcan:
	\begin{displaymath}
		\begin{aligned}
			(1-t'_{i_0})\,g_{i_0j_0} & \,=\,0 \text{ ,} \\
			t'_{i_0}\,g_{i_0j} & \,=\,0 \text{ ,}\quad
				\text{si }j\neq j_0 \text{ .}
		\end{aligned}
		%
	\end{displaymath}
	%
	Como $k$ verifica $ab=0\Rightarrow a=0\text{ o }b=0$, debe cumplirse
	que $t'_{i_0}=1$ y que $g_{i_0j}=0$ para $j\neq j_0$. En particular, en
	cada fila hay, a lo sumo, una coordenada no nula. Es decir, $g$ es una
	matriz monomial.

	Las matrices monomiales se escriben de manera \'unica como producto de
	una matriz de permutaci\'on por una matriz diagonal $m=pt$.
	% \footnote{
		% Y de manera \'unica como $m=t'p$ (aunque $t'$ puede ser
		% distinto de $t$ en este caso, $p$ est\'a determinado).
	% }
	En consecuencia, $pt\mapsto p$ induce un isomorfismo entre
	$\normaliza[G](T)/T$ y $\simetrico[m]$.
\end{solucion}

\begin{ejerAcciones}\label{ejer:acciones:clases-subgrupo-propio}
	Sea $G$ un grupo finito.
	\begin{enumerate}[(i)]
		\item\label{item:clases-subgrupo-propio:union}
			Si $H<G$ es un subgrupo propio, entonces $G$ no es
			uni\'on de conjugados de $H$.
		\item\label{item:clases-subgrupo-propio:clases}
			Si $\lista C{m}$ es una lista completa de las clases de
			conjugaci\'on en $G$ y $g_i\in C_i$, entonces
			$G=\generado{\lista g{m}}$.
	\end{enumerate}
	%
\end{ejerAcciones}

\begin{solucion}
	Sea $R$ un subconjunto de un sistema de representantes de las clases en
	$G/\normaliza[G](H)$, de manera que $\{gHg^{-1}\,:g\in R\}$ sea un
	subconjunto de todos los conjugados de $H$ en $G$. Si
	\begin{displaymath}
		G\,=\,\bigcup_{g\in R}\,gHg^{-1}
		\text{ ,}
	\end{displaymath}
	%
	entonces tenemos la cota
	\begin{equation}
		\label{eq:clases-subgrupo-propio}
		\indice{G:H}\,\ordenvert H\,=\,\ordenvert G\,\leq\,
			\ordenvert R\,\ordenvert H
		\text{ ,}
	\end{equation}
	%
	acotando el cardinal de la uni\'on por el de una uni\'on disjunta. Como
	$H$ es finito,
	\begin{displaymath}
		\begin{aligned}
			\ordenvert R\,\geq\,\indice{G:H} & \,=\,
					\indice{G:\normaliza[G]H}\,
						\indice{\normaliza[G]H:H}
				\,=\,\ordenvert{H^G}\,\indice{\normaliza[G]H:H}
				\,\geq\,\ordenvert R\,\indice{\normaliza[G]H:H}
		\end{aligned}
		%
		\text{ ,}
	\end{displaymath}
	%
	donde $H^G$ denota el conjunto de conjugados de $H$ en $G$. La \'ultima
	desigualdad es consecuencia de que
	\begin{displaymath}
		\ordenvert{H^G}\,=\,\indice{G:\normaliza[G]H}\,\geq\ordenvert R
		\text{ .}
	\end{displaymath}
	%
	En particular, $\indice{\normaliza[G]H:H}\leq 1$. Se deduce, entonces,
	que $\normaliza[G](H)=H$ y que
	$\ordenvert R=\indice{G:\normaliza[G]H}$. As\'{\i}, necesariamente, si
	$G$ es uni\'on de conjugados de $H$, debe ser la uni\'on de
	\emph{todos} los conjugados. Pero, ahora, usando que la desigualdad
	\eqref{eq:clases-subgrupo-propio} debe ser estricta, se llega a una
	contradicci\'on:
	\begin{displaymath}
		\ordenvert G\,=\,\indice{G:H}\,\ordenvert H\,=\,
			\indice{G:\normaliza[G]H}\,\ordenvert H\,=\,
			\sum_{g\in R}\,\ordenvert{gHg^{-1}}\,>\,
			\ordenvert{\bigcup_{g\in R}\,gHg^{-1}}\,=\,
			\ordenvert G
		\text{ .}
	\end{displaymath}
	%
	La desigualdad es estricta, \emph{porque $H$ es subgrupo propio}.%
	\footnote{
		En realidad, no usamos lo anterior, la cuesti\'on de que $R$
		debe ser un sistema completo de representantes. Con saber que
		una desigualdad es estricta es suficiente:
		\begin{displaymath}
			\ordenvert G\,=\,\indice{G:H}\,\ordenvert H\,\geq\,
			\indice{G:\normaliza[G]H}\,\ordenvert H\,\geq\,
			\sum_{g\in R}\,\ordenvert{gHg^{-1}}\,>\,
			\ordenvert{\bigcup_{g\in R}\,gHg^{-1}}\,=\,
			\ordenvert G
			\text{ .}
		\end{displaymath}
		%
	}

	Si $G=C_1\sqcup\,\cdots\,\sqcup C_m$ y $g_i\in C_i$, definimos
	$H=\generado{\lista g{m}}$. Como $G$ es la uni\'on de los conjugados
	de $H$ (todo elemento de $G$ es conjugado a alg\'un $g_i$ y, en
	particular, a alg\'un elemento de $H$), $H$ no puede ser un subgrupo
	propio.
\end{solucion}

