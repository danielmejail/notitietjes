\theoremstyle{plain}
\newtheorem{teoGritsenko}{\teoname}[subsection]

\theoremstyle{definition}
\newtheorem{defGritsenko}[teoGritsenko]{\defname}

%-------------

\begin{defGritsenko}\label{def:jacobi}
	Una funci\'on
	\begin{math}
		\phi:\,\semiplano[1]\times\Complejos\rightarrow\Complejos
	\end{math}
	es una \emph{forma de Jacobi de \'{\i}ndice $m$ y peso $k$}, si
	\begin{enumerate}[label=(J\arabic*)]
		\item\label{item:jacobi:holomorfia}
			es holomorfa,
		\item\label{item:jacobi:transformacion}
			la funci\'on
			$\tilde\phi:\,\semiplano[2]\rightarrow\Complejos$
			definida por
			\begin{math}
				\tilde\phi(Z)=\phi(\tau,z)\,
					\varexp^{2\pi\raizcuarta\omega m}
			\end{math}
			es una forma modular de Siegel de g\'enero $2$
			con respecto al grupo
			\begin{displaymath}
				C_{2,1}\,=\,\left\{
					\begin{bmatrix}
						* & * & * & * \\
						& * & * & * \\
						& & * & \\
						& * & * & *
					\end{bmatrix}\right\}\,\cap\,
					\modulgruppe[2]
				\dispcomma
			\end{displaymath}
			%
			es decir, $\tilde \phi\operador[k] M=\tilde \phi$,%
			\footnote{
				Por un lado,
				\begin{displaymath}
					\phi\Big(\frac{a\tau+b}{c\tau+d},
						\frac z{c\tau+d}\Big)\,=\,
						(c\tau+d)^k\,
						\varexp^{2\pi\raizcuarta m%
							\frac{cz^2}{c\tau+d}}\,
							\phi(\tau,z)
				\end{displaymath}
				%
				y, por otro,
				\begin{displaymath}
					\phi(\tau,z+\lambda\tau+\mu)\,=\,
						\varexp^{-2\pi\raizcuarta m\,%
						(\lambda^2\tau+2\lambda z)}\,
							\phi(\tau,z)
					\dispstop
				\end{displaymath}
				%	
			}
			si $M\in C_{2,1}$ y
		\item\label{item:jacobi:crecimiento}
			y, adem\'as, la funci\'on $\phi$ admite un desarrollo
			en serie de Fourier del tipo
			\begin{displaymath}
				\phi(\tau,z)\,=\,
					% \sum_{\binaria{n,r,m}\geq 0}\,
					\sum_{n\geq 0,r\in\Enteros}\,c(n,r)\,
					\varexp^{2\pi\raizcuarta\,(n\tau+rz)}
				\dispcomma
			\end{displaymath}
			%
			con $c(n,r)=0$, a menos que $r^2-4mn\leq 0$.
	\end{enumerate}
	%
	Si $c(n,r)\neq 0$ implica $r^2-4mn<0$,
	entonces $\phi$ es \emph{cuspidal}.
\end{defGritsenko}

% Existe un embedding de $\Heckerng(\modulgruppe[2])$ en
% $\Heckerng(C_{2,1})$ y dos de $\Heckerng(\modulgruppe[1])$.
% Esto nos permite asociar, a una forma de Jacobi de \'{\i}ndice $t$,
% una forma paramodular de nivel $t$. El espacio generado por aquellas
% formas paramodulares obtenidas de esta manera se denomina
% \emph{espacio de lifts} y cada una de ellas es un \emph{lift de Gritsenko}.
% Entonces, si $\phi$ es una forma de Jacobi de peso $k$ (si $c(0,0)\neq 0$,
% pedimos $k\geq 4$) e \'{\i}ndice $t\geq 1$, la funci\'on
% \begin{displaymath}
	% \Gritsenko(\phi)(\tau,z,\tau')\,=\,c(0,0)\,\Eis[k](\tau)\,+\,
		% \sum_{m\geq 1}\,m^{2-k}\,\big(
			% \phi\operador[k]{T_-(m)}\big)(\tau,z)\,
			% \varexp^{2\pi\raizcuarta tm\tau'}
% \end{displaymath}
% %
% es una forma modular de peso $k$ con respecto al grupo paramodular
% $\varGrupoParamodular(t)$ de nivel $t$.
\begin{teoGritsenko}\label{teo:gritsenko}
	Sea
	\begin{math}
		\phi(\tau,z)=\sum_{n>0,r\in\Enteros}\,c(n,r)\,
			\varexp^{2\pi\raizcuarta\,(n\tau+rz)}
	\end{math}
	una forma de Jacobi de peso $k$ e \'{\i}ndice $N$.
	La expresi\'on
	\begin{displaymath}
		\Gritsenko(\tau,z,\omega)\,=\,\sum_{\binaria{n,r,m}\geq 0}\,
			\Big(\sum_{\delta\mid\mcd{n,r,m}}\,
				\delta^{k-1}\,
				c\big(\tfrac{mn}{\delta^2},
					\tfrac r \delta\big)\Big)\,
				\varexp^{n\tau+rz+mN\omega}
	\end{displaymath}
	%
	es una forma paramodular cuspidal de peso $k$ y nivel $N$.
	Adem\'as,
	\begin{displaymath}
		\Gritsenko(\phi)\operador[k]\modularInvolution\,=\,
			(-1)^k\,\Gritsenko(\phi)
		\dispstop
	\end{displaymath}
	%
\end{teoGritsenko}

