\theoremstyle{plain}

\theoremstyle{definition}
\newtheorem{defFuncionesPara}{\defname}[subsection]

%-------------

Dada $M$, \emph{operador de peso $k$} en una funci\'on $F$ definida en
$\semiplano[g]$ asocia, a $F$, la funci\'on
\begin{equation}
	\label{eq:funciones:para:operador}
	F\operador[k] M(\Omega)\,=\,
		% \frac{\multiplier(M)^{kg/2}}{\det(C\Omega+D)^k}\,
		\frac{\multiplier(M)^{kg-\frac{g(g+1)}{2}}}{\det(C\Omega+D)^k}\,
		F(M\accion \Omega)
	\dispstop
\end{equation}
%
% Otra versi\'on del operador asocia a $F$ la funci\'on
% \begin{equation}
	% \label{eq:funciones:para:operador:bis}
	% F\varoperador[k] M(\Omega)\,=\,
		% \frac{\multiplier(M)^{kg-\frac{g(g+1)}{2}}}{\det(C\Omega+D)^k}\,
		% F(M\accion \Omega)
	% \dispstop
% \end{equation}
% %
% La relaci\'on entre ambos est\'a dada por
% \begin{displaymath}
	% F\varoperador[k] M\,=\,
		% \multiplier(M)^{kg/2-\frac{g(g+1)} 2}\,F\operador[k] M
	% \dispstop
% \end{displaymath}
% %

\begin{defFuncionesPara}\label{def:funciones:para}
	Si $K$ es un subgrupo de congruencia, una
	\emph{forma modular (de Siegel) de peso $k$ y g\'enero $g$ %
	con respecto a $K$} es una funci\'on
	$F:\,\semiplano[g]\rightarrow\Complejos$
	\begin{enumerate}[label=(P\arabic*)]
		\item\label{item:funciones:para:holomorfia}
			holomorfa,
		\item\label{item:funciones:para:transformacion}
			que verifica $F\operador[k] M=F$,
			para toda $M\in K$, y
		\item\label{item:funciones:para:crecimiento}
			tal que
			la funci\'on $F\operador[k] M$ est\'a acotada
			en regiones $\{Y\geq c\Id[g]\}$,
			para toda $M\in\modulgruppe[g]$.
	\end{enumerate}
	%
	La forma $F$ es \emph{cuspidal}, si adem\'as satisface
	\begin{equation}
		\label{eq:funciones:para:cuspidal}
		\Siegel F\operador[k] M\,=\,0
		\dispcomma
	\end{equation}
	%
	para toda $M\in\modulgruppe[g]$.
\end{defFuncionesPara}

% Al espacio de formas modulares que verifican
% \eqref{eq:funciones:para:transformacion} lo denotamos $\modulformen[k](K)$.
% Usando el operador de peso $k$, podemos tambi\'en definir formas cuspidales:
% una forma $F\in\modulformen[k](K)$ es \emph{cuspidal}, si
% \emph{para toda $M\in\modulgruppe[g]$},
% \begin{equation}
	% \label{eq:funciones:para:cuspidal}
	% \Siegel F\operador[k] M\,=\,0
	% \dispstop
% \end{equation}
% %

Toda forma modular $F$ admite un desarrollo en serie de Fourier.
Para describirlo, introducimos la siguiente notaci\'on:
\begin{displaymath}
	\begin{aligned}
		\Simetricas g(\Racionales) & \,=\,
			\Big\{S\in\Mat(g\times g,\Racionales)\,:\,\trnsp S=S\Big\} \\
		\Positivas{g,\semi}(\Racionales) & \,=\,
			\Big\{S\in\Simetricas g(\Racionales)\,:\,S\geq 0\Big\} \\
		\Positivas g(\Racionales) & \,=\,
			\Big\{S\in\Simetricas g(\Racionales)\,:\,S>0\Big\} \\
		\EnterasPositivas{g,\semi} & \,=\,
			\Big\{T\in\Positivas{g,\semi}(\Racionales)\,:\,
				2T\text{ es par}\Big\} \\
		\EnterasPositivas g & \,=\,
			\Big\{T\in\Positivas g(\Racionales)\,:\,
				2T\text{ es par}\Big\} \\
		n(S) & \,=\,
			\begin{bmatrix} \Id[g] & S \\ & \Id[g] \end{bmatrix}
			\qquad\text{(\phantom)}
			n \,:\,\Simetricas g(\Racionales)\,\rightarrow\,
				\Sp(g,\Racionales)
			\text{\phantom()} \\
		u(U) & \,=\,
			\begin{bmatrix} U & \\ & \trnsp U^{-1}\end{bmatrix}
			\qquad\text{(\phantom)}
			u \,:\,\GL(g,\Racionales)\,\rightarrow\,
				\Sp(g,\Racionales)
			\text{\phantom()}
	\end{aligned}
	%
\end{displaymath}
%
Las matrices sim\'etricas constituyen un espacio vectorial, $\Simetricas g$;
las matrices definidas positivas y semidefinidas positivas,
$\Positivas g$ y $\Positivas{g,\semi}$ son conos en $\Simetricas g$. Los
subconjuntos $\EnterasPositivas g$ y $\EnterasPositivas{g,\semi}$ est\'an
conformados por aquellas matrices que representan formas cuadr\'aticas enteras,
definidas positivas y semidefinidas positivas, respectivamente; son
ret\'{\i}culos en los respectivos conos. Las aplicaciones $X\mapsto n(X)$ y
$U\mapsto u(U)$ son morfismos de grupos.

Con esta notaci\'on, podemos expresar el desarrollo en serie de una forma de
Siegel para $\modulgruppe[g]$ de la siguiente manera:
\begin{displaymath}
	F(\Omega)\,=\,\sum_{T\in\EnterasPositivas{g,\semi}}\,
		a(T)\,e^{2\pi i\traza(T\Omega)}
	\text{ ;}
\end{displaymath}
%
si $F$ es cuspidal, indexamos sobre $\EnterasPositivas g$.
En general, si $F\in\modulformen[k](K)$,
% Intersecamos con $\Positivas{g,\semi}$, porque $F$ es holomorfa.
\begin{equation}
	\label{eq:funciones:para:fourier}
	F(\Omega)\,=\,\sum_{T\in\dual R\cap\Positivas{g,\semi}}\,a(T)\,
		\varexp^{2\pi i\traza(T\Omega)}
		\dispcomma
\end{equation}
%
donde $\dual R$ denota el ret\'{\i}culo dual con respecto a la forma traza en
$\Simetricas g$ del ret\'{\i}culo
\begin{displaymath}
	R\,:=\,R(K)\,=\,\big\{S\in\Simetricas g\,:\,n(S)\in K\big\}
	\dispstop
\end{displaymath}
%
Por ejemplo, si $g=2$, con respecto a los subgrupos de congruencia mencionados,
\begin{itemize}
	\item $R(\varGrupoSiegel(N))=\Simetricas 2(\Racionales)\cap\Mat(2\times 2,\Enteros)$ y
		$\dual R\cap\Positivas{2,\semi}=\EnterasPositivas{2,\semi}$;
	\item $R(\varGrupoKlingen(N))=R(\varGrupoBorel(N))=R(\varGrupoSiegel(N))$;
	\item
		\begin{math}
			R(\varGrupoParamodular(N))=\Big\{
				\sbmatrix{ a/N & b \\ b & c }\,:\,
				a,b,c\in\Enteros\Big\}
		\end{math} y
		\begin{math}
			\dual R=\Big\{
				\sbmatrix{ m & r/2 \\ r/2 & n }\,:\,
				m,r,n\in\Enteros,\,N|m\Big\}
		\end{math}.
\end{itemize}
%
En este \'ultimo caso, escribimos
\begin{math}
	\EnterasPositivas[N]{2,\semi}:=\dual R\cap\Positivas{2,\semi}
\end{math}; es el conjunto de formas cuadr\'aticas enteras semidefinidas
positivas $mx^2+rxy+ny^2$ con $m\in N\Enteros$.

Las transformaciones $u(U)$ vinculan los varios coeficientes de Fourier. Si
$F\in\modulformen[k](K)$, entonces
\begin{equation}
	\label{eq:funciones:para:coeficientes}
	a(T[U])\,=\,\det(U)^k\,a(T)
	\text{ ,}
\end{equation}
%
para toda $U\in\GL(g,\Racionales)$ tal que $u(U)\in K$. Por ejemplo,
\begin{itemize}
	\item $u(U)\in\varGrupoSiegel(N)$, si y s\'olo si
		$U\in\GL(2,\Enteros)$;
	\item $u(U)\in\varGrupoParamodular(N)$, si y s\'olo si
		$U\in\generado{\Gamma_0(N),\sbmatrix{ 1 & \\ & -1 }}$.%
		\footnote{
			Aqu\'{\i}, $\Gamma_0(N)$ denota el subgrupo de
			congruencia de nivel $N$ en $\SL(2,\Enteros)$.
		}
\end{itemize}
%

Los operadores de Hecke $T(m)$, $m\in\Naturales$, son operadores de
doble coclase, pero se pueden describir en t\'erminos de su efecto
en los coeficientes de Fourier de una forma paramodular.
Si $N=p$ es primo y $q\neq p$ es primo y $T\geq 0$,
\begin{displaymath}
	\begin{aligned}
		a(T;F\operador[k]{T(q)}) & \,=\,a(q\,T(q);F)
				\,+\,q^{2k-3}\,a\big(\tfrac 1 q\,T;F\big) \\
		& \qquad\,+\,q^{k-2}\,\sum_{j\tmodulo[q]}\,
			a\big(\tfrac 1 q\,T[\sbmatrix{ 1 & 0 \\ jp & q }];F
				\big) \\
		& \qquad\,+\,q^{k-2}\,
			a\big(\tfrac 1 q\,T[\sbmatrix{ 1 & 0 \\ 0 & q }];F
				\big)
		\dispstop
	\end{aligned}
	%
\end{displaymath}
%

