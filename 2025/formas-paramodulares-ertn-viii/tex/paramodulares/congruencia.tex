Dado un n\'umero natural $N\geq 1$, el \emph{subgrupo principal de %
congruencia de nivel $N$} es el siguiente subgrupo del grupo modular:
% \footnote{
	% Se omite la referencia al g\'enero, $g$.
% }
\begin{displaymath}
	\GrupoPrincipal(N) \,=\,\Big\{M\in\modulgruppe[g]\,:\,
		M\equiv\Id[g]\tmodulo[N]\Big\}
	\dispstop
\end{displaymath}
%
Este subgrupo es el n\'ucleo del morfismo
$\modulgruppe[g]\rightarrow \Sp(g,\Enterosmod[N])$ dado por reducir las
coordenadas m\'odulo $N$; es un subgrupo normal de \'{\i}ndice finito. Un
\emph{subgrupo de congruencia} es un subgrupo $K\leq\GSp(g,\Reales)^+$ que
cumple con:
\begin{itemize}
	\item
		%\label{item:subgrupos:conmensurabilidad}
		ser \emph{conmensurable con $\modulgruppe[g]$}, es decir,
		$K\cap\modulgruppe[g]$ tiene \'{\i}ndice finito en
		$\modulgruppe[g]$ y en $K$, y
	\item
		%\label{item:subgrupos:nivel}
		contener un subgrupo principal de congruencia.
\end{itemize}
%
Por ejemplo, si $g=2$, los subgrupos
\begin{displaymath}
	\begin{aligned}
		\varGrupoSiegel(N) & \,:=\,\bigg\{
			\begin{bmatrix} A & B \\ C & D \end{bmatrix}\in
				\modulgruppe[2]\,:\,C\equiv0\tmodulo[N]
			\bigg\}\,=\,\modulgruppe[2]\,\cap\,
				\begin{bmatrix}
					\Enteros & \Enteros & \Enteros & \Enteros \\
					\Enteros & \Enteros & \Enteros & \Enteros \\
					N\Enteros & N\Enteros & \Enteros & \Enteros \\
					N\Enteros & N\Enteros & \Enteros & \Enteros
				\end{bmatrix}
			\dispand \\
		\varGrupoParamodular(N) & \,:=\,\Sp(2,\Racionales)\,\cap\,
			\begin{bmatrix}
				\Enteros & \Enteros & \tfrac 1 N\Enteros & \Enteros \\
				N\Enteros & \Enteros & \Enteros & \Enteros \\
				N\Enteros & N\Enteros & \Enteros & N\Enteros \\
				N\Enteros & \Enteros & \Enteros & \Enteros
			\end{bmatrix}
	\end{aligned}
	%
\end{displaymath}
%
son subgrupos de congruencia. El grupo $\varGrupoSiegel(N)$ se conoce como
\emph{grupo de congruencia de Siegel de nivel $N$}; el grupo
$\varGrupoParamodular(N)$ es el \emph{grupo paramodular de nivel $N$}.
% \footnote{
	% En general,
	% \begin{displaymath}
		% \Sp[2g](\bb Q)\,\cap\,
			% \begin{bmatrix}
				% \Id[g] & \\ & P
			% \end{bmatrix}\,\MM[g\times g](\Enteros)\,
			% \begin{bmatrix}
				% \Id[g] & \\ & P^{-1}
			% \end{bmatrix}
	% \end{displaymath}
	% donde $P=\diag(\lista t{g})$, $t_i\in\Enteros$, $t_i|t_{i+1}$.
% }
Los subgrupos
\begin{displaymath}
	\begin{aligned}
		\varGrupoKlingen(N) & \,:=\, \modulgruppe[2]\,\cap\,
				\begin{bmatrix}
					\Enteros & \Enteros & \Enteros & \Enteros \\
					N\Enteros & \Enteros & \Enteros & \Enteros \\
					N\Enteros & N\Enteros & \Enteros & N\Enteros \\
					N\Enteros & \Enteros & \Enteros & \Enteros
				\end{bmatrix}
			\,=\,\modulgruppe[2]\,\cap\,\varGrupoParamodular(N)
			\dispand \\
		\varGrupoBorel(N) & \,:=\, \modulgruppe[2]\,\cap\,
				\begin{bmatrix}
					\Enteros & \Enteros & \Enteros & \Enteros \\
					N\Enteros & \Enteros & \Enteros & \Enteros \\
					N\Enteros & N\Enteros & \Enteros & N\Enteros \\
					N\Enteros & N\Enteros & \Enteros & \Enteros
				\end{bmatrix}
			% \,=\,\varGrupoSiegel(N)\,\cap\,\varGrupoParamodular(N)
			\,=\,\varGrupoSiegel(N)\,\cap\,\varGrupoKlingen(N)
			\dispcomma
	\end{aligned}
	%
\end{displaymath}
%
el \emph{subgrupo de congruencia de Klingen} y
el \emph{subgrupo de congruencia de Borel}, respectivamente,
tambi\'en son subgrupos de congruencia.
Todos estos grupos contienen al subgrupo principal de nivel $N$.
\begin{figure}[h]
	\centering
	\begin{tikzcd}[row sep=small]
		& \modulgruppe[2] & & \\
		& & & \varGrupoParamodular(N) \\
		\varGrupoSiegel(N) \arrow[uur,-] & & & \\
		& & \varGrupoKlingen(N) \arrow[uur,-] \arrow[uuul,-] & \\
		& \varGrupoBorel(N) \arrow[ur,-] \arrow[uul,-] & &
	\end{tikzcd}
	\caption{Diagrama de las relaciones de inclusi\'on entre los %
		subgrupos de congruencia.}
	\label{fig:icongruencia:subgrupos}
\end{figure}

Los subgrupos de congruencia admiten una noci\'on de forma modular; la
definici\'on es an\'aloga a la de formas de Siegel para el grupo
$\modulgruppe[g]$.
