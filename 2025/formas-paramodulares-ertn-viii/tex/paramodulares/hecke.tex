A diferencia de lo que ocurre en el caso de nivel $N=1$,
el \'algebra de Hecke paramodular de nivel $N>1$ no es,
en general, conmutativa.
El \'algebra de Hecke paramodular est\'a generada por las \'algebras
``locales''. Si $p\mid N$ es primo, entonces el \'algebra local en $p$
est\'a generada por
\begin{displaymath}
	\begin{aligned}
		V & \,=\,T(w) \dispcomma \\
		X & \,=\,T(\diag(1,1,p,p))\dispcomma \\
		Y_1 & \,=\,T(\diag(1,p,p^2,p))\dispand \\
		Y_2 & \,=\,T(\diag(p,1,p,p^2))
		\dispstop
	\end{aligned}
	%
\end{displaymath}
%
El elemento $w\in\GSp(2,\Racionales)$ es la matriz
\begin{displaymath}
	\begin{bmatrix} & 1 & & \\ p & & & \\ & & & p \\ & & 1 & \end{bmatrix}
	\dispstop
\end{displaymath}
%
El \'algebra en cuesti\'on es un cociente de
$\Enteros\{X,V,Y_1,Y_2\}$ por el ideal de ciertas relaciones expl\'{\i}citas
entre estos elementos.

