\theoremstyle{plain}
\newtheorem{teoHecke}{\teoname}[subsection]
\newtheorem{lemaHecke}[teoHecke]{\lemaname}

\theoremstyle{definition}
\newtheorem{defHecke}[teoHecke]{\defname}

%-------------

Empezamos definiendo el anillo de Hecke en abstracto y enunciando algunas
de sus propiedades. Usaremos la siguiente notaci\'on:
$G=\GSp(g,\Racionales)$, $G^+=G\cap\{\multiplier>0\}$,
% $\Sigma^+=G^+\cap\Mat(2g\times 2g,\Enteros)$
y $\modulgruppe=\Sp(g,\Enteros)$.

Sea $\cal L$ el espacio vectorial (m\'odulo libre) cuyos elementos son las
sumas formales de coclases a derecha
\begin{equation}
	\label{eq:hecke:elementos}
	\sum_{\modulgruppe M\subseteq G^+}\,
		c_{\modulgruppe M}\,\modulgruppe M
	\dispcomma
\end{equation}
%
con $c_{\modulgruppe M}=0$ para toda coclase salvo una cantidad finita.
El subgrupo $\modulgruppe$ act\'ua en $\cal L$:
$(\modulgruppe M)\cdot\gamma=\modulgruppe M\gamma$.
En el subespacio (subm\'odulo) $\cal L^{\modulgruppe}$, podemos definir
una operaci\'on:
\begin{equation}
	\label{eq:hecke:operacion}
	\Big(\sum_i\,c_i\,\modulgruppe M_i\Big)\,
		\Big(\sum_j\,c_j'\,\modulgruppe M_j'\Big)\,=\,
		\sum_{i,j}\,c_ic_j'\,\modulgruppe M_iM_j'
	\dispstop
\end{equation}
%
Esta nueva expresi\'on no depende de los representantes
$M_i\in\modulgruppe M_i$, ni $M_j'\in\modulgruppe M_j'$ y pertenece a
$\cal L^{\modulgruppe}$.
Por otro lado, cada doble coclase $\modulgruppe M\modulgruppe$,
$M\in G^+$ se descompone como uni\'on finita (disjunta) de coclases a
derecha: $\modulgruppe M\modulgruppe=\bigsqcup_i\,\modulgruppe M_i$ y
determina, por lo tanto, un elemento de $\cal L$.
Denotamos por $\Heckerng(G^+,\modulgruppe)$ el espacio vectorial
(m\'odulo libre) con base las dobles coclases
$\modulgruppe M\modulgruppe$, $M\in G^+$. La aplicaci\'on
\begin{displaymath}
	\modulgruppe M\modulgruppe\,=\,\bigsqcup_i\,\modulgruppe M_i
		\,\mapsto\,\sum_i\,\modulgruppe M_i
\end{displaymath}
%
determina un isomorfismo
$\Heckerng(G^+,\modulgruppe)\simeq\cal L^{\modulgruppe}$.

\begin{defHecke}\label{def:hecke}
	El \emph{\'algebra de Hecke del par $(G^+,\modulgruppe)$} es el
	\'algebra cuyo espacio subyacente es $\Heckerng(G^+,\modulgruppe)$
	con producto dado por traslaci\'on de estructura de
	$\cal L^{\modulgruppe}$.
\end{defHecke}

Para simplificar la notaci\'on, suponemos $g=2$.

\begin{lemaHecke}[Divisores simpl\'ecticos]
	\label{lema:hecke:divisores}
	Sea $\Sigma^+=G^+\cap\Mat(2g\times 2g,\Enteros)$.
	Si $M\in\Sigma^+$, entonces $\modulgruppe[2] M\modulgruppe[2]$ posee
	un \'unico representante de la forma
	\begin{displaymath}
		\spdiv(M)\,=\,\diagonal{a_1,a_2,d_1,d_2}
		\dispcomma
	\end{displaymath}
	%
	donde $a_i,d_i\in\Enteros$, $a_1\mid a_2\mid d_2\mid d_1$ y
	$a_1d_1=a_2d_2=\multiplier(M)\in\Enteros>0$.
\end{lemaHecke}

\begin{teoHecke}\label{teo:hecke}
	El \'algebra $\Heckerng(G^+,\modulgruppe[2])$ es conmutativa.
\end{teoHecke}

\begin{proof}
	La aplicaci\'on $M\mapsto\adjunta(M)=\multiplier(M)\,M^{-1}$ es
	una antiinvoluci\'on en $G^+$ que tambi\'en preserva $\modulgruppe[2]$.
	Adem\'as, se verifica
	\begin{math}
		\modulgruppe[2]\adjunta(M)\modulgruppe[2]=
			\modulgruppe[2] M\modulgruppe[2]
	\end{math}
	(si $M$ es diagonal, $\adjunta(M)=JMJ^{-1}$, pero
	$J\in\modulgruppe[2]$).
	En particular, dicha antiinvoluci\'on es trivial en
	$\Heckerng(G^+,\modulgruppe[2])$.
\end{proof}

Utilizaremos la siguiente notaci\'on para referirnos a ciertas
(uniones de) dobles coclases o, equivalentemente, elementos del
\'algebra de Hecke:
$T(M)=\modulgruppe[2] M\modulgruppe[2]$,
\begin{displaymath}
	% T(\lista a g,\,\lista d g)\,=\,T(\diagonal{\lista a g,\,\lista d g})
	T(a_1,a_2,d_1,d_2)\,=\,T(\diag{a_1,a_2,d_1,d_2})
	\dispcomma
\end{displaymath}
%
y $\diagonal d=T(d\Id[2g])$.

Los elementos $T(M)$ poseen la siguiente propiead:
si $\mcd{d_1/a_1,d_1'/a_1'}=1$, entonces
\begin{displaymath}
	\begin{aligned}
		% & T(\lista a g,\,\lista d g)\,
			% T(a_1',\,\dots,\,a_g',\,d_1',\,\dots,\,d_g') \\
		% & \qquad\,=\, T(a_1a_1',\,\dots,\,a_ga_g',\,
			% d_1d_1',\,\dots,\,d_gd_g')
		T(a_1,a_2,d_1,d_2)\,T(a_1',a_2',d_1',d_2')\,=\,
			T(a_1a_1',a_2a_2',d_1d_1',d_2d_2')
		\dispstop
	\end{aligned}
	%
\end{displaymath}
%
En particular,
\begin{displaymath}
	\begin{aligned}
		% \diagonal d\,T(\lista a g,\,\lista d g) & \,=\,
			% T(da_1,\,\dots,\,da_g,\,dd_1,\,\dots,\,dd_g) \\
		% & \,=\, T(\lista a g,\,\lista d g)\,\diagonal d
		\diagonal d\,T(a_1,a_2,d_1,d_2)\,=\,
			T(da_1,da_2,dd_1,dd_2)
		\dispstop
	\end{aligned}
	%
\end{displaymath}
%
Adem\'as, si $m\geq 1$ y
\begin{displaymath}
	T(m)\,=\,\cal O_g(m)\,=\,\big\{M\in\Sigma^+\,:\,\multiplier(M)=m\big\}
	\dispcomma
\end{displaymath}
%
entonces se cumple
\begin{displaymath}
	\begin{aligned}
		T(m) & \,=\,\sum\Big\{
			% T(\lista a g,\,\lista d g)\,:\,
			% a_id_i=m,\,
			% a_i\mid a_{i+1},\,a_g\mid d_g,\,
			% d_{i+1}\mid d_i
			T(a_1,a_2,d_1,d_2)\,:\,a_1d_1=a_2d_2=m,\,
				a_1\mid a_2\mid d_2\mid d_1\Big\} \\
		& \,=\,\sum\Big\{\modulgruppe[2] M\,:\,
			\modulgruppe[2] M\subseteq\Sigma^+,\,
			\multiplier(M)=m\Big\}
		\dispstop
	\end{aligned}
	%
\end{displaymath}
%
Si $\mcd{m,m'}=1$, entonces $T(m)T(m')=T(mm')$.

An\'alogamente, podemos definir versiones ``locales'' de estos anillos.
Usamos la siguiente notaci\'on: dado un primo $p$,
$G_p=\GSp(g,\polinomios[1/p]\Enteros)$,
$G_p^+=G_p\cap G^+$ y $\Sigma_p^+=\Sigma^+\cap G_p^+$.

\begin{lemaHecke}\label{lema:hecke:locales}
	El \'algebra $\Heckerng(G^+,\modulgruppe[2])$ est\'a
	generada por todas las sub\'algebras
	$\Heckerng(G_p^+,\modulgruppe[2])$.
\end{lemaHecke}

\begin{proof}
	Hay una noci\'on de divisor simpl\'ectico \emph{en $p$},
	$\spdiv[p]$ y se verifica
	\begin{displaymath}
		\spdiv(M)\,=\,\prod_p\,\spdiv[p](M)
		\dispstop
	\end{displaymath}
	%
	Entonces, $T(M)=T(\spdiv(M))=\prod_p\,T(\spdiv[p](M))$.
\end{proof}

\begin{lemaHecke}\label{lema:hecke:locales:bis}
	Con $m=p,p^2$, se verifica
	\begin{displaymath}
		% T(p)\,=\,T(1,\,\dots,\,1,\,p,\,\dots,\,p)
		T(p)\,=\,T(1,1,p,p)
		\dispand
		% T(p^2)\,=\,\sum_{r=0}^g\,T_r(p^2)
		T(p^2)\,=\,T_0(p^2)\,+\,T_1(p^2)\,+\,T_2(p^2)
		\dispcomma
	\end{displaymath}
	%
	donde
	\begin{displaymath}
		% T_r(p^2)\,:=\,
			% T(1,\,\dots,\,1,\,
				% \underbrace{p,\,\dots,\,p}_{r},\,
				% p^2,\,\dots,\,p^2,\,
				% p,\,\dots,\,p)
		% \dispstop
		\begin{aligned}
			T_0(p^2) & \,=\,T(1,1,p^2,p^2) \dispcomma \\
			T_1(p^2) & \,=\,T(1,p,p^2,p) \dispand \\
			T_2(p^2) & \,=\,T(p,p,p,p)\,=\,\diagonal p
			\dispstop
		\end{aligned}
	\end{displaymath}
	%
\end{lemaHecke}

% Notamos que $T_g(p^2)=\diagonal p$.

\begin{teoHecke}\label{teo:hecke:algebra}
	Los elementos $T(p)$, $T_0(p^2)$, $T_1(p^2)$ y $T_2(p^2)$
	%\dots, $T_g(p^2)$
	generan $\Heckerng(\Sigma_p^+,\modulgruppe[2])$.
\end{teoHecke}

El \teoname~\ref{teo:hecke:algebra} implica que existen ciertas relaciones
entre los operadores $T(p^k)$, $k\geq 0$. Estas relaciones se pueden
encapsular en una serie de potencias.

\begin{teoHecke}\label{teo:hecke:serie}
	% Si $g=1$, entonces, en
	% $\Heckerng(\Sigma_p^+,\modulgruppe[1])[\![\indet]\!]$,
	% \begin{displaymath}
		% \sum_{k\geq 0}\,T(p^k)\,\indet^k\,=\,
			% \big(1-T(p)\,X+p\diagonal p\,X^2\big)^{-1}
		% \dispstop
	% \end{displaymath}
	% %
	Si $g=2$, entonces, en
	$\Heckerng(\Sigma_p^+,\modulgruppe[2])[\![\indet]\!]$,
	\begin{displaymath}
		\sum_{k\geq 0}\,T(p^k)\,\indet^k\,=\,
			\big(1-p^2\diagonal p\,\indet^2\big)\,
			\big(1-Q_1(p)\,\indet+Q_2(p)\,\indet^2
				-Q_3(p)\,\indet^3+Q_4(p)\,\indet^4\big)^{-1}
		\dispcomma
	\end{displaymath}
	%
	donde
	\begin{displaymath}
		\begin{aligned}
			Q_1(p) & \,=\,T(p)\dispcomma \\
			Q_2(p) & \,=\,pT_1(p^2)+p\,(p^2+1)\diagonal p
				\,=\,T(p)^2-T(p^2)-p^2\diagonal p \dispcomma \\
			Q_3(p) & \,=\,p^3\diagonal pT(p) \dispand \\
			Q_4(p) & \,=\,p^6\diagonal p^2
			\dispstop
		\end{aligned}
		%
	\end{displaymath}
	%
\end{teoHecke}

