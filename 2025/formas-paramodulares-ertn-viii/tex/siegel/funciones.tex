\theoremstyle{plain}
\newtheorem{teoFunciones}{\teoname}[subsection]
\newtheorem{coroFunciones}[teoFunciones]{\coroname}

\theoremstyle{definition}
\newtheorem{obsFunciones}[teoFunciones]{\obsname}

%-------------

El \emph{semiplano de Siegel (de g\'enero $g$)} es el conjunto conformado por
las matrices sim\'etricas, de tama\~no $g\times g$ y de coordenadas complejas,
cuya parte imaginaria es definida positiva:
\begin{displaymath}
	\semiplano[g]\,=\,\Big\{ \Omega=X+iY\in\Mat(g\times g,\Complejos)\,:\,
		\trnsp\Omega=\Omega,\,\,Y>0\Big\}
	\dispstop
\end{displaymath}
%
Si $g=1$, $\semiplano[1]$ es el semiplano complejo. Si $g=2$,
$\Omega=\sbmatrix{ \omega & z \\ z & \tau }\in\semiplano[2]$, si y s\'olo si
\begin{displaymath}
	\Imag(\omega),\,\Imag(\tau)\,>\,0
	\quad\vardispand\quad
	\Imag(z)^2\,<\,\Imag(\omega)\,\Imag(\tau)
	\dispstop
\end{displaymath}
%
En general, el semiplano de Siegel es un abierto convexo dentro del
espacio de matrices complejas sim\'etricas; es una variedad compleja
y su dimensi\'on es $g(g+1)/2$.

El grupo $\GSp(g,\Reales)^+$ act\'ua en $\semiplano[g]$ v\'{\i}a:
\begin{equation}
	\label{eq:funciones:moebius}
	M\accion\Omega\,=\,\big(A\Omega+B\big)\,\big(C\Omega+D\big)^{-1}
	\dispcomma
\end{equation}
%
para cada $M=\sbmatrix{A & B \\ C & D}\in\GSp[2g](\bb R)^+$ y
$\Omega\in\semiplano[g]$; la acci\'on es transitiva y cada $M$ induce una
transformaci\'on biholomorfa cuya inversa est\'a dada por
\eqref{eq:simplecticas:adjunta}.
Las matrices \eqref{eq:simplecticas:ejemplos} act\'uan por:
\begin{equation}
	\label{eq:funciones:ejemplos}
	\Omega\,\mapsto\,U\Omega\trnsp U\dispcomma\quad
		\Omega\,\mapsto\,\Omega+S\quad\dispand\quad
		\Omega\,\mapsto\,-\Omega^{-1}
		\dispstop
\end{equation}
%

Una \emph{forma modular (de Siegel) de g\'enero $g$ y peso $k$ con respecto %
al grupo $\modulgruppe[g]$} es una funci\'on
$F:\,\semiplano[g]\rightarrow\Complejos$
\begin{enumerate}[label=(M\arabic*)]
	\item\label{item:funciones:holomorfia}
		holomorfa en $\semiplano[g]$,%
		\footnote{
			Una funci\'on en varias variables complejas es
			holomorfa, si localmente admite un desarrollo en
			serie de potencias; equivalentemente, dicha
			funci\'on es holomorfa en su dominio de
			definici\'on, si es holomorfa en cada variable
			por separado \cite{Gunning}.
		}
	\item\label{item:funciones:transformacion}
		que, si $M=\sbmatrix{ * & * \\ C & D }\in\modulgruppe[g]$ y
		$\Omega\in\semiplano[g]$, cumple
		\begin{equation}
			\label{eq:funciones:transformacion}
			F(M\accion\Omega)\,=\,\det(C\Omega+D)^k\,F(\Omega)
			\dispand
		\end{equation}
		%
	\item\label{item:funciones:crecimiento}
		% si $g=1$,
		acotada en regiones de la forma $\{Y\geq c\Id[g]\}$,
		$c\in\Reales$, $c>0$.
\end{enumerate}
%

Las formas modulares de Siegel de peso $k$ constituyen un espacio vectorial
de dimensi\'on finita \cite[Ch.~II, \S~4, Thm.~2]{Klingen}, que denotamos
$\modulformen[k](\modulgruppe[g])$
De \eqref{eq:funciones:transformacion}, se deducen las siguientes reglas
de transformaci\'on para una forma modular $F$:
\begin{enumerate}[label=(\roman*)]
	\item\label{item:funciones:unimodular}
		$F(U\Omega\trnsp U)=\det(U)^k\,F(\Omega)$,
		si $U\in\GL(g,\Enteros)$;
	\item\label{item:funciones:traslacion}
		$F(\Omega+S)=F(\Omega)$,
		si $\trnsp S=S\in\Mat(g\times g,\Enteros)$;
	\item\label{item:funciones:reflexion}
		$F(-\Omega^{-1})=\det(\Omega)^k\,F(\Omega)$.
\end{enumerate}
%

De \ref{item:funciones:traslacion} y de \ref{item:funciones:crecimiento},
se deduce que toda forma modular admite un desarrollo en serie de Fourier
del tipo:
\begin{equation}
	\label{eq:funciones:fourier}
	F(\Omega)\,=\,\sum_{T\geq 0}\,a(T)\,
		\varexp^{2\pi\raizcuarta\traza(T\Omega)}
	\dispcomma
\end{equation}
%
La sumatoria se realiza sobre las matrices sim\'etricas $\trnsp T=T$
tales que: $2T$ tiene coordenadas enteras y pares en la diagonal y
$T$ es semidefinida positiva.
Si $g=1$, \eqref{eq:funciones:fourier} es el desarrollo de Fourier
usual, donde $T\in\Enteros$. Si $g=2$, entonces
$T=\sbmatrix{ m & r/2 \\ r/2 & n }$, $n,r,m\in\Enteros$, representa
formas cuadr\'aticas binarias con coeficientes enteros.

La serie \eqref{eq:funciones:fourier} converge absoluta y uniformemente
sobre compactos de $\semiplano[g]$ y podemos recuperar los coeficientes:
\begin{equation}
	\label{eq:funciones:coeficientes}
	a(T)\,=\,\int_{X\tmodulo[1]}\,F(\Omega)\,
		\varexp^{-2\pi\raizcuarta\traza(T\Omega)}\,\de X
	\dispstop
\end{equation}
%

De \ref{item:funciones:unimodular} y de \eqref{eq:funciones:coeficientes},
se deduce que
\begin{displaymath}
	a(UT\trnsp U)\,=\,(\det\,U)^k\,a(T)
	\dispstop
\end{displaymath}
%
En particular, $kg\not\equiv 0\tmodulo[2]$ implica
$\modulformen[k](\modulgruppe[g])=0$.

\begin{teoFunciones}\label{teo:koecher}
	Supongamos que $g>1$. Si $F$ satisface
	% \ref{item:funciones:unimodular} y \ref{item:funciones:traslacion}
	% y la serie \eqref{eq:funciones:fourier} \emph{converge puntualmente},
	\ref{item:funciones:holomorfia} y \ref{item:funciones:transformacion},
	entonces $F$
	satisface \ref{item:funciones:crecimiento}.
	% est\'a acotada en regiones de la forma $\{Y\geq c\Id[g]\}$.
\end{teoFunciones}

\begin{proof}
	Como \eqref{eq:funciones:fourier} converge en
	$\Omega=\raizcuarta\Id[g]$, existe $\alpha>0$ tal que
	\begin{displaymath}
		|a(T)|\,\leq\,\alpha\varexp^{2\pi\raizcuarta\traza(T)}
		\dispstop
	\end{displaymath}
	%
	Afirmamos que $a(T)\neq 0$ implica $T\geq 0$. Asumiendo esto,
	\begin{displaymath}
		F(\Omega)\,=\,\sum_{T\geq 0}\,a(T)\,
			\varexp^{2\pi\raizcuarta\traza(T\Omega)}\,\ll\,
			\sum_{T\geq 0}\,|a(T)|\,\varexp^{-2\pi c\traza(T)}
		\dispcomma
	\end{displaymath}
	%
	uniformemente en $\{Y\geq c\Id[g]\}$.
	Pero el lado derecho converge, pues es la serie de Fourier
	(que converge a.u./c. de $\semiplano[g]$)
	evaluada en el punto $\raizcuarta c\Id[g]$.
	% La cota se puede deducir asumiendo la hip\'otesis m\'as d\'ebil
	% (a priori) de convergencia puntual:
	% Por otro lado, como la serie \eqref{eq:funciones:fourier}
	% converge en $\Omega=\raizcuarta \frac c 2\Id[g]$, existe $\beta>0$
	% tal que
	% \begin{displaymath}
		% |a(T)|\,\leq\,\beta\,\varexp^{\pi c\traza(T)}
		% \dispstop
	% \end{displaymath}
	% %
	% En particular, de la cota anterior para $F(\Omega)$, obtenemos
	% \begin{displaymath}
		% F(\Omega)\,\ll\,\sum_{T\geq 0}\,\varexp^{-\pi c\traza(T)}\,\ll\,
			% 1\,+\,\sum_{d\geq 1}\,d^{g(g+1})/2\,
				% \varexp^{-\pi cd}\,<\,\infty
		% \dispcomma
	% \end{displaymath}
	% %
	% donde, para la \'ultima cota, estamos usando que
	% \begin{math}
		% \cardinal{\big\{T\geq 0\,:\,%
			% 2T\text{ par y }\traza(T)=d\big\}}=
			% \BigO{d^{g(g+1)/2}}
	% \end{math}.

	Con respecto a la afirmaci\'on, si $T$ \emph{no es} semidefinida
	positiva, existe $V_1$ de coordenadas enteras y contenido $1$
	tal que $T[V_1]<0$. Completando $V_1$ a una $V\in\GL(g,\Enteros)$,
	cambiando $T$ por $T[V]$ y usando $a(T[V])=(\det\,V)^k\,a(T)$,
	podemos suponer que $T_{11}<0$. Ahora, sea
	$U=\sbmatrix{ U^* & \\ & \Id[g-2] }$, con
	$U^*=\sbmatrix{ 1 & m \\ & 1 }$ y $m\in\Enteros$. El coeficiente
	\begin{displaymath}
		|a(T)|\,=\,|a(T[U])|\,\leq\,\alpha\,
			\varexp^{2\pi\traza(T[U])}
		\dispcomma
	\end{displaymath}
	%
	pero $\traza(T[U])=\traza(T)+2T_{12}m+T_{11}m^2$ tiende a
	$-\infty$ ($m\to\infty$), pues $T_{11}<0$.
\end{proof}

% \begin{obsFunciones}\label{obs:koecher}
	% En la demostraci\'on del \teoname~\ref{teo:koecher},
	% probamos la siguiente afirmaci\'on.
	% Supongamos que $F(\Omega)$ satisface
	% % \ref{item:funciones:unimodular}, \ref{item:funciones:traslacion}
	% % y la serie \eqref{eq:funciones:fourier} converge (puntualmente).
	% \ref{item:funciones:holomorfia} y \ref{item:funciones:transformacion}.
	% Si $a(T)\neq 0$ implica $T\geq 0$, \emph{entonces} $F$ est\'a
	% acotada en regiones de la forma $\{Y\geq c\Id[g]\}$
	% (es decir, cumple \ref{item:funciones:crecimiento}).
	% Rec\'{\i}procamente, si $F(\Omega)$ est\'a acotada en regiones de la
	% forma $\{Y\geq c\Id[g]\}$, entonces $a(T)=0$ para $T\not\geq 0$:
	% de la expresi\'on \eqref{eq:funciones:coeficientes},
	% si $|F(\Omega)|\leq C$ en la regi\'on $\{Y\geq c\Id[g]\}$, entonces
	% \begin{displaymath}
		% |a(T)|\,\leq\,C\,\varexp^{2\pi\traza(TY)}
		% \dispcomma
	% \end{displaymath}
	% %
	% si $Y\geq c\Id[g]$. Si $T\not\geq 0$ y
	% $T=\sbmatrix{ T^* & T_2 \\ \trnsp T_2 & T_4 }$, $T^*\in\Enteros$,
	% cambiando $T$ por $T[V]$, $V\in\GL(g,\Enteros)$, podemos asumir
	% $T^*<0$. Eligiendo $Y=\sbmatrix{ Y^* & Y_2 \\ \trnsp Y_2 & Y_4 }$
	% tal que $Y_2=0$, tenemos $\traza(TY)=T^*Y^*+\traza(T_4Y_4)$.
	% Si $Y^*=mc$ e $Y_4=c\Id[g-1]$, se cumple
	% \begin{displaymath}
		% \traza(TY)\,=\,m\,T^*+\traza(T_4)\,<\,0
		% \dispcomma
	% \end{displaymath}
	% %
	% para $m>0$ suficientemente grande,
	% y, por lo tanto,
	% % con $m$ tendiendo a $\infty$,
	% deducimos que $a(T)=0$.
% \end{obsFunciones}

Si $F$ es una forma modular, entonces su expansi\'on de Fourier es
del tipo
\begin{displaymath}
	F(\Omega)\,=\,\sum_{T\geq 0}\,a(T)\,
		\varexp^{2\pi\raizcuarta\traza(T\Omega)}
	\dispcomma
\end{displaymath}
%
donde, ahora sabemos, $a(T)\neq 0$, s\'olo si $T\geq 0$, es decir,
$T$ representa una forma cuadr\'atica semidefinida positiva,
con coeficientes enteros.
En particular, si $g=2$,
% y escribimos $T=\sbmatrix{n & r/2 \\ r/2 & m }$ con $n,r,m\in\Enteros$,
% la condici\'on $T\geq 0$ se traduce en
% \begin{displaymath}
	% n,m\,\geq\,0\quad\dispand\quad r^2-4mn\,\leq\,0
	% \dispstop
% \end{displaymath}
% %
% Usando estas coordenadas,
podemos reescribir la expansi\'on de Fourier de una forma
de Siegel de g\'enero $2$ de la siguiente manera:
\begin{equation}
	\label{eq:funciones:fourier:dos}
	F(\tau,z,\omega)\,=\,\sum_{\binaria{n,r,m}\geq 0}\,a(n,r,m)\,
			\varexp^{2\pi\raizcuarta\,(n\tau+rz+m\omega)}
	\dispcomma
\end{equation}
%
donde la condici\'on $\binaria{n,r,m}\geq 0$ quiere decir que la
forma es semidefinida positiva: $n,m,4mn-r^2\geq 0$.
Si agrupamos los t\'erminos en funci\'on de $m\geq 0$,
\begin{equation}
	\label{eq:funciones:fourier-jacobi}
	F(\tau,z,\omega)\,=\,\sum_{m\geq 0}\,f_m(\tau,z)\,
		\varexp^{2\pi\raizcuarta \omega m}
	\dispstop
\end{equation}
%
Las funciones $f_m$ son \emph{formas de Jacobi de \'{\i}ndice $m$}
y la expansi\'on \eqref{eq:funciones:fourier-jacobi} se conoce como
la \emph{expansi\'on de Fourier-Jacobi de $F$} \cite{EichlerZagier}.

% \begin{coroFunciones}\label{coro:peso-negativo}
	% Si $k<0$, $\modulformen[k](\modulgruppe[g])=0$.
% \end{coroFunciones}
% 
% \begin{proof}
	% Si $F$ es de peso $k$ invariante por $\modulgruppe[g]$, entonces
	% la funci\'on $h(\Omega)=(\det\,Y)^{k/2}\,F(\Omega)$ es invariante por
	% $\modulgruppe[g]$. Como $(\det\,Y)^{-1}$ est\'a acotada en
	% $\{Y\geq c\Id[g]\}$, si $F(\Omega)$ est\'a acotada en estas regiones y
	% $k<0$, tambi\'en lo estar\'a $h(\Omega)$. En particular, en tal caso,
	% $h(\Omega)$ estar\'a acotada en $\semiplano[g]$:
	% $(\det\,y)^{k/2}\,|F(\Omega)|\leq C$. Entonces,
	% \emph{para toda $Y$},
	% \begin{displaymath}
		% \begin{aligned}
			% \big|a(T)\,\varexp^{-2\pi\raizcuarta\traza(TY)}\big|
				% & \,=\, \left|\int_{X\tmodulo[1]}\,F(\Omega)\,
					% \varexp^{-2\pi\raizcuarta\traza(TX)}\,
						% \de X \right| \\
				% & \,\leq\, \sup\big\{|F(\Omega)|\,:\,
					% X\tmodulo[1]\big\}\,\leq\,
					% C\,(\det\,Y)^{-k/2}
		% \dispstop
		% \end{aligned}
		% %
	% \end{displaymath}
	% %
	% De lo que se deduce ($Y\to 0$) que $a(T)=0$.
% \end{proof}
% 
% Haciendo un juego de este estilo, se puede probar que, dada $c>0$,
% existen un compacto $K$ (contenido en el dominio fundamental)
% y una constante $\alpha$ tales que
% \begin{displaymath}
	% \sup\big\{|F(\Omega)|\,:\,Y\geq c\Id[g]\big\}\,\leq\,\alpha^k\,
		% \sup\big\{|F(\Omega)|\,:\,\Omega\in K\big\}
	% \dispstop
% \end{displaymath}
% %
% En particular, con $c>0$ suficientemente chico, $\{Y\geq c\Id[g]\}$
% contiene al dominio fundamental en su interior.
% 
% \begin{coroFunciones}\label{coro:peso-cero}
	% Si $k=0$, $\modulformen[0](\modulgruppe[g])=\Complejos$.
% \end{coroFunciones}
% 
% \begin{proof}
	% Aplicar principio del m\'aximo.
% \end{proof}

