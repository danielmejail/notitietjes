\theoremstyle{plain}
\newtheorem{teoEisenstein}{\teoname}[subsection]

\theoremstyle{definition}
\newtheorem{defEisenstein}[teoEisenstein]{\defname}
\newtheorem{obsEisenstein}[teoEisenstein]{\obsname}

%-------------

Dada $F\in\modulformen[k](\modulgruppe[g])$ definimos una forma
de g\'enero menor. Esto nos permitir\'a definir la noci\'on de forma
cuspidal. El espacio $\semiplano[g]$ tiene un borde.
Vemos c\'omo restringir una forma a ese ``borde''.
Dadas $M\in\Mat(g\times g)$, $\trnsp M=M$, y $0\leq r\leq g$,
escribimos
\begin{displaymath}
	M\,=\,\begin{bmatrix} M^* & M_2 \\ \trnsp M_2 & M_4 \end{bmatrix}
	\dispcomma
\end{displaymath}
%
con $M^*\in\Mat(r\times r)$.
La siguiente identidad puede ser \'util:
\begin{displaymath}
	\begin{bmatrix} M^* & M_2 \\ \trnsp M_2 & M_4 \end{bmatrix}\,=\,
		\begin{bmatrix}
			M^*-M_4^{-1}[\trnsp M_2] & \\ & M_4
		\end{bmatrix}[
		\begin{bmatrix}
			\Id[r] & \\ -M_4^{-1}\trnsp M_2 & \Id[g-r]
		\end{bmatrix}]
	\dispstop
\end{displaymath}
%
Si $g=0$, definimos $\modulformen[k](\modulgruppe[0])$ como $\Complejos$,
si $k\geq 0$, y como $0$, si $k<0$.

\begin{teoEisenstein}\label{teo:siegel}
	Sean $g\geq 1$, $0\leq r\leq g$.
	Si $Z^{(\nu)}\in\semiplano[g]$ es una sucesi\'on tal que
	$Z^*\in\semiplano[r]$ est\'a fijo, $Z_2^{(\nu)}$ est\'a
	acotada y los autovalores de $Y_4^{(\nu)}$ tienden a $\infty$
	($Z_4^{(\nu)}\in\semiplano[g-r]$), y
	$F\in\modulformen[k](\modulgruppe[g])$, entonces el l\'{\i}mite
	\begin{displaymath}
		\lim_\nu\,F(Z^{(\nu)})
	\end{displaymath}
	%
	existe y su valor depende de $Z^*$, pero no de la sucesi\'on.
	La funci\'on resultante, $\Siegel^{g-r}F(Z^*)$ define una forma
	de Siegel de g\'enero $g-r$ y peso $k$.
\end{teoEisenstein}

\begin{proof}
	La sucesi\'on $Z^{(\nu)}$ estar\'a contenida en alguna regi\'on
	de la forma $\{Y\geq c\Id[g]\}$, eventualmente, en donde la
	serie de Fourier de $F$ converge absoluta y uniformemente.
	La igualdad
	\begin{displaymath}
		% TZ\,=\,
			% \begin{bmatrix}
				% T^*Z^*+T_2\trnsp Z_2 & T^*Z_2+T_2Z_4 \\
				% \trnsp T_2Z^*+T_4\trnsp Z_2 &
					% \trnsp T_2Z_2+T_4Z_4
			% \end{bmatrix}
			% \quad\dispand\quad
		\traza(TZ)\,=\,\traza(T^*Z^*)\,+\,2\,\traza(T_2\trnsp Z_2)
			\,+\,\traza(T_4Z_4)
		\dispstop
	\end{displaymath}
	%
	implica que, para $T\geq 0$ con $T_4>0$,
	como $Y^*>0$ est\'a fijo, $Y_2^{(\nu)}$ est\'a acotada
	e $Y_4^{(\nu)}\to\infty$,
	\begin{displaymath}
		\big|\varexp^{2\pi\raizcuarta\traza(TZ^{(\nu)})}\big|\,\leq\,
			\varexp^{-2\pi\{%
				\traza(T^*Y^*)+2\traza(T_2\trnsp Y_2)
					+\traza(T_4Y_4)\}}
	\end{displaymath}
	%
	tiende a $0$. Tomando l\'{\i}mite t\'ermino a t\'ermino en la
	serie de Fourier, se deduce que, si $T_4\neq 0$, $a(T)=0$.
	Entonces, el l\'{\i}mite $\lim_\nu\,F(Z^{(\nu)})$ existe y es
	igual a
	\begin{displaymath}
		\sum_{T^*\geq 0}\,a\spmatrix{ T^* & 0 \\ 0 & 0 }\,
			\varexp^{2\pi\raizcuarta\traza(T^*Z^*)}
		\dispstop
	\end{displaymath}
	%
	La nueva serie converge absoluta y uniformemente sobre compactos
	de $\semiplano[r]$, $\Siegel^{g-r}F$ es holomorfa en
	$\semiplano[r]$ y acotada en regiones de la forma
	$\{Y^*\geq c\Id[r]\}$. Para ver que es de peso $k$ invariante,
	sean $M^*=\sbmatrix{ A^* & B^* \\ C^* & D^* }\in\modulgruppe[r]$,
	$Z^*\in\semiplano[r]$ y
	\begin{displaymath}
		Z\,=\,\begin{bmatrix}
			Z^* & \\ & \raizcuarta\lambda\Id[g-r]
		\end{bmatrix}\quad\dispand\quad
		M\,=\,\begin{bmatrix} A & B \\ C & D \end{bmatrix}
		\dispcomma
	\end{displaymath}
	%
	donde $A=\sbmatrix{ A^* & \\ & \Id[g-r] }$,
	$B=\sbmatrix{ B^* & \\ & \phantom{0} }$,
	$C=\sbmatrix{ C^* & \\ & \phantom{0} }$ y
	$D=\sbmatrix{ D^* & \\ & \Id[g-r] }$.
	Se cumple
	\begin{displaymath}
		F(M\accion Z)\,=\,\det(CZ+D)^k\,F(Z)
	\end{displaymath}
	%
	y, tomando l\'{\i}mite $\lambda\to+\infty$,
	\begin{displaymath}
		\Siegel^{g-r}F(M^*\accion{Z^*})\,=\,
			\det(C^*Z^*+D^*)^k\,\Siegel^{g-r}F(Z^*)
		\dispstop
	\end{displaymath}
	%
\end{proof}

\begin{defEisenstein}\label{def:cuspidal}
	Una forma de Siegel $F$ es \emph{cuspidal}, si
	$\Siegel F=0$.
\end{defEisenstein}

Dado que toda $T$ singular se puede hacer equivalente a
$T[U]=\sbmatrix{ T^* & 0 \\ 0 & 0 }$ y que
$a(T[U])=\pm a(T)$, deducimos que $\Siegel F=0$, si y s\'olo si
$a(T)\neq 0$ implica $T>0$.

\begin{teoEisenstein}\label{teo:cuspidal}
	Sea $F\in\modulformen(\modulgruppe[g])$. Las siguientes afirmaciones
	acerca de $F$ son equivalentes:
	\begin{enumerate}[label=(\alph*)]
		\item\label{item:eisenstein:siegel}
			$\Siegel F=0$;
		\item\label{item:eisenstein:coeficientes}
			\begin{math}
				F(Z)=\sum_{T>0}\,a(T)\,
				\varexp^{2\pi\raizcuarta\traza(TZ)}
			\end{math};
		\item\label{item:eisenstein:crecimiento}
			para todo $c>0$, existen $c_1,c_2$ tales que,
			para toda $Z\in\semiplano[g]$ con $Y\geq c\Id[g]$
			\emph{reducida},
			\begin{displaymath}
				|F(Z)|\,\leq\,c_1\varexp^{-c_2\traza(Y)}
				\text{ ;}
			\end{displaymath}
			%
		\item\label{item:eisenstein:crecimiento:bis}
			existen $a_1,a_2>0$ tales que,
			en $\semiplano[g]$,
			\begin{displaymath}
				(\det\,Y)^{k/2}\,|F(Z)|\,\leq\,
					a_1\,\varexp^{-a_2\,(\det Y)^{1/n}}
				\dispstop
			\end{displaymath}
	\end{enumerate}
	%
\end{teoEisenstein}

Sea $C_{g,r}$ el siguiente subgrupo de $\Sp(g,\Reales)$:
\begin{displaymath}
	C_{g,r}\,=\,\bigg\{M=\begin{bmatrix} A & B \\ C & D \end{bmatrix}
		\in\Sp(g,\Reales)\,:\,
		A=\begin{bmatrix} A^* & \\ * & * \end{bmatrix},\,
		C=\begin{bmatrix} C^* & \\ & \phantom{0} \end{bmatrix},\,
		D=\begin{bmatrix} D^* & * \\ & * \end{bmatrix}
		\bigg\}
	\dispstop
\end{displaymath}
%
Tambi\'en definimos las siguientes operaciones:
\begin{displaymath}
	\begin{aligned}
	*\,:\,\semiplano[g]\,\rightarrow\,\semiplano[r]
	\disptext{dada por} &
	Z\,\mapsto\,Z^*\dispcomma\dispif
		Z\,=\,\begin{bmatrix} Z^* & * \\ * & * \end{bmatrix}
		\dispand \\
	*\,:\,C_{g,r}\,\rightarrow\,\Sp(r,\Reales)
	\disptext{dada por} &
	M\,\mapsto\,M^*\,=\,
		\begin{bmatrix} A^* & B^* \\ C^* & D^* \end{bmatrix}
	\dispstop
	\end{aligned}
	%
\end{displaymath}
%
Vale que $M^*\accion{Z^*}=M\accion Z^*$.

\begin{defEisenstein}\label{def:eisenstein}
	Si $F\in\spitzenformen[k](\modulgruppe[r])$, $k>0$ par,
	$0\leq r\leq g$ y $Z\in\semiplano[g]$, la
	\emph{serie de Eisenstein (asociada a $F$, en $Z$)} es
	\begin{displaymath}
		\Eis[g,r,k](Z;F)\,=\,
			\sum_{M\in C_{g,r}\backslash\modulgruppe[g]}\,
				\frac{F(M\accion Z^*)}{%
					\det(CZ+D)^k}
		\dispstop
	\end{displaymath}
	%
\end{defEisenstein}

Usando $M\accion Z^*=M^*\accion{Z^*}$, las propiedades del factor
$\det(CZ+D)$ y de $F$ con respecto a transformaciones de M\"obius,
ek t\'ermino general de la serie s\'olo depende de la coclase de $M$
en $C_{g,r}\backslash\modulgruppe[g]$.

\begin{teoEisenstein}\label{teo:eisenstein}
	Sean $g\geq 1$, $0\leq r\leq g$, $k>g+r+1$ par. Si
	$F\in\spitzenformen[k](\modulgruppe[r])$, la serie de
	Eisenstein $\Eis[g,r,k](Z;F)$ converge absoluta y uniformemente
	en bandas verticales.%
	\footnote{
		Regiones de la forma
		$\{\traza(X)\leq c^{-1},\,Y\geq c\Id[g]\}$.
	}
	Adem\'as,
	\begin{displaymath}
		\Siegel^{g-r}\Eis[g,r,k](-;F)\,=\,F
	\end{displaymath}
	%
	y el espacio $\modulformen[k](\modulgruppe[g])$ est\'a generado por
	\begin{displaymath}
		\big\{\Eis[g,r,k](-;F)\,:\,
			F\in\spitzenformen[k](\modulgruppe[r]),\,
			0\leq r\leq g\big\}
		\dispstop
	\end{displaymath}
	%
\end{teoEisenstein}

\begin{obsEisenstein}\label{obs:eisenstein}
	Si $F\in\img\,\Siegel$ y $F\neq 0$, entonces $k\equiv 0\tmodulo[2]$,
	pues $F=\Siegel G\neq 0$ implica $G\neq 0$ y $G\neq 0$ implica
	$(g+1)k\equiv 0\tmodulo[2]$. Como $F\neq 0$ tambi\'en implica
	$gk\equiv 0\tmodulo[2]$, debe ser $k$ par.
\end{obsEisenstein}

