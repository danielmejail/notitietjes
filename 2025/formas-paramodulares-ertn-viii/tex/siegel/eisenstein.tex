\theoremstyle{plain}
\newtheorem{teoEisenstein}{\teoname}[subsection]

\theoremstyle{definition}
\newtheorem{defEisenstein}[teoEisenstein]{\defname}
\newtheorem{obsEisenstein}[teoEisenstein]{\obsname}

%-------------

Dada $F\in\modulformen[k](\modulgruppe[g])$ definimos una forma
de g\'enero menor mediante el operador de Siegel.
El espacio $\semiplano[g]$ tiene un borde.
Vemos c\'omo restringir una forma a ese ``borde''.
Esto nos permitir\'a definir la noci\'on de forma cuspidal como el
n\'ucleo del operador de Siegel: aquellas formas que son cero en el borde.
Rec\'{\i}procamente, dada una forma cuspidal de g\'enero $r\leq g$,
definida en una componente de este borde, si el peso es suficientemente
grande (con respecto al g\'enero), es posible extenderla a $\semiplano[g]$
mediante series de Eisenstein-Klingen.
Para simplificar la exposici\'on y no sobrecargar la notaci\'on,
supondremos $g=2$. En este caso,
\begin{displaymath}
	\semiplano[2]\,=\,
		\bigg\{\Omega=
			\begin{bmatrix} \omega & z \\ z & \tau \end{bmatrix}
			\,:\,\omega,\tau\in\semiplano[1],\,z\in\Complejos,\,
			\Imag(\omega)\,\Imag(\tau)>\Imag(z)^2\bigg\}
		\dispstop
\end{displaymath}
%
Escribimos $\Omega=(\tau,z,\omega)\in\semiplano[2]$.

\begin{teoEisenstein}\label{teo:siegel}
	Sean $\tau\in\semiplano[1]$ y $F\in\modulformen[k](\modulgruppe[2])$.
	Si $\Omega_\nu=(\tau_\nu,z_\nu,\omega_\nu)\in\semiplano[2]$ es una
	sucesi\'on que cumple:
	$\omega_\nu=\omega$ est\'a fijo,
	$z_\nu$ est\'a acotada e
	$\Imag(\tau_\nu)\to\infty$,
	entonces el l\'{\i}mite
	\begin{displaymath}
		\lim_\nu\,F(\Omega_\nu)
	\end{displaymath}
	%
	existe y su valor depende de $\omega$, pero no de la sucesi\'on.
	La funci\'on resultante, $\Siegel F(\omega)$ define una
	forma de Siegel de g\'enero $1$ y peso $k$ (una forma modular
	el\'{\i}ptica).
\end{teoEisenstein}

\begin{proof}
	La sucesi\'on $\Omega_\nu$ estar\'a contenida en alguna regi\'on
	de la forma $\{Y\geq c\Id[2]\}$, eventualmente. All\'{\i}, la
	serie de Fourier de $F$ converge a.u./c.
	Si $T=\binaria{n,r,m}$ y $\Omega=(\tau,z,\omega)$, entonces
	$\traza(T\Omega)=n\tau+rz+\omega m$. Si $n>0$,
	\begin{displaymath}
		\big|\varexp^{2\pi\raizcuarta\traza(T\Omega_\nu)}\big|\,\leq\,
			\varexp^{-2\pi\{n\tau_\nu+rz_\nu+m\omega\}}
	\end{displaymath}
	%
	tiende a $0$. Tomando l\'{\i}mite t\'ermino a t\'ermino en
	\eqref{eq:funciones:fourier:dos}, se deduce que, $a(n,r,m)=0$ cuando
	$n\neq 0$. En particular, en el l\'{\i}mite, s\'olo sobreviven
	los t\'erminos con $n=r=0$: el l\'{\i}mite $\nu\to\infty$ existe
	y es igual a
	\begin{displaymath}
		\lim_{\nu\to\infty}\,F(\tau_\nu,z_\nu,\omega)\,=\,
			\sum_{m\geq 0}\,a(0,0,m)\,
				\varexp^{2\pi\raizcuarta \omega m}
		\dispstop
	\end{displaymath}
	%
	La nueva serie converge a.u./c. de $\semiplano[1]$.
	La funci\'on resultante, $\Siegel F(\omega)$ es holomorfa
	en $\semiplano[1]$ y acotada en regiones de la forma
	$\{\omega\geq\raizcuarta c\}$.
	Veamos que es de peso $k$ invariante para
	$\modulgruppe[1]=\SL(2,\Enteros)$.
	Dado $\omega\in\semiplano[1]$, elegimos la sucesi\'on
	$(\raizcuarta\nu,0,\omega)$.
	Dada $\gamma=\sbmatrix{ a & b \\ c & d }\in\SL(2,\Enteros)$,
	la matriz
	\begin{displaymath}
		\tilde\gamma\,=\,
			\begin{bmatrix}
				a & & b & \\
				& 1 & & \\
				c & & d & \\
				& & & 1
				% 1 & & & \\
				% & a & & b \\
				% & & 1 & \\
				% & c & & d
			\end{bmatrix}
	\end{displaymath}
	%
	pertenece a $\modulgruppe[2]$. Actuando en un t\'ermino de la
	sucesi\'on por este elemento,
	\begin{displaymath}
		\begin{aligned}
			\tilde\gamma\accion{(\raizcuarta\nu,0,\omega)} & \,=\,
				\Big(\sbmatrix{ a & \\ & 1 }\,
				\sbmatrix{ \omega & \\ & \raizcuarta\nu }
				\,+\,\sbmatrix{ b & \\ & \phantom{0} }\Big)\,
				\Big(\sbmatrix{ c & \\ & \phantom{0} }\,
				\sbmatrix{ \omega & \\ & \raizcuarta\nu }
				\,+\,\sbmatrix{ d & \\ & 1 }\Big)^{-1} \\
			& \,=\,\big(\raizcuarta\nu,0,
				\tfrac{a\omega+b}{c\omega+d}\big)
			\dispstop
		\end{aligned}
		%
	\end{displaymath}
	%
	Tomando l\'{\i}mite $\nu\to\infty$ en la igualdad
	\begin{displaymath}
		F(\tilde\gamma\accion{(\raizcuarta\nu,0,\omega)})=
			\det\Big(\sbmatrix{ c & \\ & \phantom{0} }\,
				\sbmatrix{ \omega & \\ & \raizcuarta\nu }
				\,+\,\sbmatrix{ d & \\ & 1 }\Big)^k\,
			F(\raizcuarta\nu,0,\omega)
			\,=\,(c\omega+d)^k\,F(\raizcuarta\nu,0,\omega)
		\dispcomma
	\end{displaymath}
	%
	se concluye que
	\begin{displaymath}
		\Siegel F(\gamma\omega)\,=\,
			\factor(\gamma,\omega)^k\,\Siegel F(\omega)
		\dispcomma
	\end{displaymath}
	%
	es decir, $\Siegel F$ es de peso $k$
	invariante para $\SL(2,\Enteros)$.
\end{proof}

\begin{defEisenstein}\label{def:cuspidal}
	Una forma de Siegel $F$ es \emph{cuspidal}, si $\Siegel F=0$.
\end{defEisenstein}

\begin{obsEisenstein}\label{obs:cuspidal}
	Dado que toda $T=\binaria{n,r,m}$ singular es equivalente
	a $\binaria{0,0,*}$ y que, en tal caso, $a(n,r,m)=a(0,0,*)$,
	deducimos que $\Siegel F=0$, si y s\'olo si
	$a(n,r,m)=0$ implica $\binaria{n,r,m}>0$.
\end{obsEisenstein}

Sea $\Delta^+\subgrpeq\modulgruppe[2]$ el subgrupo de matrices de la forma
\begin{displaymath}
	\begin{bmatrix}
		U & S\trnsp U^{-1} \\ & \trnsp U^{-1}
	\end{bmatrix}
	\dispcomma\quad U\in\GL(2,\Enteros) \dispand \trnsp S=S
	\dispstop
\end{displaymath}
%
Sea $\Gamma_\infty\subgrpeq\modulgruppe[2]$ el subgrupo de matrices de la forma
\begin{displaymath}
	M\,=\,\begin{bmatrix}
		a & & b & * \\
		* & * & * & * \\
		c & & d & * \\
		& & & *
	\end{bmatrix}
	\dispcomma\quad
	\begin{bmatrix}
		a & b \\ c & d
	\end{bmatrix}\,\in\,\SL(2,\Enteros)
	\dispstop
\end{displaymath}
%
Si $\Omega=(\tau,z,\omega)\in\semiplano[2]$, sea $\Omega^*=\omega$.

\begin{defEisenstein}\label{def:eisenstein}
	Sea $f\in\spitzenformen[k](\modulgruppe[1])$, $k>0$ par ($r=1$),
	o bien una constante ($r=0$), y sea $\Omega\in\semiplano[2]$.
	La \emph{serie de Eisenstein asociada a $f$ en $\Omega$} es
	\begin{displaymath}
		\Eis[2,r,k](\Omega;f)\,=\,
			\sum_{M\in C_{2,r}\backslash\modulgruppe[2]}\,
				\frac{f(M\accion \Omega^*)}{\det(C\Omega+D)^k}
		\dispcomma
	\end{displaymath}
	%
	donde $C_{2,0}=\Delta^+$ y $C_{2,1}=\Gamma_\infty$.
\end{defEisenstein}

\begin{teoEisenstein}\label{teo:eisenstein}
	Sean $r=0,1$, $k>r+3$ par. Si
	$f\in\spitzenformen[k](\modulgruppe[1])$ ($r=1$),
	o constante ($r=0$), la serie de Eisenstein $\Eis[g,r,k](\Omega;f)$
	converge absoluta y uniformemente en bandas verticales.%
	\footnote{
		Regiones de la forma
		$\{\traza(X)\leq c^{-1},\,Y\geq c\Id[2]\}$.
	}
	Adem\'as,
	\begin{displaymath}
		\Siegel^{2-r}\Eis[2,r,k](-;f)\,=\,f
		\dispstop
	\end{displaymath}
	%
	El espacio $\modulformen[k](\modulgruppe[2])$ est\'a generado por
	las series de Eisenstein $\Eis[2,r,k](-;f)$ ($r=0,1$)
	y por las formas cuspidales $\spitzenformen[k](\modulgruppe[2])$.
\end{teoEisenstein}

