\theoremstyle{plain}
\newtheorem{teoEjemplo}{\teoname}[subsection]

\theoremstyle{definition}

%-------------

Dados $\Omega\in\semiplano[2]$,
$\epsilon=\sbmatrix{ a\,\big|\, b }\in\Mat(2\times 2,\Enteros)$,
\begin{displaymath}
	\vartheta(\Omega;\epsilon)\,:=\,\sum_g\,
		\varexp^{\pi\raizcuarta\{\Omega[g+a/2]+\trnsp bg\}}
		\qquad\text{(\phantom)}g=\sbmatrix{ * \\ * },\,*\in\Enteros
			\text{\phantom()}
\end{displaymath}
%
define una funci\'on holomorfa.
Estas funciones verifican
\begin{enumerate}[label=(\roman*)]
	\item $\vartheta(\Omega;a+2g,b)=(-1)^{\trnsp bg}\,\vartheta(\Omega;a,b)$
	\item $\vartheta(\Omega;a,b+2g)=\vartheta(\Omega;a,b)$.
	\seti
\end{enumerate}
%
En particular, podemos restringirnos a los casos en que $\epsilon$
($a$ y $b$) tiene coordenadas $0$ y $1$. Hay diecis\'eis posibles
$\vartheta$. Adem\'as, cambiando $g$ por $-g-a$ en la definici\'on
de $\vartheta$,
\begin{enumerate}[label=(\roman*)]
	\conti
	\item $\vartheta(\Omega;a,b)=(-1)^{\trnsp ab}\,\vartheta(\Omega;a,b)$.
\end{enumerate}
%
En particular, si $\trnsp ab\not\equiv 0\tmodulo[2]$, la funci\'on
$\vartheta(\Omega;a,b)=0$ en $\semiplano[2]$. Como
\begin{displaymath}
	\trnsp\epsilon\epsilon\,=\,
		\begin{bmatrix}
			\trnsp aa & \trnsp ab \\ \trnsp ba & \trnsp bb
		\end{bmatrix}
	\dispcomma
\end{displaymath}
%
la condici\'on equivale a que $\trnsp\epsilon\epsilon$ sea par.
Diez matrices $\epsilon$ son pares y permiten definir la siguiente
funci\'on:
\begin{equation}
	\label{eq:ejemplo:theta}
	\Theta(\Omega)\,:=\,\prod_{\trnsp\epsilon\epsilon\text{ par}}\,
		\vartheta(\Omega;\epsilon)
	\dispstop
\end{equation}
%

La funci\'on \eqref{eq:ejemplo:theta} es holomorfa y verifica
\begin{displaymath}
	\Theta(M\accion \Omega)\,=\,v(M)\,\factor(M,\Omega)^5\,\Theta(\Omega)
	\dispcomma
\end{displaymath}
%
donde $v:\,\modulgruppe[2]\rightarrow\cuadratico$ es cierto morfismo
de grupos. Esta funci\'on no es id\'enticamente nula, se anula en
\begin{displaymath}
	N\,=\,\bigg\{
		\begin{bmatrix} \omega & 0 \\ 0 & \tau \end{bmatrix}\,:\,
		\omega,\tau\in\semiplano[1]\bigg\}
	\dispcomma
\end{displaymath}
%
no posee otros ceros m\'odulo $\modulgruppe[2]$ y cada cero es de
orden $1$: $\Theta(\Omega)/z$ es holomorfa y no se anula en cierto
dominio fundamental para $\modulgruppe[2]$.

\begin{teoEjemplo}\label{teo:ejemplo}
	Las funciones
	\begin{math}
		\chi_{10}(\Omega):=\Theta(\Omega)^2
	\end{math} y
	\begin{displaymath}
		\chi_{35}(\Omega)\,:=\,\Theta(\Omega)\,
			\sum_{\epsilon_1+\epsilon_2+\epsilon_3\text{ impar}}\,
			\pm\big(\vartheta(\Omega;\epsilon_1)\,
				\vartheta(\Omega;\epsilon_2)\,
				\vartheta(\Omega;\epsilon_3)\big)^{20}
	\end{displaymath}
	%
	son formas cuspidales $\neq 0$, de pesos $10$ y $35$, respectivamente.
	El anillo de formas de Siegel de g\'enero $2$ es un \'algebra
	finitamente generada y, si $\Eis[k]:=\Eis[2,0,k]$,
	\begin{displaymath}
		\modulformen(\modulgruppe[2])\,\simeq\,
			\polinomios[%
				{\Eis[4],\Eis[6],\Eis[12],\chi_{10},\chi_{35}}%
				]{%
				\Complejos}/\generado{\chi_{35}^2-R}
		\dispcomma
	\end{displaymath}
	%
	donde $R$ es una expresi\'on polinomial en las formas de peso
	par.
\end{teoEjemplo}

