El grupo modular $\SL(2,\Enteros)$ es el grupo de automorfismos del
ret\'{\i}culo $\Enteros^2$ equipado con
% pertrechado\footnote{equipado} de
la forma bilineal alternada:
\begin{displaymath}
	\langle (u,v),(w,z)\rangle \,=\,uz-vw
	\dispstop
\end{displaymath}
%

Un poco m\'as en general, si $g\geq 1$, introducimos, fijada en el
ret\'{\i}culo $\Enteros^{2g}$ una base $\lista e{g},\,\lista f{g}$,
la forma bilineal alternada:
\begin{equation}
	\label{eq:simplecticas:bilineal}
	\langle e_i,e_j\rangle \,=\,0
		\dispcomma\quad
		\langle f_i,f_j\rangle \,=\,0
		\quad\dispand\quad
		\langle e_i,f_j\rangle \,=\,
			\begin{cases}
				1 \dispcomma & \dispif i= j \dispcomma \\
				0 \dispcomma & \dispifnot\dispstop
			\end{cases}
\end{equation}
%
El \emph{grupo modular de Siegel} se define como el grupo de automorfismos de
$\Enteros^{2g}$ que preserva la forma~\eqref{eq:simplecticas:bilineal}.
En la base elegida, la matriz de la forma bilineal est\'a dada por
\begin{equation}
	\label{eq:simplecticas:matriz}
	J \,=\, \begin{bmatrix}
			& & & 1 & & \\
			& & & & \ddots & \\
			& & & & & 1 \\
			-1 & & & & & \\
			& \ddots & & & & \\
			& & -1 & & &
		\end{bmatrix}
	\,=\,\begin{bmatrix} & \Id[g] \\ -\Id[g] & \end{bmatrix}
	\dispcomma
\end{equation}
%
donde $\Id[g]$ denota la matriz identidad de tama\~no $g\times g$.%
\footnote{
	Espacios en blanco indican que la coordenada correspondiente es nula.
}
V\'{\i}a esta elecci\'on de base, este grupo se identifica con el subgrupo de
matrices
\begin{displaymath}
	\modulgruppe[g] \,:=\,\Sp(g,\Enteros)\,=\,
		\big\{M\in\Mat(2g\times 2g,\Enteros) \,:\,
			MJ\trnsp M =J\big\}
	\dispstop
\end{displaymath}
%
Es decir, si $M=\sbmatrix{A & B \\ C & D}$,%
\footnote{$A$, $B$, $C$ y $D$, matrices cuadradas.}
entonces
\begin{displaymath}
	MJ\trnsp M \,=\,
		\begin{bmatrix}
			A\trnsp B-B\trnsp A & A\trnsp D-B\trnsp C \\
			C\trnsp B-D\trnsp A & C\trnsp D-D\trnsp C
		\end{bmatrix}
\end{displaymath}
%
y $M\in\modulgruppe[g]$, si y s\'olo si
\begin{displaymath}
	A\trnsp B\,=\,B\trnsp A\dispcomma\quad
	C\trnsp D\,=\,D\trnsp C\quad\dispand\quad
	A\trnsp D-B\trnsp C\,=\,D\trnsp A-C\trnsp B\,=\,\Id[g]
	\dispstop
\end{displaymath}
%
En particular, si $M\in\modulgruppe[g]$, entonces
\begin{displaymath}
	\begin{bmatrix}
		A & B \\ C & D
	\end{bmatrix}\,
	\begin{bmatrix}
		\trnsp D & -\trnsp B \\ -\trnsp C & \trnsp A
	\end{bmatrix} \,=\,
	\begin{bmatrix}
		\Id[g] & \\ & \Id[g]
	\end{bmatrix}
	\dispcomma
\end{displaymath}
%
lo que muestra que $M$ es inversible con inversa
% (a ambos lados)
dada por
\begin{math}
	M^{-1}=\sbmatrix{\trnsp D & -\trnsp B \\ -\trnsp C & \trnsp A}
\end{math} y, en consecuencia, que tambi\'en se verifican las relaciones
\begin{displaymath}
	\trnsp D A-\trnsp BC\,=\,\trnsp AD-\trnsp CB\,=\,\Id[g]
	\dispcomma\quad
	\trnsp DB\,=\,\trnsp BD\quad\dispand\quad
	\trnsp AC\,=\,\trnsp CA
	\dispcomma
\end{displaymath}
%
o sea $\trnsp MJM=J$, tambi\'en.

Un poco \emph{m\'as} en general, una matriz $M\in\Mat(2g\times 2g,\Complejos)$
es \emph{simpl\'ectica}, si verifica:
\begin{displaymath}
	MJ\trnsp M\,=\,\multiplier(M)\,J
	\dispcomma
\end{displaymath}
%
para cierto escalar inversible $\multiplier(M)$. Si
$M=\sbmatrix{A & B \\ C & D}$ es simpl\'ectica y
\begin{equation}
	\label{eq:simplecticas:adjunta}
	\adjunta(M)\,=\,
		\begin{bmatrix}
			\trnsp D & -\trnsp B \\ -\trnsp C & \trnsp A
		\end{bmatrix}
	\dispcomma
\end{equation}
%
entonces $M\adjunta(M)=\adjunta(M)M=\multiplier(M)\,\Id[2g]$.
Las matrices simpl\'ecticas con coordenadas reales forman un grupo
que denotamos $\GSp(g,\Reales)$ y
$\multiplier:\,\GSp(g,\Reales)\rightarrow\Unidades{\Reales}$
es un morfismo de grupos.
El grupo modular de Siegel est\'a contenido en el subgrupo
\begin{displaymath}
	\GSp(g,\Reales)^+\,=\,\Big\{M\in\GSp(g,\Reales)\,:\,
		\multiplier(M)>0 \Big\}
	\dispcomma
\end{displaymath}
%
de aquellas matrices simpl\'ecticas reales cuyo factor
$\multiplier(M)$ es positivo.
A modo de ejemplo, las matrices
\begin{equation}
	\label{eq:simplecticas:ejemplos}
	\begin{bmatrix} U & \\ & \trnsp U^{-1} \end{bmatrix}
		\quad\text{,}\quad
	\begin{bmatrix} \Id[g] & S \\ & \Id[g] \end{bmatrix}
		\quad\text{y}\quad
	J\,=\,\begin{bmatrix} & \Id[g] \\ -\Id[g] & \end{bmatrix}
\end{equation}
%
son simpl\'ecticas con factor $\multiplier=1$,
si $U\in\GL(g,\Enteros)$ y $\trnsp S=S$; pertenecen a $\modulgruppe[g]$.
 
