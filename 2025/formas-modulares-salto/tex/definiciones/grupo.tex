El grupo lineal (especial) de rango $2$ jugar\'a un rol fundamental
en toda la discusi\'on. El grupo con coeficientes reales est\'a
definido de la siguiente manera:
\begin{displaymath}
	\SL(2,\Reales)\,=\,\bigg\{
		\begin{bmatrix} a & b \\ c & d \end{bmatrix}\,:\,
			a,b,c,d\in\Reales,\,ad-bc=1
		\bigg\}
	\dispstop
\end{displaymath}
%
Como transformaciones del espacio eucl\'{\i}deo $\Reales^2$,
estas matrices preservan volumen y orientaci\'on, aunque no todas
preservan las distancias entre puntos del plano.
Dentro del plano, tenemos el ret\'{\i}culo $\Enteros^2$ de puntos
con coordenadas enteras. El \emph{grupo modular} es el subgrupo
del grupo lineal especial que preserva este ret\'{\i}culo:
\begin{displaymath}
	\modulgruppe\,=\,\SL(2,\Enteros)\,=\,
		% \SL(2,\Reales)\,\cap\, \Mat(2\times 2,\Enteros)
		\bigg\{
		\begin{bmatrix} a & b \\ c & d \end{bmatrix}\,:\,
			a,b,c,d\in\Enteros,\,ad-bc=1
		\bigg\}
	\dispstop
\end{displaymath}
%
Algunas matrices pertenecientes a $\SL(2,\Enteros)$ son:
\begin{displaymath}
	\begin{bmatrix} 1 & h \\ & 1 \end{bmatrix}
	\dispcomma\quad
	\begin{bmatrix} & -1 \\ 1 & \end{bmatrix}
	\dispand
	\pm\,\begin{bmatrix} 1 & \\ & 1 \end{bmatrix}
	\dispstop
\end{displaymath}
%
Si $\gamma\in\Gamma$, entonces $\trnsp\gamma\in\Gamma$ tambi\'en.

