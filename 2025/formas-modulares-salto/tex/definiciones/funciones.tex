El \emph{semiplano complejo superior} est\'a conformado por los
n\'umeros complejos con parte imaginaria definida positiva:
\begin{displaymath}
	\semiplano\,=\,\big\{z\in\Complejos\,:\,\Imag(z)>0\big\}
	\dispstop
\end{displaymath}
%
El grupo $\SL(2,\Reales)$ act\'ua en $\semiplano$ por
\emph{transformaciones de M\"obius}: si $z\in\semiplano$ y
$\gamma\in\SL(2,\Reales)$, definimos un nuevo punto de la siguiente
manera:
\begin{equation}
	\label{eq:definiciones:moebius}
	\gamma\accion z\,=\,\frac{az+b}{cz+d}
	\dispstop
\end{equation}
%
Dado que $\Imag(z)>0$ y $c,d\in\Reales$ no ambos nulos, se deduce
que $cz+d\neq 0$. Adem\'as,
\begin{equation}
	\label{eq:definiciones:imaginaria}
	\Imag(\gamma\accion z)\,=\,\frac{\Imag(z)}{|cz+d|^2}
	\dispcomma
\end{equation}
%
de lo que se deduce que $\gamma\accion z\in\semiplano$.
Que \eqref{eq:definiciones:moebius} defina una \emph{acci\'on}
quiere decir que, si $\gamma,\gamma'\in\modulgruppe$ y $z\in\semiplano$,
entonces $(\gamma\gamma')\accion z=\gamma\accion{\gamma'\accion z}$.
Las matrices escalares, $\pm\,\sbmatrix{ 1 & \\ & 1 }$, act\'uan
trivialmente, es decir, no mueven ning\'un punto.
% Con lo cual, la acci\'on \eqref{eq:definiciones:moebius} es, en verdad,
% una acci\'on del grupo
% \begin{displaymath}
	% \PSL(2,\Reales)\,=\,\SL(2,\Reales)/\{\pm\Id\}
	% \dispstop
% \end{displaymath}
% %

% La acci\'on \eqref{eq:definiciones:moebius} es transitiva: si
% $z=x+\raizcuarta y$, $y>0$, entonces
% \begin{equation}
	% \label{eq:definiciones:transitividad}
	% z\,=\,
	% \begin{bmatrix}
		% \sqrt y & x\sqrt y^{-1} \\ & \sqrt y^{-1}
	% \end{bmatrix}\accion\raizcuarta
	% \,=\,
	% \bigg(
	% \begin{bmatrix} 1 & x \\ & 1 \end{bmatrix}\,
	% \begin{bmatrix} \sqrt y & \\ & \sqrt y^{-1} \end{bmatrix}
	% \bigg)\accion\raizcuarta
	% \dispcomma
% \end{equation}
% %
% una dilaci\'on seguida de una traslaci\'on. Las matrices que
% dejan fijo el punto $\raizcuarta\in\semiplano$ forman un subgrupo,
% el \emph{grupo ortogonal especial}:
% \begin{displaymath}
	% \SO(2,\Reales)\,=\,\bigg\{
		% \begin{bmatrix} a & b \\ -b & a \end{bmatrix}\,:\,
			% a,b\in\Reales,\,a^2+b^2=1
		% \bigg\}
	% \dispstop
% \end{displaymath}
% %
% Esto permite reinterpretar el espacio $\semiplano$ como un cociente:
% \begin{displaymath}
	% \semiplano\,=\,\SL(2,\Reales)/\SO(2,\Reales)
	% \dispstop
% \end{displaymath}
% %

\begin{defDefiniciones}\label{def:definiciones:modular}
	Una \emph{forma modular} es una funci\'on
	$f:\,\semiplano\rightarrow\Complejos$ que
	\begin{enumerate}[label=(M\arabic*)]
		\item\label{item:modular:holomorfia}
			es holomrfa,
		\item\label{item:modular:transformaciones}
			existe $k$ tal que, para toda
			$\gamma=\sbmatrix{ * & * \\ c & d }\in\modulgruppe$
			y $z\in\semiplano$, cumple
			\begin{equation}
				\label{eq:definiciones:transformacion}
				f(\gamma\accion z)\,=\,(cz+d)^k\,f(z)
				\dispand
			\end{equation}
			%
		\item\label{item:modular:crecimiento}
			est\'a acotada en regiones de la forma
			$\{y\geq \delta\}$, para todo $\delta>0$.
	\end{enumerate}
	%
	El n\'umero $k$ es el \emph{peso} de la forma.
\end{defDefiniciones}

\begin{obsDefiniciones}\label{obs:definiciones:modular:desarrollo}
	Si $f$ es una forma modular, por \ref{item:modular:transformaciones},
	la funci\'on es, en particular, peri\'odica de per\'{\i}odo $1$:
	\begin{math}
		f(z+1)=f(z)
	\end{math},
	si $z\in\semiplano$. Esto implica que existe una funci\'on
	$g$ tal que
	\begin{displaymath}
		f(z)\,=\,g(\varexp^{2\pi\raizcuarta z})
		\dispstop
	\end{displaymath}
	%
	El dominio de definici\'on de $g$ es $D\setmin\{0\}$, donde
	\begin{displaymath}
		D\,=\,\big\{ q\in\Complejos\,:\,|q|<1 \big\}
		\dispstop
	\end{displaymath}
	%
	Como, por \ref{item:modular:holomorfia}, $f$ es holomorfa en
	$\semiplano$, la funci\'on $g$ es holomorfa en el disco pinchado.
	La transformaci\'on $z\mapsto q=\varexp^{2\pi\raizcuarta z}$
	es holomorfa y la imagen de un subconjunto $\{y\geq\delta\}$,
	$\delta>0$, es un disco (pinchado y con borde) de radio
	$\varexp^{-2\pi\delta}$. Ahora, por \ref{item:modular:crecimiento},
	$f(z)$ est\'a acotada en $\{y\geq\delta\}$. En consecuencia,
	$g$ est\'a acotada en un entorno de $q=0$. Pero esto quiere decir
	que $g$ se extiende a una funci\'on holomorfa definida en todo
	el disco $D$. El desarrollo de Taylor de $g$ en $q=0$ da lugar a
	un desarrollo de $f$ ``en $\infty$'':
	\begin{equation}
		\label{eq:definiciones:desarrollo}
		f(z)\,=\,\sum_{n\geq 0}\,a_n\,\varexp^{2\pi\raizcuarta n z}
		\dispstop
	\end{equation}
	%
	Se dice que la serie en \eqref{eq:definiciones:desarrollo} es
	el \emph{desarrollo de Fourier de $f$ en el infinito}.
	La serie comienza en $n=0$, porque la funci\'on $g$ es holomorfa.
	Por esta raz\'on, se suele decir que
	$f$ es ``holomorfa en $\infty$''.
\end{obsDefiniciones}

A veces se escribe $q$, en lugar de $\varexp^{2\pi\raizcuarta z}$.
Es la existencia de estos desarrollos lo que hace relevantes a las
formas modulares en el contexto de Teor\'{\i}a de n\'umeros.

