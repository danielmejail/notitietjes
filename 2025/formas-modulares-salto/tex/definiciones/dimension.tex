Si bien una forma modular no define una funci\'on en el espacio
de \'orbitas $\modulgruppe\backslash\semiplano$, s\'{\i} tiene
sentido hablar de los ceros de una forma modular en este cociente:
el factor de automorf\'{\i}a $cz+d$ no tiene ni ceros ni polos
en $\semiplano$, si $\sbmatrix{ * & * \\ c & d }\in\modulgruppe$.
M\'as precisamente, si $z\in\semiplano$, $\gamma\in\modulgruppe$ y
$f$ es una forma modular, entonces
\begin{displaymath}
	\orden[{\gamma\accion z}](f)\,=\,\orden[z](f)
	\dispstop
\end{displaymath}
%
El hecho de que los espacios de formas modulares $\modulformen[k]$
son de dimensi\'on finita, es consecuencia de que la cantidad de ceros
(con multiplicidad) de $f\in\modulformen[k]$ que pertenecen a distintas
\'orbitas (es decir, los ceros en $\modulgruppe\backslash\semiplano$)
depende s\'olo de $k$ (y de $\modulgruppe$).
T\'ecnicamente, tambi\'en es necesario tener en cuenta la contribuci\'on
del orden de anulaci\'on de $f$ en $\infty$ y contar adecuadamente la
contribuci\'on de ciertos puntos especiales, los puntos el\'{\i}pticos.

\begin{defDefiniciones}\label{def:definiciones:elipticos}
	Un \emph{punto el\'{\i}ptico} en $\semiplano$ es un punto
	que queda fijo por alguna matriz $\gamma\in\modulgruppe$,
	$\gamma\neq\pm\Id$.
\end{defDefiniciones}

\begin{obsDefiniciones}\label{obs:definiciones:elipticos}
	Los \'unicos puntos el\'{\i}pticos en $\cFundamental$ son
	$\raizcuarta$, $\raizcubica$ y $\raizcubica+1$. Con respecto
	a estos puntos, los estabilizadores (en $\modulgruppe$)
	tienen \'ordenes $4$ y $6$, respectivamente ($\raizcubica$ y
	$\raizcubica+1$ pertenecen a la misma \'orbita, con lo que sus
	estabilizadores son conjugados).
	El resto de los puntos tiene estabilizador $\pm\Id$.
	Para cada punto $z\in\semiplano$, definimos un entero $n_z$
	de la siguiente manera:
	\begin{displaymath}
		n_z\,=\,
			\begin{cases}
				2 \dispcomma & \dispif
					z\in\modulgruppe\accion\raizcuarta
					\dispcomma \\
				3 \dispcomma & \dispif
					z\in\modulgruppe\accion\raizcubica
					\dispand \\
				1 & \text{en otro caso.}
			\end{cases}
			%
	\end{displaymath}
	%
\end{obsDefiniciones}

\begin{defDefiniciones}\label{def:definiciones:orden-en-infinito}
	El \emph{orden de anulaci\'on en $\infty$} de una forma
	modular es el menor $n\geq 0$ tal que el coeficiente $a_n$
	en su desarrollo de Fourier es distinto de cero.
\end{defDefiniciones}

\begin{teoDefiniciones}\label{teo:definiciones:cota}
	Sea $f\in\modulformen[k]$, $f\neq 0$. Entonces,
	\begin{displaymath}
		\sum_{z\in\modulgruppe\backslash\semiplano}\,
			\frac 1{n_z}\,\orden[z](f)\,+\,
			\orden[\infty](f)\,=\,\frac k{12}
		\dispstop
	\end{displaymath}
	%
\end{teoDefiniciones}

La sumatoria en el \teoname~\ref{teo:definiciones:cota} se realiza
sobre cualquier sistema de representantes de la \'orbitas de
$\modulgruppe$ en $\semiplano$.

\begin{proof}
	Sea $D=D(\epsilon)$ el cerrado que queda despu\'es de
	eliminar de $\cFundamental$ discos (abiertos) de radio
	$\epsilon>0$ (suficientemente chico) centrados en cada
	cero de $f$, en cada punto el\'{\i}ptico, as\'{\i} como
	el abierto $\{y>\epsilon^{-1}\}$. Integrando a lo largo
	del borde de este cerrado $f'(z)/f(z)$, el resultado es $0$.
	Ahora calculamos la integral en distintos segmentos, sumamos
	y comparamos con el resultado anterior.

	El contorno de los entornos removidos de los ceros y de
	los puntos el\'{\i}pticos son:
	\begin{itemize}
		\item un c\'{\i}rculo entero, si es alrededor
			de un punto no el\'{\i}ptico,
		\item medio c\'{\i}rculo, si es alrededor de
			$\raizcuarta$ y
		\item un tercio de c\'{\i}rculo, si es alrededor
			de $\raizcubica$.
	\end{itemize}
	%
	La integral a lo largo de un borde vertical se compensa con
	la integral a lo largo del borde opuesto ($f(z+1)=f(z)$ y
	la orientaci\'on se invierte). La integral a lo largo
	del borde $\{y=\epsilon^{-1}\}$ aporta
	$2\pi\raizcuarta\orden[\infty](f)$, pues, v\'{\i}a el cambio
	de variables $z\mapsto q$, integramos alrededor de un
	c\'{\i}rculo peque\~no centrado en $q=0$ (recorrido una vez,
	en sentido antihorario). La contribuci\'on de la integral a
	lo largo de un c\'{\i}rculo centrado en un punto $z$ es
	$2\pi\raizcuarta\orden[z](f)/n_z$. Por \'ultimo, la integral
	a lo largo del arco inferior de $\cFundamental$ da como
	resultado $\pi\raizcuarta k/6$, pues las dos mitades del arco
	se identifican v\'{\i}a $S$ y
	\begin{displaymath}
		% \frac{f'(S\accion z)}{f(S\accion z)}\,=\,
			% \frac{f'(z)}{f(z)}\,+\,\frac k z
		\big(\log\,f(S\accion z)\big)'\,=\,
			\big(\log\,f(z)\big)\,+\,\frac k z
		\dispstop
	\end{displaymath}
	%
\end{proof}

\begin{coroDefiniciones}\label{coro:definiciones:cota}
	La dimensi\'on de $\modulformen[k]$ es $0$ si $k<0$ o
	impar. Si $k\geq 0$ es par, entonces
	\begin{displaymath}
		\dim\,\modulformen[k]\,\leq\,
			\begin{cases}
				\piso{k/12}+1\dispcomma & \dispif
					k\not\equiv 2\tmodulo[12]
					\dispand \\
				\piso{k/12}\dispcomma & \dispif
					k\equiv 2\tmodulo[12]
					\dispstop
			\end{cases}
			%
	\end{displaymath}
	%
\end{coroDefiniciones}

\begin{proof}
	Tomar $m=\piso{k/12}+1$ puntos no el\'{\i}pticos y
	$m+1$ formas $\lista f{m+1}\in\modulformen[k]$.
	Alguna combinaci\'on lineal de ellas, $f$, se anula en los
	$m$ puntos. Pero como $m>k/12$, por el \teoname~%
	\ref{teo:definiciones:cota}, $f=0$.
	Si $k\equiv 2\tmodulo[12]$, se puede mejorar la cota
	teniendo en cuenta la f\'ormula del \teoname~%
	\ref{teo:definiciones:cota}.
\end{proof}

\begin{coroDefiniciones}\label{coro:definiciones:parametrizacion}
	El espacio $\modulformen[12]$ tiene dimensi\'on $\leq 2$ y,
	si $f,g\in\modulformen[12]$ son linealmente independientes,
	entonces $z\mapsto f(z)/g(z)$ es un isomorfismo entre
	$\modulgruppe\backslash\semiplano\cup\{\infty\}$ y
	$\bb P^1(\Complejos)$.
\end{coroDefiniciones}

\begin{proof}
	La cota se deduce del \coroname~\ref{coro:definiciones:cota}.
	Sean $f,g\in\modulformen[12]$ dos formas linealmente
	independientes. Si $\lambda,\mu\in\Complejos$, no ambos nulos,
	entonces la cantidad de ceros en
	$\modulgruppe\backslash\semiplano\cup\{\infty\}$ de la diferencia
	$\lambda f-\mu g$ es exactamente $1$ (tiene peso $12$).
	En particular, el cociente $\psi(z)=f(z)/g(z)$ toma cada valor
	de $\Complejos\cup\{\infty\}$ exactamente una vez.
\end{proof}

\begin{obsDefiniciones}\label{obs:definiciones:cota}
	Toda terna de formas modulares es algebraicamente
	dependiente: si $f$, $g$ y $h$ fuesen formas modulares
	algebraicamente independientes, entonces, para $k$
	suficientemente grande, la dimensi\'on de $\modulformen[k]$
	ser\'{\i}a, al menos, la cantidad de monomios en las
	variables $f$, $g$ y $h$ de peso total $k$, que es cuadr\'atica
	en $k$.
\end{obsDefiniciones}

Una consecuencia importante es que, para comparar dos formas modulares
de igual peso (y grupo), basta con comparar una cantidad finita de
coeficientes.

