La serie dada por \eqref{eq:ejemplos:eisenstein:desarrollo}
converge tambi\'en para $k=2$ y define una funci\'on
holomorfa en $\semiplano$.
\begin{defEjemplos}\label{def:ejemplos:eisenstein:dos}
	La \emph{serie de Eisenstein de peso $2$} es
	la serie
	\begin{displaymath}
		\varvarEis[2](z)\,=\,-\frac 1{24}\,+\,
		\sum_{n\geq 1}\,\divisores[1](n)\,q^n
		\dispcomma
	\end{displaymath}
	%
	o bien cualquiera de las renormalizaciones
	$\varEis[2](z)=-4\pi^2\varvarEis[2](z)$ o
	$\Eis[2](z)=\frac 6 \pi^2\,\varEis[2](z)$.
\end{defEjemplos}

Sin embargo, la serie de Eisenstein de peso $2$ no es una
forma modular (como ya sabemos por la cota del \coroname~%
\ref{coro:definiciones:cota}).

\begin{teoEjemplos}\label{teo:ejemplos:dos}
	Si $z\in\semiplano$ y
	$\gamma=\sbmatrix{ * & * \\ c & d }\in\modulgruppe$,
	entonces
	\begin{displaymath}
		% \varEis[2](\gamma\accion z)\,=\,
		% (cz+d)^2\varEis[2](z)\,-\,\pi\raizcuarta c\,
			% (cz+d)
		\big(\varEis[2]\baroperador[2]\gamma\big)(z)\,=\,
			\varEis[2](z)\,-\,\frac{2\pi\raizcuarta c}{cz+d}
		\dispstop
	\end{displaymath}
	%
\end{teoEjemplos}


