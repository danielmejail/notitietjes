Las seres de Eisenstein son ejemplos cl\'asicos de formas modulares.
Una manera de definirlas es mediante una especie de promedio de la
acci\'on de $\modulgruppe$ sobre una funci\'on particular que, en
principio, podr\'{\i}a no ser modular. El resultado cumplir\'a con
la condici\'on \ref{item:modular:transformaciones}, al menos formalmente.
El problema es verificar que la definici\'on es buena.

Dadas una funci\'on $f:\,\semiplano\rightarrow\Complejos$,
$g\in\SL(2,\Reales)$, $z\in\semiplano$ y $k$, definimos
una nueva funci\'on de la siguiente manera:
\begin{equation}
	\label{eq:ejemplos:operador}
	\big(f\baroperador[k] g\big)(z)\,=\,
		(cz+d)^{-k}\,f(g\accion z)
	\dispcomma
\end{equation}
%
si $g=\sbmatrix{ * & * \\ c & d }$. De esta manera, podemos
ver que la condici\'on \ref{item:modular:transformaciones} es
equivalente a $f\baroperador[k]\gamma=f$ para toda
$\gamma\in\modulgruppe$.

Si $f=1$ es la funci\'on constante con valor $1$
y $g=\sbmatrix{ * & * \\ c & d }$, entonces
\begin{displaymath}
	1\baroperador[k] g\,=\,
		\frac 1{(cz+d)^k}
	\dispstop
\end{displaymath}
%
En particular, si $k$ es par, $1\baroperador[k]\gamma=1$
para toda $\gamma$ en el subgrupo
\begin{displaymath}
	\Gamma_\infty\,=\,\bigg\{
		\pm\begin{bmatrix} 1 & h \\ & 1 \end{bmatrix}\,:\,
			h\in\Enteros
			\bigg\}
	\dispstop
\end{displaymath}
%
Este subgrupo se puede entender como el estabilizador del punto en el
infinito.
M\'as aun, si $\gamma\in\Gamma_\infty$, entonces
\begin{math}
	1\baroperador[k]{(\gamma g)}=
		1\baroperador[k] g
\end{math}.
Es decir, $1\baroperador[k] g$ s\'olo depende de la coclase
$\Gamma_\infty g\subseteq\SL(2,\Reales)$.

\begin{defEjemplos}\label{def:ejemplos:eisenstein:E}
	Dado $k\in\Enteros$, par, la \emph{serie de Eisenstein de peso $k$}
	es la serie
	\begin{displaymath}
		\Eis[k](z)\,=\,
		\sum_{\gamma\in\Gamma_\infty\backslash\modulgruppe}\,
		1\baroperador[k]\gamma
		\dispstop
	\end{displaymath}
	%
\end{defEjemplos}

\begin{obsEjemplos}\label{obs:ejemplos:eisenstein:E}
	La serie $\Eis[k]$ coincide con%
	\footnote{
		Todo par $(c,d)=1$ es la fila inferior de alguna
		$\gamma\in\modulgruppe$.
		Adem\'as, dos matrices $\gamma$ y $\gamma'$ cumplen
		$\gamma'\gamma^{-1}\in\Gamma_\infty$, si y s\'olo si
		$\det\,\gamma=\det\,\gamma'$ y sus filas inferiores
		son iguales.
	}
	\begin{equation}
		\label{eq:ejemplos:eisenstein:E}
		\Eis[k](z)\,=\,\frac 1 2\,\sum_{\mcd{c,d}=1}\,
			\frac 1{(cz+d)^k}
		\dispstop
	\end{equation}
	%
	La serie \eqref{eq:ejemplos:eisenstein:E} es absoluta y
	uniformemente convergente sobre compactos de $\semiplano$,
	si $k>2$: fijado $z$, la cantidad de pares $(c,d)$ que
	cumplen $N\leq |cz+d|<N+1$ es del orden de $N$.
	En particular, si $k>2$, entonces $\Eis[k]$ es una
	forma modular de peso $k$.
\end{obsEjemplos}

Otra versi\'on de la serie de Eisenstein --otra normalizaci\'on--
es la serie
\begin{equation}
	\label{eq:ejemplos:eisenstein:G}
	\varEis[k](z)\,=\,\frac 1 2\,\sum_{(m,n)\neq(0,0)}\,
		\frac 1{(mz+n)^k}
	\dispstop
\end{equation}
%
Esta serie es un m\'ultiplo de $\Eis[k]$ dada por
\eqref{eq:ejemplos:eisenstein:E}:
\begin{displaymath}
	\varEis[k](z)\,=\,\zeta(k)\,\Eis[k](z)
	\dispcomma
\end{displaymath}
%
donde $\zeta(k)=\sum_{r\geq 1}\,r^{-k}$. Esta normalizaci\'on
surge naturalmente al considerar sumas sobre los puntos de
ret\'{\i}culos $L\subset\Complejos$. Las series $\varEis[k](z)$
corresponden a los ret\'{\i}culos $L=\Enteros\,z+\Enteros\,1$.

Una tercera normalizaci\'on de las series de Eisenstein
es la siguiente:
\begin{equation}
	\label{eq:ejemplos:eisenstein:GG}
	\varvarEis[k](z)\,=\,\frac{(k-1)!}{(2\pi\raizcuarta)^k}\,
		\varEis[k](z)
	\dispstop
\end{equation}
%
La serie \eqref{eq:ejemplos:eisenstein:GG} tiene coeficientes
racionales.

\begin{teoEjemplos}\label{teo:ejemplos:eisentein:anillo}
	El anillo de formas modulares es un anillo de
	polinomios en $\Eis[4]$ y $\Eis[6]$.
\end{teoEjemplos}

\begin{coroEjemplos}\label{coro:ejemplo:eisenstein:anillo}
	La desigualdad del \coroname~%
	\ref{coro:definiciones:cota} es una igualdad.
\end{coroEjemplos}

\begin{proof}
	El argumento es el siguiente:
	\begin{enumerate}[label=\arabic*.]
		\item\label{item:eisenstein:independencia}
			las formas modulares $\Eis[4]$ y
			$\Eis[6]$ son algebraicamente
			independientes,
		\item\label{item:eisenstein:generado}
			el subespacio generado por $\Eis[4]$
			y por $\Eis[6]$ en peso $k$ tiene
			dimensi\'on igual a la cota del
			\coroname~%
			\ref{coro:definiciones:cota},
		\item\label{item:eisenstein:cota-inferior}
			la dimensi\'on de $\modulformen[k]$
			es, al menos, la dimensi\'on del
			subespacio de
			\ref{item:eisenstein:generado}
		\item\label{item:eisenstein:coro}
			de la coincidencia de la cota superior
			con la cota inferior para la
			dimensi\'on de $\modulformen[k]$,
			se deduce el \coroname,
		\item\label{item:eisenstein:teo}
			de \ref{item:eisenstein:coro} y de
			\ref{item:eisenstein:independencia},
			se deduce el \teoname.
	\end{enumerate}
	%
	Veamos c\'omo se deduce cada parte.

	\ref{item:eisenstein:independencia}
	Las series $\Eis[4](z)^3$ y $\Eis[6](z)^2$ son ambas
	de peso $12$, pero no son proporcionales (no son l.i.).
	En caso de que lo fueran, existir\'{\i}a
	$\lambda\in\Unidades \Complejos$ tal que
	$\Eis[6](z)^2=\lambda\Eis[4](z)^3$. La funci\'on
	$f(z)=\Eis[6](z)/\Eis[4](z)$ es meromorfa en
	$\semiplano$ y en $\infty$. Adem\'as, cumple
	\ref{item:modular:transformaciones} con $k=2$.
	Pero la existencia de $\lambda$ implicar\'{\i}a
	que $f(z)^2=\lambda\Eis[4](z)$ (y tambi\'en que
	$f(z)^3=\lambda^{-1}\Eis[6](z)$). En particular,
	$f$ ser\'{\i}a una forma modular no nula de peso $2$.
	Esto contradice la cota $\dim\,\modulformen[2]\leq 0$.
	De esta independencia lineal, se deduce la independencia
	algebraica por un argumento general.

	Si $f$ y $g$ son formas modulares de igual peso que
	no son proporcionales, entonces son algebraicamente
	independientes. Si $P(\indet,\varindet)$ es un polinomio
	con coeficientes complejos tal que
	$P(f,g)$ es la funci\'on nula, entonces para cada
	componente homog\'enea $P_d$ de $P$, la funci\'on
	$P_d(f,g)$ debe ser la funci\'on nula. Deshomogeizando,
	$P_d(f,g)/g^d=p(f/g)$, par cierto polinomio $p$ en una
	\'unica variable. Pero $p$ tiene, a lo sumo, finitas
	ra\'{\i}ces, o bien es el polinomio nulo. Como el
	cociente $f(z)/g(z)$ toma infinitos valores\dots
	$p$ debe ser el polinomio nulo. En particular,
	$P_d$ es nulo, por lo tanto, $P$ es nulo y, en
	consecuencia, $f$ y $g$ deben ser algebraicamente
	independientes.

	\ref{item:eisenstein:generado}
	La funci\'on $\Eis[4](z)^x\Eis[6](z)^y$ es una forma
	modular de peso $4x+6y$ (si $x$ e $y$ son enteros
	no negativos). El subespacio generado por $\Eis[4]$
	y por $\Eis[6]$ en peso $k$ es el generado por el
	subconjunto
	\begin{equation}
		\label{eq:eisenstein:monomios}
		\cal S_k\,=\,\big\{\Eis[4](z)^x\Eis[6](z)^y\,:\,
			4x+6y=k\big\}
		\dispstop
	\end{equation}
	%
	Ahora, si $k\in\Enteros$, la ecuaci\'on $4x+6y=k$
	tiene soluci\'on $x,y\in\Enteros$, si y s\'olo si
	$2\mid k$ y, en tal caso, las soluciones est\'an
	dadas por los pares
	\begin{displaymath}
		\big(-k/2+3t,k/2-2t\big)
		\dispcomma
	\end{displaymath}
	%
	con $t\in\Enteros$. La condici\'on $x,y\geq 0$
	equivale a $k/6\leq t\leq k/4$. Si escribimos
	$k=12l+r$ ($r\in\{0,2,4,6,8,10\}$), entonces la
	cantida de enteros $t$ en el rango es igual a $l+1$,
	si $r\neq 2$, y es igual a $l$, si $r=2$.

	\ref{item:eisenstein:cota-inferior}
	$\modulformen[k]$ contiene un subespacio de la
	dimensi\'on especificada.

	\ref{item:eisenstein:coro}
	Las cotas coinciden y, por lo tanto, vale la igualdad.

	\ref{item:eisenstein:teo}
	Toda forma pertenece a alg\'un espacio
	$\modulformen[k]$ y, por lo tanto, es combinaci\'on
	lineal de los monomios $\Eis[4]^x\Eis[6]^y$,
	$4x+6y=k$. O sea, $\Eis[4]$ y $\Eis[6]$ generan el
	anillo $\modulformen$. La independencia algebraica
	de estos generadores implica que
	$\modulformen\simeq\polinomios[\indet,\varindet]{\Complejos}$.
\end{proof}


