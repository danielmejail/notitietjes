\theoremstyle{plain}
\newtheorem{teoEjemplos}{\teoname}[section]
\newtheorem{coroEjemplos}[teoEjemplos]{\coroname}

\theoremstyle{definition}
\newtheorem{defEjemplos}[teoEjemplos]{\defname}
\newtheorem{obsEjemplos}[teoEjemplos]{\obsname}

%-------------

\subsection{Series de Eisenstein}\label{subsec:eisenstein}
\theoremstyle{plain}
\newtheorem{teoEisenstein}{\teoname}[subsection]

\theoremstyle{definition}
\newtheorem{defEisenstein}[teoEisenstein]{\defname}
\newtheorem{obsEisenstein}[teoEisenstein]{\obsname}

%-------------

Dada $F\in\modulformen[k](\modulgruppe[g])$ definimos una forma
de g\'enero menor mediante el operador de Siegel.
El espacio $\semiplano[g]$ tiene un borde.
Vemos c\'omo restringir una forma a ese ``borde''.
Esto nos permitir\'a definir la noci\'on de forma cuspidal como el
n\'ucleo del operador de Siegel: aquellas formas que son cero en el borde.
Rec\'{\i}procamente, dada una forma cuspidal de g\'enero $r\leq g$,
definida en una componente de este borde, si el peso es suficientemente
grande (con respecto al g\'enero), es posible extenderla a $\semiplano[g]$
mediante series de Eisenstein-Klingen.
Para simplificar la exposici\'on y no sobrecargar la notaci\'on,
supondremos $g=2$. En este caso,
\begin{displaymath}
	\semiplano[2]\,=\,
		\bigg\{\Omega=
			\begin{bmatrix} \omega & z \\ z & \tau \end{bmatrix}
			\,:\,\omega,\tau\in\semiplano[1],\,z\in\Complejos,\,
			\Imag(\omega)\,\Imag(\tau)>\Imag(z)^2\bigg\}
		\dispstop
\end{displaymath}
%
Escribimos $\Omega=(\tau,z,\omega)\in\semiplano[2]$.

\begin{teoEisenstein}\label{teo:siegel}
	Sean $\tau\in\semiplano[1]$ y $F\in\modulformen[k](\modulgruppe[2])$.
	Si $\Omega_\nu=(\tau_\nu,z_\nu,\omega_\nu)\in\semiplano[2]$ es una
	sucesi\'on que cumple:
	$\omega_\nu=\omega$ est\'a fijo,
	$z_\nu$ est\'a acotada e
	$\Imag(\tau_\nu)\to\infty$,
	entonces el l\'{\i}mite
	\begin{displaymath}
		\lim_\nu\,F(\Omega_\nu)
	\end{displaymath}
	%
	existe y su valor depende de $\omega$, pero no de la sucesi\'on.
	La funci\'on resultante, $\Siegel F(\omega)$ define una
	forma de Siegel de g\'enero $1$ y peso $k$ (una forma modular
	el\'{\i}ptica).
\end{teoEisenstein}

\begin{proof}
	La sucesi\'on $\Omega_\nu$ estar\'a contenida en alguna regi\'on
	de la forma $\{Y\geq c\Id[2]\}$, eventualmente. All\'{\i}, la
	serie de Fourier de $F$ converge a.u./c.
	Si $T=\binaria{n,r,m}$ y $\Omega=(\tau,z,\omega)$, entonces
	$\traza(T\Omega)=n\tau+rz+\omega m$. Si $n>0$,
	\begin{displaymath}
		\big|\varexp^{2\pi\raizcuarta\traza(T\Omega_\nu)}\big|\,\leq\,
			\varexp^{-2\pi\{n\tau_\nu+rz_\nu+m\omega\}}
	\end{displaymath}
	%
	tiende a $0$. Tomando l\'{\i}mite t\'ermino a t\'ermino en
	\eqref{eq:funciones:fourier:dos}, se deduce que, $a(n,r,m)=0$ cuando
	$n\neq 0$. En particular, en el l\'{\i}mite, s\'olo sobreviven
	los t\'erminos con $n=r=0$: el l\'{\i}mite $\nu\to\infty$ existe
	y es igual a
	\begin{displaymath}
		\lim_{\nu\to\infty}\,F(\tau_\nu,z_\nu,\omega)\,=\,
			\sum_{m\geq 0}\,a(0,0,m)\,
				\varexp^{2\pi\raizcuarta \omega m}
		\dispstop
	\end{displaymath}
	%
	La nueva serie converge a.u./c. de $\semiplano[1]$.
	La funci\'on resultante, $\Siegel F(\omega)$ es holomorfa
	en $\semiplano[1]$ y acotada en regiones de la forma
	$\{\omega\geq\raizcuarta c\}$.
	Veamos que es de peso $k$ invariante para
	$\modulgruppe[1]=\SL(2,\Enteros)$.
	Dado $\omega\in\semiplano[1]$, elegimos la sucesi\'on
	$(\raizcuarta\nu,0,\omega)$.
	Dada $\gamma=\sbmatrix{ a & b \\ c & d }\in\SL(2,\Enteros)$,
	la matriz
	\begin{displaymath}
		\tilde\gamma\,=\,
			\begin{bmatrix}
				a & & b & \\
				& 1 & & \\
				c & & d & \\
				& & & 1
				% 1 & & & \\
				% & a & & b \\
				% & & 1 & \\
				% & c & & d
			\end{bmatrix}
	\end{displaymath}
	%
	pertenece a $\modulgruppe[2]$. Actuando en un t\'ermino de la
	sucesi\'on por este elemento,
	\begin{displaymath}
		\begin{aligned}
			\tilde\gamma\accion{(\raizcuarta\nu,0,\omega)} & \,=\,
				\Big(\sbmatrix{ a & \\ & 1 }\,
				\sbmatrix{ \omega & \\ & \raizcuarta\nu }
				\,+\,\sbmatrix{ b & \\ & \phantom{0} }\Big)\,
				\Big(\sbmatrix{ c & \\ & \phantom{0} }\,
				\sbmatrix{ \omega & \\ & \raizcuarta\nu }
				\,+\,\sbmatrix{ d & \\ & 1 }\Big)^{-1} \\
			& \,=\,\big(\raizcuarta\nu,0,
				\tfrac{a\omega+b}{c\omega+d}\big)
			\dispstop
		\end{aligned}
		%
	\end{displaymath}
	%
	Tomando l\'{\i}mite $\nu\to\infty$ en la igualdad
	\begin{displaymath}
		F(\tilde\gamma\accion{(\raizcuarta\nu,0,\omega)})=
			\det\Big(\sbmatrix{ c & \\ & \phantom{0} }\,
				\sbmatrix{ \omega & \\ & \raizcuarta\nu }
				\,+\,\sbmatrix{ d & \\ & 1 }\Big)^k\,
			F(\raizcuarta\nu,0,\omega)
			\,=\,(c\omega+d)^k\,F(\raizcuarta\nu,0,\omega)
		\dispcomma
	\end{displaymath}
	%
	se concluye que
	\begin{displaymath}
		\Siegel F(\gamma\omega)\,=\,
			\factor(\gamma,\omega)^k\,\Siegel F(\omega)
		\dispcomma
	\end{displaymath}
	%
	es decir, $\Siegel F$ es de peso $k$
	invariante para $\SL(2,\Enteros)$.
\end{proof}

\begin{defEisenstein}\label{def:cuspidal}
	Una forma de Siegel $F$ es \emph{cuspidal}, si $\Siegel F=0$.
\end{defEisenstein}

\begin{obsEisenstein}\label{obs:cuspidal}
	Dado que toda $T=\binaria{n,r,m}$ singular es equivalente
	a $\binaria{0,0,*}$ y que, en tal caso, $a(n,r,m)=a(0,0,*)$,
	deducimos que $\Siegel F=0$, si y s\'olo si
	$a(n,r,m)=0$ implica $\binaria{n,r,m}>0$.
\end{obsEisenstein}

Sea $\Delta^+\subgrpeq\modulgruppe[2]$ el subgrupo de matrices de la forma
\begin{displaymath}
	\begin{bmatrix}
		U & S\trnsp U^{-1} \\ & \trnsp U^{-1}
	\end{bmatrix}
	\dispcomma\quad U\in\GL(2,\Enteros) \dispand \trnsp S=S
	\dispstop
\end{displaymath}
%
Sea $\Gamma_\infty\subgrpeq\modulgruppe[2]$ el subgrupo de matrices de la forma
\begin{displaymath}
	M\,=\,\begin{bmatrix}
		a & & b & * \\
		* & * & * & * \\
		c & & d & * \\
		& & & *
	\end{bmatrix}
	\dispcomma\quad
	\begin{bmatrix}
		a & b \\ c & d
	\end{bmatrix}\,\in\,\SL(2,\Enteros)
	\dispstop
\end{displaymath}
%
Si $\Omega=(\tau,z,\omega)\in\semiplano[2]$, sea $\Omega^*=\omega$.

\begin{defEisenstein}\label{def:eisenstein}
	Sea $f\in\spitzenformen[k](\modulgruppe[1])$, $k>0$ par ($r=1$),
	o bien una constante ($r=0$), y sea $\Omega\in\semiplano[2]$.
	La \emph{serie de Eisenstein asociada a $f$ en $\Omega$} es
	\begin{displaymath}
		\Eis[2,r,k](\Omega;f)\,=\,
			\sum_{M\in C_{2,r}\backslash\modulgruppe[2]}\,
				\frac{f(M\accion \Omega^*)}{\det(C\Omega+D)^k}
		\dispcomma
	\end{displaymath}
	%
	donde $C_{2,0}=\Delta^+$ y $C_{2,1}=\Gamma_\infty$.
\end{defEisenstein}

\begin{teoEisenstein}\label{teo:eisenstein}
	Sean $r=0,1$, $k>r+3$ par. Si
	$f\in\spitzenformen[k](\modulgruppe[1])$ ($r=1$),
	o constante ($r=0$), la serie de Eisenstein $\Eis[g,r,k](\Omega;f)$
	converge absoluta y uniformemente en bandas verticales.%
	\footnote{
		Regiones de la forma
		$\{\traza(X)\leq c^{-1},\,Y\geq c\Id[2]\}$.
	}
	Adem\'as,
	\begin{displaymath}
		\Siegel^{2-r}\Eis[2,r,k](-;f)\,=\,f
		\dispstop
	\end{displaymath}
	%
	El espacio $\modulformen[k](\modulgruppe[2])$ est\'a generado por
	las series de Eisenstein $\Eis[2,r,k](-;f)$ ($r=0,1$)
	y por las formas cuspidales $\spitzenformen[k](\modulgruppe[2])$.
\end{teoEisenstein}



\subsection{El desarrollo de las series de Eisenstein}\label{subsec:fourier}
\newcommand{\Bernoulli}[1][]{\ensuremath{\mathit{B}_{#1}}}
\newcommand{\divisores}[1][]{\ensuremath{\sigma_{#1}}}

\begin{teoEjemplos}\label{teo:ejemplos:eisenstein:desarrollo}
	La serie de Eisenstein $\varvarEis[k](z)$, $k>2$, par,
	admite el siguiente desarrollo:
	\begin{equation}
		\label{eq:ejemplos:eisenstein:desarrollo}
		\varvarEis[k](z)\,=\,-\frac{\Bernoulli[k]}{2k}
			\,+\,\sum_{n\geq 1}\,\divisores[k-1](n)\,q^n
	\end{equation}
	%
\end{teoEjemplos}

Los n\'umeros de Bernoulli $\Bernoulli[k]$ est\'an dados por la
siguiente f\'ormula
\begin{equation}
	\label{eq:ejemplos:bernoulli}
	\big(\varexp^\indet-1\big)\,
	\sum_{k\geq 0}\,\frac{\Bernoulli[k]}{k!}\,\indet^k
	\,=\,\indet
	\dispstop
\end{equation}
%
Recursivamente, se pueden calcular los valores de $\Bernoulli[k]$.

\begin{proof}
	El \teoname~\ref{teo:ejemplos:eisenstein:desarrollo}
	es consecuencia de la siguiente identidad:%
	\footnote{
		La serie $\sum_{n\in\Enteros}$ se interpreta como
		$\lim_{N\to\infty}\,\sum_{n=-N}^N$ (o de $-M$ a $N$,
		con $M$ y $N$ tendiendo a $\infty$ de manera que
		$|M-N|$ est\'e acotado).
	}
	\begin{equation}
		\label{eq:ejemplos:identidad}
		\sum_{n\in\Enteros}\,\frac 1{z+n}\,=\,
		\frac \pi{\tan\,\pi z}
		\dispstop
	\end{equation}
	%
	La funci\'on $\pi/\tan\,\pi z$, que tiene per\'{\i}odo $1$,
	admite el siguiente desarrollo:%
	\footnote{
		Escribir la definici\'on de tangente, de seno y coseno\dots
	}
	\begin{displaymath}
		\frac \pi{\tan\,\pi z}\,=\,-2\pi\raizcuarta\,
		\Big(\frac 1 2+
			\sum_{r\geq 1}\,q^r\Big)
		\dispstop
	\end{displaymath}
	%
	Derivar $k-1$ veces dentro de la sumatoria (esto es correcto para
	$k\geq 2$ y $z\in\semiplano$) y dividir por $(-1)^{k-1}\,(k-1)!$
	para deducir la f\'ormula de Lipschitz:
	\begin{displaymath}
		\sum_{n\in\Enteros}\,\frac 1{(z+n)^k}\,=\,
			\frac{(-2\pi\raizcuarta)^k}{(k-1)!}\,
				\sum_{r\geq 1}\,r^{k-1}\,q^r
		\dispstop
	\end{displaymath}
	%
	Separar la sumatoria $\varEis[k](z)$ ($k>2$) en $m=0$ y $m\neq 0$
	y usar que
	\begin{math}
		\zeta(k)=-\frac{(2\pi\raizcuarta)^k}{(k-1)!}\,
			\frac{\Bernoulli[k]}{2k}
	\end{math}.
\end{proof}

\begin{obsEjemplos}\label{obs:ejemplos:fourier}
	Para $k\in\{4,6,8\}$, los desarrollos de Fourier de las series
	$\Eis[k](z)$ comienzan de la siguiente manera:
	\begin{displaymath}
		\begin{aligned}
			\Eis[4](z) & \,=\,1\,+\,240\,q\,+\,2160\,q^2\,+\,
				\dots\dispcomma \\
			\Eis[6](z) & \,=\,1\,-\,504\,q\,-\,16632\,q^2\,-\,
				\dots\dispand \\
			\Eis[8](z) & \,=\,1\,+\,480\,q\,+\,61920\,q^2\,+\,
				\dots\dispstop
		\end{aligned}
		%
	\end{displaymath}
	%
	Si notamos que $\dim\,\modulformen[k]=1$, si $k\in\{4,6,8,10,14\}$,
	podemos deducir algunas identidades entre las series de Eisenstein,
	as\'{\i} como identidades que involucran sumas de potencias de
	divisores:
	\begin{displaymath}
		\begin{aligned}
			& \Eis[4](z)^2\,=\,\Eis[8](z)\dispcomma
			\Eis[4](z)\Eis[6](z)\,=\,\Eis[10](z)\dispand \\
			& \Eis[6](z)\Eis[8](z)\,=\,\Eis[4](z)\Eis[10](z)\,=\,
				\Eis[14](z)
			\dispstop
		\end{aligned}
		%
	\end{displaymath}
	%
	An\'alogamente, como $\dim\,\modulformen[12]=2$, debe existir
	una relaci\'on lineal entre
	$\Eis[4](z)\Eis[8](z)$, $\Eis[6](z)^2$ y $\Eis[12](z)$:
	\begin{displaymath}
		441\,\Eis[4](z)\Eis[8](z)\,+\,250\,\Eis[6](z)^2\,=\,
			691\,\Eis[12](z)
		\dispstop
	\end{displaymath}
	%
\end{obsEjemplos}



\subsection{La serie de Eisenstein de peso $2$}\label{subsec:dos}
La serie dada por \eqref{eq:ejemplos:eisenstein:desarrollo}
converge tambi\'en para $k=2$ y define una funci\'on
holomorfa en $\semiplano$.
\begin{defEjemplos}\label{def:ejemplos:eisenstein:dos}
	La \emph{serie de Eisenstein de peso $2$} es
	la serie
	\begin{displaymath}
		\varvarEis[2](z)\,=\,-\frac 1{24}\,+\,
		\sum_{n\geq 1}\,\divisores[1](n)\,q^n
		\dispcomma
	\end{displaymath}
	%
	o bien cualquiera de las renormalizaciones
	$\varEis[2](z)=-4\pi^2\varvarEis[2](z)$ o
	$\Eis[2](z)=\frac 6 \pi^2\,\varEis[2](z)$.
\end{defEjemplos}

Sin embargo, la serie de Eisenstein de peso $2$ no es una
forma modular (como ya sabemos por la cota del \coroname~%
\ref{coro:definiciones:cota}).

\begin{teoEjemplos}\label{teo:ejemplos:dos}
	Si $z\in\semiplano$ y
	$\gamma=\sbmatrix{ * & * \\ c & d }\in\modulgruppe$,
	entonces
	\begin{displaymath}
		% \varEis[2](\gamma\accion z)\,=\,
		% (cz+d)^2\varEis[2](z)\,-\,\pi\raizcuarta c\,
			% (cz+d)
		\big(\varEis[2]\baroperador[2]\gamma\big)(z)\,=\,
			\varEis[2](z)\,-\,\frac{2\pi\raizcuarta c}{cz+d}
		\dispstop
	\end{displaymath}
	%
\end{teoEjemplos}




\subsection{La funci\'on discriminante}\label{subsec:discriminante}
\begin{defEjemplos}\label{def:ejemplos:discriminante}
	La \emph{funci\'on discriminante} es la funci\'on
	$\Delta:\,\semiplano\rightarrow\Complejos$ dada por
	\begin{equation}
		\label{eq:ejemplos:discriminante}
		\Delta(z)\,=\,\varexp^{2\pi\raizcuarta z}\,
		\prod_{n\geq 1}\,\big(1-
			\varexp^{2\pi\raizcuarta nz}\big)^{24}
		\dispstop
	\end{equation}
	%
\end{defEjemplos}

\begin{teoEjemplos}\label{teo:ejemplos:discriminante}
	El producto \eqref{eq:ejemplos:discriminante} converge
	y define una funci\'on holomorfa en $\semiplano$.
	M\'as aun, $\Delta(z)$ es una forma modular de peso
	$12$ y $a_0(\Delta)=0$.
\end{teoEjemplos}

\begin{proof}
	Tomando el logaritmo $\log\,\Delta(z)$ y derivando, deducimos
	\begin{displaymath}
		\frac 1{2\pi\raizcuarta}\,\big(\log\,\Delta(z)\big)'\,=\,
			1-24\,\sum_{n\geq 1}\,\frac{n\,q^n}{1-q^n}\,=\,
			1-24\,\sum_{m\geq 1}\,\divisores[1](m)\,q^m\,=\,
			\Eis[2](z)
		\dispstop
	\end{displaymath}
	%
	Si $\gamma=\sbmatrix{ * & * \\ c & d }\in\modulgruppe$ y
	$z\in\semiplano$, entonces
	\begin{displaymath}
		\frac 1{2\pi\raizcuarta}\,\bigg(\log\,\bigg\{
			\frac{\Delta(\gamma\accion z)}{%
				(cz+d)^{12}\,\Delta(z)}\bigg\}\bigg)' \,=\,
			\frac 1{(cz+d)^2}\,\Eis[2](\gamma\accion z)\,-\,
			\frac{12}{2\pi\raizcuarta}\,\frac c{cz+d}\,-\,
			\Eis[2](z)
		\dispcomma
	\end{displaymath}
	%
	que es $=0$, por el \teoname~\ref{teo:ejemplos:dos}. En particular,
	para cada $\gamma\in\modulgruppe$, existe una constante
	$C(\gamma)\neq 0$ tal que
	$\big(\Delta\baroperador[12]\gamma\big)(z)=C(\gamma)\,\Delta(z)$,
	si $z\in\semiplano$. La funci\'on
	$C:\,\modulgruppe\rightarrow\Unidades\Complejos$ es morfismo de
	grupos y verifica $C(T)=C(S)=1$,%
	\footnote{
		En el caso de $T$, se deduce de que
		$\varexp^{2\pi\raizcuarta z}$ es peri\'odica.
		En el caso de $S$, se deduce de reemplazar $z=\raizcuarta$
		en la identidad $\Delta(-1/z)=C(S)\,z^{12}\,\Delta(z)$ y
		notar que $\Delta(\raizcuarta)\neq 0$.
	}
	con lo que $C(\gamma)=1$ para toda $\gamma\in\modulgruppe$.
\end{proof}

\begin{obsEjemplos}\label{obs:ejemplos:discriminante}
	Dado que $\dim\,\modulformen[12]=2$, debe existir una relaci\'on
	lineal entre $\Eis[4](z)^3$, $\Eis[6](z)^2$ y $\Delta(z)$.
	Mirando los coeficientes de Fourier,
	\begin{displaymath}
		1728\,\Delta(z)\,=\, \Eis[4](z)^3-\Eis[6](z)^2
		\dispstop
	\end{displaymath}
	%
\end{obsEjemplos}



