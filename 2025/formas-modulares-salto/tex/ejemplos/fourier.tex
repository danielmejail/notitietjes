\newcommand{\Bernoulli}[1][]{\ensuremath{\mathit{B}_{#1}}}
\newcommand{\divisores}[1][]{\ensuremath{\sigma_{#1}}}

\begin{teoEjemplos}\label{teo:ejemplos:eisenstein:desarrollo}
	La serie de Eisenstein $\varvarEis[k](z)$, $k>2$, par,
	admite el siguiente desarrollo:
	\begin{equation}
		\label{eq:ejemplos:eisenstein:desarrollo}
		\varvarEis[k](z)\,=\,-\frac{\Bernoulli[k]}{2k}
			\,+\,\sum_{n\geq 1}\,\divisores[k-1](n)\,q^n
	\end{equation}
	%
\end{teoEjemplos}

Los n\'umeros de Bernoulli $\Bernoulli[k]$ est\'an dados por la
siguiente f\'ormula
\begin{equation}
	\label{eq:ejemplos:bernoulli}
	\big(\varexp^\indet-1\big)\,
	\sum_{k\geq 0}\,\frac{\Bernoulli[k]}{k!}\,\indet^k
	\,=\,\indet
	\dispstop
\end{equation}
%
Recursivamente, se pueden calcular los valores de $\Bernoulli[k]$.

\begin{proof}
	El \teoname~\ref{teo:ejemplos:eisenstein:desarrollo}
	es consecuencia de la siguiente identidad:%
	\footnote{
		La serie $\sum_{n\in\Enteros}$ se interpreta como
		$\lim_{N\to\infty}\,\sum_{n=-N}^N$ (o de $-M$ a $N$,
		con $M$ y $N$ tendiendo a $\infty$ de manera que
		$|M-N|$ est\'e acotado).
	}
	\begin{equation}
		\label{eq:ejemplos:identidad}
		\sum_{n\in\Enteros}\,\frac 1{z+n}\,=\,
		\frac \pi{\tan\,\pi z}
		\dispstop
	\end{equation}
	%
	La funci\'on $\pi/\tan\,\pi z$, que tiene per\'{\i}odo $1$,
	admite el siguiente desarrollo:%
	\footnote{
		Escribir la definici\'on de tangente, de seno y coseno\dots
	}
	\begin{displaymath}
		\frac \pi{\tan\,\pi z}\,=\,-2\pi\raizcuarta\,
		\Big(\frac 1 2+
			\sum_{r\geq 1}\,q^r\Big)
		\dispstop
	\end{displaymath}
	%
	Derivar $k-1$ veces dentro de la sumatoria (esto es correcto para
	$k\geq 2$ y $z\in\semiplano$) y dividir por $(-1)^{k-1}\,(k-1)!$
	para deducir la f\'ormula de Lipschitz:
	\begin{displaymath}
		\sum_{n\in\Enteros}\,\frac 1{(z+n)^k}\,=\,
			\frac{(-2\pi\raizcuarta)^k}{(k-1)!}\,
				\sum_{r\geq 1}\,r^{k-1}\,q^r
		\dispstop
	\end{displaymath}
	%
	Separar la sumatoria $\varEis[k](z)$ ($k>2$) en $m=0$ y $m\neq 0$
	y usar que
	\begin{math}
		\zeta(k)=-\frac{(2\pi\raizcuarta)^k}{(k-1)!}\,
			\frac{\Bernoulli[k]}{2k}
	\end{math}.
\end{proof}

\begin{obsEjemplos}\label{obs:ejemplos:fourier}
	Para $k\in\{4,6,8\}$, los desarrollos de Fourier de las series
	$\Eis[k](z)$ comienzan de la siguiente manera:
	\begin{displaymath}
		\begin{aligned}
			\Eis[4](z) & \,=\,1\,+\,240\,q\,+\,2160\,q^2\,+\,
				\dots\dispcomma \\
			\Eis[6](z) & \,=\,1\,-\,504\,q\,-\,16632\,q^2\,-\,
				\dots\dispand \\
			\Eis[8](z) & \,=\,1\,+\,480\,q\,+\,61920\,q^2\,+\,
				\dots\dispstop
		\end{aligned}
		%
	\end{displaymath}
	%
	Si notamos que $\dim\,\modulformen[k]=1$, si $k\in\{4,6,8,10,14\}$,
	podemos deducir algunas identidades entre las series de Eisenstein,
	as\'{\i} como identidades que involucran sumas de potencias de
	divisores:
	\begin{displaymath}
		\begin{aligned}
			& \Eis[4](z)^2\,=\,\Eis[8](z)\dispcomma
			\Eis[4](z)\Eis[6](z)\,=\,\Eis[10](z)\dispand \\
			& \Eis[6](z)\Eis[8](z)\,=\,\Eis[4](z)\Eis[10](z)\,=\,
				\Eis[14](z)
			\dispstop
		\end{aligned}
		%
	\end{displaymath}
	%
	An\'alogamente, como $\dim\,\modulformen[12]=2$, debe existir
	una relaci\'on lineal entre
	$\Eis[4](z)\Eis[8](z)$, $\Eis[6](z)^2$ y $\Eis[12](z)$:
	\begin{displaymath}
		441\,\Eis[4](z)\Eis[8](z)\,+\,250\,\Eis[6](z)^2\,=\,
			691\,\Eis[12](z)
		\dispstop
	\end{displaymath}
	%
\end{obsEjemplos}

