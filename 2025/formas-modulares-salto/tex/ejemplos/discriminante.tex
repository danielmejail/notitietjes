\begin{defEjemplos}\label{def:ejemplos:discriminante}
	La \emph{funci\'on discriminante} es la funci\'on
	$\Delta:\,\semiplano\rightarrow\Complejos$ dada por
	\begin{equation}
		\label{eq:ejemplos:discriminante}
		\Delta(z)\,=\,\varexp^{2\pi\raizcuarta z}\,
		\prod_{n\geq 1}\,\big(1-
			\varexp^{2\pi\raizcuarta nz}\big)^{24}
		\dispstop
	\end{equation}
	%
\end{defEjemplos}

\begin{teoEjemplos}\label{teo:ejemplos:discriminante}
	El producto \eqref{eq:ejemplos:discriminante} converge
	y define una funci\'on holomorfa en $\semiplano$.
	M\'as aun, $\Delta(z)$ es una forma modular de peso
	$12$ y $a_0(\Delta)=0$.
\end{teoEjemplos}

\begin{proof}
	Tomando el logaritmo $\log\,\Delta(z)$ y derivando, deducimos
	\begin{displaymath}
		\frac 1{2\pi\raizcuarta}\,\big(\log\,\Delta(z)\big)'\,=\,
			1-24\,\sum_{n\geq 1}\,\frac{n\,q^n}{1-q^n}\,=\,
			1-24\,\sum_{m\geq 1}\,\divisores[1](m)\,q^m\,=\,
			\Eis[2](z)
		\dispstop
	\end{displaymath}
	%
	Si $\gamma=\sbmatrix{ * & * \\ c & d }\in\modulgruppe$ y
	$z\in\semiplano$, entonces
	\begin{displaymath}
		\frac 1{2\pi\raizcuarta}\,\bigg(\log\,\bigg\{
			\frac{\Delta(\gamma\accion z)}{%
				(cz+d)^{12}\,\Delta(z)}\bigg\}\bigg)' \,=\,
			\frac 1{(cz+d)^2}\,\Eis[2](\gamma\accion z)\,-\,
			\frac{12}{2\pi\raizcuarta}\,\frac c{cz+d}\,-\,
			\Eis[2](z)
		\dispcomma
	\end{displaymath}
	%
	que es $=0$, por el \teoname~\ref{teo:ejemplos:dos}. En particular,
	para cada $\gamma\in\modulgruppe$, existe una constante
	$C(\gamma)\neq 0$ tal que
	$\big(\Delta\baroperador[12]\gamma\big)(z)=C(\gamma)\,\Delta(z)$,
	si $z\in\semiplano$. La funci\'on
	$C:\,\modulgruppe\rightarrow\Unidades\Complejos$ es morfismo de
	grupos y verifica $C(T)=C(S)=1$,%
	\footnote{
		En el caso de $T$, se deduce de que
		$\varexp^{2\pi\raizcuarta z}$ es peri\'odica.
		En el caso de $S$, se deduce de reemplazar $z=\raizcuarta$
		en la identidad $\Delta(-1/z)=C(S)\,z^{12}\,\Delta(z)$ y
		notar que $\Delta(\raizcuarta)\neq 0$.
	}
	con lo que $C(\gamma)=1$ para toda $\gamma\in\modulgruppe$.
\end{proof}

\begin{obsEjemplos}\label{obs:ejemplos:discriminante}
	Dado que $\dim\,\modulformen[12]=2$, debe existir una relaci\'on
	lineal entre $\Eis[4](z)^3$, $\Eis[6](z)^2$ y $\Delta(z)$.
	Mirando los coeficientes de Fourier,
	\begin{displaymath}
		1728\,\Delta(z)\,=\, \Eis[4](z)^3-\Eis[6](z)^2
		\dispstop
	\end{displaymath}
	%
\end{obsEjemplos}

