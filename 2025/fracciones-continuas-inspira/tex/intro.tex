\begin{pregunta}\label{preg:intro}
	Dado un n\'umero (real), aproximarlo por fracciones.
\end{pregunta}

\begin{ejemplo}\label{ejem:intro}
	Sea $\alpha\in\Reales$ un n\'umero cuyo desarrollo decimal
	comienza de la siguiente manera: $\alpha=1,13732\dots$
	?`Cu\'an cerca queremos (podemos) aproximarlo?
	Digamos que queremos aproximarlo a tres decimales.
	La fracci\'on $\frac{1137}{1000}$ conincide con $\alpha$
	en los primeros tres decimales.

	Otra manera de aproximar es indicar un intervalo, rango,
	en donde tengamos la certeza de que $\alpha$ se encuentra.
	Por ejemplo, $\alpha\in\aaIntervalo{1,1373}{1,1374}$.
	Hay otras fracciones en este intervalo;
	la $\frac{11373}{10^4}$ est\'a justo en el borde.
	Talvez haya una ``mejor''. Adem\'as, su denominador es
	m\'as grande que el de $\frac{1137}{10^3}$
	?`Podemos aproximarlo con la misma (o mejor) precisi\'on,
	pero usando denominadores m\'as chicos?

	\paragraph{Primera aproximaci\'on}
	Una aproximaci\'on sencilla es el piso: $\piso\alpha=1$.
	La diferencia (el error) es la parte fraccionaria:
	$\fraccionaria\alpha=\alpha-1=0,1373\dots<1$.
	\paragraph{Segunda aproximaci\'on}
	La idea es, en una segunda instancia, aproximar
	$\fraccionaria\alpha$. Pero $\fraccionaria\alpha<1$,
	as\'{\i} que, en su lugar, intentamos aproximar
	$\alpha_1=\frac 1{\fraccionaria\alpha}>1$. Podemos hacer
	lo mismo que hicimos con $\alpha$ ?`Cu\'anto es
	$\piso{\alpha_1}$? Para encontrar este valor, buscamos
	un intervalo que contenga $\alpha_1$ y sea lo
	suficientemente bueno (estrecho) como para determinar
	el valor que buscamos. Ahora,
	\begin{displaymath}
		\fraccionaria\alpha\in
		\aaIntervalo{0,1373}{0,1374}\,=\,
		\aaIntervalo{\frac{1373}{10^4}}{\frac{1374}{10^4}}
		\dispcomma
	\end{displaymath}
	%
	con lo que
	\begin{math}
		\alpha_1\in
		\aaIntervalo{\frac{10^4}{1374}}{\frac{10^4}{1373}}
	\end{math}.
	Pero
	\begin{displaymath}
		\begin{aligned}
			10^4 \,=\,7\cdot 1374+382
			\dispcomma\quad &
			\frac{10^4}{1374}\,=\,7+\frac{382}{1374}
			\dispcomma \\
			10^4\,=\,7\cdot 1373+389
			\dispand &
			\frac{10^4}{1373}\,=\,7+\frac{389}{1373}
			\dispstop
		\end{aligned}
		%
	\end{displaymath}
	%
	O sea, $\piso{\alpha_1}=7$ y conseguimos la siguiente
	aproximaci\'on para $\alpha$:
	\begin{displaymath}
		\alpha\,=\,1+\frac 1{\alpha_1}\,=\,
			1+\cfrac 1 {7+\fraccionaria{\alpha_1}}
			\,\in\,
			\aaIntervalo{1+\frac 1 8}{1+\frac 1 7}
		\dispstop
	\end{displaymath}
	%
	Los extremos del intervalo son $1+1/8=9/8$ y $1+1/7=8/7$.
	En particular, el extremo derecho, $8/7$,
	obtenido despreciando la parte fraccionaria de $\alpha_1$,
	aproxima bien a un decimal. Adem\'as, sabemos que
	\begin{math}
		\fraccionaria{\alpha_1}\in
		\aaIntervalo{\frac{382}{1374}}{\frac{389}{1373}}
	\end{math}.
	\paragraph{Tercera aproximaci\'on}
	Ahora bien, podemos seguir este procedimiento, aproximando
	$\alpha_2=\frac 1 {\fraccionaria{\alpha_1}}$. Sabemos que
	\begin{math}
		\alpha_2\in
		\aaIntervalo{\frac{1373}{389}}{\frac{1374}{382}}
	\end{math}.
	Pero
	\begin{displaymath}
		\begin{aligned}
			1373 \,=\,3\cdot 389+206 \dispcomma\quad &
			\frac{1373}{389}\,=\,3+\frac{206}{389}
			\dispcomma \\
			1374 \,=\,3\cdot 382+228 \dispand &
			\frac{1374}{382}\,=\,3+\frac{228}{382}
			\dispstop
		\end{aligned}
		%
	\end{displaymath}
	%
	En particular, $\piso{\alpha_2}=3$ y
	\begin{math}
		\fraccionaria{\alpha_2}\in
		\aaIntervalo{\frac{206}{389}}{\frac{228}{382}}
	\end{math}.
	As\'{\i}, notando que $\fraccionaria{\alpha_2}\in\aaIntervalo 0 1$,
	deducimos que
	\begin{displaymath}
		\alpha\,=\,1+\cfrac 1{7+\cfrac 1{3+\fraccionaria{\alpha_2}}}
		\,\in\,
		\aaIntervalo{1+\cfrac 1{7+\cfrac 1 3}}{1+\cfrac 1{7+\cfrac 1 4}}
		\dispstop
	\end{displaymath}
	%
	Los extremos del intervalo son
	\begin{displaymath}
		1+\cfrac 1 {7+\cfrac 1 3} \,=\,\frac{25}{22}
		\dispand
		1+\cfrac 1 {7+\cfrac 1 4} \,=\,\frac{33}{29}
		\dispstop
	\end{displaymath}
	%
	El extremo \emph{izquierdo}, $25/22$, obtenido despreciando
	la parte fraccionaria de $\alpha_2$, aproxima
	correctamente a dos decimales.
	\paragraph{Cuarta aproximaci\'on}
	Seguimos con $\alpha_3=\frac 1 {\fraccionaria{\alpha_2}}$,
	que se encuentra en el rango
	\begin{math}
		\alpha_3\in
		\aaIntervalo{\frac{382}{228}}{\frac{389}{206}}
	\end{math}.
	Haciendo las divisiones correspondientes,
	\begin{displaymath}
		\begin{aligned}
			382 \,=\,1\cdot 228+154 \dispcomma\quad &
				\frac{382}{228}\,=\,1+\frac{154}{228}
				\dispcomma \\
			389\,=\,1\cdot 206+183 \dispand &
				\frac{389}{206}\,=\,1+\frac{183}{206}
			\dispstop
		\end{aligned}
		%
	\end{displaymath}
	%
	El piso de $\alpha_3$ es $\piso{\alpha_3}=1$,
	mientras que
	\begin{math}
		\fraccionaria{\alpha_3}\in
		\aaIntervalo{\frac{154}{228}}{\frac{183}{206}}
	\end{math}.
	Entonces, podemos encerrar $\alpha$ en el siguiente
	intervalo:
	\begin{displaymath}
		\alpha\,=\,
		1+\cfrac 1{7+\cfrac 1 {3+ \cfrac 1 {\fraccionaria{\alpha_3}}}}
		\,\in\,
		\aaIntervalo{%
		1+\cfrac 1{7+\cfrac 1{3+\cfrac 1 2}}}{%
			1+\cfrac 1{7+\cfrac 1{3+\cfrac 1 1}}}
		\dispstop
	\end{displaymath}
	%
	Los extremos del intervalo son:
	\begin{displaymath}
		1+\cfrac 1{7+\cfrac 1{3+\cfrac 1 2}} \,=\,
			\frac{58}{51}\dispand
		1+\cfrac 1{7+\cfrac 1{3+\cfrac 1 1}} \,=\,
			\frac{33}{29}
		\dispstop
	\end{displaymath}
	%
	El extremo \emph{derecho}, $33/29$, obtenido despreciando
	la parte fraccionaria de $\alpha_3$ coincide con $\alpha$
	en los primeros tres decimales.
	\paragraph{Continuaci\'on}
	Podr\'{\i}amos seguir ?`Hasta qu\'e punto? ?`Cu\'antos pasos
	podemos hacer con lo que sabemos de $\alpha$ hasta ahora,
	es decir, que $\alpha=1,13732\dots$?
	Para empezar, $\alpha_4=\frac 1{\fraccionaria{\alpha_3}}$
	pertenece al intervalo
	$\aaIntervalo{\frac{206}{183}}{\frac{228}{154}}$.
	En particular, $\piso{\alpha_4}=1$ y
	\begin{math}
		\fraccionaria{\alpha_4}\in
		\aaIntervalo{\frac{23}{206}}{\frac{74}{228}}
	\end{math}.
	De $\alpha$, podemos decir que
	\begin{displaymath}
		\alpha\,=\,
		1+\cfrac 1{%
			7+\cfrac 1{%
				3+\cfrac 1{%
					1+\cfrac 1 {%
						1+\fraccionaria{\alpha_4}}}}}
		\,\in\,
		\aaIntervalo{%
			1+\cfrac 1{%
			7+\cfrac 1{%
			3+\cfrac 1{%
			1+\cfrac 1 {%
			1+0}}}}}{%
			1+\cfrac 1{%
			7+\cfrac 1{%
			3+\cfrac 1{%
			1+\cfrac 1 {%
			1+1}}}}}
		\dispcomma
	\end{displaymath}
	%
	o sea,
	\begin{math}
		\alpha\in
		\aaIntervalo{\frac{58}{51}}{\frac{91}{80}}
	\end{math}.
	Notemos que no conseguimos una aproximaci\'on correcta
	a m\'as de tres decimales en este paso.

	Ahora, $\alpha_5=\frac 1{\fraccionaria{\alpha_4}}$
	pertenece al intervalo
	$\aaIntervalo{\frac{228}{74}}{\frac{206}{23}}$.
	Si hacemos las divisiones, vemos que
	$228=3\cdot 74+6$ y que
	$206=8\cdot 23+22$ !`No podemos precisar el valor
	de $\piso{\alpha_5}$! !`S\'olo sabemos que
	$\piso{\alpha_5}\in\iIntervalo 3 8$! En todo caso,
	podemos decir que $\alpha$ se encuentra contenido
	en el intervalo
	\begin{displaymath}
		\aaIntervalo{
			1+\cfrac 1{7+\cfrac 1{3+\cfrac 1{1+\cfrac 1{1+\cfrac 1 8}}}}}{
			1+\cfrac 1{7+\cfrac 1{3+\cfrac 1{1+\cfrac 1{1+\cfrac 1 3}}}}}
		\,=\,\aaIntervalo{\frac{497}{437}}{\frac{207}{182}}
		\dispstop
	\end{displaymath}
	%
	En decimales, $\frac{497}{437}=1,13729\dots$ y
	$\frac{207}{182}=1,13736\dots$.
	Tiene sentido. Con la aproximaci\'on inicial de $\alpha$
	con la que contamos, no podemos esperar conseguir una
	``mejor'' aproximaci\'on, en el sentido de tener m\'as
	decimales correctos.
\end{ejemplo}

\begin{observacion}\label{obs:intro}
	La fracci\'on $\frac{33}{29}$ aproxima $\alpha$ a tres decimales,
	igual que $\frac{1137}{1000}$. Los denominadores son m\'as chicos;
	la aproximaci\'on es, al menos, tan buena como $\frac{1137}{1000}$,
	pero es ``m\'as sencilla''. Obtuvimos, adem\'as, una sucesi\'on de
	aproximaciones:
	\begin{displaymath}
		1
		% \,=\,\frac 1 1
		\dispcomma\quad
		\frac 8 7
		% \,=\,1 + \frac 1 7
		\dispcomma\quad
		\frac{25}{22}
		% \,=\,1 + \frac 1 { 7 + \frac 1 3 }
		\dispcomma\quad
		\frac{33}{29}
		% \,=\,1 + \frac 1 { 7 + \frac 1 { 3 + \frac 1 1 } }
		\dispstop
	\end{displaymath}
	%
	Estas aproximaciones satisfacen:
	\begin{displaymath}
		1\,<\,\alpha\dispcomma\quad
		\alpha\,<\,\frac 8 7\dispcomma\quad
		\frac{25}{22}\,<\,\alpha\dispcomma\quad
		\alpha\,<\,\frac{33}{29}
		\dispstop
	\end{displaymath}
	%
	Las diferencias entre las sucesivas aproximaciones est\'an dadas por:
	\begin{displaymath}
		\frac 8 7 - 1 \,=\,\frac 1 7\dispcomma\quad
		\frac{25}{22} - \frac 8 7 \,=\,\frac{-1}{154}\dispcomma\quad
		\frac{33}{29} - \frac{25}{22} \,=\,\frac 1 {638}
		% \frac{58}{51}-\frac{33}{29} \,=\,\frac{-1}{1479}
		% \frac{91}{80}-\frac{33}{92} \,=\,\frac{-1}{2320}
		% \frac{497}{437}-\frac{58}{51} \,=\,\frac 1{22287}
		% \frac{207}{182}-\frac{58}{51} \,=\,\frac 1{9282}
		\dispstop
	\end{displaymath}
	%
\end{observacion}

