\begin{definicion}\label{def:fraccion}
	Una expresi\'on de la forma
	\begin{equation}\label{eq:fraccion}
		a_0 + \cfrac 1 { a_1 + \cfrac 1 { a_2 + \dots } }
	\end{equation}
	%
	es una \emph{fracci\'on continua}. Abreviamos
	\eqref{eq:fraccion} por $\contfrac{a_0,a_1,a_2,\dots}$.
\end{definicion}

\begin{ejemplo}\label{ejem:definiciones:otra}
	$\contfrac{1,2}=1+\cfrac 1 2=\frac 3 2$.
\end{ejemplo}

\begin{ejemplo}\label{ejem:definiciones:intro}
	El \ejemname~\ref{ejem:intro} nos provee de algunos ejemplos
	de fracciones continuas:
	\begin{displaymath}
		\begin{aligned}
			\contfrac{1} & \,=\, 1\dispcomma \\
			\contfrac{1,7} & \,=\,1+\cfrac 1 7
				\,=\,\frac 8 7 \dispcomma \\
			\contfrac{1,7,3} & \,=\,
				1 + \cfrac 1 { 7 + \cfrac 1 3 }
				\,=\,\frac{25}{22} \dispcomma \\
			\contfrac{1,7,3,1} & \,=\,
			1 + \cfrac 1 {7 + \cfrac 1 { 3 + \cfrac 1 1 } }
				\,=\,\frac{33}{29}
				\dispstop
		\end{aligned}
		%
	\end{displaymath}
	%
	Las fracciones del lado derecho de cada igualdad son los
	\emph{valores} de la fracciones continuas.
\end{ejemplo}

\begin{ejemplo}\label{ejem:definiciones:pi}
	Al \emph{resolver} la fracci\'on continua
	$\contfrac{3,7,5,1,292}$ obtenemos $\frac{103993}{33102}$
	?`Qu\'e n\'umero parece estar aproximando?
\end{ejemplo}

\begin{definicion}\label{def:convergente}
	Si $\contfrac{a_0,a_1,\dots,a_n}$ es una fracci\'on continua
	y $m\leq n$, podemos ``truncar'' y obtener
	$\contfrac{a_0,a_1,\dots,a_m}$, la
	\emph{convergente parcial $m$-\'esima}.
\end{definicion}

\begin{ejemplo}\label{ejem:definiciones:convergente}
	La primera convergente parcial de $\contfrac{1,7,3,1}$ es
	$\contfrac{1,7}$.
\end{ejemplo}

\begin{observacion}\label{obs:definiciones:finitas}
	Las fracciones continuas ``finitas'' representan n\'umeros
	racionales (fracciones).
\end{observacion}

\begin{pregunta}\label{preg:definiciones:finitas}
	?`Podemos representar todo n\'umero racional como
	fracci\'on continua?
\end{pregunta}

\begin{ejemplo}\label{ejem:definiciones:finitas}
	La fracci\'on $\frac{10} 7$ es representada por
	$\contfrac{1,2,3}$; el valor de $\contfrac{1,2,3}$
	es $\frac{10} 7$:
	\begin{displaymath}
		\begin{aligned}
			10\,=\,1\cdot 7+3 \dispcomma\quad &
				\frac{10} 7\,=\,1+\frac 3 7
				\,=\,1+\cfrac 1 { 7/3 } \dispcomma \\
			7\,=\,2\cdot 3+1 \dispcomma\quad &
				\frac 7 3\,=\,2+\cfrac 1 3 \dispand
				\frac{10} 7 \,=\,
					1+\cfrac 1 { 2 + \cfrac 1 3 }
			\dispstop
		\end{aligned}
		%
	\end{displaymath}
	%
	La fracci\'on $\frac{43}{30}$ es representada por
	$\contfrac{1,2,3,4}$; el valor de $\contfrac{1,2,3,4}$
	es $\frac{43}{30}$:
	\begin{displaymath}
		\begin{aligned}
			43\,=\, 1\cdot 30+13 \dispcomma\quad &
				\frac{43}{30}\,=\, 1+\frac{13}{30}
				\,=\,1+\cfrac 1 {30/13} \dispcomma \\
			30\,=\, 2\cdot 13+4 \dispcomma\quad &
				\frac{30}{13}\,=\,2+\frac 4 {13}
				\,=\,2+\cfrac 1 {13/4} \dispcomma \\
			13\,=\, 3\cdot 4+1 \dispcomma\quad &
				\frac{13} 4\,=\, 3+\frac 1 4 \dispand
				\frac{43}{30}\,=\,
				1+\cfrac 1 {2+\cfrac 1 {3+\cfrac 1 4}}
			\dispstop
		\end{aligned}
		%
	\end{displaymath}
	%
	La fracci\'on $\frac{225}{157}$ es representada por
	$\contfrac{1,2,3,4,5}$; el valor de $\contfrac{1,2,3,4,5}$
	es $\frac{225}{157}$:
	\begin{displaymath}
		\begin{aligned}
			225 \,=\, 1\cdot 157+68 \dispcomma\quad &
				\frac{225}{157} \,=\,1+\frac{68}{157}
				\,=\,1+\cfrac 1{157/68}
				\dispcomma \\
			157 \,=\, 2\cdot 68+21 \dispcomma\quad &
				\frac{157}{68} \,=\,2+\frac{21}{68}
				\,=\,2+\cfrac 1{68/21}
				\dispcomma \\
			68 \,=\, 3\cdot 21+5 \dispcomma\quad &
				\frac{68}{21} \,=\,3+\frac 5{21}
				\,=\,3+\cfrac 1{21/5}
				\dispcomma \\
			21 \,=\, 4\cdot 5+1 \dispcomma\quad &
				\frac{21} 5 \,=\,4+\frac 1 5 \dispand
				\frac{225}{157}\,=\,
				1+\cfrac 1{2+\cfrac 1{3+\cfrac 1{4+\cfrac 1 5}}}
			\dispstop
		\end{aligned}
		%
	\end{displaymath}
	%
\end{ejemplo}

